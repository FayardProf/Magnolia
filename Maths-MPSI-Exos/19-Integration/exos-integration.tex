\documentclass{magnolia}

\magtex{tex_driver={pdftex}}
\magfiche{document_nom={Exercices sur l'intégration},
          auteur_nom={François Fayard},
          auteur_mail={fayard.prof@gmail.com}}
\magexos{exos_matiere={maths},
         exos_niveau={mpsi},
         exos_theme={Intégration}}
\magmisenpage{misenpage_presentation={tikzvelvia},
              misenpage_format={a4},
              misenpage_nbcolonnes={1},
              misenpage_sol={non}}
\maglieudiff{lieu_lycee={Aux Lazaristes},
             lieu_classe={MPSI 1},
             lieu_annee={2019--2020}}
\magprocess

\begin{document}
%BEGIN_BOOK


\magsection{Intégration}

% \exercice{nom={Calcul de quelques intégrales}, utile=1, difficulte=1}
% Calculer les intégrales suivantes~:
% \[\integ{n}{m}{E(x)}{x} \quad \text{pour $n,m\in\Z$}\]
% \[\integ{-1}{2}{x\abs{x}}{x} \qquad \integ{-1}{1}{x\abs{x}}{x} \qquad
%   \integ{-1}{1}{\frac{\sin x}{1+x^2}}{x}\]

\exercice{nom={Calcul de limites}, utile=2, difficulte=2}
\begin{questions}
\question Calculer les limites des expressions suivantes lorsque $n$ tend vers
  $+\infty$
  \[\frac{1}{n!} \integ{0}{1}{\arcsin^n t}{t}, \qquad
    \integ{0}{1}{x^n\ln(1+x^2)}{x}, \qquad \integ{0}{1}{\ln\p{1+x^n}}{x}.\]
\question Calculer la limite de
  \[\integ{0}{1}{x^2\sqrt{1+ax^2}}{x}\]
  lorsque $a$ tend vers $0$.
\end{questions}

\exercice{nom={Calcul de limites}, utile=2, difficulte=2}
\begin{questions}
\question
  \begin{questions}
  \question Donner la limite, lorsque $t$ tend vers 1 de
    \[\frac{1}{\ln t}-\frac{1}{t\ln t}.\]
  \question En déduire la limite lorsque $x$ tend vers $1$ de
    \[\integinv{x}{x^2}{\ln t}{t}.\]
  \end{questions}
\question Donner la limite lorsque $x$ tend vers $0$ de
  \[\integ{x}{3x}{\frac{\sin t}{t^2}}{t}.\]
\end{questions}
\begin{sol}
$\quad$
\begin{questions}
\question
  \begin{questions}
  \question 
    \[\frac{1}{\ln t}-\frac{1}{t\ln t}\tendvers{t}{1}1\]
  \question 
    \[\integinv{x}{x^2}{\ln t}{t}\tendvers{x}{1}\ln 2\]    
  \end{questions}
\question 
  \[\integ{x}{3x}{\frac{\sin t}{t^2}}{t}\tendvers{x}{0}\ln 3\]
\end{questions}
\end{sol}

\exercice{nom={Calcul de limite}, utile=1, difficulte=3}
Soit $f$ une fonction continue sur le segment $\interf{0}{1}$ à valeurs
strictement positives. Pour tout $\alpha>0$, on définit
$$I(\alpha)\defeq \p{\integ{0}{1}{f^\alpha(t)}{t}}^\frac{1}{\alpha}.$$
\begin{questions}
\question Montrer que $I(\alpha)$ converge vers la borne supérieure de
  $f$ lorsque $\alpha$ tend vers $+\infty$.
\question Le but de cet question est de montrer que lorsque $\alpha$ tend vers
  $0$, $I(\alpha)$ tend vers
  \[\exp\p{\integ{0}{1}{\ln(f(t))}{t}}.\]
  \begin{questions}
  \question On suppose dans cette question que $\forall x\in\interf{0}{1}
    \qsep f(x)\geq1$.
    \begin{questions}
    \question Soit $\varepsilon>0$. Montrer qu'il existe $\eta>0$ tel que
      $$\forall x\in\interf{0}{\eta} \qsep 1+(1-\varepsilon)x\leq \e^x
        \leq 1+(1+\varepsilon)x.$$
    \question En déduire qu'il existe $\eta'>0$ tel que
      \[\forall \alpha\in\interof{0}{\eta'} \qsep
        1+(1-\varepsilon)\alpha\ln(f(t)) \leq f^{\alpha} (t) \leq 
        1+(1+\varepsilon)\alpha\ln(f(t)).\]
    \question Conclure
    \end{questions}
  \question Montrer le cas général.
  \end{questions}
\end{questions}


\exercice{nom={Fonction d'intégrale nulle}, utile=2, difficulte=1}
Soit $f$ une fonction continue sur un intervalle $I$. On suppose que
quels que soient $a$ et $b$ dans $I$
\[\integ{a}{b}{f(t)}{t}=0.\]
Montrer que $f$ est nulle.


\exercice{nom={Égalité dans l'inégalité triangulaire}, utile=2, difficulte=2}
Soit $f$ une fonction continue sur le segment $\interf{a}{b}$. On suppose
que
\[\abs{\integ{a}{b}{f(x)}{x}}=\integ{a}{b}{\abs{f(x)}}{x}.\]
Montrer que~:
\begin{questions}
\question Si $f$ est réelle, $f$ garde un signe constant.
\question Si $f$ est complexe, $f$ garde un argument constant.
\end{questions}

% \subsection{Inégalité de Jensen}
% Soit $f$ une fonction réelle continue par morceaux sur $\interf{0}{1}$ et
% $g$ une fonction réelle continue et convexe. On se propose de démontrer
% l'inégalité de Jensen :
% \[g\p{\integ{0}{1}{f(t)}{t}} \leq \integ{0}{1}{g\p{f(t)}}{t}\]
% \begin{questions}
% \question Démontrer que le resultat est vrai lorsque $f$ est une fonction
%   en escalier.
% \question En déduire le cas général.
% \end{questions}

\exercice{nom={Inégalité de Gronwall}, utile=3, difficulte=2}
\begin{questions}
\question Soit $f$ une fonction positive et continue sur $\RP$. On suppose
  qu'il existe un nombre réel $k$ positif tel que
  \[\forall x \in\RP \qsep f(x) \leq k\integ{0}{x}{f(t)}{t}.\]
  Montrer que la fonction $f$ est nulle.
\question Soit $c\in\RP$, $u$ et $v$ deux applications continues et positives
  de $\RP$ dans $\R$ telles que
  \[\forall x\in\RP \qsep u(x)\leq c+\integ{0}{x}{u(t)v(t)}{t}.\]
  Montrer que
  \[\forall x \in\RP \qsep u(x)\leq c\exp\p{\integ{0}{x}{v(t)}{t}}.\]
\end{questions}


\magsection{Intégration et dérivation}

\exercice{nom={Calcul de primitives}, utile=2}
Donner le domaine de définition et calculer les primitives suivantes~:
\[\prim{\p{x^2+x+1}\e^x}{x}, \qquad \prim{\p{x^2-1}\cos x}{x}, \qquad
  \prim{x^3\ln x}{x},\]
\[\prim{\sin^2 x\cos^3 x}{x}, \qquad \prim{\sin x\cos^2 x}{x}, \qquad
  \prim{\sin^2 x\cos^2 x}{x},\]
\[\prim{\frac{x}{x^2+1}}{x}, \qquad \prim{\frac{1}{x\ln x}}{x}, \qquad
  I_n=\prim{\ln^n x}{x} \quad (\text{pour $n\in\N$}).\]
\begin{sol}
\[I_n=x\sum_{k=0}^n (-1)^k \frac{n!}{(n-k)!}\ln^{n-k} x\]   
\end{sol}

\exercice{nom={Intégrales de Wallis}, utile=3}
Pour tout entier $n\in\N$, on définit $I_n$ et $J_n$ par
\[I_n\defeq\integ{0}{\frac{\pi}{2}}{\sin^n t}{t} \quad\text{et}\quad
  J_n\defeq\integ{0}{\frac{\pi}{2}}{\cos^n t}{t}.\]
\begin{questions}
\question Montrer que pour tout $n\in\N$, $I_n=J_n$.
\question Montrer que les suites $(I_n)$ et $(J_n)$ sont positives et
  décroissantes. Calculer $I_0$ et $I_1$.
\question Montrer que pour tout $n\in\N$
  \[I_{n+2}=\frac{n+1}{n+2}I_n.\]
\question En déduire que pour tout $n\in\N$
  \[I_n I_{n+1}=\frac{\pi}{2(n+1)}.\]
\question En déduire que
  \[I_n\equiS\sqrt{\frac{\pi}{2n}}.\]
\end{questions}


\exercice{nom={Étude de fonctions}, utile=3, difficulte=2}
Étudier le domaine de définition, les symétries, la monotonie et les limites
aux bornes du domaine de définition des fonctions d'expressions
\[x\mapsto \integ{1}{1+x^2}{\ln(t)}{t}, \qquad
  x\mapsto \integinv{x}{2x}{\ln(1+t^2)}{t},\]
\[x\mapsto \integinv{x}{x^2}{\ln(t)}{t}, \qquad
  x\mapsto \integ{x}{2x}{\e^{t^2}}{t}.\]
\begin{sol}
$\quad$
\begin{questions}
\question La fonction est paire, croissante sur $\RP$ et tend vers $+\infty$
  en $+\infty$ (Calculer $f$ pour cette dernière limite).
\question La fonction est définie sur $\Rs$ et est impaire. Elle
  est décroissante sur $\interof{0}{\sqrt{2}}$ et croissante sur
  $\interfo{\sqrt{2}}{+\infty}$. En 0 et en $+\infty$, elle tend vers $+\infty$
  (il suffit de minorer l'intégrale en minorant la fonction à intégrer
  par son minimum).
\question La fonction est définie sur $\RPs\setminus\ens{1}$. Elle est
  croissante sur $\intero{0}{1}$ et sur $\intero{1}{+\infty}$. La limite en
  1 est difficile à calculer (voir un autre exercice). On trouve $\ln 2$.
  La limite en $+\infty$ est $+\infty$ et en 0 est 0 (ces deux limites
  se font simplement en utilisant la première inégalit de la moyenne).
\question La fonction est définie sur $\R$, impaire et tend trivialement vers
  $+\infty$ en $+\infty$.
\end{questions}
\end{sol}

\exercice{nom={Étude d'une fonction définie par une intégrale}, utile=2, difficulte=2}
Soit $f$ une fonction continue et positive sur $\interf{0}{\frac{\pi}{2}}$. On
définit la fonction $g$ d'expression
$$g(x)\defeq\integ{0}{\frac{\pi}{2}}{\frac{f(t)}{1+x\sin t}}{t}.$$
\begin{questions}
\question Montrer que $g$ est définie sur $\intero{-1}{+\infty}$.
\question Montrer que $g$ est décroissante.
\question Étant donné $a>-1$, montrer que $g$ est lipschitzienne sur
  $\interfo{a}{+\infty}$. En déduire que $g$ est continue sur
  $\intero{-1}{+\infty}$.
\question Montrer que $g$ est dérivable sur $\intero{-1}{+\infty}$ et que
  \[\forall x\in\intero{-1}{+\infty} \qsep
    g'(x)=
    -\integ{0}{\frac{\pi}{2}}{\frac{f(t)\sin t}{\p{1+x\sin(t)}^2}}{t}.\]
\end{questions}


\exercice{nom={Calcul numérique}, utile=2, difficulte=1}
Donner une majoration de l'erreur commise en prenant $x-\frac{x^2}{2}$
comme valeur approchée de $\ln(1+x)$. En déduire une valeur approchée de
$\ln(1,003)$ à $10^{-8}$ près.


\exercice{nom={Calcul de limites}, utile=3, difficulte=1}
Étudier la convergence des suites de terme général
\[\frac{1}{n}\sum_{k=1}^n \sin\p{\frac{k\pi}{n}}, \qquad
  n\sum_{k=1}^n \frac{1}{\p{n+k}^2}, \qquad
  \frac{1}{n\sqrt{n}}\sum_{k=1}^{n-1} \sqrt{k},\]
\[\sqrt[n]{\prod_{k=1}^n \p{1+\p{\frac{k}{n}}^2}}, \qquad
  \sum_{k=1}^{n-1} \frac{1}{\sqrt{n^2-k^2}}.\]

\exercice{nom={Calcul de limites}, utile=2}
\begin{questions}
\question Montrer que
  \[\forall x \in\RP \qsep x^2-\frac 13 x^4 \leq \sin^2( x)\leq x^2.\]
\item En déduire la limite de la suite de terme général
  \[\sum\limits_{k=1}^n \sin^2 \left(\frac 1{\sqrt{k+n}} \right).\]
\end{questions}

\exercice{nom={Calcul de limites}, utile=2}
Soit $f \in\classec{1}([0,1],\R)$. Déterminer la limite de la suite de terme général
\[\frac 1n \sum\limits_{k=0}^{n-1} f\p{\frac{k+1}n} f'\p{\frac kn}.\]

\exercice{nom={Suite}}
\begin{questions}
\question Montrer que pour tout $x\in[0,1]$
  \[\abs{\e^x-1-x}\leq\frac{\e}{2}x^2.\]
\question En déduire la limite de la suite de terme général
  \[u_n\defeq \p{\sum_{k=1}^n \e^{\frac{1}{n+k}}} -n.\]
\end{questions}


\exercice{nom={Lemme de Lebesgue}, utile=2, difficulte=3}
Soit $f$ une fonction continue par morceaux sur le segment $\interf{a}{b}$.
Le but de cet exercice est de montrer que
\[\integ{a}{b}{f(t)\sin(nt)}{t} \tendvers{n}{+\infty} 0.\]
\begin{questions}
\question En effectuant une intégration par parties, montrer que le résultat
  est vrai lorsque $f$ est supposé $\classec{1}$.
\question Le but de cette question est de démontrer que le résultat est vrai
  dans le cas général.
  \begin{questions}
  \question Montrer que le résultat est vrai lorsque $f$ est une fonction en
    escalier.
  \question En déduire le cas général.
  \end{questions}
\end{questions}


\exercice{nom={Calcul de $\zeta(2)$}}
\begin{questions}
\question Montrer qu'il existe un unique polynôme $P$ de degré
  inférieur ou égal à 2 tel que $P(0)=0$ et
  \[\forall k\in\Ns \qsep \integ{0}{1}{P(t)\cos(kt)}{t}=\frac{1}{k^2}.\]
\question En utilisant le lemme de Lebesgue, en déduire que
  \[\sum_{k=1}^n \frac{1}{k^2}\tendvers{n}{+\infty}\frac{\pi^2}{6}.\]
\end{questions}

\exercice{nom={Généralisation du lemme de Lebesgue}, utile=2, difficulte=3}
Soit $f$ une fonction continue par morceaux sur le segment $\interf{a}{b}$
et $g$ une fonction définie sur $\R$, continue par morceaux et $T$-périodique.
Le but de cet exercice est de montrer que
\[\integ{a}{b}{f(t)g\p{nt}}{t} \tendvers{n}{+\infty}
  \p{\frac{1}{T} \integ{0}{T}{g(t)}{t}} \integ{a}{b}{f(t)}{t}.\]
\begin{questions}
\question Démontrer le résultat lorsque $f$ est constante puis lorsque $f$
  est une fonction en escalier.
\question En déduire le résultat général.
\question Soit $f$ une fonction de classe $\classec{1}$ sur
  $\interf{0}{1}$. On définit la suite $\p{u_n}$ par
  \[\forall n\geq 1 \qsep u_n\defeq \frac{1}{n}
    \p{\frac{1}{2}f\p{0}+\sum_{k=1}^{n-1} f\p{\frac{k}{n}}+\frac{1}{2}f\p{1}}.\]
  \begin{questions}
  \question Montrer que pour tout $k\in\intere{0}{n-1}$
    \[\integ{\frac{k}{n}}{\frac{k+1}{n}}{f(t)}{t}=\frac{1}{n}f\p{\frac{k}{n}}
      +\integ{\frac{k}{n}}{\frac{k+1}{n}}{\p{\frac{k+1}{n}-t}f'(t)}{t}.\]
  \question En déduire que
    \[u_n=\integ{0}{1}{f(t)}{t}+\petito{n}{+\infty}{\frac{1}{n}}.\] 
  \end{questions}
\end{questions}

\exercice{nom={Limite différentielle}, utile=1, difficulte=3}
Soit $f$ une fonction de classe $\mathcal{C}^1$ sur $\R$ telle que
$$f'(x)+f(x) \tendvers{x}{+\infty} 0.$$
Le but de cet exercice est de montrer que $f(x)$ tend vers $0$ lorsque $x$
tend vers $+\infty$.
\begin{questions}
\question On note $\epsilon$ la fonction $\epsilon\defeq f+f'$. Montrer que si $a$
  est un réel
  \[f(x)=f(a)\e^{a-x}+\e^{-x}\integ{a}{x}{\epsilon(t)\e^t}{t}.\]
\question Conclure.
\question Que dire si la condition de départ est changée en
  \[f'(x)+\lambda f(x) \tendvers{x}{+\infty} 0,\]
  où $\lambda$ est un réel~?
\end{questions}
%END_BOOK
\end{document}

