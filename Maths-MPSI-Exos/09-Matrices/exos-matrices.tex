\documentclass{magnolia}

\magtex{tex_driver={pdftex}}
\magfiche{document_nom={Exercices sur les matrices},
          auteur_nom={François Fayard},
          auteur_mail={fayard.prof@gmail.com}}
\magexos{exos_matiere={maths},
          exos_niveau={mpsi},
          exos_chapitre_numero={9},
          exos_theme={Matrices}}
\magmisenpage{}
\maglieudiff{}
\magprocess

\begin{document}
%BEGIN_BOOK

\magsection{Matrice}
\magsubsection{Matrice}
\magsubsection{Matrice carrée}
\magsection{Opérations sur les matrices}
\magsubsection{Combinaison linéaire}

\exercice{nom={Sous-structures de $\mat{n}{\K}$}}
Montrer que l'ensemble des matrices
\[\begin{pmatrix}
  x   & y  & z\\
  2z  & x  & y\\ 
  2y  & 2z & x
  \end{pmatrix}\]
pour $x,y,z \in \Q$ est un sous-espace vectoriel de $(\mat{3}{\Q},+)$.

% \question On dit qu'une matrice $A$ est triangulaire inférieure si :
%   \[\forall i,j \in\intere{1}{n} \quad i<j \Longrightarrow a_{i,j}=0\]
%   Montrer que l'ensemble des matrices triangulaires inférieures est une
%   sous-algèbre de $\mat{n}{\K}$.

\magsubsection{Produit}


\exercice{nom={Sous-structures de $\mat{n}{\K}$}}
Soit $E$ l'ensemble des matrices de $\mat{3}{\C}$ de la forme :
\[\begin{pmatrix}
    a & 0 & b\\
    0 & a & c\\
    0 & 0 & d
    \end{pmatrix}\]
où $a,b,c,d$ sont des nombres complexes.
\begin{questions}
\question Montrer que $E$ est une sous-espace vectoriel de $\mat{3}{\C}$, stable par produit.
\question Les éléments de $E$ commutent-ils entre eux~?
\question Soit $A,B\in E$. Est-il possible d'avoir $AB=0$ sans que $A=0$ ou
  $B=0$~?
% \question Donner la dimension de cette algèbre.
\end{questions}


\exercice{nom={Produit}}
Soient $A$ et $B$ deux matrices de $\mat{n}{\K}$ telles que
\[\forall X\in\mat{n}{\K} \qsep AXB=0.\]
Montrer que $A=0$ ou $B=0$.


\magsubsection{Calcul dans l'algèbre $\mat{n}{\K}$}

\exercice{nom={Calcul de puissances successives}}
Calculer la puissance n-ième des matrices
  \[\begin{pmatrix}
    2 & 2 & 4\\
    0 & 2 & 1\\
    0 & 0 & 2
    \end{pmatrix} \et
    \begin{pmatrix}
    \cos \theta & -\sin \theta & 0\\
    \sin \theta & \cos \theta & 0\\
    0 & 0 & 2
    \end{pmatrix}\]
\begin{sol}
\[\begin{pmatrix}
  2^n & n2^n & n2^{n-2}\p{n+7}\\
  0 & 2^n & n2^{n-1} \\
  0 & 0 & 2^n
  \end{pmatrix} \et
  \begin{pmatrix}
  \cos\p{n\theta} & -\sin\p{n\theta} & 0\\
  \sin\p{n\theta} & \cos\p{n\theta} & 0\\
  0 & 0 & 2^n
  \end{pmatrix}\]
\end{sol}



\magsubsection{Matrice inversible}

\exercice{nom={Déterminant}}
Soit $A$ la matrice
\[A=\begin{pmatrix}
    a_{1,1} & a_{1,2}\\
    a_{2,1} & a_{2,2}
  \end{pmatrix}\]
\begin{questions}
\question Montrer que $A^2-(a_{1,1}+a_{2,2})A+(a_{1,1}a_{2,2}-a_{1,2}a_{2,1})I=0$.
\question En déduire une condition nécessaire et suffisante pour que $A$ soit
  inversible.
\end{questions}






% \question Soit $A,B,C$ les matrices définies par :
%   \[\begin{pmatrix}
%     1 & 0\\
%     0 & 0
%     \end{pmatrix}\quad
%     \begin{pmatrix}
%     0 & 1\\
%     0 & 0
%     \end{pmatrix}\quad
%     \begin{pmatrix}
%     0 & 0\\
%     1 & 0
%     \end{pmatrix}\]
%   \begin{questions}
%   \question Comparer $\tr\p{ABC}$ et $\tr\p{BAC}$.
%   \question Comparer $\tr\p{ABC}$, $\tr\p{CAB}$ et $\tr\p{BCA}$.
%   \end{questions}
% \end{questions}









% \exercice{nom={Groupe des rotations}}
% Montrer que l'ensemble des matrices~:
% \[\begin{pmatrix}
%   \cos \theta & -\sin \theta\\
%   \sin \theta & \cos \theta
%   \end{pmatrix}\]
% pour $\theta\in\R$ est un sous-groupe de $(\gl{2}{\R},\times)$.





\exercice{nom={Calcul d'inverse}}
Soit $n\in\Ns$ et $a\in\R$. On définit la matrice $A\in\mat{n}{\K}$ par
\[\forall i,j\in\intere{1}{n} \quad a_{i,j}=
  \begin{cases}
  a^{j-i} & \text{si $j\geq i$}\\
  0 & \text{sinon}
  \end{cases}\]
En introduisant la matrice $N$ définie par
\[\forall i,j\in\intere{1}{n} \quad n_{i,j}=
  \begin{cases}
  1 & \text{si $j=i+1$}\\
  0 & \text{sinon}
  \end{cases}\]
montrer que $A$ est inversible et calculer $A^{-1}$.
\begin{sol}
On remarque, en calculant $AN$ que $A=I_n+aAN$, donc $A(I_n-aN)=I_n$. Donc
$A$ est inversible et $A^{-1}=I_n-aN$.
\end{sol}

% \exercice{nom={Calcul de rang}}
% Résoudre le système linéaire dont la matrice $A$ est définie par :
% \[\forall i,j\in\intere{1}{n} \quad a_{i,j}=
%   \begin{cases}
%   1 & \text{si} \abs{i-j}\leq 1\\
%   0 & \text{sinon}
%   \end{cases}\]
% Quel est son rang ?
% \begin{sol}
% On peut voir qu le rang de la matrice est au moins $n-1$ puisque les $n-1$
% premières colonnes de la matrice forment une famille libre. On calcule ensuite
% le déterminant et on trouve une relation de récurrence $u_{n+2}=u_{n+1}-u_n$
% puis
% \[u_n=\frac{2}{\sqrt{3}}\sin\p{\frac{(n+1)\pi}{3}}\]
% Donc $A$ est inversible si et seulement si $n$ n'est pas congru à 2 modulo 3.
% Sinon, elle est de rang $n-1$.
% \end{sol}

\magsubsection{Calcul par bloc}

\exercice{nom={Inversibilité des matrices triangulaires supérieures}}
Soit $n\in\N$ et $A\in\mat{n+1}{\K}$ une matrice de la forme
\[A=\begin{pmatrix}
  \alpha & L\\
  0      & B
  \end{pmatrix}\]
où $\alpha\in\K$, $L\in\mat{1,n}{\K}$ et $B\in\mat{n}{\K}$.
\begin{questions}
\question Soit $A'\in\mat{n+1}{\K}$ une matrice que l'on décompose sous la forme
  \[A'=\begin{pmatrix}
    \alpha' & L'\\
    C'       & B'
    \end{pmatrix}\]
  où $\alpha'\in\K$, $L'\in\mat{1,n}{\K}$, $C'\in\mat{n,1}{\K}$ et $B'\in\mat{n}{\K}$.
  Montrer que $AA'=I_{n+1}$ et $A'A=I_{n+1}$ si et seulement si
  \[\alpha\alpha'=1, \qquad BB'=I_n, \qquad B'B=I_n, \qquad C'=0\quad\text{et}\quad L'=-\alpha' L B'.\]
\question En déduire que $A$ est inversible si et seulement si $\alpha\neq 0$ et $B\in\gl{n}{\K}$.
\question Montrer qu'une matrice triangulaire supérieure est inversible si et seulement si tous ses
  coefficients diagonaux sont non nuls et que si tel est le cas, son inverse est triangulaire
  supérieure et ses coefficients diagonaux sont les inverses des coefficients diagonaux de $A$.
\end{questions}

\exercice{nom={Matrices symplectiques}}
Soit $n\in\N$ un entier pair et $m\in\N$ tel que $n=2m$. On pose
\[J\defeq\begin{pmatrix}
  0   & -I_m\\
  I_m & 0 \end{pmatrix}\in\mat{n}{\R}\]
On appelle matrice symplectique toute matrice $M\in\mat{n}{\R}$ telle que
\[M^\top J M =J.\]
\begin{questions}
\question Montrer que $J^\top=-J$.
\question Montrer que $J\in\gl{n}{\R}$ et que $J^{-1}=-J$.
\question Montrer que $I_n$ et $J$ sont symplectiques.
\question Montrer que le produit de deux matrices symplectiques est symplectique.
\end{questions}

\magsection{Matrice et système linéaire}
\magsubsection{Interprétation matricielle}
\magsubsection{Calcul d'inverse, système de \nom{Cramer}}

\exercice{nom={Calcul d'inverse}}
Calculer l'inverse de la matrice
  \[\begin{pmatrix}
    1 & -1 & 2\\
    0 & 1 & 1\\
    0 & 0 & 1
    \end{pmatrix}\]
\begin{sol}
\[\begin{pmatrix}
  1 & 1 & -3\\
  0 & 1 & -1\\
  0 & 0 & 1
  \end{pmatrix}\]
\end{sol}


\magsubsection{Opérations élémentaires par produit matriciel}

% \exercice{nom={Étude d'un système affine}}
% Soit $a,b,c$ trois réels deux à deux distincts.
% \begin{questions}
% \question Montrer que le système~:
%   \[\left\lbrace
%     \begin{array}{l}
%     x+ay+a^2z=0\\
%     x+by+b^2z=0\\
%     x+cy+c^2z=0
%     \end{array}\right.\]
%   est de Cramer.
% \question Résoudre le système :
%   \[\left\lbrace
%   \begin{array}{l}
%   x+ay+a^2z=a^4\\
%   x+by+b^2z=b^4\\
%   x+cy+c^2z=c^4
%   \end{array}
%   \right.\]
% \end{questions}

\exercice{nom={Exercice}}
Résoudre le système
\[\syslin{-x_1&+x_2&+x_3&+\cdots&+x_n&=&1\hfill\cr
           x_1&-x_2&+x_3&+\cdots&+x_n&=&2\hfill\cr
           \vdots\ \,& &    &       &\vdots\ \,&=&\,\vdots\hfill\cr
           x_1&+x_2&+x_3&+\cdots&-x_n&=&n\hfill}\]
% $$\begin{cases} -x_1+x_2+\dots+x_n=1 \\ x_1-x_2+x_3+\dots+x_n=2 \\ \dots \\
% x_1+x_2+\dots+x_{n-1}-x_n=n \end{cases}$$
\begin{sol}
On trouve
\[x_i=\frac{n(n+1)}{4(n-2)}-\frac{i}{2}\]
\end{sol}

\exercice{nom={Exercice}}
Soit $m\in\R$. Résoudre le système suivant.
\[\syslin{x&+y&+z&+t&=&3\hfill\cr
          x&+my&+z&-mt&=&m+2\hfill\cr
          mx&-y&-mz&-t&=&-1.\hfill}\]

\exercice{nom={Exercice}}
Soit $a,b,c\in\C$. Résoudre le système
\[\syslin{x&+y&+z&=&a\hfill\cr
          x&+\jj y&+\jj^2 z&=&b\hfill\cr
          x&+\jj^2y&+\jj z&=&c.\hfill}\]
Donner une condition nécessaire et suffisante
sur $a,b,c$ pour que les solutions soient réelles.

% \exercice{nom={Exercice}}
% Soit $A = \begin{pmatrix}
% 2&2&0\\1&2&1\\0&2&2\\
% \end{pmatrix}$. Déterminer les $\lambda \in \R$ tels que $ \exists X \in \R^3$
% non nul tel que $AX = \lambda X$. Pour chaque $\lambda$
% déterminer $E_{\lambda} = \left\{ X \in \R^3 / AX = \lambda X\right\}$. Conclusion ?

\magsubsection{Matrice échelonnée}

%END_BOOK
\end{document}
















