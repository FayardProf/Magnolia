\documentclass{magnolia}

\magtex{tex_driver={pdftex}}
\magfiche{document_nom={Exercices sur les fractions rationnelles},
          auteur_nom={François Fayard},
          auteur_mail={fayard.prof@gmail.com}}
\magexos{exos_matiere={maths},
         exos_niveau={mpsi},
         exos_chapitre_numero={11},
         exos_theme={Fractions rationnelles}}
\magmisenpage{}
\maglieudiff{}
\magprocess
 
\begin{document}

%BEGIN_BOOK
\magsection{Fraction rationnelle}

\exercice{nom={Valeurs prises par une fraction rationnelle}}
% Gourdon Algèbre, page 75
Soit $F\in\fracC$ une fraction rationnelle. On note $\mathcal{D}_F\defeq\C\setminus\ens{a_1,\ldots,a_n}$,
où $a_1,\ldots,a_n\in\C$ sont les pôles de $F$, et on définit la fonction
\[\dspappli{f}{\mathcal{D}_F}{\C}{z}{F(z).}\]
On suppose que $F$ n'est pas constante. Montrer que $f(\mathcal{D}_F)=\C$ ou qu'il existe $\alpha\in\C$
tel que $f(\mathcal{D}_F)=\C\setminus\ens{\alpha}$.

\exercice{nom={Primitive}}
Montrer que la fraction rationnelle $F\defeq 1/X$ n'a pas de primitive dans $\fracC$.

\magsection{Décomposition en éléments simples}


\exercice{nom={Décomposition en éléments simples}}
Décomposer en éléments simples sur $\R$ les fractions suivantes
\[\frac{10X^3}{(X^2+1)(X^2-4)}, \qquad \frac{X^4+1}{X^4+X^2+1},
  \qquad \frac{X^3-1}{(X-1)(X-2)(X-3)},\]
\[\frac{X^2}{(X^2+X+1)^2}, \qquad \frac{(X^2+4)^2}{(X^2+1)(X^2-2)^2}.\]
\begin{sol}
$\quad$
\begin{questions}
\question 
  \[\frac{10X^3}{(X^2+1)(X^2-4)}=
    \frac{4}{X-2}+\frac{4}{X+2}+\frac{2X}{X^2+1}\]
\question 
  \[\frac{X^4+1}{X^4+X^2+1}=
    1-\frac{1}{2}\cdot\frac{X}{X^2-X+1}+\frac{1}{2}\cdot\frac{X}{X^2+X+1}\]
\question 
  \[\frac{X^3-1}{(X-1)(X-2)(X-3)}=
    1-\frac{7}{X-2}+\frac{13}{X-3}\]
\question 
  \[\frac{X^2}{(X^2+X+1)^2}=
    \frac{1}{X^2+X+1}-\frac{X+1}{\p{X^2+X+1}^2}\]\
\question 
  \begin{eqnarray*}
  \frac{(X^2+4)^2}{(X^2+1)(X^2-2)^2}&=&
    \frac{1}{X^2+1}+\frac{3\sqrt{2}}{4}\cdot\frac{1}{X+\sqrt{2}}
    +\frac{3}{2}\cdot\frac{1}{\p{X+\sqrt{2}}^2}\\
   & &
    -\frac{3\sqrt{2}}{4}\cdot\frac{1}{X-\sqrt{2}}
    +\frac{3}{2}\cdot\frac{1}{\p{X-\sqrt{2}}^2}
  \end{eqnarray*}
\end{questions}
\end{sol}

\exercice{nom={Décomposition en éléments simples}}
Décomposer en éléments simples sur $\C$ les fractions suivantes
\[\frac{1}{X^n-1}, \qquad \frac{X^{n-1}}{X^n-1}, \qquad
  \frac{1}{(X-1)(X^n-1)}.\]
\begin{sol}
$\quad$
\begin{questions}
\question 
  \[\frac{1}{X^n-1}=\sum_{k=0}^{n-1} \frac{1}{n}\cdot\frac{\omega^k}{X-\omega^k}\]
\question 
  \[\frac{X^{n-1}}{X^n-1}=\frac{1}{n}\sum_{k=0}^{n-1} \frac{1}{X-\omega^k}\]
\question 
  \[\frac{1}{\p{X-1}\p{X^n-1}}=-\frac{n-1}{2n}\cdot\frac{1}{X-1}+
    \frac{1}{n}\cdot\frac{1}{\p{X-1}^2}+\sum_{k=1}^{n-1}
    \frac{1}{n}\cdot\frac{\omega^k}{\omega^k-1}\cdot\frac{1}{X-\omega^k}\]
\end{questions}
\end{sol}

\exercice{nom={Une somme}}
Soit $n\in\Ns$. Simplifier
\[\sum_{k=1}^n \frac{1}{k(k+1)(k+2)}.\]

\exercice{nom={Une somme de série}}
Calculer la limite lorsque $n$ tend vers $+\infty$ de
\[\sum_{k=2}^n \frac{3k^2-1}{(k-1)^2k^2(k+1)^2}.\]
\begin{sol}
On trouve
\[\frac{3X^2-1}{(X-1)^2X^2(X+1)^2}=\frac{1}{2}\cdot\frac{1}{(X-1)^2}
  -\frac{1}{X^2}+\frac{1}{2}\cdot\frac{1}{(X+1)^2}\]
(Astuce de Anis Marrakchi : Pour connaître le coefficient au dessus de
$1/(X+1)$ et $-1/(X-1)$ on regroupe ces éléments simples, on multiplie par
$X^2$ et on fait tendre $x$ vers $+\infty$ // Autre astuce de Anis Marrakchi :
on peut remarquer que $F=P'/P^2$).
On obtient donc une somme télescopique. La somme de la série est donc
$\frac{3}{8}$.
\end{sol}

\exercice{nom={Décomposition en éléments simples}}
On considère la fraction
\[F\defeq\frac{1}{(X^3-1)^3}\]
que l'on souhaite décomposer en éléments simples sur $\fracC$.
\begin{questions}
\question Calculer la partie polaire de $F$ relative au pôle 1.
\question En étudiant les symétries de $F$, en déduire sa décomposition en
  éléments simples.
\end{questions}
\begin{sol}
$\quad$
\begin{questions}
\question On trouve
  \[\frac{5}{27}\cdot\frac{1}{X-1}-\frac{1}{9}\cdot\frac{1}{(X-1)^2}
    +\frac{1}{27}\cdot\frac{1}{(X-1)^3}\]
\question On a enfin~:
  \begin{eqnarray*}
  \frac{1}{(X^3-1)^3}&=&
    \frac{5}{27}\cdot\frac{1}{X-1}-\frac{1}{9}\cdot\frac{1}{(X-1)^2}
    +\frac{1}{27}\cdot\frac{1}{(X-1)^3}+\\
    &&\sum_{\alpha\in\ens{j,j^2}}\cro{
    \frac{5}{27}\cdot\frac{\alpha}{X-\alpha}+\frac{1}{9}\cdot\frac{1+\alpha}{(X-\alpha)^2}
    +\frac{1}{27}\cdot\frac{1}{(X-\alpha)^3}}
  \end{eqnarray*}
\end{questions}
\end{sol}

\exercice{nom={Autour de \nom{Chebyshev}}}
\begin{questions}
\question Montrer que pour tout entier $n\in\N$, il existe un unique polynôme
  $P_n\in\polyR$ tel que
  \[X^n+\frac{1}{X^n}=P_n\p{X+\frac{1}{X}}.\]
\question Soit $n\in\Ns$.
  \begin{questions}
  \question Factoriser $P_n$ dans $\polyC$.
  \question Décomposer $1/P_n$ en éléments simples dans $\fracC$.
  \end{questions}
\end{questions}
\begin{sol}
$\quad$
\begin{questions}
\question $P_{n+1}=XP_n-P_{n-1}$
\question
  \begin{questions}
  \question 
    \[P_n=\prod_{k=0}^{n-1} \p{X-2\cos\p{\frac{\pi\p{2k+1}}{2n}}}\]
  \question
    \[\frac{1}{P_n}=\sum_{k=0}^{n-1} \frac{\p{-1}^k}{n}\sin\p{\frac{\pi\p{2k+1}}{2n}}
      \frac{1}{X-2\cos\p{\frac{\pi\p{2k+1}}{2n}}}\]
  \end{questions}
\end{questions}
\end{sol}

% \exercice{nom={Racines de P'}}
% Soit $P\in\polyC$ de degré $n$.
% \begin{questions}
% \question Décomposer $P'/P$ en éléments simples.
% \question En déduire que les racines de $P'$ sont dans l'enveloppe convexe des
%   racines de $P$ c'est-à-dire que toute racine de $P'$ s'écrit comme barycentre
%   à coefficients positifs des racines de $P$.
% \end{questions}


\exercice{nom={Racines multiples de $P'$}}
Soit $P\in\polyR$ un polynôme non constant.
\begin{questions}
\question On suppose que $P$ est scindé sur $\polyR$.
\begin{questions}
\question Calculer la décomposition en éléments simples de $P'/P$.
\question En déduire que
  \[\forall x\in\R\qsep P'(x)^2\geq P(x)P''(x)\]
  avec égalité si et seulement si $x$ est racine multiple de $P$.
\question Rappeler pourquoi $P'$ est scindé, puis montrer que toute
  racine multiple de $P'$ est racine de $P$.
\end{questions}
\question
\begin{questions}
\question Est-il vrai en général que toute racine multiple de $P'$ est racine de $P$~?
\question Est-il vrai, sous l'hypothèse que $P$ est scindé sur $\R$, que toute racine de $P'$ est racine de $P$~?
\end{questions}
\question On suppose que
  \[\forall x\in\R\qsep P'(x)^2\geq P(x)P''(x).\]
  Montrer que $P$ possède une racine réelle.
\end{questions}

\exercice{nom={Théorème de \nom{Gauss}-\nom{Lucas}}}
\begin{questions}
\question Soit $P\in\fracC$ non nul. Calculer la décomposition en éléments simples de $F\defeq P'/P$.
\question Soit $n\in\Ns$. Déterminer la valeur de
  \[\sum_{z\in\mathbb{U}_n\setminus\ens{1}} \frac{1}{1-z}.\]
\question Soit $P\in\polyC$ un polynôme non constant dont les racines sont $z_1,\ldots,z_r\in\C$.
\begin{questions}
\question Montrer que pour toute racine $\alpha\in\C$ de $P'$, il existe $\lambda_1,\ldots,\lambda_r \geq 0$
  tels que $\lambda_1+\cdots+\lambda_r=1$ et \[\alpha=\lambda_1 z_1+\cdots+\lambda_r z_r.\]
\question Interpréter géométriquement ce résultat.
\end{questions}
\end{questions}


\exercice{nom={Calcul de dérivées $n$-ièmes}}
Calculer la dérivée $n$-ième de
\[x\mapsto\ln(x^2-x+2).\]

\exercice{nom={Majoration d'une dérivée $n$-ième}}
Montrer que
\[\forall n\in\Ns \qsep \forall x\in\R \qsep \abs{\arctan^{(n)} x}\leq(n-1)!.\]
\begin{sol}
On trouve grâce à $\arctan'(x)=\dfrac{1}{1+x^2}=\dfrac{\ii}{2}\p{\dfrac{1}{x+i}-\dfrac{1}{x-i}}$, 
\[\forall n\in\Ns \quad \forall x\in\R \quad \arctan^{(n)} x=
  (-1)^{n+1}\frac{(n-1)!}{2}\cro{\frac{1}{(x+i)^n}-\frac{1}{(x-i)^n}}i\]
  puis on majore en passant aux valeurs absolues.
\end{sol}

\exercice{nom={Dérivée $n$-ième}}
On pose
\[R\defeq\frac{1}{X^2+1}.\]
\begin{questions}
\question Décomposer $R$ en éléments simples, puis calculer $R^{(n)}$ pour tout $n\in\N$.
\question Montrer que pour tout $n\in\N$
  \[R^{(n)}=\frac{(-1)^n (n+1)!}{(X^2+1)^{n+1}}\prod_{k=1}^n \p{X-\cotan\frac{k\pi}{n+1}}.\]
\end{questions}

\magsection{Primitive d'expression rationnelle}
% \exercice{nom={Primitives de polynôme-exponentielles}}
% Calculer les primitives des fonctions dont les expressions sont :
% \[(1+2x+x^2)e^x \qquad (1+2x)e^x \sin x\]
% \begin{sol}
% $\quad$
% \begin{questions}
% \question
%   \[\prim{(1+2x+x^2)e^x}{x}=\p{1+x^2}e^x\]
% \question
%   \[\prim{(1+2x)e^x \sin x}{x}=
%     \p{\frac{1}{2}-x}e^x\cos x-\p{\frac{1}{2}+x}e^x\sin x\]
% \end{questions}
% \end{sol}

% \exercice{nom={Primitives de polynômes en $\cos,\sin$}}
% Calculer les primitives des fonctions dont les expressions sont :
% \[\sin^2 x\cos^3 x \qquad \sin^3 x\cos^4 x \qquad \sin^3 x\cos^5 x\]

\exercice{nom={Primitives de fractions rationnelles}}
\begin{questions}
\question Calculer sur un intervalle $I$ que l'on précisera les primitives des
  fonctions dont les expressions sont
  \[\frac{1}{1-x^4}, \qquad \frac{x}{1+x^4}, \qquad \frac{x^2}{x^3-1}.\]
\question Calculer
  \[\integ{0}{1}{\frac{x^3+x}{(x^2+x+1)^2}}{x}\]
\end{questions}
\begin{sol}
$\quad$
\begin{questions}
\question
  \[\priminv{1-x^4}{x}=\frac{1}{2}\arctan x+\frac{1}{2}\argth x\]
  \[\prim{\frac{x}{1+x^4}}{x}=\frac{1}{2}\arctan\p{x^2}\]
  \[\prim{\frac{x^2}{x^3-1}}{x}=\frac{1}{3}\ln\abs{x^3-1}\]
\question Le mieux est de faire cet exercice en cherchant une primitive sous
  une forme donnée.
  \[\prim{\frac{x^3+x}{(x^2+x+1)^2}}{x}=\frac{1}{2} \ln\p{1+x+x^2}
    -\frac{7\sqrt{3}}{9}
    \arctan\p{\frac{2 x+1}{\sqrt{3}}}+\frac{x-1}{3 \p{1+x+x^2}}\]
  On trouve
  \[\frac{1}{54}\p{18+27\ln 3-7\sqrt{3}\pi}\]
\end{questions}
\end{sol}

\exercice{nom={Primitives de fractions rationnelles en $\cos,\sin$}}
Calculer sur un intervalle $I$ que l'on précisera les primitives des
fonctions dont les expressions sont
\[\frac{1}{\cos x\sin^3 x}, \qquad \frac{1}{\sin x+\cos x}.\]
\begin{sol}
$\quad$
\begin{questions}
  \question On doit décomposer en éléments simples~:
  \[\frac{1}{X \left(X^2-1 \right)^2}=
    \frac{1}{4 (X-1)^2}-\frac{1}{2 (X-1)}+\frac{1}{X}-\frac{1}{4 \
    (X+1)^2}-\frac{1}{2 (X+1)}\]
  Puis~:
  \[\priminv{\cos x\sin^3 x}{x}=\ln\abs{\tan x}-\frac{1}{2\sin^2 x}\]
% \question
%   \[\priminv{1+2\cos x}{x}=
%     \frac{2\sqrt{3}}{3}\arctan\p{\frac{\sqrt{3}\tan\p{x/2}}{3}}\]
\question
  \[\priminv{\sin x+\cos x}{x}=\sqrt{2}
    \argth\p{\frac{\tan\p{\frac{x}{2}}-1}{\sqrt{2}}}\]
\end{questions}
\end{sol}

\exercice{nom={Calcul de primitives}}
Calculer sur un intervalle $I$ que l'on précisera les primitives des
fonctions dont les expressions sont
\[\frac{1}{2}x\sqrt{\frac{x-1}{x+1}}, \qquad \frac{x}{\sqrt{-x^2+x+2}}.\]

\exercice{nom={Une intégrale impropre}}
Calculer la limite lorsque $x$ tend vers $\pi/2$ par la gauche de
\[\integ{0}{x}{\sqrt{\tan t}}{t}.\]
\begin{sol}
On se ramène au calcul d'une primitive de $u^2/(1+u^4)$. Après avoir
factorisé $u^4+1$ en $(u^2+\sqrt{2}u+1)(u^2-\sqrt{2}u+1)$, on cherche
l'allure d'une primitive. On utilise ensuite l'imparité de la primitive
cherchée pour se ramener au calcul de 2 coefficients. Remarquer qu'un des
coefficients est inutile, puis calculer le dernier par substitution. On trouve
\[\integ{0}{\frac{\pi}{2}}{\sqrt{\tan t}}{t}=\frac{\pi}{\sqrt{2}}\]
\end{sol}
%END_BOOK

\end{document}