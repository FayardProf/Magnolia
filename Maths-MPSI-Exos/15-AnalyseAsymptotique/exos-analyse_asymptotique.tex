\documentclass{magnolia}

\magtex{tex_driver={pdftex}}
\magfiche{document_nom={Exercices sur les développements limités},
          auteur_nom={François Fayard},
          auteur_mail={fayard.prof@gmail.com}}
\magexos{exos_matiere={maths},
         exos_niveau={mpsi},
         exos_chapitre_numero={7},
         exos_theme={Développements limités}}
\magmisenpage{}
\maglieudiff{}
\magprocess



\begin{document}

%BEGIN_BOOK

\magsection{Suites équivalentes, suite négligeable devant une autre}


\exercice{nom={Équivalents et composition par $\ln$}}
Soit $(u_n)$ et $(v_n)$ deux suites équivalentes strictement positives.
\begin{questions}
\question On suppose que $(u_n)$ et $(v_n)$ tendent vers $0$ ou $+\infty$ en
$+\infty$. Montrer que
\[\ln u_n \equi{n}{+\infty} \ln v_n.\]
En déduire des équivalents simples de $\ln(n^2-1)$ et $\ln(\sqrt{n+1}-\sqrt{n})$.
\question Que peut-on dire si $(u_n)$ et $(v_n)$ convergent vers 1~?
\end{questions}

\exercice{nom={Calcul d'équivalents}}
Donner un équivalent simple des suites de terme général
\[\ln(n+1)-\ln n, \qquad n(\sqrt[n]{3}-1), \qquad \sum_{k=1}^n k^{k^2}.\]

\begin{sol}
$$1/n$$
$$n(\sqrt[n]{3}-1)=n\p{\e^{\frac{1}{n}\ln(3)}-1}\equi{n}{+\infty}n\cdot \frac{1}{n}\ln(3)=\ln(3).$$
Montrons que $$\sum_{k=1}^n k^{k^2}\equi{n}{+\infty}n^{n^2}.$$
On a $$1\leq \frac{\sum_{k=1}^n k^{k^2}}{n^{n^2}}=1+\sum_{k=1}^{n-1} \frac{k^{k^2}}{n^{n^2}}\leq 1+(n-1)\frac{(n-1)^{(n-1)^2}}{n^{n^2}}\leq 1+\frac{(n-1)^{(n-1)^2+1}}{n^{(n-1)^2+1}}$$
Or : $$\frac{(n-1)^{(n-1)^2+1}}{n^{(n-1)^2+1}}=\e^{\overbrace{\p{(n-1)^2+1}\ln\p{1-\frac{1}{n}}}^{\equi{n}{+\infty}-\frac{n^2}{n}}}\tendvers{n}{+\infty}0$$ d'où le résultat d'après le théorème des gendarmes.
\end{sol}


\exercice{nom={Équivalents}}
Soit $(u_n)$ une suite réelle de limite nulle telle que
\[u_n+u_{n+1} \equi{n}{+\infty} \frac{1}{n}.\]
\begin{questions}
\question Montrer que si $u_n$ est décroissante alors $u_n \sim 1/(2n)$.
\question Étudier le cas de la suite \[u_n\defeq\frac 1{2n}+\frac{(-1)^n}{\sqrt{n}}.\]
\end{questions}
\begin{sol}
$\quad$
\begin{questions}
\question Écrire $u_n+u_{n+1}\leq 2u_n\leq u_{n-1}+u_{n}$.
\end{questions}
\end{sol}

\exercice{nom={Constante d'\nom{Euler}}}
Soit $(H_n)$ la suite définie par
\[\forall n\in\Ns \qsep H_n\defeq\sum_{k=1}^n \frac{1}{k}.\]
\begin{questions}
\question En utilisant une intégrale, montrer que
  \[\forall n\in\Ns\qsep \frac{1}{n + 1} \leq \ln(n + 1)-\ln (n) \leq\frac{1}{n}.\]
\question En déduire que $\ln (n + 1) \leq H_n \leq \ln (n) + 1$.
\question Déterminer la limite, puis un équivalent de $H_n$.
\question Montrer que $u_n\defeq H_n-\ln (n)$ est décroissante et positive. En
  déduire qu'il existe une constante (la constante d'\nom{Euler}) $\gamma$ telle que
 \[\sum_{k = 1}^n \frac 1k = \ln n + \gamma + \petito{n}{+\infty}{1}.\]
\end{questions}



\exercice{nom={Développement asymptotique d'une suite}}
On considère la suite définie par
\[u_0\defeq 1 \et u_{n+1}\defeq \sqrt{n+u_n}\]
\begin{questions}
\question Montrer que
 \[\forall n\in\Ns \qsep \sqrt{n-1}\leq u_n\leq 2\sqrt{n}.\]
\question En déduire que $u_n\equi{n}{+\infty}\sqrt{n}$, puis que
  \[u_n=\sqrt{n}+\frac{1}{2}+\petito{n}{+\infty}{1}.\]
\question Prouver enfin que \[u_n=\sqrt{n}+\frac{1}{2}-\frac{3}{8\sqrt{n}}
  + \petito{n}{+\infty}{\frac{1}{\sqrt{n}}}.\]
\end{questions}

\begin{sol}
\begin{questions}
\question La récurrence passe bien.
\question Au passage dans la preuve de l'inégalité précédente, on a obtenu $\sqrt{n}\leq u_n\leq \sqrt{n+2\sqrt{n-1}}$ d'où le résultat par théorème des gendarmes une fois divisé par $\sqrt{n}$.
Or,
$$u_n-\sqrt{n}=\frac{u_n^2-n}{u_n+\sqrt{n}}$$ avec $u_{n-1}\equi{n}{+\infty}\sqrt{n-1}\equi{n}{+\infty}\sqrt{n}$ et $u_n+\sqrt{n}=\sqrt{n}+\petito{n}{+\infty}{\sqrt{n}}+\sqrt{n}\equi{n}{+\infty}2\sqrt{n}$ donc $$u_n-\sqrt{n}\tendvers{n}{+\infty}\frac{1}{2}.$$
\question Comme on a aussi $u_{n-1}-\sqrt{n}\tendvers{n}{+\infty}\frac{1}{2}$, on a en fait $$u_{n-1}=\sqrt{n}+\frac{1}{2}+\petito{n}{+\infty}{1}$$. Ainsi,

\begin{eqnarray*}
u_n&=&\sqrt{n-1+\sqrt{n}+\frac{1}{2}+\petito{n}{+\infty}{1}}\\
&=&\sqrt{n+\sqrt{n}-\frac{1}{2}+\petito{n}{+\infty}{1}}\\
&=&\sqrt{n}\p{1+\frac{1}{\sqrt{n}}-\frac{1}{2n}+\petito{n}{+\infty}{\frac{1}{n}}}^{\frac{1}{2}}\\
&=&\sqrt{n}\p{1+\frac{1}{2\sqrt{n}}-\frac{1}{4n}-\frac{1}{8}\p{\frac{1}{\sqrt{n}}-\frac{1}{2n}}^2+\petito{n}{+\infty}{\frac{1}{n}}}\\
&=&\sqrt{n}\p{1+\frac{1}{2\sqrt{n}}-\frac{3}{8n}+\petito{n}{+\infty}{\frac{1}{n}}}\\
&=&\sqrt{n}+\frac{1}{2}-\frac{3}{8\sqrt{n}}
  + \petito{n}{+\infty}{\frac{1}{\sqrt{n}}}.
\end{eqnarray*}

\end{questions}
\end{sol}

\exercice{nom={Théorème des valeurs intermédiaires}}
Pour $n\in\Ns$, soit $f_n$ la fonction définie par 
\[\forall x\in\R \qsep f_n(x)\defeq nx^{n+1}-(n+1)x^n-\frac{1}{2}.\]
\begin{questions}
\question Démontrer que $f_n$ admet une unique racine positive, notée $x_n$.
\question Montrer que la suite $(x_n)_{n\in\Ns}$ converge vers 1 (on pourra
  déterminer le signe de $f_{n+1}(x_n)$ pour en déduire le sens de variation de
  la suite).
\question Montrer que la fonction $g$, d'expression $\e^y(y-1)-\frac{1}{2}$,
  possède une unique racine, notée $\gamma$, dans $\intero{0}{+\infty}$. Donner
  une valeur numérique de $\gamma$ à $10^{-3}$ près.
\question Établir que, si $\alpha$ est une constante strictement positive,
  alors $(f_n(1+\frac{\alpha}{n}))_{n\in\Ns}$ converge vers $g(\alpha)$.
\question Soit $\epsilon>0$. Montrer que $f_n(1+\frac{\gamma+\epsilon}{n})$
  est ultimement positif, et que $f_n(1+\frac{\gamma-\epsilon}{n})$ est
  ultimement négatif.
\question En déduire que
  \[x_n=1+\frac{\gamma}{n}+\petito{n}{+\infty}{\frac{1}{n}}.\]
\end{questions}

\exercice{nom={Une suite implicite}}
\begin{questions}
\question Montrer qu'il existe une unique suite $(x_n)_{n\in\N}$ à termes
  strictement positifs telle que  $x_n^n\ln(x_n)=1$ pour tout entier $n$.
\question Montrer que cette suite est  décroissante et qu'elle tend vers $1$. 
\question Montrer que
  \[x_n-1\equi{n}{+\infty}\frac{w(n)}{n}\]
  où $w$ est la fonction de Lambert, c'est-à-dire la fonction réciproque de
  $x\mapsto x\e^x$ sur~$\RP$.
\end{questions}

\begin{sol}
\begin{questions}
\question Le tableau de variation de $f_n : x\mapsto x^n \ln (x)$ permet d'affirmer que l'équation
$f_n(x) = 1$ possède une unique solution $x_n$ sur $\RPs$ et que de plus $x_n \in [1;+\infty[$.
\question On a $$f_n(x_{n+1})=x_{n+1}^{n}\ln(x_{n+1})=\frac{1}{x_{n+1}}x_{n+1}^{n+1}\ln(x_{n+1})=\frac{1}{x_{n+1}}\leq 1=f_n(x_n)$$ donc $x_{n+1}\leq x_n$ car $f$ est strictement croissante sur $[1;+\infty[$. La suite $(x_n)$ est donc décroissante et minorée par $1$ donc elle converge vers $\ell \geq 1$. \\
Supposons un instant que $\ell>1$, alors comme $f_n$ est croissante et que $(x_n)$ décroît, $f_n(x_n)\geq f_n(\ell)$, i.e. $$1=x_n^n\ln(x_n)\geq \underbrace{\ell^n\ln(\ell)}_{\tendvers{n}{+\infty}+\infty}$$ d'où la contradiction. Ainsi, $\ell=1$.
\question On a $$\e^{n\ln(x_n)}n\ln(x_n)=n$$ donc en posant $\phi : x\mapsto x\e^x$ on a :
$$\phi(n\ln(x_n))=n$$ donc $$n\ln(x_n)=w(n)$$ mais comme $(x_n)$ tend vers $1$, $\ln(x_n)=\dfrac{w(n)}{n}\equi{n}{+\infty}x_n-1$.
\end{questions}
\end{sol}





\magsection{Comparaison de fonctions}

\exercice{nom={Composition d'équivalents}}

Soit $f$ et $g$ deux fonctions définies sur $\R$. On suppose que
\[f(x)\equi{x}{+\infty}g(x)\]
et que ces fonctions admettent une limite commune notée $l\in\Rbar$ lorsque
$x$ tend vers $+\infty$.
\begin{questions}
\question On suppose dans cette question que $f$ et $g$ sont à valeurs
  strictement positives.
  \begin{questions}
  \question Montrer que si $l\neq 1$, alors
    \[\ln\p{f(x)}\equi{x}{+\infty}\ln\p{g(x)}.\]
  \question Que pouvez-vous dire lorsque $l=1$~?
  \end{questions}
\question Parmi les équivalents suivants, lesquels sont systématiquement vrais~?
  (on pourra discuter selon les valeurs de $l$).
  \[\arctan\p{f(x)}\equi{x}{+\infty}\arctan\p{g(x)}, \qquad 
    \e^{f(x)}\equi{x}{+\infty} \e^{g(x)},\]
  \[\sin\p{f(x)}\equi{x}{+\infty}\sin\p{g(x)}.\]
  
\end{questions}

\begin{sol}
\begin{questions}
\question 
  \begin{questions}
  \question \[\frac{\ln\p{f(x)}}{\ln\p{g(x)}}=\frac{\ln\p{g(x)}+\ln\p{\frac{f(x)}{g(x)}}}{\ln\p{g(x)}}=1+\frac{\ln\p{\frac{f(x)}{g(x)}}}{\ln\p{g(x)}}\tendvers{x}{+\infty}{1}.\]
  \question Si $f(x)\equi{x}{+\infty}1$, alors on n'a pas nécessairement $\ln\p{f(x)}\equi{x}{+\infty}0$.
  \end{questions}
\question 
\begin{itemize}
\item [$\bullet$] Si $l\neq 0$, $\dfrac{\arctan\p{f(x)}}{\arctan\p{g(x)}}\tendvers{x}{+\infty}{\dfrac{\arctan(l)}{\arctan(l)}}=1$. Si $l=0$, $\arctan\p{f(x)}\equi{x}{+\infty}f(x)$ et $\arctan\p{g(x)}\equi{x}{+\infty}g(x)$ par composition à droite. Donc cela marche dans tous les cas pour $\arctan$.
\item [$\bullet$] 
$\e^{f(x)}\equi{x}{+\infty} \e^{g(x)}$ si et seulement si le rapport tend vers $1$ donc ssi $f-g$ tend vers $0$.
\item [$\bullet$] Si $l\in \R\setminus\pi\Z$, $\dfrac{\sin\p{f(x)}}{\sin\p{g(x)}}\tendvers{x}{+\infty}{\dfrac{\sin(l)}{\sin(l)}}=1$.\\
Si $l=0$, $\sin\p{f(x)}\equi{x}{+\infty}f(x)\equi{x}{+\infty}g(x)\equi{x}{+\infty}\sin\p{g(x)}$.\\
Si $l\in \pi\Z\setminus{\set{0}}$, posons $k\in \Zs$ tel que $l=k\pi$. Et considérons $f(x)=l+1/x$ et $g(x)=l+1/x^2$. En fonction de la parité de $k$, $\sin(f(x))=\pm\sin(1/x)\equi{x}{+\infty}\pm 1/x$ tandis que $\sin(g(x))=\pm\sin(1/x^2)\equi{x}{+\infty}\pm 1/x^2$ donc on ne peut rien dire en général.\\
Si $l=\pm \infty$, on peut prendre $f(x)=x$ et $g(x)=\pi+x$ pour montrer que cela ne fonctionne pas.
\end{itemize}
\end{questions}

\end{sol}


\exercice{nom={Existence de développement limité}}
\begin{questions}
\question $\sqrt{x}$ admet-elle un développement limité d'ordre $n\geq 1$ en
  0~?
\question À quels ordres $x^{\frac{13}{3}}$ admet-elle un développement
  limité en 0~?
\question Soit $n\in\N$. $\abs{x}^n$ admet-elle un développement limité
  d'ordre $n$ en 0~? 
\end{questions}

\begin{sol}
\begin{enumerate}
\item Non, non dérivable en $0$.
\item  $x^{\frac{13}{3}} = o(x^4)$ mais pas plus.
\item Si $n$ est pair, $\abs{x}^n = x^n$ admet un DL à tout ordre.

  Si $n$ est impair, $\abs{x}^n =o(x^{n-1})$ mais pas d'ordre $n$.
\end{enumerate}
\end{sol}

\magsection{Développement limité}


\exercice{nom={Calcul}}
Calculer les développements limités suivants.
\[\e^{\cos x} \quad \text{en 0 à l'ordre 4}, \qquad\qquad
  \ln\p{\frac{1}{\cos x}} \quad \text{en 0 à l'ordre 7},\]
\[\frac{1}{\cos x} \quad \text{en 0 à l'ordre 5}, \qquad\qquad
  \ln\p{1+\ch x} \quad \text{en 0 à l'ordre 4},\]
% \[\frac{\sin x-x}{\cos x-1} \quad \text{en 0 à l'ordre 2} \qquad
%   \frac{\ln\p{1+x}}{\sin x} \quad \text{en 0 à l'ordre 3}\]
\[\frac{1}{\sin x}-\frac{1}{\sh x} \quad \text{en 0 à l'ordre 3},\qquad\qquad
  \ln\p{\tan x} \quad \text{en $\pi/4$ à l'ordre 3},\]
\[\e^{\arcsin x} \quad \text{en 0 à l'ordre 4}\qquad\qquad
  \arctan\p{\e^x} \quad \text{en 0 à l'ordre 3}\]
\[\arccos\p{\frac{1+x}{2+x}} \quad \text{en 0 à l'ordre 2},\qquad\qquad
 \arctan\p{2\sin x} \quad \text{en $\pi/3$ à l'ordre 3}.\]
\begin{sol}
Ne pas oublier d'écrire $\ln(1/(\cos x))=-\ln(\cos x)$.
\[e^{\cos x}=e-\frac{e}{2}x^2+\frac{e}{6}x^4+\petitozero{x}{x^4} \qquad
  \ln\p{\frac{1}{\cos x}}=\frac{1}{2}x^2+\frac{1}{12}x^4+\frac{1}{45}x^6+
  \petitozero{x}{x^7}\]
Il peut être intéressant de dériver $\ln(1+\ch x)$ car on obtient du $\sh x$
ce qui permet de bien réduire l'ordre des calculs.
\[\frac{1}{\cos x}=1+\frac{1}{2}x^2+\frac{5}{24}x^4+\petitozero{x}{x^5} \qquad
  \ln(1+\ch x)=\ln 2+\frac{1}{4}x^2-\frac{1}{96}x^4+\petitozero{x}{x^4}\]
Pour $\ln(\tan x)$, il peut être intéressant de dériver l'expression.
\[\frac{1}{\sin x}-\frac{1}{\sh x}=\frac{1}{3}x+\petitozero{x}{x^3} \qquad
  \ln\p{\tan\p{\frac{\pi}{4}+x}}=2x+\frac{4}{3}x^3+\petitozero{x}{x^3}\]
Pour $\arctan\p{e^x}$, il peut être intéressant de dériver l'expression.
\[e^{\arcsin x}=1+x+\frac{1}{2}x^2+\frac{1}{3}x^3+\frac{5}{24}x^4+
  \petitozero{x}{x^4} \qquad
  \arctan\p{e^x}=\frac{\pi}{4}+\frac{1}{2}x-\frac{1}{12}x^3+\petitozero{x}{x^3}\]
\[\arccos\p{\frac{1+x}{2+x}}=\frac{\pi}{3}-\frac{\sqrt{3}}{6}x+
  \frac{5\sqrt{3}}{72}x^2+\petitozero{x}{x^2}\]
Pour $\arctan(2\sin x)$, on peut dériver l'expression en entier. On se
retrouve à faire des DL de $\sin$ et $\cos$ en des points autres que 0. Pour
cela, on utilise Taylor-Young.
\[\arctan\p{2\sin\p{\frac{\pi}{3}+x}}=\frac{\pi}{3}+\frac{1}{4}x
  -\frac{3\sqrt{3}}{16}x^2+\frac{3}{16}x^3+\petitozero{x}{x^3}\]
\end{sol}

\exercice{nom={Fonction définie par morceaux}}
Soit $f$ la fonction définie sur $\R$ par
\[\forall x\in\R \quad f(x)=
  \begin{cases}
  \cos\sqrt{\abs{x}} & \text{si $x<0$}\\
  1 & \text{si $x=0$}\\  
  \ch\sqrt{x} & \text{si $x>0$.}
  \end{cases}\]
À quels ordres $f$ admet-elle un développement limité en $0$~?

\begin{sol}
On a égalité des DL à gauche de $0$, en $0$ et à droite de $0$, donc on peut écrire ces DL à tout ordre pour tout $x$.
\end{sol}

\exercice{nom={Développement limité de $\tan x$}}
\begin{questions}
\question Démontrer que $\tan x$ et $\tan' x$ admettent un développement
  limité en $0$ à tout ordre. Expliquer comment obtenir le développement
  limité de $\tan'x$ à partir de celui de $\tan x$.
\question En exploitant la relation $\tan'x=1+\tan^2 x$, donner le
  développement limité de $\tan x$ en 0 à l'ordre 7.
\end{questions}

\begin{sol}
\begin{questions}
\question Ces fonctions sont $\mathcal{C}^{\infty}$.
 Quand on sait que le DL(n) de $f$ existe et le DL(n+1) de $f'$ existe alors on passe bien du premier au deuxième en dérivant. Pour le voir concrètement, on écrit les deux DLs, on primitive celui de $f'$ et on utilise l'unicité du DL...
\question On part de $\tan(x)=x+\petitozero{x}{x}$. Alors $\tan'(x)=1+(x+\petitozero{x}{x})^2=1+x^2+\petitozero{x}{x^2}$ donc en primitivant :
$$\tan(x)=x+\dfrac{1}{3}x^3+\petitozero{x}{x^3}.$$
Et on recommence :
$$\tan'(x)=1+(x+\dfrac{1}{3}x^3+\petitozero{x}{x^3})^2=1+x^2(1+\dfrac{1}{3}x^2+\petitozero{x}{x^2})^2=1+x^2(1+\dfrac{2}{3}x^2+\petitozero{x}{x^2})=1+x^2+\dfrac{2}{3}x^4+\petitozero{x}{x^4}$$
Ainsi,
$$\tan(x)=x+\dfrac{1}{3}x^3+\dfrac{2}{15}x^5+\petitozero{x}{x^5}$$
Et on recommence :
\begin{eqnarray*}
\tan'(x)&=& 1+(x+\dfrac{1}{3}x^3+\dfrac{2}{15}x^5+\petitozero{x}{x^5})^2\\
&=& 1+x^2(1+\dfrac{1}{3}x^2+\dfrac{2}{15}x^4+\petitozero{x}{x^4})^2\\
&=& 1+x^2(1+\dfrac{2}{3}x^2+\dfrac{17}{45}x^4+\petitozero{x}{x^4})\\
&=& 1+x^2+\dfrac{2}{3}x^4+\dfrac{17}{45}x^6+\petitozero{x}{x^6}
\end{eqnarray*}
Ainsi,
$$\tan(x)=x+\dfrac{1}{3}x^3+\dfrac{2}{15}x^5+\dfrac{17}{315}x^7+\petitozero{x}{x^7}$$
et on s'arrête ! ENFIN !

\end{questions}



\end{sol}

\exercice{nom={Calcul}}
\begin{questions}
\question Donner le développement limité de
  \[\integinv{x}{x^2}{\sqrt{1+t^2}}{t}\]
  en 0 à l'ordre 4.
\question Sur le même modèle, donner un développement limité de
  \[\integ{x}{\frac{1}{x}}{\e^{-t^2}}{t}\]
  en 1 à l'ordre 3.
\end{questions}
\begin{sol}
$\quad$
\begin{questions}
\question On écrit $F(x)=H(x^2)-H(x)$ qu'on dérive :
$$F'(x)=2xh(x^2)-h(x)=2x\frac{1}{\sqrt{1+x^4}}-\frac{1}{\sqrt{1+x^2}}$$ dont on va faire un développement limité en $0$ à l'ordre $3$ pour le primitiver pour conclure.
$$\frac{1}{\sqrt{1+u}}=1-\frac{1}{2}u+\frac{3}{8}u^2+\petitozero{u}{u^2}$$ donc 
$$\frac{1}{\sqrt{1+x^2}}=1-\frac{1}{2}x^2+\petitozero{x}{x^3}$$
et $$2x\frac{1}{\sqrt{1+x^4}}=2x\p{1-\frac{1}{2}x^4+\petitozero{x}{x^4}}=2x+\petitozero{x}{x^3}$$
d'où :
$$F'(x)=-1+2x+\frac{1}{2}x^2+\petitozero{x}{x^3}$$

On trouve finalement en primitivant et parce-que $F(0)=0$ :
  \[\integinv{x}{x^2}{\sqrt{1+t^2}}{t}=-x+x^2+\frac{1}{6}x^3+\petitozero{x}{x^4}\]
\question On cherche un DL d'ordre $3$ en $0$ de 
$$\integ{1+h}{\frac{1}{1+h}}{e^{-t^2}}{t}=G\p{\frac{1}{1+h}}-G(1+h)$$ qu'on dérive :
$$\frac{d}{dh}\p{\integ{1+h}{\frac{1}{1+h}}{e^{-t^2}}{t}}=\frac{-1}{(1+h)^2}\e^{-\frac{1}{(1+h)^2}}-\e^{-(1+h)^2}$$ que l'on va développer à l'ordre $2$.

Or, $$\frac{1}{(1+h)^2}=\frac{1}{1+2h+h^2}=1-(2h+h^2)+(2h+h^2)^2+\petitozero{h}{h^2}=1-2h+3h^2+\petitozero{h}{h^2}$$
donc
$$\e^{-\frac{1}{(1+h)^2}}=\frac{1}{e}\e^{2h-3h^2+\petitozero{h}{h^2}}=\frac{1}{e}\p{1+2h-3h^2+\frac{(2h-3h^2)^2}{2}+\petitozero{h}{h^2}}=\frac{1}{e}\p{1+2h-h^2+\petitozero{h}{h^2}}$$
Enfin,
$$\e^{-(1+h)^2}=\frac{1}{e}\e^{-2h-h^2}=\frac{1}{e}\p{1-2h-h^2+\frac{(-2h-h^2)^2}{2}+\petitozero{h}{h^2}}=\frac{1}{e}\p{1-2h+h^2+\petitozero{h}{h^2}}$$.
Lorsqu'on rassemble les trois DL précédents, on obtient :
\begin{eqnarray*}
\frac{d}{dh}\p{\integ{1+h}{\frac{1}{1+h}}{e^{-t^2}}{t}}&=&-\p{1-2h+3h^2+\petitozero{h}{h^2}}\p{\frac{1}{e}\p{1+2h-h^2+\petitozero{h}{h^2}}}-\frac{1}{e}\p{1-2h+h^2+\petitozero{h}{h^2}}\\
&=&\frac{-1}{e}\p{1-2h^2+\petitozero{h}{h^2}+\p{1-2h+h^2+\petitozero{h}{h^2}}}\\
&=&\frac{-1}{e}\p{2-2h-h^2+\petitozero{h}{h^2}}
\end{eqnarray*}
On trouve finalement en primitivant et parce-que $F(0)=0$ :
  \[\integ{1+h}{\frac{1}{1+h}}{e^{-t^2}}{t}=-\frac{2}{e}h+\frac{1}{e}h^2+\frac{1}{3e}h^3+\petitozero{h}{h^3}\]
\end{questions}
\end{sol}

\exercice{nom={Calcul}}
Donner le développement limité en 0 à l'ordre $n+1$ de
\[\ln\p{1+x+\frac{x^2}{2!}+\dots+\frac{x^n}{n!}}.\]

\begin{sol}
$\displaystyle \sum_{k=0}^{n}\dfrac{x^k}{k!}=e^x-\dfrac{x^{n+1}}{(n+1)!}+o(x^{n+1})$ d'où :\\$\ln\p{
\displaystyle \sum_{k=0}^{n}\dfrac{x^k}{k!}}=\ln\p{e^x-\dfrac{x^{n+1}}{(n+1)!}+o(x^{n+1})}=\ln(e^x)+\ln\p{1-\dfrac{x^{n+1}e^{-x}}
{ (n+1)! }+o(x^{(n+1)})}=x-\dfrac{x^{n+1}}{(n+1)!}+o(x^{n+1})$.
\end{sol}

\magsection{Développement asymptotique}
\exercice{nom={Calcul}}
Calculer les développements asymptotiques suivants
\[\sqrt[3]{x^3+x^2}-\sqrt[3]{x^3-x^2} \quad\text{en $+\infty$ à 2 termes},
  \qquad \ln\p{\sqrt{1+x}} \quad\text{en $+\infty$ à 2 termes}.\]
\begin{sol}
On trouve
\[\sqrt[3]{x^3+x^2}-\sqrt[3]{x^3-x^2}=x\p{\sqrt[3]{1+\frac1x}-\sqrt[3]{1-\frac1x}}=\frac{2}{3}+\frac{10}{81}\cdot\frac{1}{x^2}
  +\petito{x}{+\infty}{\frac{1}{x^2}}\]
\[\ln\p{\sqrt{1+x}}=\ln(\sqrt{x})+\frac{1}{2}\ln(1+x)=\frac{1}{2}\ln(x)+\frac{1}{2}\p{\frac{1}{x}+\petito{x}{+\infty}{\frac{1}{x}}}\frac{1}{2}\ln x+\frac{1}{2}\cdot\frac{1}{x}
  +\petito{x}{+\infty}{\frac{1}{x}}\]
\end{sol}


\exercice{nom={Développement de $\arcsin x$ en $-1$}}
\begin{questions}
\question Établir une relation entre
  $$\arcsin\sqrt{x} \et \frac{\pi}{4}+\frac{1}{2}\arcsin\p{2x-1}.$$
\question En déduire un développement asymptotique de $\arcsin x$ en $-1$
  à la précision $x^2$.
\end{questions}
\begin{sol}
\begin{enumerate}
\item On dérive ces deux fonctions : elles ont la même dérivée. Donc elles sont égales à une constante près. Mais ces deux fonctions sont égales en 0, donc elles sont égales sur leur domaine de $\mathcal{D}_f$ (au fait, quel est ce domaine ?).
\item Soit $h$ proche de 0, avec $h>0$. Avec la question précédente, on a : $\arcsin(h-1)=2\p{\arcsin(\sqrt{h/2})-\dfrac\pi4}$. Il ne reste plus qu'à faire un développement de $\arcsin(\sqrt{h/2})$ quand $h$ tend vers 0 : ce ne sera pas un DL en $h$, mais un ``DL en $\sqrt h$\ ''. On le veut à la précision $h^2$, donc à l'ordre 4 en $\sqrt h$.
Or, $$\arcsin(x)=x+\frac{x^3}{6}+\petitozero{x}{x^4}$$ d'où $$\arcsin\p{\sqrt{\frac{h}{2}}}=\sqrt{\frac{h}{2}}+\frac{1}{6}\sqrt{\frac{h}{2}}\frac{h}{2}+\petitozero{h}{h^2}$$
\[\arcsin(-1+h)=-\frac{\pi}{2}+\sqrt{2}\sqrt{h}+\frac{\sqrt{2}}{12}h\sqrt{h}+
  \petitozero{h}{h^2}\]
\end{enumerate}

\end{sol}

\exercice{nom={Calcul de limites}}
Calculer les limites des expressions suivantes lorsqu'elles existent.
\[\p{\tan x}^{\tan 2x} \quad \text{en $\frac{\pi}{4}$}, \qquad\qquad
  \frac{1}{x}-\frac{1}{\ln\p{1+x}} \quad \text{en 0},\]
\[\frac{\p{1+x}^{\frac{1}{x}}-\e}{x} \quad \text{en 0}, \qquad\qquad
  \frac{1}{2\p{1-\sqrt{x}}}-\frac{1}{3\p{1-\sqrt[3]{x}}} \quad \text{en 1},\]
\[\frac{1}{\sin^4 x}\p{\sin\frac{x}{1-x}-\frac{\sin x}{1-\sin x}}
  \quad\text{en 0}, \qquad\qquad
  \frac{\p{1+x}^{\frac{\ln x}{x}}-x}{x\p{x^x-1}} \quad \text{en 0}.\]
\begin{sol}
\begin{enumerate}
\item Grâce à $\tan'=1+\tan^2$ on a d'après la formule de T-Y :
$$\tan\p{\frac{\pi}{4}+h}=1+h+\petitozero{h}{h},$$
et on a aussi $\tan(h)=h+\petitozero{h}{h^2}$.
Mais alors :
\begin{eqnarray*}
\p{\tan\p{\frac{\pi}{4}+h}}^{\tan\p{\frac{\pi}{2}+2h}}&=&\p{\tan\p{\frac{\pi}{4}+h}}^{-\frac{1}{\tan(2h)}}\\
&=&\e^{-\frac{1}{\tan(2h)}\ln\p{\tan\p{\frac{\pi}{4}+h}}}\\
&=&\e^{-\frac{1}{2h+\petitozero{h}{h^2}}\ln\p{1+2h+\petitozero{h}{h}}}\\
&=&\e^{-\frac{1}{2h}(1+\petitozero{h}{h})\p{2h+\petitozero{h}{h}}}\\
&=&\e^{-\frac{1}{2h}(2h+\petitozero{h}{h})}\\
&=&\boxed{\e^{-1+\petitozero{h}{1}}\tendvers{h}{0}{\frac{1}{e}}.}
\end{eqnarray*}
D'où :
\[\boxed{\p{\tan x}^{\tan 2x}\tendvers{x}{\frac{\pi}{4}}\frac{1}{e}.}\]
\item \begin{eqnarray*}
\frac{1}{x}-\frac{1}{\ln\p{1+x}}&=&\frac{1}{x}-\frac{1}{x-\frac{x^2}{2}+\petitozero{x}{x^2}}\\
&=&\frac{1}{x}\p{1-\frac{1}{1-\frac{x}{2}+\petitozero{x}{x}}}\\
&=&\frac{1}{x}\p{1-\p{1+\frac{x}{2}+\petitozero{x}{x}}}\\
&=&\boxed{-\dfrac12+\petitozero{x}{1}.}
\end{eqnarray*}
\item 
\begin{eqnarray*}
\frac{\p{1+x}^{\frac{1}{x}}-\e}{x}&=&\frac{\e^{\frac{1}{x}\ln(1+x)}-\e}{x}\\
&=&\frac{\e^{\frac{1}{x}\p{x-\frac{x^2}{2}+\petitozero{x}{x^2}}}-\e}{x}\\
&=&\frac{\e^{1-\frac{x}{2}+\petitozero{x}{x}}-\e}{x}\\
&=&\frac{e\p{1-\frac{x}{2}+\petitozero{x}{x}-1}}{x}\\
&=&\boxed{e\p{-\frac{1}{2}+\petitozero{x}{1}}\tendvers{x}{0}{-\frac{e}{2}}.}
\end{eqnarray*}
\item 
\begin{eqnarray*}
\frac{1}{2\p{1-\sqrt{1+h}}}-\frac{1}{3\p{1-\p{1+h}^{\frac{1}{3}}}}&=&\frac{1}{2\p{1-\p{1+\frac{1}{2}h-\frac{1}{8}h^2+\petitozero{h}{h^2}}}}-\frac{1}{3\p{1-\p{1+\frac{1}{3}h-\frac{1}{9}h^2+\petitozero{h}{h^2}}}}\\
&=&\frac{1}{2\p{-\frac{1}{2}h+\frac{1}{8}h^2+\petitozero{h}{h^2}}}-\frac{1}{3\p{-\frac{1}{3}h+\frac{1}{9}h^2+\petitozero{h}{h^2}}}\\
&=&\frac{1}{h}\p{\frac{-1}{1-\frac{1}{4}h+\petitozero{h}{h}}+\frac{1}{1-\frac{1}{3}h+\petitozero{h}{h}}}\\
&=&\frac{1}{h}\p{1+\frac{1}{3}h+\petitozero{h}{h}-\p{1+\frac{1}{4}h+\petitozero{h}{h}}}\\
&=&\frac{1}{h}\p{\frac{1}{12}h+\petitozero{h}{h}}\\
&=&\boxed{\frac{1}{12}+\petitozero{h}{1}\tendvers{h}{0}{\frac{1}{12}}.}
\end{eqnarray*}
\item On a $$\sin(x)=x-\frac{x^3}{6}+\petitozero{x}{x^4}$$
et 
\begin{eqnarray*}
\frac{1}{1-\sin(x)}&=&\frac{1}{1-\p{x-\frac{x^3}{6}}+\petitozero{x}{x^3}}\\
&=&1+\p{x-\frac{x^3}{6}}+\p{x-\frac{x^3}{6}}^2+\p{x-\frac{x^3}{6}}^3+\petitozero{x}{x^3}\\
&=&1+x+x^2+\frac{5}{6}x^3+\petitozero{x}{x^3}
\end{eqnarray*}
donc 
$$\frac{\sin(x)}{1-\sin(x)}=x+x^2+\frac{5}{6}x^3+\frac{2}{3}x^4+\petitozero{x}{x^4}.$$
De plus, $$\frac{x}{1-x}=x+x^2+x^3+x^4+\petitozero{x}{x^4},$$
donc $$\sin\p{\frac{x}{1-x}}=x+x^2+x^3+x^4-\frac{1}{6}\p{x+x^2+x^3+x^4}^3+\petitozero{x}{x^4}=x+x^2+\frac{5}{6}x^3+\frac{1}{2}x^4+\petitozero{x}{x^4},$$
mais alors

$$\boxed{\frac{1}{\sin^4 x}\p{\sin\frac{x}{1-x}-\frac{\sin x}{1-\sin x}}=\frac{1}{\sin^4 x}\p{-\frac{1}{6}x^4+\petitozero{x}{x^4}}\equi{x}{0}{-\frac{1}{6}}.}$$

\item $$x\p{x^x-1}=x\p{e^{x\ln(x)}-1}\equi{x}{0}{x^2\ln(x)}$$
et 
\begin{eqnarray*}
\p{1+x}^{\frac{\ln(x)}{x}}-x&=& \e^{\frac{\ln(x)}{x}\ln(1+x)}-x\\
&=& \e^{\frac{\ln(x)}{x}\p{x-x^2/2+\petitozero{x}{x^2}}}-x\\
&=&\e^{\ln(x)}\e^{-x\ln(x)/2+\petitozero{x}{x\ln(x)}}-x\\
&=&x\p{\e^{-x\ln(x)/2+\petitozero{x}{x\ln(x)}}-1}\\
&\equi{x}{0}&x\p{-\frac{x}{2}\ln(x)}
\end{eqnarray*}
\fbox{Ainsi, le quotient étudié équivaut à $-\dfrac12 $ en $0$.}
\end{enumerate}

\end{sol}

\exercice{nom={Fonction de classe $\classec{1}$}}
Soit $f$ la fonction définie par
\[\forall x\in\interf{-\pi/2}{\pi/2} \qsep f(x)\defeq
  \begin{cases}
  \frac{1}{\sin x}-\frac{1}{x} & \text{si $x\neq 0$}\\
  0 & \text{si $x=0$.}
  \end{cases}\]
Montrer que $f$ est de classe $\classec{1}$ sur $\interf{-\pi/2}{\pi/2}$.

\begin{sol}
$\forall x \in \intero{-\pi/2}{\pi/2}\setminus{0}$, on a :
\begin{eqnarray*}
\frac{1}{\sin x}-\frac{1}{x}&=&\frac{1}{x}\p{\frac{x}{\sin x}-1}\\
&=&\frac{1}{x}\p{\frac{1}{1-\frac{x^2}{6}+\petitozero{x}{x^2}}-1}\\
&=&\frac{1}{x}\p{1+\frac{x^2}{6}+\petitozero{x}{x^2}-1}\\
&=&\frac{1}{6}x+\petitozero{x}{x}.
\end{eqnarray*}
Donc $f(x)=\frac{1}{6}x+\petitozerop{x}{x}$. Or, $f(0)=0$ donc $f(x)=\frac{1}{6}x+\petitozero{x}{x}$. On en déduit que $f$ est continue en $0$ et dérivable en $0$ de dérivée $1/6$.

$\forall x \in \intero{-\pi/2}{\pi/2}\setminus{0}$, on a :
\begin{eqnarray*}
f'(x)&=&\frac{-\cos(x)}{\sin^2 x}+\frac{1}{x^2}\\
&=&\frac{1}{x^2}\p{1-\frac{x^2}{\sin^2 x}\cos(x)}\\
&=&\frac{1}{x^2}\p{1-\p{1+\frac{1}{6}x^2+\petitozero{x}{x^2}}^2\p{1-\frac{1}{2}x^2+\petitozero{x}{x^2}}}\\
&=&\frac{1}{x^2}\p{1-\p{1+\frac{1}{3}x^2+\petitozero{x}{x^2}}\p{1-\frac{1}{2}x^2+\petitozero{x}{x^2}}}\\
&=&\frac{1}{x^2}\p{1-\p{1-\frac{1}{6}x^2+\petitozero{x}{x^2}}}\\
&=&\frac{1}{6}+\petitozero{x}{1}
\end{eqnarray*}
Donc $f'$ est bien continue en $0$ et finalement, $f$ est bien $\classec{1}$ sur $\intero{-\pi/2}{\pi/2}$.

\end{sol}


\exercice{nom={Asymptotes}}
Étudier les asymptotes des graphes des fonctions suivantes
\[\sqrt[3]{\p{x^2-2}\p{x+3}}, \qquad \frac{1}{x}\p{2x^2-1}\e^{\frac{1}{x}}.\]
\begin{sol}
On a~:
\[\sqrt[3]{\p{x^2-2}\p{x+3}}=x+1-\frac{5}{3}\cdot\frac{1}{x}+\petito{x}{\pm\infty}{\frac{1}{x}}\]
\[\frac{1}{x}\p{2x^2-1}e^{\frac{1}{x}}=2x+2-\frac{2}{3}\cdot\frac{1}{x^2}+\petito{x}{\pm\infty}{\frac{1}{x^2}}\]
\end{sol}

\exercice{nom={Étude d'une fonction}}
Soit $f$ la fonction définie par
\[f(x)\defeq\frac{1}{x}\ln\frac{\e^x-1}{x}\]
\begin{questions}
\question Calculer $f\p{-x}$ en fonction de $f(x)$. Que peut-on en déduire
  sur le graphe de $f$~?
\question Donner l'allure du graphe de $f$ au voisinage de 0.
\question Étudier le comportement asymptotique de $f$.
\question Montrer que la fonction d'expression $xf(x)$ est convexe (sa dérivée seconde est positive) sur $\RPs$.
  En déduire les variations de $f$ et tracer sa courbe représentative.
\end{questions}
\begin{sol}
$\quad$
\begin{questions}
\question Le domaine de définition est $\Rs$ et on trouve $f(-x)=1-f(x)$.
  La symétrie par rapport au point $O$ de coordonnées $(0,1/2)$ est donc une
  symétrie du graphe de $f$.
\question Inutile de faire un développement limité à l'ordre 2 car
  $f$ est \og impaire \fg. Il faut faire un développement limité de $f$
  à l'ordre 3. On trouve
  \[f(x)=\frac{1}{2}+\frac{1}{24}x-\frac{1}{2880}x^3+\petitozero{x}{x^3}\]
\question $y=1$ est une asymptote et le graphe de $f$ est en dessous.
\question Il faut montrer que $g(x)=\ln((e^x-1)/x)$ est convexe. On dérive deux
  fois et on doit montrer que $(e^x-1)^2\geq x^2 e^x$ c'est-à-dire que
  $x\leq 2\sh(x/2)$ ce qui est trivial.
\end{questions}
\end{sol}

\exercice{nom={La formule de \nom{Stirling}}}
Le but de cet exercice est calculer de calculer un équivalent de $n!$.
On considère la suite $u$ définie par
$$\forall n \geq 0 \qsep u_n\defeq\frac{n^{n+\frac{1}{2}}}{\e^n n!}.$$
\begin{questions}
\question Montrer que
  $$\forall n \geq 1 \qsep \ln\left(\frac{u_{n+1}}{u_n}\right)=
    n\cro{\ln\left(1+\frac{1}{n}\right)-\frac{1}{n}}+
    \frac{1}{2}\ln\left(1+\frac{1}{n}\right).$$
\question En déduire que
  $$\ln\left(\frac{u_{n+1}}{u_n}\right)
    \underset{n\rightarrow +\infty}{\sim}\frac{1}{12n^2}.$$
\question En déduire que la suite de terme général
  \[S_n\defeq\sum_{k=1}^{n-1} \ln\left(\frac{u_{k+1}}{u_k}\right)\]
  est monotone à partir d'un certain
  rang. Montrer qu'il existe un rang à partir duquel
  \[\ln\left(\frac{u_{k+1}}{u_k}\right)\leq\frac{1}{6k^2},\]
  puis en déduire que $(S_n)$ est convergente.
\question En déduire que la suite de terme général $\ln(u_n)$ est convergente.
\question En déduire l'existence d'un réel $a>0$ tel que 
  $$n!\underset{n\rightarrow+\infty}{\sim}a\sqrt{n}\left(\frac{n}{\e}\right)^n.$$
\question En utilisant les résultats de l'exercice sur les intégrales de \nom{Wallis}
  (compléments d'analyse), montrer que $a=\sqrt{2\pi}$.
\end{questions}
%END_BOOK

\end{document}