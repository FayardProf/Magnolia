\documentclass{magnolia}

\magtex{tex_driver={pdftex}}
\magfiche{document_nom={Exercices sur les fonctions de deux variables},
          auteur_nom={François Fayard},
          auteur_mail={fayard.prof@gmail.com}}
\magexos{exos_matiere={maths},
         exos_niveau={mpsi},
         exos_chapitre_numero={3},
         exos_theme={Fonctions de deux variables}}
\magmisenpage{misenpage_presentation={tikzvelvia},
              misenpage_format={a4},
              misenpage_nbcolonnes={1},
              misenpage_sol={non}}
\maglieudiff{lieu_lycee={Aux Lazaristes},
             lieu_classe={MPSI 1},
             lieu_annee={2019--2020}}
\magprocess

\begin{document}

%BEGIN_BOOK
\magsection{Limite, continuité}

\exercice{nom={Calcul de limites}}
Déterminer les limites, si elles existent, des fonctions suivantes en $\p{0,0}$.
\[f\p{x,y}\defeq\p{x+y}\sin\p{\frac{1}{x^2+y^2}}, \qquad
  f\p{x,y}\defeq\frac{x^2-y^2}{x^2+y^2}, \qquad
  f\p{x,y}\defeq\frac{\abs{x+y}}{x^2+y^2},\]
\[f\p{x,y}\defeq\frac{x^3+y^3}{x^2+y^2}, \qquad
  f\p{x,y}\defeq\frac{x^2+y^2-1}{x}\sin x, \qquad
  f\p{x,y}\defeq x^y.\]

\exercice{nom={Fonction définie par morceaux}}
Soit $f$ la fonction définie sur $\R^2$ par
\[\forall (x,y)\in\R^2 \qsep f(x,y)\defeq
  \begin{cases}
  \frac{1}{2}x^2+y^2-1 & \text{si $x^2+y^2>1$,}\\
  -\frac{1}{2}x^2 & \text{si $x^2+y^2\leq 1$.}
  \end{cases}
\]
Montrer que $f$ est continue en tout point de $\R^2$.

\exercice{nom={L'image d'un convexe}}
Soit $f$ une fonction continue sur un convexe $P$ de $\R^2$. Montrer que $f(P)$
est un intervalle.

\magsection{Dérivation}

\exercice{nom={Régularité d'une fonction}}
Soit $f$ la fonction définie sur $\R^2$ par
\[\forall\p{x,y}\in\R^2 \qsep
  f\p{x,y}\defeq
  \begin{cases}
  \frac{\sin\p{x^2}+\sin\p{y^2}}{\sqrt{x^2+y^2}} &
    \text{si $\p{x,y}\neq\p{0,0}$,}\\
  0 &  \text{si $\p{x,y}=\p{0,0}$.}
  \end{cases}\]
\begin{questions}
\question $f$ admet-elle des dérivées partielles au point $\p{0,0}$~?
\question $f$ est-elle continue en $\p{0,0}$~?
\question $f$ est-elle de classe $\classec{1}$~?   
\end{questions}

\exercice{nom={Dérivées partielles et continuité}}
Soit $f$ la fonction définie sur $\R^2$ par
\[\forall (x,y)\in\R^2 \qsep f(x,y)\defeq
  \begin{cases}
  \frac{x^2y}{x^4+y^2} & \text{si $(x,y)\not=(0,0)$,}\\
  0                    & \text{si $(x,y)=(0,0)$.}  
  \end{cases}\]
Montrer que $f$ admet une dérivée selon tout vecteur en $(0,0)$ mais n'est pas
continue en $0$.

\exercice{nom={Calcul}}
Soit $f:\p{x,y}\mapsto xy^2$.
\begin{questions}
\question Déterminer la dérivée de $f$ au point $a\defeq \p{-1,3}$ selon le vecteur
  $h\defeq \p{2,-3}$.
\question Écrire le développement limité de $f$ à l'ordre 1 en $a$.
\end{questions}
% \begin{solution}
% $\quad$
% \begin{questions}
% \question On trouve 36.
% \question C'est $-9+9h_1-6h_2$.
% \end{questions}
% \end{solution}

\exercice{nom={Calcul de dérivées}}
Soit $f$ une fonction de classe $\classec{1}$ sur $\R^2$. Calculer les dérivées
ou dérivées partielles des fonctions suivantes
\[g_1\p{x,y}\defeq f(y,x), \qquad g_2(x)\defeq f(x,x),\]
\[g_3\p{x,y}\defeq f(y,f(x,x)), \qquad g_4(x)\defeq f(x,f(x,x)).\]

% \exercice{nom={Équation aux dérivées partielles}}
% Soit $h$ une fonction de la variable $x$.
% \begin{questions}
% \question Montrer que, pour qu'il existe une fonction
%   $f\in\classec{2}\p{\R^2,\R}$ telle que~:
%   \[\parfrac{f}{x}\p{x,y}=-x\p{y^2+1}h(x) \qquad
%     \parfrac{f}{y}\p{x,y}=2x^2yh(x) \quad \p{1}\]
%   il faut que $h$ vérifie~:
%   \[xh'(x)+3h(x)=0 \quad \p{2}\]
% \question Déterminer les solutions de $\p{2}$, puis trouver $f$ de classe
%   $\classec{2}$ satisfaisant $\p{1}$.
% \end{questions}

% \exercice{nom={Laplacien}}
% Soit $u$ une fonction réelle des variables réelles $x$ et $y$ définie par
% $u\p{x,y}=\p{F\circ r}\p{x,y}$ où $r\p{x,y}=\sqrt{x^2+y^2}$ et $F$ est une
% fonction réelle d'une variable réelle. On pose~:
% \[\Delta u=\frac{\partial^2 u}{\partial x^2}
%           +\frac{\partial^2 u}{\partial y^2}\]
% \begin{questions}
% \question Calculer~:
%   \[\parfrac{r}{x} \qquad \parfrac{r}{y} \qquad
%     \frac{\partial^2 r}{\partial x^2} \qquad \frac{\partial^2 r}{\partial y^2}\]
% \question Prouver que~:
%   \[\Delta u=F''(r)+\frac{F'(r)}{r}\]
% \question En déduire $\Delta u$ lorsque $u\p{x,y}=\ln\p{x^2+y^2}$.
% \end{questions}

% \exercice{nom={Équations aux dérivées partielles}}
% Résoudre les équations aux dérivées partielles suivantes~:
% \[x^2\frac{\partial^2 f}{\partial x\partial y}=1 \qquad
%   \frac{\partial^2 f}{\partial x\partial y}=\frac{x}{y}+a\]
% \[\frac{\partial^2 f}{\partial x^2}=xy \qquad
%   \frac{\partial f}{\partial x}=af\]

\exercice{nom={Équation aux dérivées partielles}}
On pose $\mathcal{U}\defeq\R^2\setminus\enstq{(x,y)\in\R^2}{y=0 \et x\leq 0}$. On souhaite
déterminer l'ensemble des fonctions $f:\mathcal{U}\mapsto\R$ de classe $\classec{1}$ telles que
\[(E) \qquad \forall (x,y)\in\mathcal{U} \qsep
  x\frac{\partial f}{\partial x}(x,y)+y\frac{\partial f}{\partial y}(x,y)=
  \sqrt{x^2+y^2}.\]
\begin{questions}
\question On pose $\mathcal{V}\defeq \RPs\times\intero{-\pi}{\pi}$ et on définit
  \[\dspappli{\phi}{\mathcal{V}}{\mathcal{U}}{(r,\theta)}{(r\cos \theta,r \sin \theta)}\]
  Montrer que $\phi$ est une bijection.
\question Soit $f:\mathcal{U}\to\R$ une fonction de classe $\classec{1}$. On
  définit la fonction $g:\mathcal{V}\to\R$ par
  \[\forall (r,\theta)\in\mathcal{V}\qsep g(r,\theta)\defeq f(r\cos \theta,r\sin\theta).\]
  Montrer que $g$ est de classe $\classec{1}$ et déterminer les dérivées partielles de $g$ en fonction de celles de $f$.
\question En déduire l'ensemble des solutions de $(E)$.
\end{questions}
% \begin{solution}
% On trouve~:
% \[\forall (x,y)\in\mathcal{U} \quad f(x,y)=
%   \sqrt{x^2+y^2}+\phi\p{2\arctan\frac{y}{\sqrt{x^2+y^2}+x}}\]
% où $\phi$ est une fonction de classe $\classec{1}$ de $\intero{-\pi}{\pi}$
% dans $\R$.    
% \end{solution}

%END_BOOK

\end{document}