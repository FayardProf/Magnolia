\documentclass{magnolia}

\magtex{tex_driver={pdftex}}
\magfiche{document_nom={Exercices sur la dérivation},
          auteur_nom={François Fayard},
          auteur_mail={fayard.prof@gmail.com}}
\magexos{exos_matiere={maths},
         exos_niveau={mpsi},
         exos_chapitre_numero={11},
         exos_theme={Dérivation}}
\magmisenpage{}
\maglieudiff{}
\magprocess

\begin{document}

%BEGIN_BOOK
\magsection{Fonction dérivable, dérivées successives}


% \exercice{nom={Une fonction dont la dérivée n'est pas continue}}
% Soit $f$ la fonction définie sur $\R$ par
% $$\forall x\in\R \qsep f(x)\defeq
%   \begin{cases}
%   x^2\sin\p{\frac{1}{x}} & \text{si $x\not=0$}\\
%   0 & \text{si $x=0$.}
%   \end{cases}
% $$
% \begin{questions}
%   \item Étudier la continuité de $f$
%   \item Montrer que $f$ est dérivable sur $\R$ et calculer sa dérivée.
%   \item Montrer que $f'$ n'est pas continue en $0$.
% \end{questions}
% \begin{sol}
% \begin{questions}
% \question Soit $f$ la fonction définie sur $\R$ par
% \[\forall x\in\R \quad f(x)=
%   \begin{cases}
%   x^2\sin\p{\frac{1}{x}} & \text{si $x\not=0$}\\
%   0 & \text{si $x=0$.}
%   \end{cases}\]
% D'après les théorèmes usuels, $f$ est continue sur $\Rs$. Montrons que $f$ est continue en 0. On a
% \begin{eqnarray*}
% \forall x\in\Rs\qsep \abs{f(x)}
% &=& \abs{x^2\sin\p{\frac{1}{x}}}\\
% &=& x^2\abs{\sin\p{\frac{1}{x}}}\\
% &\leq& x^2\tendvers{x}{0} 0
% \end{eqnarray*}
% donc
% \[f(x)\tendversp{x}{0}0=f(0).\]
% On en déduit que $f$ est continue en 0.
% \question D'après les théorèmes usuels, $f$ est dérivable sur $\Rs$ et
% \[\forall x\in\Rs\qsep f'(x)=2x\sin\p{\frac{1}{x}}-\cos\p{\frac{1}{x}}.\]
% Montrons que $f$ est dérivable en 0 et que $f'(0)=0$. On a
% \begin{eqnarray*}
% \forall x\in\Rs\qsep \abs{\frac{f(x)-f(0)}{x-0}}
% &=& \abs{\frac{f(x)}{x}}\\
% &=& \abs{x\sin\p{\frac{1}{x}}}\\
% &=& \abs{x}\abs{\sin\p{\frac{1}{x}}}\\
% &\leq& \abs{x}\tendvers{x}{0} 0
% \end{eqnarray*}
% donc 
% \[\frac{f(x)-f(0)}{x-0}\tendvers{x}{0}0\]
% ce qui prouve que $f$ est dérivable en 0 et que $f'(0)=0$. En conclusion, $f$ est dérivable sur $\R$ et
% \[\forall x\in\R\qsep f'(x)=
%   \begin{cases}
%   2x\sin\p{\frac{1}{x}}-\cos\p{\frac{1}{x}} & \text{si $x\not=0$}\\
%   0 & \text{si $x=0$.}
%   \end{cases}\]
% \question Montrons que $f'$ n'est pas continue en 0. On raisonne par l'absurde et on suppose que $f'$ est continue en 0. Alors, comme $f'(0)=0$
% \[f'(x)\tendvers{x}{0}0\]
% donc, puisque
% \[2x\sin\p{\frac{1}{x}}\tendvers{x}{0}0\]
% on en déduit que
% \[\cos\p{\frac{1}{x}}=2x\sin\p{\frac{1}{x}}-f'(x)\tendvers{x}{0}0.\]
% On définit la suite $(u_n)$ par
% \[\forall n\in\Ns\qsep u_n\defeq\frac{1}{2\pi n}.\]
% Alors
% \[u_n\tendvers{n}{+\infty}0\]
% mais
% \[\cos\p{\frac{1}{u_n}}=\cos(2\pi n)=1\tendvers{n}{+\infty}1\neq 0.\]
% C'est absurde. Donc $f'$ n'est pas continue en 0.
% \end{questions}
% \end{sol}

\exercice{nom={Dérivabilité}}
Soit $\alpha\in\R$. Étudier la dérivabilité sur $\R$ de la fonction $f_\alpha$
définie par
\[\forall x\in\R \qsep f_\alpha(x)\defeq
  \begin{cases}
  \abs{x}^\alpha \sin\p{\frac{1}{x}} & \text{si $x\neq 0$}\\
  0 & \text{si $x=0$}
  \end{cases}\]
Préciser les nombres $\alpha$ pour lesquels $f_\alpha$ est de classe
$\classec{1}$ sur $\R$.

\exercice{nom={Suite}}
Soit $f:\R\to\R$ une fonction dérivable en 0 telle que $f\p{0}=0$.
Déterminer la limite éventuelle de la suite de terme général
\[u_n\defeq \sum_{k=0}^n f\p{\frac{k}{n^2}}.\]

\begin{sol}
Fixons $\epsilon>0$. On sait qu'il existe $\delta>0$ tel que, pour tout $x\in[-\delta,\delta]$, on a
$$|f(x)-f'(0)x|\leq \epsilon\times |x|.$$
Soit $n$ assez grand pour que $\frac{1}{n}\leq \delta$. On applique
alors la formule précédente à $x=\frac k{n^2}$, pour $k\in\{1,\dots,n\}$. On obtient
$$-\epsilon\frac k{n^2}\leq f\left(\frac k{n^2}\right)-f'(0)\frac{k}{n^2}\leq \epsilon\frac k{n^2}.$$
On somme ces inégalités pour $k$ allant de $1$ à $n$ et on trouve
$$-\epsilon\frac{n+1}{2n}\leq \sum_{k=1}^n f\left(\frac k{n^2}\right)-f'(0)\frac{n+1}{2n}\leq \epsilon\frac{n+1}{2n}.$$
En écrivant $\frac{n+1}{n}=1+\frac{1}{n}$ et en remarquant que $(n+1)/2n\leq 1$, on trouve
$$-\epsilon+\frac{f'(0)}{2n}\leq \sum_{k=1}^n f\left(\frac k{n^2}\right)-\frac{f'(0)}2\leq \epsilon+\frac{f'(0)}{2n}.$$
Pour $n$ assez grand, on en déduit que 
$$-2\epsilon\leq \sum_{k=1}^n f\left(\frac k{n^2}\right)-\frac{f'(0)}2\leq 2\epsilon.$$
On en conclut que la suite converge avec $$\lim_{n\to+\infty}\sum_{k=1}^n f\left(\frac k{n^2}\right)=\frac{f'(0)}{2}.$$

\underline{Solution privilégiée par Victor :}
$f$ est dérivable en $0$ donc admet un DL en $0$ d'ordre $1$ et il existe une fonction $\epsilon$ tel que $\epsilon(h)\tendvers{h}{0}0$ et :
$$f(h)=\underbrace{f(0)}_{=0}+f'(0)h+h\epsilon(h).$$
Ainsi, pour tout $n\in \N$ et pour tout $k\in \llbracket0,n\rrbracket$, en appliquant cela en $h=\dfrac{k}{n^2}$, il vient :
$$f\p{\frac{k}{n^2}}=f'(0)\frac{k}{n^2}+\frac{k}{n^2}\epsilon\p{\frac{k}{n^2}}$$ et en sommant :
$$\sum_{k=0}^nf\p{\frac{k}{n^2}}=\sum_{k=0}^nf'(0)\frac{k}{n^2}+\sum_{k=0}^n\frac{k}{n^2}\epsilon\p{\frac{k}{n^2}}.$$
Pour le premier terme 
$$\sum_{k=0}^nf'(0)\frac{k}{n^2}=\frac{f'(0)}{n^2}\sum_{k=0}^n k=f'(0) \frac{n+1}{2n}\tendvers{n}{+\infty}\frac{f'(0)}{2}.$$
Montrons maintenant que le deuxième terme tend vers $0$. Soit alors $\delta>0$. Comme $\epsilon(h)\tendvers{h}{0}0$, il existe $\eta>0$ tel que $\forall h \in [-\eta,\eta]$, $\abs{\epsilon(h)}\leq \delta$. Soit alors $N_0 \in \N$ tel que $\forall n\geq N_0$, $\dfrac{1}{n}\leq \eta$. Pour un tel $n$, on a alors $\forall k\in \llbracket 0, n\rrbracket$, $\dfrac{k}{n^2}\leq \dfrac{n}{n^2}=\dfrac{1}{n}\leq \delta$, donc :
$$\abs{\sum_{k=0}^n\frac{k}{n^2}\epsilon\p{\frac{k}{n^2}}}\leq \sum_{k=0}^n\frac{k}{n^2}\underbrace{\abs{\epsilon\p{\frac{k}{n^2}}}}_{\leq \delta}\leq \delta \frac{n(n+1)}{2n^2}=\frac{\delta}{2}\p{1+\frac{1}{n}}\leq \delta.$$
On vient de montrer que le deuxième terme tendait vers $0$ donc $u_n\tendvers{n}{+\infty}\dfrac{f'(0)}{2}$.
 
\end{sol}

% \exercice{nom={Dérivabilité}}
% Étudier la dérivabilité des fonctions $f$ et $g$ définies sur $\R$ par~:
% \[\forall x\in\R \quad f(x)=
%   \begin{cases}
%   x^2\sin\p{\frac{\pi}{x}} & \text{si $x\neq 0$}\\
%   0 & \text{sinon}
%   \end{cases}\]
% \[\forall x\in\R \quad g(x)=
%   \begin{cases}
%   f(x)\sin\p{\frac{\pi}{\sin\p{\frac{\pi}{x}}}} & \text{si
%     $\sin\p{\frac{\pi}{x}}\neq 0$}\\
%   0 & \text{sinon}
%   \end{cases}\]

\exercice{nom={Dérivabilité}}
Pour tout entier naturel $n$ on définit la fonction $f_n$ sur
$\R\setminus\pi\Z$ par
\[\forall x\in\R\setminus\pi\Z \qsep
  f_n(x)\defeq \frac{\sin\p{\p{2n+1}x}}{\sin x}.\]
\begin{questions}
\question Étudier le domaine de définition de $f_n$, sa parité, sa périodicité.
\question Montrer que $f_n$ se prolonge par continuité sur~$\R$. 
\question Montrer que la fonction ainsi prolongée est de classe
  $\classec{\infty}$ sur $\R$ (on pourra établir une relation entre $f_{n+1}$
  et $f_n$).
\end{questions}

\exercice{nom={Dérivabilité}}
Soit $f$ une fonction de classe $\classec{2}$ sur $\interf{0}{1}$ et
$a\in[0,1]$. 
\begin{questions}
\question Montrer que la fonction
  \[\dspappli{\tau_a}{\interf{0}{1}\setminus\ens{a}}{\R}{x}{%
    \frac{f(x)-f(a)}{x-a}}\]
  se prolonge par continuité en $a$.
\question Montrer que la fonction ainsi prolongée est de classe $\classec{1}$ sur
  $[0,1]$.
\end{questions}

\exercice{nom={Calcul de dérivées $n$-ièmes}}
Calculer les dérivées à l'ordre $n$ des fonctions d'expressions
\[\sin^5(x), \qquad x^2\e^x, \qquad
  x^{n-1}\ln(x), \qquad x^{n-1}\e^{\frac{1}{x}}.\]
\begin{sol}
\[\frac{\text{d}^5}{\text{d}x^5}\p{\sin^5 x}=
  \frac{1}{16}\cro{5^n\sin\p{5x+\frac{n\pi}{2}}
  -5\times 3^n\sin\p{3x+\frac{n\pi}{2}}+10\sin\p{x+\frac{n\pi}{2}}}\]
\[\frac{\text{d}^n}{\text{d}x^n}\p{\ln\p{x^2-x+2}}=
  (-1)^{n+1}(n-1)!\frac{\sum_{k=0}^n \binom{n}{k}(-1)^k
  2^{\frac{k}{2}+1}\cos\p{k\arctan\sqrt{7}}x^{n-k}}{(x^2-x+2)^n}\]
\[\frac{(n-1)!}{x}, \qquad (-1)^n\frac{\e^{1/x}}{x^{n+1}}\]
Les deux derniers se font par récurrence.
\end{sol}



\magsection{Théorème de \nom{Rolle} et applications}


\exercice{nom={Application directe du théorème de \nom{Rolle}}}
\begin{questions}
\question Soit $f$ une fonction dérivable sur $\R$. On suppose que $f$ est
  1-périodique et admet $n$ zéros sur l'intervalle $\interfo{0}{1}$. Montrer
  que $f'$ admet $n$ zéros sur ce même intervalle.
\question Soit $n\in\N$ et $P\in\polyR$ un polynôme de degré inférieur ou égal à $n$. Majorer le nombre de zéros de la
  fonction $g$ définie sur $\RPs$ par
  \[\forall x\in\RPs \qsep g(x)\defeq P(x)-\ln x.\]
\end{questions}

\begin{sol}
\begin{questions}
\question On suppose que $f$ est $1$-périodique et admet $n$ zéros $a_1,\ldots a_n$ sur l'intervalle $\interfo{0}{1}$. D'après le théorème de Rolle, $\forall i \in \set{1,\ldots,n-1}$, $f'$ s'annule sur $\intero{a_i}{a_{i+1}}$ ce qui donne $n-1$ zéros $b_i$. De plus, par $1$-périodicité, il y en a un autre $b_n$ sur $\intero{a_n}{a_1+1}$. Ou bien, il est plus petit strict que $1$ auquel cas on gagné, ou bien, il est supérieur ou égal à $1$ auquel cas $b_n-1$ convient. 
\question 
\end{questions}

\end{sol}


\begin{questions}
\question Soit $f$ une fonction dérivable $n$ fois sur un intervalle $I$. On
  suppose que $f$ admet $n+1$ zéros distincts dans $I$. Montrer qu'il existe
  $c\in I$ tel que $f^{(n)}(c)=0$.
\question Soit $n\in\N$. On définit la fonction $P_n$ sur $\R$ par
  \[\forall x\in\R \qsep P_n(x)\defeq\p{x^2-1}^n.\]
  Montrer que $P_n^{(n)}$ admet $n$ racines distinctes dans $\intero{-1}{1}$.
\end{questions}

\begin{sol}
\begin{questions}
\question Classique. $\mathcal{H}_k$ : "$f^{(k)}$ admet $n+1-k$ zéros distincts".
\question $-1$ et $1$ sont racines de multiplicités $n$ de $P_n$ donc $P_n^{(k)}(1)=0,  \forall k \in \set{0,1,\ldots,n-1}$. De même pour $-1$.\\
$\mathcal{H}_k$ : "$f^{(k)}$ admet $k$ zéros distincts sur $]-1;1[$". On arrive toujours par augmenter d'un le nombre grâce à la première remarque.
\end{questions}

\end{sol}

\exercice{nom={Application récursive du théorème de \nom{Rolle}}}
Soit $n\in\Ns$ et $f$ la fonction définie sur $\interf{0}{\frac{\pi}{2}}$ par
  \[\forall x\in\interf{0}{\frac{\pi}{2}} \qsep f(x)\defeq x^n\cos x.\]
  \begin{questions}
  \question Montrer que $f$ est de classe $\classec{\infty}$ et que, pour tout $k\in\intere{0}{n-1}$, $f^{(k)}(0)=0$.
  \question En déduire qu'il existe $c\in\interof{0}{\frac{\pi}{2}}$ tel que $f^{(n)}(c)=0$.
  \end{questions}

\exercice{nom={Accroissements finis généralisés}}
Soit $f$ et $g$ deux fonctions réelles, continues sur $\interf{a}{b}$ et
dérivables sur $\intero{a}{b}$. Montrer qu'il existe $c\in\intero{a}{b}$ tel
que
\[(f(b)-f(a))g'(c)=(g(b)-g(a))f'(c).\]

\begin{sol}
Si $g(b)-g(a)=0$, on peut appliquer le théorème de Rolle à $g$ sur $\intero{a}{b}$ ce qui donne le résultat. Sinon, $g(b)-g(a)\neq 0$.
Définissons alors $h$ sur $\interf{a}{b}$ par $h(x)=f(b)-f(x)-\frac{f(b)-f(a)}{g(b)-g(a)}\p{g(b)-g(x)}$. Cette fonction s'annule en $a$ et en $b$. On peut donc lui appliquer le théorème de Rolle. Il existe $c\in \intero{a}{b}$ tel que $h'(c)=0$, i.e $f'(c)=\frac{f(b)-f(a)}{g(b)-g(a)}g'(c)$.
\end{sol}

\exercice{nom={Théorème de \nom{Darboux}}}
Le but de cet exercice est de montrer que toute fonction dérivée vérifie la
propriété des valeurs intermédiaires (alors que l'on sait bien qu'elle peut
ne pas être continue). On se donne donc $f$ une fonction dérivable sur un
intervalle $I$, $a,b\in I$ et $y_0\in\interf{f'(a)}{f'(b)}$. On souhaite
montrer qu'il existe $x_0\in\interf{a}{b}$ tel que $f'\p{x_0}=y_0$.
\begin{questions}
\question Résoudre le problème lorsque $y_0=0$.
\question En déduire le cas général.
\end{questions}

\begin{sol}
\begin{questions}
\question 
Comme dans cette question, $0\in \interf{f'(a)}{f'(b)}$ et qu'on cherche $x_0$ tel que $f'(x_0)=0$  si l'une des deux bornes vaut $0$, on l'a trouvé. Ainsi, on peut supposer $f'(a)<0$ et $f'(b)>0$. Supposons de plus $a<b$ (on s'intéresserait au max au lieu du min dans le cas contraire)\\
$f$ est continue sur $[a,b]$ donc y est bornée et atteint ses bornes. Notons $c$ le point de $[a,b]$ tel que $f$ est minimale en $c$. Montrons alors que $c\in ]a,b[$, on aura alors un extremum local dans l'intérieur de $[a,b]$ et comme $f$ est dérivable en $c$, on aura $f'(c)=0$.\\
\begin{itemize}
\item[$\bullet$] On a $f'(a)<0$, i.e. $\underset{x\underset{>}{\to}a}{\lim}\dfrac{f(x)-f(a)}{x-a}<0$. On peut alors fixer $\delta>0$ tel que $\forall x \in ]a,a+\delta]$, $\dfrac{f(x)-f(a)}{x-a}<0$, d'où, comme $x>a$, $f(x)<f(a)$ et donc $f(a)$ n'est pas le minimum.
\item[$\bullet$] On effectue la même chose à gauche de $b$ ce qui montrer que $f(b)$ n'est pas non plus le minimum. 
\end{itemize}
Finalement, $c\in ]a,b[$ et $f'(c)=0$. On choisit donc $x_0=c$.
\question Définissons $\dspappli{\phi}{I}{\R}{x}{f(x)-y_0 x}$. $\phi$ est continue sur $I$, dérivable sur $I$ car $f$ l'est. On a $\phi'(a)=f'(a)-y_0\leq 0$ et $\phi'(b)=f'(b)-y_0\geq 0$. D'après la première question, il existe $x_0 \in "[a,b]"$ tel que $\phi'(x_0)=0=f'(x_0)-y_0$.
\end{questions}

\end{sol}

\exercice{nom={Une équation différentielle non linéaire}}
Rechercher les solutions réelles de l'équation
différentielle $y'=\abs{y}$.


\exercice{nom={Théorème de la corde}}
Soit $f$ une fonction continue et dérivable sur $\interf{0}{1}$ telle que
\[f\p{0}=0, \quad f\p{1}=1, \quad f'\p{0}=0, \et f'\p{1}=0.\]
Le but de cet exercice est de montrer qu'il existe $c\in\intero{0}{1}$ tel
que
\[f'(c)=\frac{f(c)}{c}.\]
\begin{questions}
\question Soit $g$ la fonction définie sur $\interf{0}{1}$ par
  \[\forall x\in\interf{0}{1} \qsep g(x)\defeq
    \begin{cases}
    \frac{f(x)}{x} & \text{si $x\neq 0$}\\
    0 & \text{sinon.}
    \end{cases}\]
  Montrer que $g$ est continue sur $\interf{0}{1}$ et dérivable sur
  $\interof{0}{1}$.
\question En remarquant que $g'\p{1}<0$, montrer qu'il existe
  $c\in\intero{0}{1}$ tel que $g'(c)=0$ puis conclure.
\end{questions}

\exercice{nom={Calcul numérique}}
À l'aide du théorème des accroissements finis, majorer l'erreur commise lorsque
l'on prend 100 comme valeur approchée de $\sqrt{10001}$.

\begin{sol}
Posons $f(x)=\sqrt x$ sur $[10000,10001]$. Alors, d'après le TAF, il existe $c \in ]10000,10001[$ tel que $$\sqrt{10001}-100=f'(c)=\frac{1}{2\sqrt c}\leq \frac{1}{2\sqrt{10000}}=\frac{1}{200}.$$
\end{sol}

\exercice{nom={Divergence d'une série}}
En appliquant le théorème des accroissements finis à la fonction
$x\mapsto\ln\abs{\ln x}$ sur $\interf{k}{k+1}$, montrer que la suite de terme
général
\[\sum_{k=2}^n \frac{1}{k\ln k}\]
est divergente.

\begin{sol}
Soit $k\geq 2$. Considérons $\dspappli{f}{[k,k+1]}{\R}{x}{\ln(\ln(|x|))}$. $f$ est continue sur le fermé et dérivable sur l'ouvert donc on peut appliquer le théorème des accroissement finis, il existe $c\in ]k,k+1[$ tel que $$f(k+1)-f(k)=f'(c)=\frac{1}{c\ln(c)}\leq \dfrac{1}{k\ln(k)}$$ d'où le résultat par sommation télescopique puis théorème de comparaison.
\end{sol}


\magsection{Convexité}

\exercice{nom={Inégalités en vrac}}
\begin{questions}
% \question Montrer que :
%   \[\forall x\in\interfo{0}{\frac{\pi}{2}} \quad \tan x \geq x+
%     \frac{x^3}{3}\]
% \question Montrer que :
%   \[\forall x\in\RP \quad 1-\frac{1}{3}x+\frac{2}{9}x^2
%     -\frac{14}{81}x^3 \leq \frac{1}{\sqrt[3]{1+x}} \leq
%     1-\frac{1}{3}x+\frac{2}{9}x^2\]
\question Donner deux méthodes différentes pour montrer que quel que soit le
  réel $x$
  \[\e^x \geq 2\e^{\frac{x}{2}}-1.\]
\question Soit $n\in\N$. Montrer que
  \[\forall x\in\RP \qsep x^{n+1}-(n+1)x+n \geq 0.\]
\question Soit $x\in\intero{1}{+\infty}$ et $n\in\Ns$. Montrer que
  \[x^n-1 \geq n\p{x^{\frac{n+1}{2}}-x^{\frac{n-1}{2}}}.\]
\end{questions}
\begin{sol}
$\quad$
\begin{questions}
\question Développer $(e^{x/2}-1)^2$ ou faire une inégalité de convexité avec
  0,$x$ et $x/2$.
\question Tangente en 1.
\question Deux méthodes permettent de conclure~:
  \begin{itemize}
  \item Factoriser des deux côtés par $x-1$ et se rendre compte qu'une inégalité
    arithmético-géométrique permet de conclure.
  \item Factoriser par $e^{(n/2)\ln x}$ et utiliser la croissance de $\sinh u/u$.
  \end{itemize}
\end{questions}
\end{sol}


\exercice{nom={Opérations élémentaires sur les fonctions convexes}}
\begin{questions}
\question Étant donné une fonction $f$ convexe sur $\R$ et une fonction $g$
  convexe et croissante sur $\R$, montrer que $g\circ f$ est convexe.
\question Soit $f$ une fonction définie sur $\R$ à valeurs strictement
  positives. Montrer que si $\ln(f)$ est convexe, alors $f$ est convexe.
\end{questions}

\exercice{nom={Fonctions convexes majorées}}
\begin{questions}
\question Soit $f$ une fonction convexe et majorée sur $\R$. Montrer que $f$
  est constante.
\question Donner un exemple de fonction convexe et majorée sur
  $\intero{0}{+\infty}$ et qui ne soit pas constante.
\end{questions}

\exercice{nom={Moyenne arithmétique, harmonique et géométrique}}
Soit $x_1,\ldots,x_n$ $n$ réels strictement positifs. On définit leurs
moyennes arithmétique, géométrique et harmonique par
\[a\defeq \frac{1}{n}\sum_{k=1}^n x_k, \quad g\defeq\sqrt[n]{\prod_{k=1}^n x_k},\]
\[\frac{1}{h}\defeq \frac{1}{n}\sum_{k=1}^n \frac{1}{x_k}.\]
Montrer que
\[h\leq g\leq a.\]

\exercice{nom={Inégalités de \nom{Hölder} et \nom{Minkowski}}}
Soit $p$ et $q$ deux réels strictement supérieurs à $1$ tels que
\[\frac{1}{p}+\frac{1}{q}=1.\]
Soit $x_1,\ldots,x_n$ et $y_1,\ldots,y_n$ des réels positifs.
\begin{questions}
\question Le but de cette question est de montrer que
  \[\sum_{k=1}^n x_k y_k \leq \p{\sum_{k=1}^n x_k^p}^{\frac{1}{p}}
                              \p{\sum_{k=1}^n y_k^q}^{\frac{1}{q}}.\]
  \begin{questions}
  \question Montrer que
    \[\forall x,y\in\RP \qsep xy\leq\frac{x^p}{p}+\frac{y^q}{q}.\]
  \question Montrer le résultat demandé lorsque
    \[\sum_{k=1}^n x_k^p=1 \quad \text{et} \quad \sum_{k=1}^n y_k^q=1.\]
  \question En déduire le cas général
  \end{questions}
\question Montrer que
  \[\p{\sum_{k=1}^n (x_k+y_k)^p}^{\frac{1}{p}} \leq
    \p{\sum_{k=1}^n x_k^p}^{\frac{1}{p}}+
    \p{\sum_{k=1}^n y_k^p}^{\frac{1}{p}}.\]
\end{questions}
\begin{sol}
$\quad$
\begin{questions}
\question Écrire $\p{x_k+y_k}^p=\p{x_k+y_k}^{p-1}\p{x_k+y_k}=x_k\p{x_k+y_k}^{p-1}
  +y_k\p{x_k+y_k}^{p-1}$.
\end{questions}
\end{sol}

%END_BOOK

\end{document}