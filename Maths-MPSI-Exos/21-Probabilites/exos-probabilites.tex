\documentclass{magnolia}

\magtex{tex_driver={pdftex},
        tex_packages={xypic,nicefrac}}
\magfiche{document_nom={Exercices sur les probabilités},
          auteur_nom={François Fayard},
          auteur_mail={fayard.prof@gmail.com}}
\magexos{exos_matiere={maths},
         exos_niveau={mpsi},
         exos_chapitre_numero={25},
         exos_theme={Probabilités}}
\magmisenpage{misenpage_presentation={tikzvelvia},
              misenpage_format={a4},
              misenpage_nbcolonnes={1},
              misenpage_sol={non}}
\maglieudiff{lieu_lycee={Aux Lazaristes},
             lieu_classe={MPSI 1},
             lieu_annee={2019--2020}}
\magprocess

\begin{document}
%BEGIN_BOOK

\magsection{Espace probabilisé}
\magsubsection{Espace probabilisé}

\exercice{nom={Exercice}}
Soit $A$ et $B$ deux évènements d'un espace probabilisé $(\Omega,\mathbb{P})$. Montrer que
\[\sqrt{\mathbb{P}(A\cap B)\mathbb{P}(A\cup B)}\leq\frac{\mathbb{P}(A)+\mathbb{P}(B)}{2}\]
et étudier les cas d'égalité.
\begin{sol}
Poser $a=\mathbb{P}(A)$, $b=\mathbb{P}(B)$ et $c=\mathbb{P}(A\cap B)$. Cette inégalité revient à prouver
que $(a+b-2c)^2\geq 0$. Il y a égalité si et seulement si $a=b=c$.
\end{sol}

\exercice{nom={Exercice}}
On permute aléatoirement les lettres du mot \og \textsc{Baobab} \fg. Avec quelle probabilité
le mot obtenu est-il encore \og \textsc{Baobab} \fg~?

\exercice{nom={Exercice}}
Expliquer pourquoi, lorsqu'on lance 3 dés simultanément, on obtient plus souvent
la somme 10 que la somme 9, alors que ces deux sommes peuvent être obtenues de 6
manières chacune.

\exercice{nom={Exercice}}
On lance 4 fois de suite un dé équilibré à 6 faces. Avec quelle probabilité
obtient-on~:
\begin{questions}
\question Au moins un 6.
\question Exactement un 6.
\question Au moins 2 faces identiques.
\end{questions}

\exercice{nom={Exercice}}
Dans un lot de 20 yaourts, il y en a 3 qui ont dépassé la date de péremption. On extrait au
hasard et simultanément 4 yaourts. Quelle est la probabilité qu'un seul de ces yaourts ait
dépassé la date de péremption ?

% \begin{reponse}
% Univers = combinaisons de 4 yaourts parmi 20 puisque le tirage est simultané. $\card \p{\Omega} = \binom{20}{4}$.
% On le muni de la probabilité uniforme.
% Dénombrement des cas favorables : on doit choisir un yaourt périmé $\binom{4}{1}$ puis 3 non périmés 
% $\binom{17}{3}$ donc il y a $\binom{4}{1}\binom{17}{3}$ choix.
% La probabilité cherchée est donc $\dfrac{\binom{4}{1}\binom{17}{3}}{ \binom{20}{4}} = 
% \boxed{\frac{8}{19}}$. 
% \end{reponse}

\exercice{nom={Exercice}}
On place aléatoirement $n$ boules indiscernables dans $n$ urnes discernables. Calculer la
probabilité qu'une seule urne soit vide.
% \begin{reponse}
% Facile puisqu'une seule urne est vide ssi une est vide, toutes les autres contiennent une
% seule boule sauf une qui en contient deux. On procède par dénombrement.
% \end{reponse}

\exercice{nom={Exercice}}
On lance $n$ fois une pièce équilibrée. On appelle changement un lancer qui donne un résultat différent du précédent.
Calculer pour tout entier $k \in \intere{0}{n-1}$ la probabilité pour qu'il y ait $k$ changements.
% \begin{reponse}
% $\Omega=\{P,F\}^n$. On prend la probabilité uniforme. On dénombre. Une réalisation avec $k$ changements correspond au choix des $k$ emplacements de changement entre $2$ et $n$ puis le premier lancer (P ou F) détermine toute la suite.
% On a donc $2 \binom{n-1}{k}$ choix. $\card \Omega = 2^n$ donc la probabilité cherchée est 
% $\boxed{\frac 1{2^{n-1}} \binom{n-1}{k}}$ (la somme fait bien $1$).
% \end{reponse}

\magsubsection{Variable aléatoire}

\magsubsection{Lois usuelles}
\magsection{Dépendance des évènements}
\magsubsection{Probabilité conditionnelle}

\exercice{nom={Exercice}}
Un train contient $n$ places numérotées et $n$ voyageurs possèdent un billet. Le premier voyageur monte dans le train, mais il a oublié son billet et se place donc au hasard. Puis chaque personne s'installe à sa place si elle est libre et choisit une place libre au hasard sinon. Quelle est la probabilité que la dernière personne se trouve à sa place ?
% \begin{reponse}
% On note $P_i=j$ l'évènement "la personne $i$ est à la place $j$" et $p_n=\pr{P_n=n}$.
% On conditionne suivant la place qu'occupe $P_1$.
% $$p_n=\pr{P_n=n}=\sum\limits_{k=1}^n \prc{P_n=n}{P_1=k} \underbrace{\pr{P_1=k}}_{=\frac 1n}
% =\frac 1n \sum\limits_{k=1}^{n-1} \prc{P_n=n}{P_1=k} $$
% Mais si $P_1=k$ alors tous les voyageurs jusqu'au $k-1$-ième inclus vont à leur place puis arrive le $k$-ième qui trouve sa place occupée et doit donc choisir au hasard une place. On se retrouve donc dans la même situation qu'au début mais il reste $n-k+1$ voyageurs. On a donc 
% $$\boxed{p_n=\frac 1n \sum\limits_{k=1}^{n-1} p_{n-k+1} =\frac 1n \sum\limits_{k=1}^{n-1} p_{k} }$$
% avec $p_1=1$, $p_2=\frac 12$ on obtient par récurrence, $\boxed{p_n=\frac 12}$.
% \end{reponse}



% \begin{reponse}
% \begin{enumerate}
% \item La plante donne une fleur rose toutes les années ssi elle donne une fleur rose la première année donc la probabilité vaut $\boxed{\frac 34}$.
% \item La plante donne une fleur blanche la première année est probabilité $\frac 14$ puis chaque année avec probabilité $\frac 12$ donc par indépendance, la probabilité cherchée est $\boxed{\frac 14 \p{\frac 12}^{n-1}}$.
% \item On $R_n$ l'évènement : la plante a donné une fleur rose l'année $n$ et $B_n$ pour blanche. On a alors
% $$\pr{R_{n+1}}=\prc{R_{n+1}}{R_n} \pr{R_n}+ \prc{R_{n+1}}{B_n} \pr{B_n}$$
% $$p_{n+1}=p_n+ \frac 12 (1-p_n)$$
% $$\boxed{p_{n+1}=\frac 12 p_n + \frac 12}$$
% Si on pose $x_{n}=p_n-1$ alors $x_{n+1}=\frac 12 x_n$ et comme $x_1=- \frac 14$ on a $x_n=-\frac 14 \p{\frac 12}^{n-1}$ et finalement
% $$\boxed{p_n = 1 -\frac 14 \p{\frac 12}^{n-1}}$$
% En particulier, $p_n \to 1$ donc à coup presque sûr la fleur deviendra définitivement rose.
% \end{enumerate}
% \end{reponse}

\exercice{nom={Exercice}}
$n$ personnes numérotées de $1$ à $n$ se transmettent dans cet ordre une
information. Chaque personne transforme l'information reçue en son contraire
avec la probabilité $p \in ]0,1[$ et la transmet fidèlement avec la probabilité
$1-p$. Quelle est la probabilité que l'information parvienne non déformée à la $n$-ième
personne ?

% \begin{reponse}
% On note $A_n$ l'évènement "la $n$-ième personne reçoit l'information non
% déformée" et $p_n=\pr{A_n}$.
% Pour $n \in \{1,\dots,N-1\}$, $\{ A_n, \co{A_n} \}$ est un système complet
% d'évènements donc d'après la \textbf{formule des probabilités totales}
% $$\pr{A_{n+1}}=\pr{A_n} \pr{A_{n+1} | A_n} + \pr{\co{A_n}} \pr{A_{n+1} | \co{A_n}}$$
% $$p_{n+1}=p_n (1-p) + (1-p_n) p =(1-2p) p_n + p$$
% On pose $p_1=1$ et on a une suite arithmético-géométrique ce qui donne
% $$\boxed{p_N = \frac 12 \left( 1+ (1-2p)^{N-1} \right)}$$
% En particulier, $p_N \to \frac 12$ et ceci indépendamment de la valeur de
% $p$ !
% \end{reponse}

\exercice{nom={Exercice}}%vincent
On dispose de $n+1$ urnes $U_0, U_1, \dots, U_n$ où l'urne $U_k$
contient $k$ boules blanches et $n-k$ boules noires. On choisit une
urne au hasard puis on réalise $p$ tirages avec remise dans l'urne
choisie.
\begin{questions}
\question Quelle est la probabilité de ne tirer que des boules blanches ?
\question Déterminer la limite de cette probabilité lorsque $n$ tend vers $+\infty$.
\end{questions}

% \begin{reponse}
% \begin{enumerate}
% \item
% On note $U_k$ l'évènement : "on a choisi l'urne $U_k$" et $B_p$ l'évènement : "on a tiré $p$ boules blanches".
% Comme le tirage de l'urne est uniforme, $\pr{U_k} = \frac 1{n+1}$.
% \medskip
% La famille $(U_0, \dots, U_n)$ est un système complet d'évènements donc la formule des probabilités totales donne
% $$\pr{B_p} = \sum\limits_{k=0}^n \prc{B_p}{U_k} \pr{U_k} = \frac 1{n+1} \sum\limits_{k=0}^n \prc{B_p}{U_k}$$
% On note $BL_i$ l'évènement : "on a tiré une boule blanche au $i$-ième tirage". On a donc 
% $$\prc{B_p}{U_k} = \prc{BL_1 \cap \dots \cap BL_p}{U_k} = \prc{BL_1}{U_k} \dots \prc{BL_p}{U_k} = 
% \prc{BL_1}{U_k}^p$$
% par indépendance des tirages successifs et par le fait qu'ils sont avec remise. Il reste donc
% $$\prc{B_p}{U_k} = \p{\frac kn}^p \text{ donc } \boxed{\pr{B_p} = \frac 1{n+1} \sum\limits_{k=0}^n \p{\frac kn}^p}$$
% \item On utilise une somme de Riemann ou une comparaison somme-intégrale pour obtenir
% $$\boxed{ \pr{B_p} \to \frac 1{p+1} \text{ quand } n \to + \infty}$$
% \end{enumerate}
% \end{reponse}





\exercice{nom={Exercice}}%\textbf{[pêche à la grenouille]}
Soit ${\cal G}$ un ensemble fixé de cardinal $N$ et ${\cal M}$ une partie fixée de ${\cal G}$ de cardinal $m$. L'ensemble des parties de ${\cal G}$ est muni de la probabilité uniforme. On fixe $n \leq N$.
\begin{enumerate}
\item Soit $k \in \intere{0}{\min(n, m)}$. Quelle est la probabilité $p_k$ qu'une partie ${\cal P}$ de ${\cal G}$ de cardinal $n$ intersecte ${\cal M}$ selon un ensemble de cardinal $k$~?
\item Le cardinal moyen de l'intersection d'une partie ${\cal P}$ de ${\cal G}$ de cardinal $n$ avec ${\cal M}$ est donc \[\sum\limits_{k=0}^{\min(n,m)} k p_k.\] Calculer ce nombre moyen.
\item Un étang contient $N$ grenouilles. On cherche à estimer $N$. Pour cela, on commence par pêcher $m$ grenouilles que l'on marque puis que l'on rejette à l'eau. Quelques jours plus tard, on pêche $n$ grenouilles. Montrer que l'on peut estimer $N$.
\end{enumerate}

\exercice{nom={Exercice}}
On considère une urne contenant $n$ boules numérotées. Les nombres inscrits sur les boules sont des réels deux à deux distincts sur lesquels on ne sait rien de plus.
On réalise un tirage sans remise dans l'urne avec l'objectif de déterminer le maximum $M$ des nombres inscrits sur les boules.
Comme on n'a pas le temps de tirer toutes les boules (disons que $n$ est trop grand), on opte pour la stratégie suivante : on choisit un entier $p \in \intere{1}{n-1}$, on tire $p$ boules successivement et sans remise et on mémorise le maximum $M_p$ que l'on a vu parmi ces $p$ boules. On continue alors à tirer des boules, mais on s'arrête dès que l'on a trouvé un nombre supérieur à $M_p$. 
\begin{enumerate}
\item Quelle est la probabilité $P_{n,p}$ que l'on trouve ainsi le maximum $M$.
\item Si $n=4$, quelle valeur de $p$ a-t-on intérêt à choisir ?
\item Soit $\alpha \in ]0,1]$. On choisit $p$ en fonction de $n$ (on le notera donc $p_n$) de manière à ce que 
\[\frac{p_n}{n} \tendvers{n}{+\infty} \alpha.\] ($\alpha$ représente donc asymptotiquement la proportion de boules que l'on commence par tirer).
Montrer que $P_{n,p_n}$ tend vers $-\alpha \ln(\alpha)$ lorsque $n \to + \infty$.
\item Quelle valeur de $\alpha$ a-t-on intérêt à choisir ?
\end{enumerate}



% \begin{sol}
% \begin{enumerate}
% \item 
% On note $A_k$ l'évènement "le maximum $M$ se trouve en position $k$ dans le tirage". On a par équiprobabilité 
% $\pr{A_k}=\frac 1n$. 
% On note $G$ l'évènement "on gagne i.e. on a trouvé le maximum".
% On a évidemment $\prc{G}{A_k}=0$ si $k \leq p$ puisqu'alors le maximum fait partie de l'échantillon des $p$ premières boules et l'on ne trouvera plus le maximum parmi les boules qui restent.
% Ainsi, puisque $(A_1,\dots,A_n)$ est un système complet, la formule des probabilités totales donne
% $$\pr{G} = \sum\limits_{k=p+1}^n \prc{G}{A_k} \pr{A_k} = \frac 1n \sum\limits_{k=p+1}^n \prc{G}{A_k}$$
% Il reste donc à calculer $\prc{G}{A_k}$. Mais, si $M$ est en position $k$, on gagne ssi entre les boules $p+1$ et $k-1$ on a trouvé aucune valeur supérieure à $M_p$ ssi le max des valeurs des boules de $1$ à $k-1$ est obtenu parmi les $p$ premières boules et ceci est réalisé avec une probabilité $\frac{p}{k-1}$ (équiprobabilité). On a donc
% $$\boxed{P_{n,p}} = \frac 1n \sum\limits_{k=p+1}^n \frac{p}{k-1} = \boxed{ \frac pn \sum\limits_{k=p}^{n-1} \frac{1}{k} }$$
% \item 
% La formule précédente montrer que 
% $$P_{4,1}=\frac{11}{24} \qquad P_{4,2}=\frac{10}{24} \qquad P_{4,3}=\frac{6}{24}$$
% de sorte que l'on a intérêt à choisir $\boxed{p=1}$.
% \item On utilise le développement asymptotique bien connu de $H_n=\sum\limits_{k=1}^n \frac 1k = 
% \ln(n) + \gamma + o(1)$ ce qui donne facilement le résultat.
% \item On maximise facilement la fonction $\alpha \mapsto - \alpha \ln(\alpha)$ ce qui donne que le maximum est atteint en 
% $\frac 1e$ et vaut $\frac 1e$. On a donc intérêt à prendre $\boxed{p \sim \frac 1e n}$.
% \end{enumerate}
% \end{sol}


%\exercice{nom={Exercice}}On dispose d'une pièce truquée pour laquelle la probabilité de faire {\it pile} est $p$. On ne connait pas $p$. Pour l'estimer, on lance la pièce 10 fois et on obtient $3$ fois {\it pile}. Quelle est la valeur de $p$ rendant cet évènement le plus probable ?
%


%\exercice{nom={Exercice}}
%On transmet un message en binaire sous forme d'une suite de $n$ bits.
%Lors de la transmission, chaque bit a une probabilité $p$ d'être modifié. Quelle est la probabilité que le message reçu comporte au plus une erreur ?
%
%\begin{reponse}
%Le nombre de bits erronés suit une loi binomiale de paramètres $n,q$.
%
%La probabilité cherchée est donc $\boxed{q^n+npq^{n-1}}$.
%\end{reponse}
%



%
%\exercice{nom={Exercice}}
%On considère un gène possédant deux caractères $a$ et $A$. Le génotype d'un individu est donc $aa$, $aA$ ou $AA$.
%On suppose qu'à la génération $0$, les proportions de porteurs $aa$, $aA$ et $AA$ sont respectivement $u_1,u_2$ et $u_3$.
%
%On fait l'hypothèse que les mariages sont aléatoires et la transmission des caractères uniforme.
%
%On note $M$ le génotype de la mère, $P$ celui du père et $E$ celui de l'enfant. On note $x=u_1+ \frac 12 u_2$. Montrer qu'à la génération 1,
%$$\pr{ E = \text{aa} } = x^2 \qquad \pr{ E = \text{aA} } = 2x(1-x) \qquad \pr{ E = \text{AA} } = (1-x)^2 \qquad$$
%En déduire que les proportions sont constantes à partir de la première génération.
%
%

\magsubsection{Formule des probabilités totales}

\exercice{nom={Monthy Hall}}
Dans une fête foraine, on vous propose le jeu suivant~: trois verres opaques sont
retournés devant vous, dont l'un seulement abrite une bille, et vous devez deviner lequel.
\begin{questions}
\question Quelle est la probabilité de deviner juste~?
\question Pris de pitié devant votre malchance à répétition, le maître du jeu décide 
  de vous donner un coup de pouce. Après votre réponse, il vous indique, parmi les deux
  verres que vous n'avez pas désignés, une verre qui ne contient pas la bille et vous
  propose de revoir votre réponse. Préférez-vous confirmer votre réponse initiale ou
  la modifier~?
\end{questions}

\exercice{nom={Exercice}}
Un horticulteur dispose d'un stock de plantes. Chacune des plantes fleurit une fois par an.
Pour chaque plante, la première année, la probabilité de donner une fleur rose est de \nicefrac{3}{4},
celle de donner une fleur blanche est de \nicefrac{1}{4}.  Puis les années suivantes, pour tout entier naturel $n$ non nul :
\begin{itemize}
\item Si l'année $n$, la plante a donné une fleur rose, alors l'année $n+1$, elle donnera
  une fleur rose.
\item Si l'année $n$, la plante a donné une fleur blanche, alors l'année $n+1$, elle
  donnera de façon équiprobable une fleur rose ou une fleur blanche.
\end{itemize}
\begin{enumerate}
\item Quelle est la probabilité que la plante ne donne que des fleurs roses pendant les $n$
premières années ?
\item Quelle est la probabilité que la plante ne donne que des fleurs blanches pendant les $n$
premières années ?
\item On note $p_n$ la probabilité de l'évènement \og
  La plante a donné une fleur rose l'année $n$ \fg.
Calculer $p_n$, puis sa limite quand $n$ tend vers $+\infty$.
\end{enumerate}


\exercice{nom={Urnes d'\nom{Ehrenfest}}}%\textbf{[urnes d'Ehrenfest = chaîne de Markov]}
On considère deux urnes $U_1$ et $U_2$. Chaque urne contient une boule. À chaque étape, on choisit une boule au hasard et on la change d'urne. Pour tout $k \in \{0,1,2\}$ et
$n\in\N$, on note $E_n^k$ l'évènement \og Il y a $k$ boules dans l'urne $U_1$ après
l'étape $n$ \fg. On note
\[a_n \defeq \mathbb{P}\p{E_n^0},\quad b_n \defeq \mathbb{P}\p{E_n^1}, \quad
  c_n \defeq \mathbb{P}\p{E_n^2}, \et 
  X_n \defeq \trans{(a_n, b_n, c_n)}.\]
\begin{questions}
\question Déterminer une matrice $A\in\mat{3}{\R}$ telle que pour tout $n\in\N$,
$X_{n+1}=A X_n$.
\question Calculer $A^2$ et $A^3$.
\question Comment interpréter ce résultat~?
\end{questions}
% \begin{reponse}
% On trouve $A=\begin{pmatrix} 0 & \frac 12 & 0 \\ 1 & 0 & 1 \\ 0 & \frac 12 & 0 \end{pmatrix}$ puis $A^3=A$.
% \end{reponse}

\exercice{nom={Jinko}}
\nom{Jinko} le chaton a trois passions dans la vie~: Manger, Dormir et Jouer. On peut
  considérer qu'il pratique ces activités par tranches de 5 minutes.
  \begin{itemize}
  \item Après 5 minutes de repas, il continue de manger les 5 minutes suivantes avec une
    probabilité de \nicefrac{1}{2}. Sinon, il se met à jouer.
  \item Après 5 minutes de sommeil, il continue de dormir les 5 minutes suivantes avec une
    probabilité de \nicefrac{3}{4}. Sinon, il a faim au réveil et va manger.
  \item Après 5 minutes de jeu, soit il est en appétit et mange les 5 minutes suivantes
    avec une probabilité de \nicefrac{1}{4}, soit il est fatigué et s'endort.
  \end{itemize}
  L'expérience que l'on considère est une journée de \nom{Jinko} de $5(m+1)$ minutes. Sur la
  tranche d'indice 0 des 5 premières minutes de la journée, \nom{Jinko} se lève et va
  petit-déjeuner. Pour tout $n\in\intere{0}{m}$, on note $A_n:\Omega\to\ens{M,D,J}$ la
  variable aléatoire donnant l'activité de \nom{Jinko} sur la tranche de 5 minutes
  d'indice $n$ de la journée. Pour tout
  $n\in\intere{0}{m}$, on pose
  \[X_n\defeq
    \begin{pmatrix}
    \mathbb{P}(A_n=M)\\
    \mathbb{P}(A_n=D)\\
    \mathbb{P}(A_n=J)
    \end{pmatrix}.\]
  \begin{questions}
  \question Déterminer une matrice $B\in\mat{3}{\R}$ telle que
    \[\forall n\in\intere{0}{m-1}\qsep X_{n+1}=B X_n.\]
  \question
    \begin{questions}
    \question Déterminer les valeurs propres de $B$, c'est-à-dire les $\lambda\in\R$ tels
      que $B-\lambda I_3$ n'est pas inversible.
    \enonce On note $\lambda_1,\lambda_2,\lambda_3\in\R$ ces valeurs et on pose
      $P\defeq(X-\lambda_1)(X-\lambda_2)(X-\lambda_3)$.
    \question Montrer que $P(B)=0$. En déduire une expression de $B^n$ valable pour tout
      $n\in\N$.
    \question En déduire les limites de
      \[\mathbb{P}(A_n=M),\qquad
        \mathbb{P}(A_n=D),\qquad
        \mathbb{P}(A_n=J)\]
      lorsque $n$ tend vers $+\infty$.
    \end{questions}
  \end{questions}

\magsubsection{Formule de \nom{Bayes}}

\exercice{nom={Exercice}}
On prend un dé au hasard parmi un lot de 100 dés dont on sait que 25
sont pipés~: pour ces dés, la probabilité d'obtenir $6$ est $1/2$.
\begin{questions}
\question On lance le dé et on obtient $6$.
  Quelle est la probabilité que ce dé soit pipé~?
\question Même question si l'on fait $n$ lancers et que l'on obtient
  $6$ à chaque fois.
\end{questions}
% \begin{reponse}
% On note $DP$ l'évènement "le dé est pipé" et $A_1$ l'évènement "on obtient
% $6$ au premier lancers".
% On a alors par la \textbf{formule de Bayes} :
% \begin{eqnarray*}
% \pr{DP | A_1 }& = &\frac{\pr{DP} \pr{A_1 | DP}}{\pr{DP} \pr{A_1 | DP} +
% \pr{\co{DP}} \pr{A_1 | \co{DP}}} \\
% & = &\frac{ \frac 14 \pr{A_1 | DP}}{  \frac 14 \pr{A_1 | DP} + 
% \frac 34 \pr{A_1 | \co{DP}}}=\frac{ \frac 14  \frac 12 }{  \frac 14 \frac 12  +
%  \frac 34  \frac 16} =\frac 12 \\
% \end{eqnarray*}
% De la même manière en notant $A_n$ l'évènement "on obtient $6$ aux $n$ lancers"
% :
% \begin{eqnarray*}
% \pr{DP | A_n }& = &\frac{\pr{DP} \pr{A_n | DP}}{\pr{DP} \pr{A_n | DP} + \pr{\co{DP}}
% \pr{A_n | \co{DP}}} \\
% & = &\frac{ \frac 14 \pr{A_n | DP}}{  \frac 14 \pr{A_n | DP} + 
% \frac 34 \pr{A_n | \co{DP}}}=\frac{ \frac 14  \frac{1}{2^n} }{  \frac 14 \frac{1}{2^n}  + %
% \frac 34  \frac{1}{6^n}} =\frac {1}{1+ \frac{1}{3^{n-1}}} \\
% \end{eqnarray*}
% On a donc $\pr{DP | A_n } \to 1$ ce qui est raisonnable\dots
% \end{reponse}

\magsubsection{Indépendance}

\exercice{nom={Exercice}}
Un concours consiste à passer 3 épreuves indépendantes. On a $80 \%$ de chances de réussir
l'épreuve $1$, $60\%$ pour l'épreuve $2$ et $25\%$ pour l'épreuve $3$. On est reçu au
concours si on réussit au moins deux épreuves sur trois (n'importe lesquelles). Quelle est
la probabilité de réussir le concours ?

% \begin{reponse}
% On note $R_k$ l'évènement "on a réussi l'épreuve $k$" et $R$ l'évènement
% "on a réussi le concours". Alors
% $$R = \left( R_1 \cap R_2 \cap \co{R_3} \right) \cup \left( R_2 \cap R_3 \cap \co{R_1} \right) \cup \left( R_1 \cap R_3 \cap \co{R_2} \right) \cup
% \left( R_1 \cap R_2 \cap R_3 \right)$$
% et c'est une partition donc en passant aux probabilités (par exemple, $R_1$
% et $R_2$ sont indépendants) on obtient
% $$\pr{R} = \pr{R_1} \pr{R_2} \left( 1 - \pr{R_3} \right) + \dots = \boxed{59
% \%}$$
% \end{reponse}

\exercice{nom={Exercice}}
On dispose d'un dé à quatre faces non pipé. On réalise $n$ lancers indépendants.
On note $T$ l'évènement  \og Lors des $n$ lancers, les quatre numéros sont sortis \fg.
En utilisant la formule du crible, calculer la probabilité de $T$, puis sa limite lorsque $n$
tend vers $+\infty$.
% \begin{reponse}
% On note $A_k$ l'évènement "lors des $n$ lancers, le numéro $k$ n'est jamais sorti". Alors 
% $$\co{T} = A_1 \cup A_2 \cup A_3 \cup A_4$$
% En utilisant la formule du crible, on a donc
% $$1- \pr{T} = \pr{A_1} + \dots + \pr{A_4} - \pr{A_1 \cap A_2} - \dots - \pr{A_3 \cap A_4} 
% + \pr{A_1 \cap A_2 \cap A_3} + \dots - \pr{A_1 \cap \dots \cap A_4}$$
% ce qui donne facilement
% $$\pr{T} = 1 - 4 \p{\frac 34}^n + 6 \p{\frac 12}^n - 4 \p{\frac 14}^n \to 1$$
% ce qui est bien naturel \dots
% \end{reponse}

\magsubsection{Loi d'une somme}


\exercice{nom={Centre d'appels}}
 Dans un centre d'appels, un employé effectue successivement $n$ appels
  téléphoniques vers $n$ correspondants distincts dont chacun décroche avec une
  probabilité $p\in\interf{0}{1}$.
  \begin{questions}
  \question Déterminer la loi du nombre $N_1$ de correspondants qui ont décroché.
  \question L'employé rappelle plus tard les $n-N_1$ correspondants qui n'ont pas décroché
    lors de la première série d'appels. On note $N_2$ le nombre de correspondants
    qui ont décroché cette fois. Enfin, on pose $N\defeq N_1+N_2$. Cette variable aléatoire donne
    le nombre total de personnes qui ont été contactées avec succès.
    \begin{questions}
    \question Pour tout $k,i\in\intere{0}{n}$, calculer $\mathbb{P}(N=k|N_1=i)$.
    \question En déduire la loi suivie par $N$.
    \end{questions}
    % Quelle est la loi du nombre total $N\defeq N_1+N_2$
    % de correspondants qui ont décroché~?
  \end{questions}
  \begin{sol}
Il faut calculer $\mathbb{P}(N=k)=\sum \mathbb{P}(N_1=i) \mathbb{P}(n=k|N_1=i)$. On utilise une relation sur
les coefficients binomiaux qui se montre facilement en passant par les factorielles. On trouve que $N$ suit une
loi binomiale de paramètre $n,p(2-p)$.
  \end{sol}



%END_BOOK
\end{document}
















