\documentclass{magnolia}


\magtex{tex_driver={pdftex}}
\magfiche{document_nom={Exercices sur les équations différentielles},
          auteur_nom={François Fayard},
          auteur_mail={fayard.prof@gmail.com}}
\magexos{exos_matiere={maths},
         exos_niveau={mpsi},
         exos_chapitre_numero={6},
         exos_theme={Équations différentielles}}
\magmisenpage{}
\maglieudiff{}
\magprocess

\begin{document}

%BEGIN_BOOK
\magsection{Équation différentielle linéaire du premier ordre}
% \magsubsection{Équation différentielle homogène}

% \exercice{nom={Équation de \nom{Bernoulli}}}
% Soit $\alpha>0$. On considère l'équation différentielle
% \[\forall t\in\interfo{\alpha}{+\infty} \qsep t^2y'(t)+y(t)+y^2(t)=0.\]
% On admet que l'unique solution s'annulant de cette équation est la fonction nulle. En effectuant le changement de fonction $z=1/y$, déterminer l'ensemble des solutions de $(E)$.

% \begin{sol}
% Les solutions sont la fonction nulle, les fonctions
% \[\forall t\geq\alpha\qsep y(t)=\frac{-1}{c\e^{-\frac{1}{t}}+1}\]
% où $c\geq 0$, ainsi que les fonctions
% \[\forall t\geq\alpha\qsep y(t)=\frac{1}{c\e^{-\frac{1}{t}}-1}\]
% où $c> \e^{1/\alpha}$.
% On doit utiliser le Théorème de Cauchy-Lipschitz (version non donné aux élèves ?). Si $y$ s'annule alors, c'est la fonction nulle. Sinon, $y$ ne s'annule jamais et on peut alors poser $z=y^{1-n}$. On a alors $z'=(1-n)y'y^{-n}$. Et en en divisant $(E)$ par $y^n$, on a alors $(E) \Longleftrightarrow t^2\dfrac{z'}{1-n}+z+1=0$.

% Disons pour simplifier que $I$ ne contient pas $0$, en résolvant cette équation différentielle, on obtient $z(t)=\lambda e^{\frac{1-n}{t}}-1$. Ainsi, $y(t)=\p{\lambda e^{\frac{1-n}{t}}-1}^{\frac{1}{1-n}}$.

% Quand j'essaie de vérifier qu'un tel $y$ fonctionne, ça ne marche pas...
% \end{sol}


\magsubsection{Équation différentielle avec second membre}


\exercice{nom={Calcul}}
Résoudre les équations différentielles suivantes sur un intervalle à préciser
\[y'+2y=x^2-2x+3, \qquad
%%\question $\p{x\ln x}y'-y=-\frac{\ln x+1}{x}$
  \p{1+x}y'+y=1+\ln (1+x),\]
\[y'+y=\frac{1}{1+\e^x}.\]
%, \qquad y'=\sqrt{1+y^2}.\]
%%\question $2xy'+y=x^n \quad n\in\N$
\begin{sol}
\begin{questions}
\question Soit $(E)$ l'équation différentielle
 \[(E) \quad \forall x\in\R\qsep y'(x)+2y(x)=x^2-2x+3\]
 On cherche une solution particulière de $(E)$ sous la forme $y(x)=ax^2+bx+c$. Soit $a, b, c\in\R$ et $y$ la fonction définie sur $\R$ par
 \[\forall x\in\R \qsep y(x)\defeq ax^2+bx+c\]
 D'après les théorèmes usuels, $y$ est dérivable sur $\R$ et
\[\forall x\in\R\qsep y'(x)= 2ax+b.\]
donc
\begin{eqnarray*}
\text{$y$ est solution de $(E)$}
&\ssi& \forall x\in\R\qsep y'(x)+2y(x)=x^2-2x+3\\
&\ssi& \forall x\in\R\qsep 2ax^2+(2b+2a)x+(2c+b)=x^2-2x+3\\
&\ssi& 2a = 1 \et 2a+2b=-2 \et b+2c=3\\
&    & \text{car un polynôme admettant une infinité de racines est nul}\\
&\ssi& a = \frac{1}{2} \et b=-\frac{3}{2} \et c=\frac{9}{4}.
\end{eqnarray*}
Donc
\[x\mapsto \frac{1}{2}x^2-\frac{3}{2}x+\frac{9}{4}\]
est une solution particulière de $(E)$. De plus, les solutions de l'équation différentielle homogène associée
\[(E_H) \quad \forall x\in\R\qsep y'(x)+2y(x)=0\]
sont les fonctions $x\mapsto \lambda\e^{-2x}$ où $\lambda\in\R$. Les solutions de $(E)$ sont donc les fonctions
\[x\mapsto \lambda\e^{-2x}+\frac{1}{2}x^2-\frac{3}{2}x+\frac{9}{4}\]
où $\lambda\in\R$.
\question $y(x)=\ln(1+x)+c/(1+x)$.
\question $y(x)=\cro{\ln(1+e^x)+c}e^{-x}$.
\question $y(x)=\sh(x+c)$.
\end{questions}
\end{sol}

\exercice{nom={Avec un second membre}}
Déterminer les fonctions dérivables $y:\R\to\R$ telles que
\[\forall x\in\R\qsep y'(x)+y(x)=\integ{0}{1}{y(t)}{t}.\]

\exercice{nom={Équations fonctionnelles}}
\begin{questions}
\question Déterminer les fonctions dérivables $y:\R\to\R$ telles que
\[\forall x,y\in\R\qsep f(x+y)=\e^x f(y)+f(x)\e^y.\]
\question Déterminer les fonctions dérivables $y:\R\to\R$ telles que $f(0)\neq 0$ et
\[\forall x,y\in\R\qsep f(x+y)=f(x)f'(y)+f'(x)f(y).\]
\end{questions}

\magsubsection{Problème de \nom{Cauchy}}
\magsubsection{Équation différentielle non résolue}

\exercice{nom={Une équation différentielle avec peu de solutions}}
Soit $I$ un intervalle de $\R$ et $(E)$ l'équation différentielle
$$\forall t\in I \qsep \abs{t}y'(t)+\p{t-1}y(t)=0.$$
\begin{questions}
\question Résoudre cette équation pour $I=\RPs$ puis $I=\RMs$.
\question En déduire les solutions de cette équation différentielle lorsque
  $I=\R$.
\question Soit $t_0\in\R$ et $y_0\in\R$. Le problème de \nom{Cauchy} $y(t_0)=y_0$
  a-t-il toujours au moins une solution~? Si oui, est-elle unique~?
\end{questions}
\begin{sol}
$\quad$
\begin{questions}
\question Sur $\RPs$, $y(t)=cte^{-t}$ et sur $\RMs$, $y(t)=ce^t/t$.
\question L'unique solution est la fonction nulle.
\question Aucune solution si $y_0\neq 0$. Une seule solution sinon. 
\end{questions}
\end{sol}

\exercice{nom={Une équation différentielle avec beaucoup de
  solutions}}
Soit $I$ un intervalle de $\R$ et $(E)$ l'équation différentielle
$$\forall t\in I \qsep ty'(t)-\p{t+2}y(t)=0.$$
\begin{questions}
\item Résoudre cette équation pour $I=\RPs$ puis $I=\RMs$.
\item En déduire les solutions de cette équation différentielle lorsque $I=\R$.
\item Soit $t_0\in\R$ et $y_0\in\R$. Le problème de \nom{Cauchy} $y(t_0)=y_0$ a-t-il
  toujours au moins une solution~? Si oui, est-elle unique~?
\end{questions}

\exercice{nom={Discontinuité des coefficients de l'équation}}
Soit $H$ la fonction de \nom{Heaviside} définie sur $\R$ par
$$\forall t\in\R \qsep H(t)\defeq
\begin{cases}
  0 & \text{si $t\leq 0$}\\
  1 & \text{si $t>0$}.
\end{cases}
$$
On considère l'équation différentielle
$$\forall t\in\R \qsep y'(t)+H(t)y(t)=0.$$
\begin{questions}
\question Résoudre cette équation différentielle.
\question Les problèmes de \nom{Cauchy} associés à cette équation ont-ils toujours
  une unique solution~?
\end{questions}


% \magsubsection{Méthode d'\nom{Euler}}


\magsection{Équation différentielle linéaire du second ordre}
\magsubsection{Équation différentielle homogène}




\exercice{nom={Équation d'\nom{Euler}}}
On considère l'équation différentielle
\[(E) \quad t^2y''-ty'+y=0.\]
\begin{questions}
\item Dans cette question, on souhaite résoudre $(E)$ sur $\RPs$.
  \begin{questions}
  \item En posant $z(u)\defeq y\p{\e^u}$, montrer que $y$ est solution de $(E)$ si et
    seulement si $z$ est solution d'une équation différentielle du second ordre
    à coefficients constants que l'on précisera.
  \item En déduire l'ensemble des solutions de $(E)$.
  \end{questions}
\item Résoudre $(E)$ sur $\RMs$ puis sur $\R$.
\end{questions}
\begin{sol}
$\quad$
\begin{questions}
\question
  \begin{questions}
  \question On pose $t=e^u$. En fait on définit $z$ sur $\R$ par $z(u)=y(e^u)$.
    On trouve $z''-2z'+z=0$.
  \question Donc $y(t)=(c_1\ln t+c_2)t$.
  \end{questions}
\question On pose $t=-e^u$. On trouve la même équation différentielle,
  donc $y(t)=(c_1\ln(-t)+c_2)t$. Sur $\R$, on trouve $y(t)=ct$.
\end{questions}
\end{sol}

\exercice{nom={Changement de variable}}
\begin{questions}
\question En posant $x\defeq \tan t$, résoudre l'équation différentielle
  \[\forall x\in\R\qsep (1+x^2)^2 y''(x)+2x(1+x^2)y'(x)+4y(x)=0.\]
\question Soit $\alpha\in\Rs$. En posant $x\defeq \sh t$, résoudre l'équation différentielle
  \[\forall x\in\R\qsep (1+x^2)y''(x)+xy'(x)-\alpha^2 y(x)=0.\]
\end{questions}

\exercice{nom={Équations fonctionnelles}}
\begin{questions}
\item Soit $\lambda\in\R$. Trouver toutes les fonctions deux fois dérivables
  sur $\R$ telles que
  $$\forall x\in\R \qsep f'(x)=f\p{\lambda-x}.$$
\item Trouver toutes les fonctions $f$ deux fois dérivables sur $\RPs$ telles
  que
  $$\forall x>0 \qsep f'(x)=f\p{\frac{1}{x}}.$$
  {\it On utilisera les résultats sur l'équation d'\nom{Euler}}
\end{questions}
\begin{sol}
$\quad$
\begin{questions}
\question On trouve
  \[y(x)=c\cos\p{x-\p{\frac{\lambda}{2}+\frac{\pi}{4}}}\]
\question On trouve
  \[y(x)=c\sqrt{x}\cos\p{\frac{\sqrt{3}}{2}\ln x-\frac{\pi}{6}}\]
\end{questions}
\end{sol}


\exercice{nom={Coefficients non constants}}
Résoudre sur $\R$ l'équation différentielle
\[(2x+1)y''+(4x-2)y'-8y=0\]
sachant qu'il existe une solution de la forme $y=\e^{\alpha x}$.

\exercice{nom={Utilisation du plan de phase}}
Le mouvement d'une particule chargée dans un champ magnétique dirigé suivant
l'axe $(Oz)$ est régi par un système différentiel de la forme
$$\left\lbrace
\begin{array}{l}
x''=\omega y'\\
y''=-\omega x'\\
z''=0
\end{array}
\right.$$
où $\omega$ dépend de la masse, de la charge de la particule et du champ
magnétique. En considérant  $u=x'+\ii y'$, résoudre ce système différentiel.

\magsubsection{Équation différentielle avec second membre}


\exercice{nom={Calcul}}
Résoudre les équations différentielles suivantes sur $\R$
\[y''+y'-6y=1-8x-30x^2, \qquad y''+3y'+2y=\e^{-x},\]
\[y''-4y'+4y=x\cosh\p{2x}, \qquad y''+y=\sin^3 x.\]
%\[y''-2y'+5y=-4e^{-x}\cos x+7e^{-x}\sin x-4e^x\sin\p{2x}\]
% \begin{enumerate}
% \item $
% \item $y''+y'=3+2x$
% \item $y''+4y=4+2x-8x^2-4x^3$
% \item $y''+3y'+2y=e^x$
% \item $y''+3y'+2y=e^{-x}$
% \item $y''+4y'+4y=\p{16x^2+16x-14}e^{2x}$
% \item $y''-3y'+2y=\p{-3x^2+10x-7}e^x$
% \item $y''+y'-2y=8\sin\p{2x}$
% \end{enumerate}
\begin{sol}
\[y(x)=c_1 e^{2x}+c_2 e^{-3x}+5x^2+3x+2\]
\end{sol}



\magsubsection{Problème de \nom{Cauchy}}
Déterminer l'unique solution $y$ sur $\R$ de l'équation différentielle
\[\forall x\in\R\qsep y''(x)+y(x)=3x^2\]
telle que $y(0)=1$ et $y'(0)=2$.


%END_BOOK
\end{document}
