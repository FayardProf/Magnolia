\documentclass{magnolia}

\magtex{tex_driver={pdftex}}
\magfiche{document_nom={Exercices sur les espaces vectoriels de dimension finie},
          auteur_nom={François Fayard},
          auteur_mail={fayard.prof@gmail.com}}
\magexos{exos_matiere={maths},
         exos_niveau={mpsi},
         exos_chapitre_numero={13},
         exos_theme={Espaces vectoriels de dimension finie}}
\magmisenpage{}
\maglieudiff{}
\magprocess

\begin{document}
%BEGIN_BOOK
\magsection{Famille libre, famille génératrice, base}
\exercice{nom={Familles de fonctions}}
Montrer que les familles de fonctions suivantes sont libres.
\begin{questions}
% \question La famille $f_0,f_1,\ldots,f_n$ définie par~:
%   \[\forall k\in\intere{0}{n} \quad \forall x\in \R \quad f_k(x)=x^k\]
\question La famille de fonctions $(f_0,\ldots,f_n)$ définies sur $\R$ par
  \[\forall k\in\intere{0}{n} \qsep \forall x\in\R \qsep f_k(x)\defeq x^k.\]
% \question $f_n(x)=x^n$, pour $A=\RPs$ et $n\in\Z$.
\question La famille $(f_{\alpha_1},\ldots,f_{\alpha_n})$
  où $\alpha_1,\ldots,\alpha_n\in\RP$ sont deux à deux distincts et
  \[\forall k\in\intere{1}{n} \qsep \forall x\in\R \qsep
    f_{\alpha_k}(x)\defeq \cos\p{\alpha_k x}.\]
\question La famille de fonctions $(f_0,\ldots,f_n)$ définies par
  \[\forall k\in\intere{0}{n} \qsep \forall x\in\R \qsep f_k(x)\defeq \sin^k x.\]
% \question $f_\alpha(x)=e^{\alpha x}$, pour $A=\R$ et $\alpha\in\R$.
\question La famille $(f_1,\ldots,f_n)$ où
  $\alpha_1,\ldots,\alpha_n\in\R$ sont deux à deux distincts et
  \[\forall k\in\intere{1}{n} \qsep \forall x\in\R \qsep
    f_{k}(x)\defeq \e^{\ii \alpha_k x}.\]
% \question $f_\alpha(x)=e^{\alpha x}$, pour $A=[1,2]$ et $\alpha\in\C$.
\question La famille $(f_{1},\ldots,f_{n})$ où
  $\alpha_1,\ldots,\alpha_n\in\R$ sont deux à deux distincts et
  \[\forall k\in\intere{1}{n} \qsep \forall x\in\R \qsep
    f_{k}(x)\defeq\abs{x-\alpha_k}.\]
\end{questions}
\begin{sol}
$\quad$
\begin{questions}
\question   
\question dériver deux fois.
\question Introduire le polynôme
  \[P=\sum_{k=0}^n X^k\]
  Alors $\forall x\in\R \quad P(\sin x)=0$ donc $P$ admet une infinité de
  racines donc $P=0$, donc tous les $\lambda_k$ sont nuls.
\question dériver.
\question Remarquer que $f_{\alpha_k}$ n'est pas dérivable en $\alpha_k$.
\end{questions}
\end{sol}


% \exercice{nom={Famille des fonction polynômes-exponentielle}}
% Pour tout entier naturel $n$ et tout nombre complexe $\alpha$, on définit la
% fonction $f_{n,\alpha}$ sur $\R$ par :
% \[\forall x \in \R \quad f_{n,\alpha}(x)=x^n e^{\alpha x}\]
% Le but de cet exercice est de montrer que la famille
% $(f_{n,\alpha})_{(n,\alpha) \in\N\times\C}$ est libre.
% \begin{questions}
% \question Montrer que la famille $(f_{n,0})_{n\in\N}$ est libre.
% \question Pour $\alpha\in\Cs$, résoudre sur $\R$ l'équation différentielle :
%   \[(E) \quad \forall t \in \R \quad y'(t)+\alpha y(t)=0\]
% \question Montrer par récurrence sur un majorant de
%   $\deg{P_1}+\dots+\deg{P_n}$ que si $\alpha_1,\ldots,\alpha_n$ sont $n$
%   complexes deux à deux distincts et si :
%   \[\forall x\in\R \quad P_1(x)e^{\alpha_1 x}+\ldots+P_n(x)e^{\alpha_n x}=0\]
%   alors $P_1=P_2=\ldots=P_n=0$.
% \question Conclure.
% \end{questions}

% \exercice{nom={Espace des fonctions trigonométriques}}
% Soit $E$ le $\R$-espace vectoriel des fonctions de $\R$ dans $\R$. Montrer que
% l'espace vectoriel engendré par les fonctions $x\rightarrow \cos^n(x)$ pour
% $n\in\N$ et celui engendré par les fonctions $x\rightarrow \cos(nx)$ pour
% $n\in\N$ coïncident.

% \exercice{nom={Nombres algébriques}}
% On considère $\R$ comme $\Q$-espace vectoriel.
% \begin{questions}
% \question Montrer que la famille $(1,\sqrt{2},\sqrt{3})$ est libre ({\it On dit
%   que cette famille de réels est $\Q$-libre}).
% \question Montrer que la famille $(\ln p)_{p\in\mathcal{P}}$ est libre ({\it
%   $\mathcal{P}$ désigne l'ensemble des nombres premiers}).
% \end{questions}

\exercice{nom={Changement de base}}
Dans $\R^3$, on considère les trois vecteurs $u\defeq(1,1,-1)$, $v\defeq(-1,1,1)$ et
$w\defeq(1,-1,1)$.
\begin{questions}
\question Montrer que $u,v,w$ forme une base de $\R^3$.
\question Donner les coordonnées de $(2,1,3)$ dans cette base.
\end{questions}
\begin{sol}
$\quad$
\begin{questions}
\question Il suffit de montrer que la famille est libre.
\question On trouve pour coordonnées $(3/2,2,5/2)$.
\end{questions}
\end{sol}

\exercice{nom={Modification d'une famille libre}}
Soit $E$ un \Kev et $x_1,\ldots,x_n$ une famille libre de $n$ vecteurs. On se
donne $n$ scalaires $\lambda_1,\ldots,\lambda_n$ et on pose
\[y\defeq\sum_{k=1}^n \lambda_k x_k\]
Pour tout $i\in\intere{1}{n}$ on pose $y_i\defeq x_i+y$. Donner une condition nécessaire
et suffisante sur les $\lambda_k$ pour que la famille $y_1,\ldots,y_n$ soit libre.
\begin{sol}
La condition nécessaire et suffisante est
\[\sum_{k=1}^n \lambda_k\neq -1\]
On voit facilement que si cette somme est différente de $-1$, la famille est
libre (il suffit d'écrire les équations obtenues en utilisant le fait que la
famille des $x_k$ est libre et de sommer toutes les équations). Si la somme
est égale à $-1$
\[\sum_{k=1}^n \lambda_k y_k=0\]
est une relation de liaison.
\end{sol}


\magsection{Dimension}

\exercice{nom={Dimension}}
Soit $A$ et $B$ deux sous-espaces vectoriels de dimension $3$ de $\R^5$.
Montrer que $A\cap B\not=\{0\}$.
\begin{sol}
Utiliser Grassman.
\end{sol}

% \subsection{Endomorphisme nilpotent}
% Soit $E$ un \Kev de dimension $n$ et $f$ un endomorphisme nilpotent de $E$,
% c'est-à-dire tel qu'il existe un entier $m$ tel que $f^m=0$.
% \begin{questions}
% \question En notant $m_0$ le plus petit entier tel que $f^{m_0}=0$, montrer
%   qu'il existe $x\in E$ tel que la famille $x,f(x),\ldots,f^{m_0-1}(x)$ soit
%   libre.
% \question Que peut-on en déduire sur $m_0$ ?
% \end{questions}

\exercice{nom={Supplémentaire commun}}
Soit $E$ un \Kev de dimension finie $n$, et $A$ et $B$ deux sous-espaces
vectoriels de $E$ de même dimension $r$. Le but de cet exercice est de montrer
que $A$ et $B$ ont un supplémentaire commun dans $E$.
\begin{questions}
\question Montrer le résultat lorsque $n=r+1$.
\question Plus généralement, montrer que si le résultat est vrai si
  $\dim A=\dim B=r+1$, alors il est vrai si $\dim A=\dim B=r$.
\question Conclure.
\end{questions}

% \subsection{Rang d'une famille de vecteurs}
% Déterminer le rang des familles suivantes :
% \begin{questions}
% \question $x_1=(1,0,1),x_2=(-1,-2,3),x_3=(-1,-1,1)$ dans $\R^3$.
% \question $x_1=(1,0,1,0),x_2=(1,1,0,0),x_3=(-1,-1,1,0),x_4=(0,0,2,0)$ dans
%   $\R^4$.
% \question $x_1=(3,2,1,0),x_2=(2,3,4,5),x_3=(0,1,2,3),x_4=(1,2,1,2)$ dans
%   $\R^4$.
% \question $\phi_1,\phi_2,\phi_3,\phi_4$ dans $\mathcal{L}(\R^4,\R)$ avec pour
%   $x,y,z,t\in\R$ :
%   \begin{itemize}
%   \item $\phi_1(x,y,z,t)=x+z$
%   \item $\phi_2(x,y,z,t)=-x+2y$
%   \item $\phi_3(x,y,z,t)=x+y-z+t$
%   \item $\phi_4(x,y,z,t)=y+t$
%   \end{itemize}
% \end{questions}

\exercice{nom={Centre de $\Endo{E}$}}
Soit $E$ un \Kev de dimension finie. Le but de cet exercice est de montrer que
le centre de $\Endo{E}$, c'est à dire l'ensemble des endomorphismes qui
commutent avec tous les endomorphismes est l'ensemble des homothéties.
\begin{questions}
\question Montrer que les homothéties sont dans le centre de $\Endo{E}$.
\question Soit $u$ un endomorphisme de $E$ tel que
  \[\forall x\in E \qsep \exists \lambda\in\K \qsep  u(x)=\lambda x.\]
  Montrer que $u$ est une homothétie.
\question Soit $u$ un endomorphisme de $E$ qui commute avec toutes les
  applications linéaires.
  \begin{questions}
  \question Soit $x\in E$. En considérant une symétrie par rapport à $\K x$,
    montrer qu'il existe $\lambda\in\K$ tel que $u(x)=\lambda x$.
  \question Conclure.
  \end{questions}
\end{questions}

\exercice{nom={Formes linéaires et trace sur $\mat{n}{\K}$}}
Soit $\phi$ une forme linéaire sur $\mat{n}{\K}$. 
\begin{questions}
\question Montrer qu'il
  existe une et une seule matrice $A\in\mat{n}{\K}$ telle que
  \[\forall X\in\mat{n}{\K} \qsep \phi(X)=\tr(AX).\]
\question On suppose que~: $\forall X,Y\in\mat{n}{\K} \qsep \phi(XY)=\phi(YX)$.
  Montrer qu'il existe $\lambda\in\K$ tel que $A=\lambda I_n$. En déduire que
  \[\forall X\in\mat{n}{\K} \qsep \phi(X)=\lambda \tr(X).\]
\end{questions}

\exercice{nom={Matrice triangulaire supérieure par blocs}}
Soit $n\in\N$, $p_1,p_2\in\N$ tels que $n=p_1+p_2$ et $A\in\mat{n}{\K}$ une matrice
triangulaire supérieure par blocs
\[A=
  \begin{pmatrix}
  A_{1,1} & A_{1,2}\\
  0 & A_{2,2}
\end{pmatrix} \qquad \text{où $A_{i,j}\in\mat{p_i,p_j}{\K}$.}\]
Montrer que $A$ est inversible si et seulement si $A_{1,1}$ et $A_{2,2}$ le
sont. Si tel est le cas, montrer que
\[A^{-1}=
  \begin{pmatrix}
  A_{1,1}^{-1} & \star \\
  0 & A_{2,2}^{-1}
\end{pmatrix}\]

\magsection{Calcul de dimension et de rang, hyperplan}


\exercice{nom={Le laplacien discret}}
Soit $\phi$ l'application de $\polyK$ dans $\polyK$ définie par
\[\forall P\in\polyK \qsep \phi(P)\defeq P\p{X+1}+P\p{X-1}-2P(X).\]
\begin{questions}
\question Calculer $\deg\cro{\phi(P)}$ en fonction de $\deg P$.
\question Quel est le noyau de $\phi$~?
\question Montrer que $\phi$ est surjective.
\end{questions}
\begin{sol}
$\quad$
\begin{questions}
\question
\question
\question Soit $Q\in\polyK$ et $n\in\N$ tel que $\deg Q\leq n$. On
  introduit~:
  \[\dspappli{\psi}{\polyK[n+2]}{\polyK[n]}{P}{\phi(P)}\]
  et on montre que $\psi$ est surjective. Pour cela, on montre que
  $(\psi(X^2),\ldots,\psi(X^{n+2}))$ est libre. Il suffit de remarquer
  que si
  \[\sum_{k=2}^{n+2} \lambda_k \psi(X^k)=0\]
  alors
  \[\sum_{k=2}^{n+2} \lambda_k X^k\in\ker\phi\]
  donc tous les $\lambda_k$ sont nuls.
\end{questions}
\end{sol}


\exercice{nom={Noyau et image en somme directe}}
Soit $E$ un \Kev de dimension finie et $f$ un endomorphisme de $E$. Montrer
que
\[E=\im f \oplus \ker f  \quad\Longleftrightarrow\quad
 \im f = \im f^2.\]
Cette équivalence est-elle vraie en dimension infinie~?


\exercice{nom={Rang et composition}}
Montrer qu'on ne change pas le rang d'une application linéaire en la composant par
la droite par une application surjective ou par la gauche par une application injective.  

\exercice{nom={Rang d'une somme d'applications linéaires}}
Soit $E$ et $F$ deux \Kev de dimension finie, $f$ et $g$ deux applications
linéaires de $E$ dans $F$.
\begin{questions}
\question Montrer que :
  \[\abs{\rg f-\rg g} \leq \rg (f+g) \leq \rg f+\rg g\]
\question On suppose dans cette question que $E=F$. Montrer que si
  $f\circ g=0$ et $f+g$ est inversible, alors $\rg f+\rg g=\dim E$.
\end{questions}

\exercice{nom={Dimension du noyau et composition}}
Soit $E$, $F$ et $G$ trois \Kevs de dimensions finies, $f\in\lin{E}{F}$ et
$g\in\lin{F}{G}$. En considérant la restriction de $f$ à
$\ker g\circ f$, montrer que~:
\[\dim \ker g\circ f \leq \dim \ker g + \dim \ker f\]
\begin{sol}
Utiliser la théorème du rang avec~:
\[\dspappli{\phi}{\ker(g\circ f)}{F}{x}{f(x)}\]
\end{sol}

\exercice{nom={Endomorphisme $f$ tel que $f^2=0$}}
Soit $E$ un \Kev de dimension 4 et $f\in\Endo{E}$ tel que $f^2=0$. Montrer
que $\rg f\leq 2$. 
\begin{sol}
Utiliser le fait que $\im f\subset\ker f$. On utilise ensuite le théorème
du rang. 
\end{sol}

\exercice{nom={Factorisation}}
Soit $E$ un \Kev de dimension finie et $f,g\in\Endo{E}$. Montrer qu'il existe
$h\in\Endo{E}$ tel que $f=h\circ g$ si et seulement si $\ker g\subset \ker f$.

% \subsection{Noyaux et images itérés}
% Soit $E$ un \Kev de dimension finie et $f\in\Endo{E}$. On définit pour tout
% entier $n$~:
% \[K_n=\ker f^n \quad \text{et} \quad I_n=\im f^n\]
% \begin{questions}
% \question 
%   \begin{questions}
%   \question Montrer que la suite $\p{K_n}$ est croissante au sens de l'inclusion et
%     que la suite $\p{I_n}$ est décroissante au sens de l'inclusion.
%   \question 
%     \begin{questions}
%     \question Montrer qu'il existe $n\in\N$ tel que $K_{n}=K_{n+1}$. Dans la
%       suite, on note $n_0$ le plus petit entier tel que $K_{n_0}=K_{n_0+1}$.
%     \question Montrer que :
%       \[\forall n \geq n_0 \quad K_n=K_{n_0}\]
%     \end{questions}
%   \question 
%     \begin{questions}
%     \question Montrer qu'il existe $n\in\N$ tel que $I_{n}=I_{n+1}$. Dans la
%       suite, on note $n_1$ le plus petit entier tel que $I_{n_1}=I_{n_1+1}$.
%     \question Montrer que :
%       \[\forall n \geq n_1 \quad I_n=I_{n_1}\]
%     \end{questions}
%   \question Montrer que $n_0=n_1$. Dans la suite, on note $r$ cette valeur commune.
%   \end{questions}
% \question 
%   \begin{questions}
%   \question Montrer que $I_r \oplus K_r = E$.
%   \question Montrer que $I_r$ et $K_r$ sont stables par $f$, puis que la
%     restriction de $f$ à $K_r$ est nilpotente et que la restriction de $f$ à
%     $I_r$ est un automorphisme.
%   \end{questions}
% \question 
%   \begin{questions}
%   \question Montrer que le suite de terme général $\dim I_{n} - \dim I_{n+1}$
%     est décroissante. ({\it On pourra pour cela considérer l'application
%     $\phi$ de $I_n$ dans $I_{n+1}$ qui à $x$ associe $f(x)$.})
%   \question Énoncer et démontrer un résultat semblable pour la suite $\p{K_n}$. 
%   \end{questions}
% \end{questions}

% \subsection{Bidual}
% Soit $E$ un \Kev de dimension finie. On note $E^{\star\star}$ le bidual de $E$,
% c'est-à-dire le dual du dual de $E$.
% \begin{questions}
% \question Montrer que $E$ et $E^{\star\star}$ sont isomorphes.
% \question Soit $\phi$ l'application de $E$ dans $E^{\star\star}$ qui à $x$ associe
%   l'application $\phi_x$ de $E^\star$ dans $\K$ qui à $\psi$ associe
%   $\psi(x)$.
%   \begin{questions}
%   \question Montrer que pour tout $x\in E$, $\phi_x$ est une forme linéaire
%     sur $E^\star$.
%   \question Montrer que $\phi$ est un isomorphisme.
%   \end{questions}
% \end{questions}

%END_BOOK
\end{document}