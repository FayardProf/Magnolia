\documentclass{magnolia}

\magtex{tex_driver={pdftex}}
\magfiche{document_nom={Exercices sur les fonctions usuelles},
          auteur_nom={François Fayard},
          auteur_mail={fayard.prof@gmail.com}}
\magexos{exos_matiere={maths},
         exos_niveau={mpsi},
         exos_chapitre_numero={5},
         exos_theme={Fonctions usuelles}}
\magmisenpage{}
\maglieudiff{}
\magprocess

\begin{document}

%BEGIN_BOOK
\magsection{Logarithme, exponentielle, puissance}
\magsubsection{Logarithme népérien}

\exercice{nom={Équations, inéquations, inégalités}}
Montrer que pour tout $x\in\intero{-1}{1}\setminus\ens{0}$, on a
\[\frac{\ln\p{1+x}}{x}\leq -\frac{\ln\p{1-\abs{x}}}{\abs{x}}.\]
\begin{sol}
Pour $x<0$, l'inégalité est une égalité. Pour $x>0$, l'inégalité se
montre facilement en résolvant l'inéquation (inutile de dériver).
\end{sol}

\exercice{nom={Études de variations}}
\begin{questions}
\question Soit $a,b\in\R$ tels que $0<a<b$. On définit la fonction $f$ sur $\RPs$
  par
  \[\forall x>0 \qsep f(x)\defeq\frac{\ln\p{1+ax}}{\ln\p{1+bx}}.\]
  Étudier la monotonie de $f$.
\question
  \begin{questions}
  \question Montrer que
    \[\forall x\geq 0 \qsep x-\frac{x^2}{2}\leq\ln\p{1+x}\leq x.\]
  \question En déduire la limite de la suite de terme général
    \[\prod_{k=1}^n \p{1+\frac{k}{n^2}}.\]
  \end{questions}
\end{questions}
\begin{sol}
$\quad$
\begin{questions}
\question $f$ est croissante sur $\RPs$. Il suffit de dériver $f$ puis d'étudier
  le numérateur (on le dérive).
\question La limite est $\sqrt{e}$.
\end{questions}
\end{sol}

\magsubsection{Exponentielle}
\magsubsection{Logarithme et exponentielle en base $a$}

\exercice{nom={Équations, inéquations, inégalités}}
\begin{questions}
\question Résoudre, avec $a\in\RPs\setminus\ens{1}$
  $$\log_a x > \log_{a^3}\p{3x-2}.$$
\question Résoudre
  $$
  \begin{cases}
  \log_y x +\log_x y=\frac{50}{7} &\\
  xy=256. &
  \end{cases}
  $$
\end{questions}
\begin{sol}
$\quad$
\begin{questions}
\question Le domaine est $]2/3,+\infty[$. Si $a>1$, l'inéquation est toujours
  vérifiée, sauf si $x=1$. Si $a<1$, l'inéquation n'est jamais vérifiée.
\question On pose $u=(\ln x)/(\ln y)$ et on obtient $u+1/u=50/7$ ce qui
  donne 7 et $1/7$ comme solutions. On trouve $x=2$ et $y=128$ ou le contraire.\\
  On peut aussi multiplier tout par $\ln x\ln y$ et obtenir une équation en
  le produit et la somme de $\ln x$ et $\ln y$.
\end{questions}
\end{sol}

\magsubsection{Fonction puissance}
\magsubsection{Calcul de limite}

\exercice{nom={Calcul de limite en $\pm\infty$}}
Déterminer les limites, si elles existent, en $+\infty$ des fonctions
d'expressions
\[\sqrt{x+1}-\sqrt{x}, \qquad \sqrt{x^2+x+1}-\sqrt{x}, \qquad
  \frac{\sqrt{2x^2+1}-\sqrt{x^2+x+1}}{x},\]
\[\sqrt{x+\sqrt{x+\sqrt{x}}}-\sqrt{x}, \qquad
  \frac{\p{x^x}^x}{x^{\p{x^x}}}, \qquad \frac{\e^{2x} \ln^3 x}{x^4},\]
\[\frac{a^{\p{b^x}}}{b^{\p{a^x}}} \quad \text{où $1<a<b$}, \qquad
  \frac{a^{\p{a^x}}}{x^{\p{x^a}}} \quad \text{où $a>1$}.\]
Déterminer la limite, si elle existe, en $-\infty$ de 
\[x^2 \e^{x} \ln^3(-x).\]

\begin{sol}
$$\sqrt{x+1}-\sqrt{x}=\dfrac{1}{\sqrt{x+1}+\sqrt{x}}\tendvers{x}{+\infty}{0}.$$
$$\sqrt{x^2+x+1}-\sqrt{x}=x\p{\sqrt{1+1/x+1/x^2}-1/\sqrt{x}}\tendvers{x}{+\infty}{+\infty}.$$
$$\frac{\sqrt{2x^2+1}-\sqrt{x^2+x+1}}{x}=\sqrt{2+1/x^2}-\sqrt{1+1/x+1/x^2}\tendvers{x}{+\infty}{\sqrt{2}-1}.$$
$$\sqrt{x+\sqrt{x+\sqrt{x}}}-\sqrt{x}=\dfrac{\sqrt{x+\sqrt{x}}}{\sqrt{x+\sqrt{x+\sqrt{x}}}+\sqrt{x}}=\dfrac{\sqrt{1+1/\sqrt{x}}}{\sqrt{1+\sqrt{1/x+1/x^{3/2}}}+1}\tendvers{x}{+\infty}{\frac{1}{2}}.$$
$$\frac{\sin x}{x}\tendvers{x}{+\infty}{0}.$$
$$\frac{\p{x^x}^x}{x^{\p{x^x}}}=\e^{x^x\p{\frac{1}{x^{x-2}}-1}\ln(x)}\tendvers{x}{+\infty}{0}.$$

$$\frac{\e^{2x} \ln^3 x}{x^4}\tendvers{x}{+\infty}{0} \text{ par croissances comparées}.$$
$$\frac{a^{\p{b^x}}}{b^{\p{a^x}}}=\e^{\e^{x\ln(b)}\p{\ln(a)-\e^{x\ln(a/b)}\ln(b)}}\tendvers{x}{+\infty}{+\infty}.$$ 
$$\frac{a^{\p{a^x}}}{x^{\p{x^a}}}\tendvers{x}{+\infty}{+\infty}.$$
\end{sol}

\exercice{nom={Calcul de limite en 0}}
Déterminer les limites, si elles existent, en $0$ des fonctions
d'expressions
\[\frac{\sqrt{1+x}-\sqrt{1-x}}{x}, \qquad
  x^x, \qquad \abs{\ln x}^x,\]
\[x^2 \ln^3(x^3), \qquad \frac{\sqrt[3]{1+x}-\sqrt[3]{1-x}}{x}.\]
\begin{sol}
\begin{questions}
\question On a
\[\frac{\sqrt{1+x}-\sqrt{1-x}}{x}=\frac{(1+x)-(1-x)}{x\p{\sqrt{1+x}+\sqrt{1-x}}}=\frac{2}{\sqrt{1+x}+\sqrt{1-x}}\tendvers{x}{0}1.\]
\question On a
\[x^x=\e^{x \ln(x)}.\]
Or, par croissances comparées
\[x\ln(x)\tendvers{x}{0}0\]
donc
\[x^x\tendvers{x}{0}1.\]
\question On a
\[\abs{\ln x}^x=\e^{x\ln\abs{\ln x}}.\]
On pose $u\defeq \abs{\ln x}=-\ln x$ pour $x\in\interof{0}{1}$. Alors
\[u=-\ln(x)\tendvers{x}{0}+\infty\]
et
\[\abs{\ln x}^x=\e^{\frac{\ln u}{\e^ u}}.\]
Comme
\[\frac{\ln u}{\e^ u}\tendvers{u}{+\infty}0\]
on en déduit que
\[\abs{\ln x}^x\tendvers{x}{0}1.\]
\question On a, par croissances comparées
  \[x^2 \ln^3(x^3)=27 x^2\ln^3(x)\tendvers{x}{0}0.\]
\question En posant $a=\sqrt[3]{1+x}$ et $b=\sqrt[3]{1-x}$, on a
\begin{eqnarray*}
 \frac{\sqrt[3]{1+x}-\sqrt[3]{1-x}}{x}=\frac{a-b}{x}
 &=&\frac{a^3-b^3}{x\p{a^2+ab+b^2}}=\frac{2}{a^2+ab+b^2}\\
 &=&\frac{2}{\p{1+x}^{2/3}+\cro{(1+x)(1-x)}^{1/3}+\p{1-x}^{2/3}}\tendvers{x}{0}\frac{2}{3}
 \end{eqnarray*}
\question On a
\[\p{\sin x}^{\frac{1}{\ln x}}=\e^{\frac{\ln(\sin x)}{\ln x}}=\e^{1+\frac{\ln\p{\frac{\sin x}{x}}}{\ln x}}\tendvers{x}{0}\e.\]

\end{questions}
\end{sol}






\magsection{Fonctions trigonométriques directes et réciproques}
\magsubsection{Fonctions trigonométriques directes}

\exercice{nom={Calcul de limite en 0}}
Déterminer les limites, si elles existent, en $0$ des fonctions
d'expressions
\[\frac{\ln\p{1+\sin x}}{x}, \qquad \p{\sin x}^{\frac{1}{\ln x}},
  \qquad \frac{\sin x}{\sqrt{1-\cos x}}.\]
\begin{sol}
\begin{questions}
\question Au voisinage de 0
  \[\frac{\ln(1+\sin x)}{x}=\frac{\ln(1+\sin x)}{\sin x}\cdot\frac{\sin x}{x}\]
  Or
  \[\sin x\tendvers{x}{0}0 \et \frac{\ln(1+u)}{u}\tendvers{u}{0}1 \quad\donc\quad
    \frac{\ln(1+\sin x)}{\sin x}\tendvers{x}{0}1.\]
  De plus
  \[\frac{\sin x}{x}\tendvers{x}{0}1\]
  donc
  \[\frac{\ln(1+\sin x)}{x}\tendvers{x}{0}1.\]
\question L'expression étant impaire, on cherche sa limite à droite en 0. On a, puisque $\sin(x/2)>0$ au voisinage à droite de 0
  \[\frac{\sin x}{\sqrt{1-\cos x}}=\frac{\sin x}{\sqrt{2\sin^2\p{\frac{x}{2}}}}=\frac{1}{\sqrt{2}}\cdot\frac{\sin x}{\sin \frac{x}{2}}=\sqrt{2}\cdot\frac{\sin x}{x}\cdot\frac{\frac{x}{2}}{\sin\frac{x}{2}}\tendversdp{x}{0}\sqrt{2}.\]
  Par imparité
  \[\frac{\sin x}{\sqrt{1-\cos x}}\tendversgp{x}{0}-\sqrt{2}\]
  donc $(\sin x)/\sqrt{1-\cos x}$ n'admet pas de limite en 0.
\end{questions}
\end{sol}

\magsubsection{Fonction $\arcsin$}
\magsubsection{Fonction $\arccos$}
\exercice{nom={Étude de fonction}}
On considère la fonction $f$ définie par
$$f(x)\defeq\arccos\sqrt{\frac{1+\sin x}{2}}-\arcsin\sqrt{\frac{1+\cos x}{2}}.$$
\begin{questions}
\question Déterminer le domaine de définition et de continuité de $f$.
\question Pour tout $x\in\R$, exprimer $f\p{\frac{\pi}{2}-x}$ en fonction de
  $f(x)$. Sur quel intervalle $I$ suffit-il de faire l'étude de $f$~?
\question Étudier la dérivabilité de $f$ sur $I$ et calculer $f'$.
\question Tracer le graphe de $f$ sur $\interf{-\pi}{\pi}$.
\end{questions}
\begin{sol}
\begin{questions}
\question On cherche le domaine de $\arccos\sqrt{(1+\sin x)/2}$. Pour tout $x\in\R$, on a
\begin{eqnarray*}
-1\leq\sin(x)\leq 1
&\donc& 0\leq 1+\sin(x)\leq 2\\
&\donc& 0\leq \frac{1+\sin(x)}{2}\leq 1\\
&\donc& 0\leq \sqrt{\frac{1+\sin(x)}{2}}\leq 1.
\end{eqnarray*}
Donc $\arccos\sqrt{(1+\sin x)/2}$ est définie sur $\R$. De même $\arcsin\sqrt{(1+\cos x)/2}$ est définie sur $\R$. Donc $f$ est définie sur $\R$. D'après les théorèmes usuels, $f$ est donc continue sur $\R$.
\question Soit $x\in\R$. Alors
\begin{eqnarray*}
f\p{\frac{\pi}{2}-x}
&=& \arccos\sqrt{\frac{1+\sin\p{\frac{\pi}{2}-x}}{2}}-\arcsin\sqrt{\frac{1+\cos\p{\frac{\pi}{2}-x}}{2}}\\
&=& \arccos\sqrt{\frac{1+\cos x}{2}}-\arcsin\sqrt{\frac{1+\sin x}{2}}\\
&=& \frac{\pi}{2}-\arcsin\sqrt{\frac{1+\cos x}{2}}-\cro{\frac{\pi}{2}-\arccos\sqrt{\frac{1+\sin x}{2}}}\\
&=& \arccos\sqrt{\frac{1+\sin x}{2}} - \arcsin\sqrt{\frac{1+\cos x}{2}}\\
&=& f(x)
\end{eqnarray*}
De plus, $f$ est $2\pi$-périodique. Il suffit donc d'étudier $f$ sur un intervalle de longueur $2\pi$. Puisque $f(\pi/2-x)=f(x)$, on choisit un un intervalle centré en $\pi/4$~: $\interf{\pi/4-\pi}{\pi/4+\pi}$. Comme $f(\pi/2-x)=f(x)$, on se limite donc à l'étude de $f$ sur $\interf{\pi/4-\pi}{\pi/4}$, c'est-à-dire $\interf{-3\pi/4}{\pi/4}$.
On obtiendra le graphe de $f$ est effectuant une symétrie par rapport à la droite d'équation $x=\pi/4$ puis des translations de vecteur $2k\pi\ve{e_x}$ où $k\in\Z$.
\question On cherche le domaine de dérivabilité de $f$. On a
\begin{eqnarray*}
\forall x\in\interf{-\frac{3\pi}{4}}{\frac{\pi}{4}}\qsep
\frac{1+\sin(x)}{2}=0
&\ssi& \sin(x)=-1\\
&\ssi& x=-\frac{\pi}{2}
\end{eqnarray*}
De plus
\begin{eqnarray*}
\forall x\in\interf{-\frac{3\pi}{4}}{\frac{\pi}{4}}\qsep
\sqrt{\frac{1+\sin(x)}{2}}=1
&\ssi& \frac{1+\sin(x)}{2}=1\\
&\ssi& \sin x=1 \quad\text{ce qui est faux.}
\end{eqnarray*}
D'autre part
\begin{eqnarray*}
\forall x\in\interf{-\frac{3\pi}{4}}{\frac{\pi}{4}}\qsep
\frac{1+\cos(x)}{2}=0
&\ssi& \cos(x)=-1 \quad\text{ce qui est faux.}
\end{eqnarray*}
et enfin
\begin{eqnarray*}
\forall x\in\interf{-\frac{3\pi}{4}}{\frac{\pi}{4}}\qsep
\sqrt{\frac{1+\cos(x)}{2}}=1
&\ssi& \cos(x)=1\\
&\ssi& x = 0.
\end{eqnarray*}
D'après les théorèmes usuels, $f$ est donc dérivable sur $\mathcal{D}\defeq\interf{-3\pi/4}{\pi/4}\setminus\ens{-\pi/2,0}$ et
\begin{eqnarray*}
\forall x\in\mathcal{D}\qsep
f'(x)&=&\frac{\cos x}{2}\cdot\frac{1}{2\sqrt{\frac{1+\sin(x)}{2}}}\cdot(-1)\cdot\frac{1}{\sqrt{1-\p{\sqrt{\frac{1+\sin(x)}{2}}}^2}}+\\
     & &\frac{\sin x}{2}\cdot\frac{1}{2\sqrt{\frac{1+\cos(x)}{2}}}\cdot\frac{1}{\sqrt{1-\p{\sqrt{\frac{1+\cos(x)}{2}}}^2}}\\
&=& \frac{1}{2}\cro{\frac{-\cos x}{\sqrt{1-\sin^2 x}}+\frac{\sin x}{\sqrt{1-\cos^2 x}}}\\
&=& \frac{1}{2}\cro{\frac{\sin x}{\abs{\sin x}}-\frac{\cos x}{\abs{\cos x}}}.
\end{eqnarray*}
\question On en déduit que
\[\forall x\in\intero{0}{\frac{\pi}{4}}\qsep f'(x)=\frac{1}{2}\cro{\frac{\sin x}{\abs{\sin x}}-\frac{\cos x}{\abs{\cos x}}}=\frac{1}{2}\p{1-1}=0\]
Il existe donc $c$ tel que $f(x)=c$ sur $]0,\pi/4[$. Puisque $f$ est continue sur $[0,\pi/4]$, on en déduit que $f(x)=c$ sur $[0,\pi/4]$. Or $f(0)=\arccos(1/\sqrt{2})-\arcsin(1)=\pi/4-\pi/2=-\pi/4$. Donc
\[\forall x\in\interf{0}{\frac{\pi}{4}}\qsep f(x)=-\frac{\pi}{4}.\]
De plus
\[\forall x\in\intero{-\frac{\pi}{2}}{0}\qsep f'(x)=\frac{1}{2}\cro{\frac{\sin x}{\abs{\sin x}}-\frac{\cos x}{\abs{\cos x}}}=\frac{1}{2}\p{(-1)-1}=-1\]
Il existe donc $c$ tel que $f(x)=-x+c$ sur $]-\pi/2,0[$. Puisque $f$ est continue sur $[-\pi/2,0]$, on en déduit que $f(x)=-x+c$ sur $[-\pi/2,0]$. Or $f(0)=-\pi/4$. Donc
\[\forall x\in\interf{-\frac{\pi}{2}}{0}\qsep f(x)=-x-\frac{\pi}{4}.\]
Enfin
\[\forall x\in\intero{-\frac{3\pi}{4}}{-\frac{\pi}{2}}\qsep f'(x)=\frac{1}{2}\cro{\frac{\sin x}{\abs{\sin x}}-\frac{\cos x}{\abs{\cos x}}}=\frac{1}{2}\p{(-1)-(-1)}=0\]
Il existe donc $c$ tel que $f(x)=c$ sur $]-3\pi/4,-\pi/2[$. Puisque $f$ est continue sur $[-3\pi/4,-\pi/2]$, on en déduit que $f(x)=c$ sur $[-3\pi/4,-\pi/2]$. Or $f(-\pi/2)=\pi/4$. Donc
\[\forall x\in\interf{-\frac{3\pi}{4}}{-\frac{\pi}{2}}\qsep f(x)=\frac{\pi}{4}.\]
Ce qui permet de tracer $f$.
% \begin{center}
% \includegraphics[width=0.7\textwidth]{graphe-f.pdf}
% \end{center}
\end{questions}
\end{sol}

\magsubsection{Fonction $\arctan$}

\exercice{nom={Simplification}}
Simplifier les expressions suivantes
$$\arccos\p{\cos\frac{2\pi}{3}}, \qquad \arccos\p{\cos\p{-\frac{2\pi}{3}}},$$
$$\arccos\p{\cos 4\pi}, \qquad \arctan\p{\tan\frac{3\pi}{4}},$$ 
$$\tan\p{\arcsin x}, \qquad \sin\p{\arccos x}, \qquad \cos\p{\arctan x}.$$
\begin{sol}
$$\arccos\p{\cos\frac{2\pi}{3}}=\frac{2\pi}{3}, \qquad \arccos\p{\cos\p{-\frac{2\pi}{3}}}=\frac{2\pi}{3},$$
$$\arccos\p{\cos 4\pi}=\arccos\p{0}=0, \qquad \arctan\p{\tan\frac{3\pi}{4}}=\arctan\p{\tan\frac{-\pi}{4}}=\frac{-\pi}{4},$$ 
$$\tan\p{\arcsin x}=\frac{\sin\p{\arcsin x}}{\cos\p{\arcsin x}}=\frac{x}{\sqrt{1-x^2}}, \qquad \sin\p{\arccos x}=\pm \sqrt{1-\cos^2(\arccos x)}=\sqrt{1-x^2} \quad (\sin \geq 0 \text{ sur } [0;\pi])$$
$$\cos\p{\arctan x}=\sqrt{\cos^2\p{\arctan x}}=\sqrt{\frac{1}{1+\tan^2\p{\arctan x}}}=\sqrt{\frac{1}{1+x^2}}.$$
\end{sol}

\exercice{nom={Étude de fonction}}
Étudier la fonction définie par
$$f(x)\defeq x^2 \arctan\frac{1}{1+x^2}.$$
\begin{sol}
La fonction est paire d'où une étude sur $\RP$. On dérive et on obtient
\[\forall x\in\RP \quad f'(x)=2xg(x) \quad\text{avec}\quad g(x)=
   \arctan\frac{1}{1+x^2}-\frac{x^2}{(1+x^2)^2+1}\]
On dérive $g$ et on obtient
\[\forall x\in\RP \quad g'(x)=-\frac{4x(2+x^2)}{\p{(1+x^2)^2+1}}\]
On en déduit facilement que $f$ est croissante sur $\RP$. Sa limite en $+\infty$
est 1 et la fonction s'annule en 0.
\end{sol}


\magsubsection{Formules de trigonométrie réciproque}

\exercice{nom={Identités}}
A-t-on égalité entre les expressions suivantes~?
$$\arcsin\sqrt{x} \et \frac{\pi}{4}+\frac{1}{2}\arcsin\p{2x-1},$$
$$\arctan\frac{x+y}{1-xy} \et \arctan x+\arctan y,$$
$$\arcsin x+\arcsin\sqrt{1-x^2} \et \frac{\pi}{2},$$
$$2\arcsin x \et \arcsin\p{2x\sqrt{1-x^2}}.$$
\begin{sol}
$\quad$
\begin{questions}
\question $\forall x\in\interf{0}{1} \quad
  \arcsin\sqrt{x}=\frac{\pi}{4}+\frac{1}{2}\arcsin\p{2x-1}$
\question
\question On trouve
  \[\forall x\in\interf{-1}{1} \quad \arcsin x+\arcsin\sqrt{1-x^2}=
    \begin{cases}
    \frac{\pi}{2} & \text{si $x\geq 0$}\\
    \frac{\pi}{2}+2\arcsin x & \text{si $x\leq 0$}
    \end{cases}\]
\question On trouve
  \[\forall x\in\interf{-1}{1} \quad \arcsin\p{2x\sqrt{1-x^2}}=
    \begin{cases}
    -\pi-2\arcsin x & \text{si $-1\leq x \leq -1/\sqrt{2}$}\\
    2\arcsin x & \text{si $-1/\sqrt{2}\leq x\leq 1/\sqrt{2}$}\\
    \pi-2\arcsin x & \text{si $1/\sqrt{2}\leq x \leq 1$}
    \end{cases}\]
\end{questions}
\end{sol}

\exercice{nom={Équations}}
Résoudre les équations suivantes
$$\arctan x+\arctan\p{2x}=\frac{\pi}{4}, \qquad
  \arcsin\p{2x}=\arcsin{x}+\arcsin\p{\sqrt{2}x}.$$
\begin{sol}
$\quad$
\begin{questions}
\question On compose par $\tan$ et on trouve comme solutions $(-3\pm\sqrt{17})/2$.
  La première solution est à éliminer car négative. La seconde solution
  est à garder. Il faut remarquer que $(-3+\sqrt{17})/2\leq1/\sqrt{3}$.
  En conclusion l'unique solution est
  \[\frac{-3+\sqrt{17}}{2}\]
\question On compose par $\sin$ et on obtient 0 et $\pm \sqrt{\frac{7}{32}}$.
  Comme on est sur $\interf{-1/2}{1/2}$, la composition par $\sin$ ne perd pas
  l'équivalent. En conclusion, les solutions sont
  \[0,\pm\sqrt{\frac{7}{32}}\]
\end{questions}
\end{sol}




\magsection{Fonctions trigonométriques hyperboliques}
\exercice{nom={Simplification}}
Simplifier les expressions suivantes
\[\frac{\ch\p{\ln x}+\sh\p{\ln x}}{x}, \qquad \sh^2x\cos^2y+\ch^2x\sin^2y,\]
\[\ln\sqrt{\frac{1+\tanh x}{1-\tanh x}}.\]
% \begin{sol}
% On trouve~:
% \[\frac{\ch\p{\ln x}+\sh\p{\ln x}}{x}=1\]
% \[\sh^2x\cos^2y+\ch^2x\sin^2y=\sh^2 x+\sin^2 y\]
% \[\argsh\frac{x^2-1}{2x}=
%  \begin{cases}
%  \ln x & \text{si $x>0$}\\
%  -\ln(-x) & \text{si $x<0$}
%  \end{cases} \quad \text{poser $y=\ln x$}\]
% \[\argch\p{2x^2-1}=
%   \begin{cases}
%   2\argch x & \text{si $x\geq 1$}\\
%   2\argch(-x) & \text{si $x\leq -1$}
%   \end{cases}\]
% \[\ln\sqrt{\frac{1+\tanh x}{1-\tanh x}}=2x\]
% \end{sol}

\exercice{nom={Identité}}
Montrer que
\[\arctan\p{\e^x}-\arctan\p{\tanh\frac{x}{2}}\]
est une constante à déterminer.
\begin{sol}
Soit $f$ la fonction définie sur $\R$ par
\[\forall x\in\R\qsep f(x)\defeq \arctan\p{\e^x}-\arctan\p{\tanh\frac{x}{2}}.\]
D'après les théorèmes usuels, $f$ est dérivable sur $\R$ et
\begin{eqnarray*}
\forall x\in\R\qsep f'(x)
&=& \frac{\e^x}{1+\e^{2x}}-\frac{1}{2}\p{1-\th^2\p{\frac{x}{2}}}\frac{1}{1+\th^2\p{\frac{x}{2}}}\\
&=& \frac{\e^x}{1+\e^{2x}}-\frac{1}{2}\cdot\frac{\ch^2\p{\frac{x}{2}}-\sh^2\p{\frac{x}{2}}}{\ch^2\p{\frac{x}{2}}+\sh^2\p{\frac{x}{2}}}\\
&=& \frac{1}{\e^{-x}+\e^{x}}-\frac{1}{2}\cdot\frac{1}{\p{\frac{\e^{\frac{x}{2}}+\e^{-\frac{x}{2}}}{2}}^2+\p{\frac{\e^{\frac{x}{2}}-\e^{-\frac{x}{2}}}{2}}^2}\\
&=& \frac{1}{\e^{-x}+\e^{x}} - \frac{1}{\e^{-x}+\e^{x}}\\
&=& 0.
\end{eqnarray*}
Il existe donc $c\in\R$ tel que
\[\forall x\in\R\qsep f(x)=c.\]
Or $f(0)=\arctan(1)-\arctan(\th(0))=\pi/4-\arctan(0)=\pi/4$, donc $c=\pi/4$. En conclusion 
\[\forall x\in\R\qsep \arctan\p{\e^x}-\arctan\p{\tanh\frac{x}{2}}=\frac{\pi}{4}.\]
\end{sol}

\exercice{nom={Calcul de somme}}
Soit $a$ et $b$ deux réels. Calculer
$$\sum_{k=0}^n \ch\p{a+kb}.$$

\exercice{nom={Produit}}
\begin{questions}
\question Déterminer la limite, lorsque $x$ tend vers 0, de
  \[\frac{\th x}{x}.\]
\question Montrer que
  \[\forall x\in\R\qsep \th(2x)=\frac{2\th x}{1+\th^2 x}.\]
\question En déduire que pour tout $x\in\Rs$
  \[\prod_{k=1}^n \p{1+\th^2 \frac{x}{2^k}}\tendvers{n}{+\infty}\frac{x}{\th x}.\]
\end{questions}
%END_BOOK

\end{document}