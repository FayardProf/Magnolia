\documentclass{magnolia}

\magtex{tex_driver={pdftex}}
\magfiche{document_nom={Exercices d'arithmétique},
          auteur_nom={François Fayard},
          auteur_mail={fayard.prof@gmail.com}}
\magexos{exos_matiere={maths},
         exos_niveau={mpsi},
         exos_chapitre_numero={11},
         exos_theme={Arithmétique}}
\magmisenpage{}
\maglieudiff{}
\magprocess

\begin{document}

%BEGIN_BOOK
\magsection{Divisibilité, division euclidienne}

\magsection{Calculs modulo $n$}

\exercice{nom={Exercice}}
Soit $n\in\Ns$ et $a,b\in\Z$ tels que $a\equiv b\ [n]$. Montrer que
\[a^n\equiv b^n\ \left[n^2\right].\]

\magsection{$\pgcd$, $\ppcm$}

\exercice{nom={Calculs des coefficients de {\sc Bézout}}}
Résoudre dans $\Z$ les équations suivantes :
\[95x+71y=1 \qquad 24x-15y=3 \qquad 12x+15y+20z=1\]
\begin{sol}
$\quad$
\begin{questions}
\question On trouve une solution particulière à l'aide de l'algorithme d'Euclide étendu puis on obtient ensuite toutes les autres : $x=3+71k$ et $y=-4-95k$ avec $k\in\Z$.
\question $x=2+5k$ et $y=3+8k$ avec $k\in\Z$.
\question $x=-2+5k$, $y=-5+4k'$ et $z=5-3\p{k+k'}$.\\
Victor (Pas convaincu du tout, j'ai l'impression d'arnaquer...) : Analyse : On commence par résoudre $12x+5t=1$. On trouve $x=-2+5k$ et $t=5+12k$ avec $k\in\Z$. $y$ et $z$ doivent être solutions de $t=3y+4z$. On résout alors $3y'+4z'=1$ de solutions $y'=-1+4u$ et $z'=1-3u$. En multipliant par $t$ ces solutions, on obtient $$y=-5+4(3k+5u-12ku) \et z=5-3(4k+5u-12ku)=5-3(k+3k+5u-12ku).$$ On pose alors $k'=3k+5u-12ku$. Un tel $k'$ a toutes les raisons de parcourir $\Z$, donc a priori, on a trouvé que les solutions peuvent être pour $k,k'\in \Z$ :
$$x=-2+5k \et y=-5+4k' \et z=5-3\p{k+k'}.$$
Synthèse : on vérifie que cela fonctionne.
\end{questions}
\end{sol}

\exercice{nom={Ordre d'un produit}}
Soit $(G,\star)$ un groupe fini et $x,y$ deux éléments de $G$ d'ordre
respectifs $\omega_x$ et $\omega_y\in\Ns$. On suppose que $x\star y=y\star x$
et que $\omega_x\wedge\omega_y=1$.
\begin{questions}
\question Montrer que
${\rm Gr}(x)\cap{\rm Gr}(y)=\ens{e}$.
\question En déduire que  $xy$ est d'ordre $\omega_x\omega_y$.
\end{questions}

\begin{sol}
Déjà $(xy)^{mn} = x^{mn}y^{mn}= (x^m)^n(y^n)^m=e.e=e$. Soit $p$
tel que $(xy)^p = e$, alors $e= (xy)^{mp} = x^{mp}y^{mp}=y^{mp}$,
et donc $mp$ est divisible par l'ordre de $y$ , c'est-à-dire $n$.
Comme $m$ et $n$ sont premiers entre eux alors d'après le théorème
de Gauss $n$ divise $p$. Un raisonnement semblable à partir de
$(xy)^{np}=e$ conduit à : $m$ divise $p$. Finalement $m|p$ et
$n|p$ donc $mn|p$ car $m$ et $n$ sont premiers entre eux.\\
Voici un contre exemple dans le cas où $m$ et $n$ ne sont pas
premiers entre eux : dans le groupe $\Z/12\Z$ : $\bar{2}$ est
d'ordre $6$, $\bar{4}$ est d'ordre $3$, mais $\bar{2}+\bar{4} =
\bar{6}$ est d'ordre $2 \not= 3\times 6$.\\
Voici un contre-exemple dans le cas où les deux éléments ne commutent pas.
$A:=\left(\begin{smallmatrix} 0&-1\\1&0
\end{smallmatrix}\right)$ et $B:=\left(\begin{smallmatrix}
0&1\\-1&-1\end{smallmatrix}\right)$ sont des
\'el\'ements de $\text{GL}(2,\R)$ d'ordres finis
et $AB$ n'est pas d'ordre fini. $A$ est d'ordre $4$, $B$ est d'ordre $3$, $(AB)^n =
\begin{pmatrix} 1&n\\0&1\\ \end{pmatrix}$ n'est jamais la matrice identité pour $n\geq 1$.
\end{sol}

\exercice{nom={Divers calculs de $\pgcd$ et $\ppcm$}}
Soit $a, b\in\Z$. Calculer
\[(15a^2+8a+6)\wedge(30a^2+21a+13), \qquad (a^3+a)\wedge(2a+1),\]
\[(a-b)^3 \wedge (a^3-b^3), \qquad (a+b)\vee (a \wedge b).\]
%% Reponse
%% |a+b| - |a-b|(a^b)(3(a^b)^(a-b))
\begin{sol}
$\quad$
\begin{questions}
\question 
\begin{francois}
François :Une utilisation du lemme d'Euclide permet de conclure. On trouve 1.
\end{francois}\\
\begin{victor}
Victor :Rappelons que si $a=bq+r$, on a $a\wedge b=b\wedge r$ et que $a\wedge b=a\wedge (a-b)$ (on écrit $a=b+(a-b)$).
Ainsi, on trouve ici
$$30n^2+21n+13=2(15n^2+8n+6)+5n+1,\ \ \ 15n^2+8n+6=3n(5n+1)+5n+6.$$
On a donc
\begin{eqnarray*}
(30n^2+21n+13)\wedge (15n^2+8n+6)&=&(15n^2+8n+6)\wedge (5n+1)\\
&=&(5n+1)\wedge (5n+6)\\
&=&(5n+1)\wedge 5\\
&=&1.
\end{eqnarray*}
\end{victor}
\question Remarquer que $a\wedge (2a+1)=1$. On en déduit que
  \begin{eqnarray*}
  (a^3+a)\wedge (2a+1)
  &=& a(a^2+1)\wedge (2a+1)\\
  &=& (a^2+1)\wedge (2a+1)\\
  &=& (a^2-2a)\wedge (2a+1)\\
  &=& a(a-2)\wedge (2a+1)\\
  &=& (a-2)\wedge (2a+1)\\
  &=& (a-2)\wedge 5
  \end{eqnarray*}
  Donc le $\pgcd$ est 5 si $a\equiv 2\ [5]$ et est égal à 1 sinon.
\question On trouve $3\abs{a-b}\p{a\wedge b}^2$ si $a\equiv b [3(a\wedge b)]$
  et $\abs{a-b}(a\wedge b)^2$ sinon.
\question On trouve $\abs{a+b}$.
\end{questions}
\end{sol}

\exercice{nom={Autour du $\pgcd$}}
\begin{questions}
\question Soient $a,b,c\in\Z$ tels que $a\wedge c=1$. Montrer que~:
  \[\p{ab}\wedge c=b\wedge c\]
\question Trouver l'ensemble des couples $\p{a,b}\in\Z^2$ tels que~:
  \[a\wedge b=a+b-1\]
\end{questions}
\begin{sol}
$\quad$
\begin{questions}
\question Soit $a,b,c\in\Z$ tels que $a\wedge c=1$. Montrons que
  $\p{ab}\wedge c=b\wedge c$. Ces deux nombres étant positifs, il suffit de
  montrer que $\p{ab}\wedge c|b\wedge c$ et $b\wedge c|\p{ab}\wedge c$.
  \begin{itemize}
  \item $b\wedge c|\p{ab}\wedge c$.\\
    En effet $b\wedge c|b$ donc $b\wedge c|ab$. De plus $b\wedge c|c$ donc
    $b\wedge c|\p{ab}\wedge c$.
  \item $\p{ab}\wedge c|b\wedge c$.\\
    En effet $\p{ab}\wedge c|c$. De plus $\p{ab}\wedge c|ab$. Comme
    \[\p{\p{ab}\wedge c}\wedge a=\p{ab}\wedge\p{c\wedge a}=\p{ab}\wedge 1=1\]
    on en déduit, d'après le lemme de Gauss, que $\p{ab}\wedge c|b$. Donc
    $\p{ab}\wedge c|b\wedge c$.
  \end{itemize}
\question Montrer que si $a$ et $b$ sont solution du problème, alors $a\wedge b=1$.
  Les solutions sont les $a=2u+1$ et $b=1-2u$ avec $u\in\Z$.
\end{questions}
\end{sol}

\exercice{nom={Autour de la suite de Fibonacci}}
On définit la suite de Fibonacci par~:
\[F_0=0 \quad F_1=1 \et \forall n\in\N \quad F_{n+2}=F_{n+1}+F_n\]
\begin{questions}
\question Démontrer que~:
  \[\forall n\in\Ns \quad F_{n+1}F_{n-1}-F_n^2=\p{-1}^n\]
  En déduire que $F_n$ et $F_{n+1}$ sont premiers entre eux.
\question Démontrer que~:
  \[\forall n\in\N \quad \forall p\in\Ns \quad
    F_{n+p}=F_p F_{n+1}+F_{p-1}F_n\]
  En déduire que $F_n \wedge F_p = F_{n+p}\wedge F_p$.
\question Montrer que~:
  \[\forall n,p\in\N \quad F_n \wedge F_p = F_{n\wedge p}\]
\end{questions}
\begin{sol}
$\quad$
\begin{questions}
\question Par récurrence sur $n$. On a ensuite une relation de Bézout.
  \begin{eqnarray*}
  F_{n+2}F_n-F_{n+1}^2
  &=& \p{F_{n+1}+F_n}\p{F_{n+1}-F_{n-1}}-F_{n+1}^2
  \end{eqnarray*}
\question Faire une récurrence sur $n$ et mettre le \og pour tout $p$ \fg
  dans la récurrence. C'est ce qu'il y a de plus rapide.
  Il faut montrer qu'il se divisent entre eux. Pour un sens il faut utiliser
  Gauss et le fait que $F_p\wedge F_{p-1}=1$.
\question C'est l'algorithme d'Euclide.
\end{questions}
\end{sol}

\exercice{nom={Reste de la division euclidienne d'une puissance}}
Soit $n$ un entier supérieur à 2 et $a$ un entier premier avec $n$. Pour tout
entier $k$ on note $r_k$ le reste de la division euclidienne de $a^k$ par $n$.
\begin{questions}
\question Montrer que la suite $r$ est périodique. Pour cela on montrera dans
  l'ordre
  \begin{questions}
  \question qu'il existe $k_1,k_2\in\N$ tels que $k_1<k_2$ et $a^{k_1}\equiv a^{k_2}\ [n]$
  \question puis, qu'il existe $T\in\Ns$ tel que $a^T\equiv 1\ [n]$
  \question et enfin conclure
  \end{questions}
\question Quel est le reste de la division euclidienne de $3^{1998}$ par $5$ ?
\question Montrer que $13$ divise $3^{126}+5^{126}$.
\end{questions}
\begin{sol}
\begin{questions}
\question
\question On cherche la période de $3^k$ modulo 5. On trouve que $3^4\equiv 1\ \cro{5}$. Or $1998=4\times 499+2$, donc $3^{1998}\equiv 4\ \cro{5}$.
\question De même, on trouve que $3^3\equiv 1\ \cro{13}$ et que $5^4\equiv 126\ \cro{13}$. Or $126=3\times 42+0$ et $126=4\times 31+2$ donc
  $3^{126}+5^{126}\equiv 1+(-1)\equiv 0\ \cro{13}$.
\end{questions}
\end{sol}
\magsection{Théorèmes classiques}
\exercice{nom={Le théorème chinois}}
On se donne $p_1, p_2\in\Ns$ premiers entre eux, et $a_1$ et $a_2\in\Z$. 
\begin{questions}
\question On souhaite montrer qu'il existe $n\in\Z$ tel que~:
  \[n\equiv a_1 \quad [p_1] \qquad \text{et} \qquad n\equiv a_2 \quad [p_2]\]
  \begin{questions}
  \question Montrer que le problème admet une solution lorsque $a_1=1$ et
    $a_2=0$ ainsi que lorsque $a_1=0$ et $a_2=1$.
  \question En déduire le cas général.
  \end{questions}
\question En déduire l'ensemble des solutions du système~:
  \[n\equiv a_1 \quad [p_1] \qquad \text{et} \qquad n\equiv a_2 \quad [p_2]\]
\question {\bf Application~:} Résoudre le système
    \[n\equiv 3 \quad [21] \qquad \text{et} \qquad n\equiv 1 \quad [5]\]
\end{questions}

\begin{sol}
\begin{questions}
\question On se munit de $u,v\in \Z$ tels que $up_1+vp_2=1$.
  \begin{questions}
  \question Lorsque $a_1=1$ et
    $a_2=0$, on observe que $n=vp_2$ est solution. Lorsque $a_1=0$ et $a_2=1$, $n=up_1$ l'est.
  \question On prend $n=a_2up_1+a_1vp_2$.
  \end{questions}
\question Analyse : Soit $m$ une solution du système. Comme $n$ est également solution, on a :
  \[m-n\equiv 0 \quad [p_1] \qquad \text{et} \qquad m-n\equiv 0 \quad [p_2]\]
  Ainsi, $p_1\mid m-n \et p_2\mid m-n$ avec $p_1\wedge p_2=1$ donc $p_1p_2\mid m-n$. Ainsi, il existe $k\in \Z$ tel que $m=n+kp_1p_2$.
  Synthèse : réciproquement, on vérifie que de tels éléments sont solutions.
\question {\bf Application~:} On a $1\times 21+(-4)\times 5=1$. On a donc $n_0=21-12\times 5=-39$ comme solution particulière. L'ensemble des solutions est alors $$\set{-39+105k, k\in \Z}.$$
\end{questions}
\end{sol}

\exercice{nom={Les pirates}}
Une bande de 17 pirates dispose d'un butin composé de $N$ pièces d'or d'égale
valeur. Ils décident de se le partager également et de donner le reste au
cuisinier (non pirate). Celui-ci reçoit 3 pièces.\\
Mais une rixe éclate et 6 pirates sont tués. Tout le butin est reconstitué et
partagé entre les survivants comme précédemment; le cuisinier reçoit alors
4 pièces.\\
Dans un naufrage ultérieur, seuls le butin, 6 pirates et le cuisinier sont
sauvés. Le butin est à nouveau partagé de la même manière et le cuisinier
reçoit 5 pièces.\\
Quelle est alors la fortune minimale que peut espérer le cuisinier lorsqu'il
décide d'empoisonner le reste des pirates~?

\begin{sol}
Il faut traduire ceci en termes de congruences. On a :
$$\left\{
\begin{array}{rcl}
x&\equiv&3\ [17]\\
x&\equiv&4\ [11]\\
x&\equiv&5\ [6]
\end{array}\right.$$
On peut résoudre d'abord les deux premières équations
ensembles, puis introduire la troisième. Ici, tout est facilité si on remarque que $37$ est tel que
$37\equiv 3\ [17]$ et $37\equiv 4\ [11]$. Puisque $17\wedge 11=1$, on sait d'après l'exercice précédent que
$$\left\{
\begin{array}{rcl}
x&\equiv&3\ [17]\\
x&\equiv&4\ [11]
\end{array}\right.
\iff x\equiv 37[187].$$
On doit donc résoudre le système
$$\left\{\begin{array}{rcl}x&\equiv&37\ [187]\\x&\equiv&5\ [6].\end{array}\right.$$
Or, $1=1\times 187-6\times 37$. L'ensemble des solutions de ce système est donc : $$\{37+187\times1\times (5-37)+1122k;\ k\in\mathbb Z\}=\{-5947+1122k;\ k\in\mathbb Z\}.$$
etit entier positif est obtenu pour $k=6$ et donne $785$. Le cuisinier est sûr d'obtenir au moins 785 pièces
d'or.

\end{sol}

\exercice{nom={Petit théorème de Fermat, système de chiffrement%
  {\sc RSA}}}
\begin{questions}
\question 
Soit $p$ un nombre premier.
  \begin{questions}
  \question  Soit $k\in\intere{1}{p-1}$. Montrer que $p$ divise $\binom{p}{k}$.
  \question  Soit $a,b \in \Z$. Montrer que :
    \[(a+b)^p\equiv a^p+b^p \quad [p]\]
  \question  En déduire que pour tout $m\in\Z$ :
    \[m^p \equiv m \quad [p]\]
  \question  En déduire que si $m\in\Z$ et $p$ sont premiers entre eux :
    \[m^{p-1} \equiv 1 \quad [p]\]
  \end{questions}
\question  On se donne deux nombres premiers $p$ et $q$ distincts, on pose
  $n\defeq pq$ et on définit
  \[\phi(n)\defeq\card\enstq{k\in\intere{0}{n-1}}{k\wedge n=1}.\]
  \begin{questions}
  \question Soit $c\in\N$ tel que $c\wedge \phi(n)=1$. Montrer qu'il existe $d\in\N$ tel que $cd\equiv 1\ [\phi(n)]$.
  \question Montrer que $\phi(n)=(p-1)(q-1)$.
  \question Montrer que si $t\in\Z$, alors $t^{cd}\equiv t \quad [n]$.
  \end{questions}
\end{questions}

\begin{sol}
\begin{questions}
\question 
Soit $p$ un nombre premier.
  \begin{questions}
  \question  Déjà, on a $$p\mid k!\binom{p}{k}=\underbrace{p(p-1)\ldots (p-k+1)}_{k \text{ termes}}.$$
Or, $p\wedge i=1, \forall i \in \intere{1}{k}$, donc $p\wedge k!$ et donc d'après le lemme de Gauss, $p\mid \binom{p}{k}$.
  \question  En développant avec le binôme de Newton et en utilisant la question précédente.
  \question  On démontre ensuite par récurrence à l'aide de ce qu'on vient de démontrer que $\forall m \in \N$, $m^p\equiv m \quad [p]$. On étend cela à $\Z$. Si $m\leq 0\in \Z$, $0\equiv (-m+m)^p\equiv (-m)^p+m^p \equiv -m+m^p \quad [p]$ donc $m^p\equiv m \quad [p]$.
  \question  Soit alors $m\in \Z$ tel que $p\nmid m$, on a alors $p\wedge m=1$ car $p$ est premier. D'après ce qu'on vient de démontrer, $p\mid m^p-m=m(m^{p-1}-1)$ et $p\wedge m=1$ donc d'après le lemme de Gauss, $p\mid m^{p-1}-1$, i.e. $m^{p-1}\equiv 1\ [p]$.
  \end{questions}
\question  
  \begin{questions}
  \question Montrons que $\phi(n)=(p-1)(q-1)$. On va compter les $k$ dans $\intere{0}{n-1}$ qui ne sont pas premiers avec $pq$. Si $k$ n'est pas premier avec $pq$ alors nécessairement $k\wedge p\neq 1$ ou $k\wedge q=1$. Il y en a $q-1$ non premiers avec $p$ ($p, 2p, \ldots, (q-1)p$) et $(p-1)$ non premiers avec $q$ ($q,2q,\ldots,(p-1)q$) et ceux -ci sont distincts (le plus petit multiple de $p$ et $q$ est $pq$). Ainsi, $$\phi(n)=pq-(p-1)-(q-1)=(p-1)(q-1).$$
  \question D'après l'hypothèse sur $c$ et $d$, il existe $k\in\Z$ tel que $cd=1+k\phi(n)=1+k(p-1)(q-1)$. Soit alors $t\in\Z$. $t$ est alors premier avec $p$ ou avec $q$. Supposons par exemple que c'est avec $p$.  On a :
  $$t^{cd}\equiv t(t^{p-1})^{k(q-1)} \equiv 1 \quad [n]$$ d'après la première question.
  \end{questions}
\end{questions}

Commentaire culturel :
 {\it L'application $g$ de $(\Z/n\Z)$ dans $(\Z/n\Z)$ qui à $\bar{t}$ associe
 $\bar{t}^c$ s'appelle fonction de chiffrement, et l'application $f$ de
 $(\Z/n\Z)$ dans $(\Z/n\Z)$ qui à $\bar{t}$ associe $\bar{t}^d$ s'appelle
 fonction de déchiffrement. L'exercice affirme que $f\circ g=Id$. On peut donc
 chiffrer un message (représenté par $\bar{t}$) par le biais de $g$, puis on le
 déchiffre par le biais de $f$. Le couple $(n,c)$ est appelé clef publique,
 l'entier $d$ la clef secrète. La sécurité de ce système repose sur le fait que
 connaissant la clef publique, il est très difficile de déterminer la clef
 secrète $d$ : il faudrait par exemple factoriser $n$ pour trouver $p$ et $q$ ce
 qui est impossible à réaliser de nos jours lorsque $p$ et $q$ sont grands,
 typiquement de l'ordre de 100 chiffres. En d'autres termes, tout le monde peut
 chiffrer, mais seuls ceux connaissant la clef secrète peuvent déchiffrer. Ce
 système de chiffrement est apparu en 1976. Il est appelé système {\sc RSA} (du
 nom de ses inventeurs {\sc Rivest}, {\sc Shamir} et {\sc Adleman}) et est
 couramment utilisé aujourd'hui, car il est extrêmement robuste. Son apparition
 explique l'intérêt que l'on porte aujourd'hui aux algorithmes de factorisation
 de primalité.}
\end{sol}

% p = 2, q = 11, n = 22, phi(n) = 10, c = 3, d = 7
% En pratique, prendre p et q de l'ordre de 512 bits (154 chiffres en base 10)
% rend le système RSA incassable


\exercice{nom={Pour les Toulousaings}}
Soit $a_1,\ldots,a_{1789}\in\Z$ tels que
\[\sum_{k=1}^{1789} a_k=0\]
Montrer que
\[\sum_{k=1}^{1789} a_k^{37} \equiv 0\ [399].\]


\magsection{Nombres premiers}

\exercice{nom={Raréfication des nombres premiers}}
Montrer qu'il existe des intervalles de $\N$ de longueur aussi grande que l'on
veut qui ne contiennent aucun nombre premier.

\exercice{nom={Encadrement du n-ième nombre premier}}
Pour tout $n\in\Ns$, on note $p_n$ le $n$-ième nombre premier.
\begin{questions}
\question  Montrer que :
  \[\forall n \in \Ns \quad p_{n+1} \leq p_1 \ldots p_n +1\]
\question  En déduire que :
  \[\forall n \in \Ns \quad p_n \leq 2^{2^n}\]
\question  Soit $x\in\RP$. On note $\pi(x)$ le nombre de nombres premiers
  inférieurs ou égaux à $x$. Montrer que pour $x$ assez grand :
  \[\ln(\ln x) \leq \pi(x) \leq x\]
  {\it On démontrera le fait que pour $n\geq3$, $\e^{\e^{n-1}} \geq 2^{2^n}$.}
\end{questions}

\begin{sol}
\begin{questions}
\question  Cela vient de la preuve d'Euclide.
\question  Ca passe TB à la récurrence.
\question  On veut tout d'abord montrer que pour $n\geq3$, $e^{e^{n-1}} \geq 2^{2^n}$. En raisonnant par équivalence avec deux passages aux log, on arrive à $$n\geq \dfrac{1+\ln(\ln(2))}{1-\ln(2)}\approx 2,065.$$
Soit $x\in\RP$. On note $\pi(x)$ le nombre de nombres premiers inférieurs ou égaux à $x$. Montrer que pour $x$ assez grand :
  \[\ln(\ln x) \leq \pi(x) \leq x\]
  L'inégalité de droite ne pose pas de soucis.
  Pour l'inégalité de gauche. Posons $$n=1+\ent{\ln(\ln(x))}.$$
  On a $$n-1\leq \ln(\ln(x))\leq n$$ donc $$p_n\leq 2^{2^n}\leq e^{e^{n-1}}\leq x \leq e^{e^n}.$$ Par croissance (évidente) de la fonction $\pi$, cela donne :
  $$\ln(\ln(x))\leq n=\pi(p_n)\leq \pi(x).$$
\end{questions}
\end{sol}

\exercice{nom={Cas particuliers du théorème de {\sc Dirichlet}}}
\begin{questions}
\question Montrer qu'il existe une infinité de nombres premiers de la forme
  $4k+3$.
\question Montrer qu'il existe une infinité de nombres premiers de la forme
  $6k+5$.
\end{questions}
{\it Le théorème de {\sc Dirichlet} affirme que si $a$ et $b$ sont premiers entre eux, il existe une infinité de nombres premiers de la forme $ak+b$.}
%END_BOOK

\end{document}