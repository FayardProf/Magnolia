\documentclass{magnolia}

\magtex{tex_driver={pdftex},
        tex_packages={epigraph,xypic}}
\magfiche{document_nom={Cours sur les séries},
          auteur_nom={François Fayard},
          auteur_mail={fayard.prof@gmail.com}}
\magcours{cours_matiere={maths},
          cours_niveau={mpsi},
          cours_chapitre_numero={21},
          cours_chapitre={Séries}}
\magmisenpage{misenpage_presentation={tikzvelvia},
          misenpage_format={a4},
          misenpage_nbcolonnes={1},
          misenpage_preuve={non},
          misenpage_sol={non}}
\magmisenpage{}
\maglieudiff{}
\magprocess

\begin{document}

%BEGIN_BOOK
\setlength\epigraphwidth{.6\textwidth}
\epigraph{\og Divergent series are the invention of the devil, and it is shameful to base on them any demonstration whatsoever. \fg}{--- {\sc Niels Abel (1802--1829)}}

\setlength\epigraphwidth{.4\textwidth}
\epigraph{$\dsp \sum_{k=1}^{+\infty} k=-\frac{1}{12}$}{--- {\sc Srinivasa Ramanujan (1887--1920)}}

\magtoc
\vspace{2ex}
Dans ce chapitre, $\K$ désignera l'un des corps $\R$ ou $\C$.

\section{Série}
\subsection{Série}

\begin{definition}
Soit $(u_n)$ une suite d'éléments de $\K$. On appelle \emph{série de terme général $u_n$}
et on note $\sum u_n$ la suite $(S_n)$ définie par
\[\forall n\in\N\qsep S_n\defeq \sum_{k=0}^n u_k.\]
Le terme $S_n$ est appelé \emph{somme partielle d'indice $n$} de la série.
\end{definition}

% \begin{remarqueUnique}
% \remarque Soit $(v_n)$ une suite à valeurs dans $\RouC$. On définit la suite $(u_n)$ par
%   \[\forall n\in\N\qsep u_n\defeq v_{n+1} - v_n.\]
%   Alors
%   \[\forall n\in\N\qsep v_n=v_0 + \sum_{k=0}^{n-1} u_k.\]
%   L'étude de la suite $(v_n)$ se ramène donc à l'étude de la série $\sum u_n$. On dit que
%   la suite $(u_n)$ est la suite \emph{dérivée} de la suite $(v_n)$. Il sera parfois utile
%   d'exprimer une suite à l'aide d'une série afin d'appliquer des techniques propres à ces
%   dernières.
% \end{remarqueUnique}

\begin{definition}
On dit qu'une série $\sum u_n$ \emph{converge} lorsque la suite de ses sommes partielles
converge. Si c'est le cas, sa limite $l\in\K$ est appelée \emph{somme} de la série. On
la note
\[\sum_{n=0}^{+\infty} u_n.\]
Dans le cas contraire, on dit qu'elle \emph{diverge}.
\end{definition}

\begin{remarques}
\remarque Si on change un nombre fini de termes de la suite $(u_n)$, on ne change pas
  la nature de la série $\sum u_n$. Par contre, si elle converge, cela peut changer sa somme.
\remarque Soit $\sum u_n$ une série convergente. Alors, quel que soit $n\in\N$
  \[\sum_{k=0}^{+\infty} u_k=\sum_{k=0}^n u_k + \sum_{k=n+1}^{+\infty} u_k.\]
\end{remarques}

% \begin{remarques}
% \remarque Lorsque la série $\sum u_n$ converge vers $l$, on définit la suite $(r_n)$ par
%   \[\forall n\in\N\qsep r_n\defeq l-s_n.\]
%   Quel que soit $n\in\N$, on a alors
%   \[\sum_{k=n+1}^m u_k \tendvers{m}{+\infty} r_n\]
%   et on note
%   \[r_n=\sum_{k=n+1}^{+\infty} u_k.\]
% \end{remarques}

\begin{exoUnique}
\exo Montrer que la série
  \[\sum_{n\geq 1} \frac{1}{n(n+1)}\]
  converge et calculer sa somme.
\end{exoUnique}


\begin{definition}
Soit $\sum u_n$ une série convergente. On définit la suite $(R_n)$ par
\begin{eqnarray*}
\forall n\in\N\qsep R_n &\defeq& \sum_{k=0}^{+\infty} u_k - \sum_{k=0}^n u_k\\
&=& \sum_{k=n+1}^{+\infty} u_k.
\end{eqnarray*}
Le terme $R_n$ est appelé \emph{reste d'indice $n$} de la série.
\end{definition}

\begin{remarqueUnique}
\remarque La suite $(R_n)$ des restes converge vers 0.
\end{remarqueUnique}

\begin{proposition}
Soit $\sum u_n$ une série. Si elle est convergente, alors
\[u_n\tendvers{n}{+\infty}0.\]
Par contraposée, si la suite $(u_n)$ ne converge pas vers 0, la série $\sum u_n$ est
divergente. On dit qu'elle diverge \emph{grossièrement}.
\end{proposition}

\begin{remarqueUnique}
\remarque Il est possible qu'une série diverge sans diverger grossièrement.
  Par exemple, si $(u_n)$ est la suite définie par
  \[\forall n\in\N\qsep u_n\defeq\sqrt{n+1}-\sqrt{n}\]
  alors, la série associée diverge alors que la suite $(u_n)$ converge vers 0.
\end{remarqueUnique}

\begin{proposition}
La suite $(u_n)$ et la série $\sum (u_{n+1}-u_n)$ sont de même nature.
\end{proposition}

\begin{remarqueUnique}
\remarque Si $(u_n)$ est une suite, la suite de terme général $u_{n+1}-u_n$ est appelée
  \emph{dérivée} de la suite $(u_n)$. Par sommation télescopique
  \[\forall n\in\N\qsep u_n=u_0 + \sum_{k=0}^{n-1} (u_{k+1}-u_k).\]
  L'étude de la suite $(u_n)$ se ramène donc à l'étude de la série
  $\sum (u_{n+1}-u_n)$.
\end{remarqueUnique}

\begin{proposition}
Soit $\sum u_n$ et $\sum v_n$ deux séries convergentes et
$\lambda,\mu\in\K$. Alors, la série \mbox{$\sum (\lambda u_n+\mu v_n)$} est convergente et
\[\sum_{n=0}^{+\infty} \p{\lambda u_n+\mu v_n}=\lambda\sum_{n=0}^{+\infty} u_n
  +\mu\sum_{n=0}^{+\infty} v_n.\]
\end{proposition}

\begin{remarques}
\remarque Attention, il est possible que la série $\sum (\lambda u_n+\mu v_n)$ soit convergente
  sans que les séries $\sum u_n$ et $\sum v_n$ le soient. Avant d'écrire
  \[\sum_{n=0}^{+\infty} \p{\lambda u_n+\mu v_n}=\lambda\sum_{n=0}^{+\infty} u_n
    +\mu\sum_{n=0}^{+\infty} v_n\]
  il faudra donc toujours vérifier que les séries $\sum u_n$ et $\sum v_n$ soient convergentes.
  Un tel oubli pourrait conduire à écrire des horreurs comme
  \begin{eqnarray*}
  0 = \sum_{n=0}^{+\infty} 0 &=& \sum_{n=0}^{+\infty} (1+(-1))\\
  &=& \sum_{n=0}^{+\infty} 1 + \sum_{n=0}^{+\infty} (-1).
  \end{eqnarray*}
  La dernière expression n'a en effet aucun sens car les deux séries sont grossièrement divergentes.
\remarque Si $\sum u_n$ est convergente et $\sum v_n$ est divergente, alors
  $\sum (u_n+v_n)$ est divergente.
\end{remarques}


% \begin{exoUnique}
% \exo Déterminer les $z\in\C$ pour lesquels la série
%   \[\sum_{n\in\N} z^n\]
%   converge. Lorsque c'est le cas, calculer sa somme.
% \end{exoUnique}

\begin{proposition}
Soit $z\in\C$. Alors la série
\[\sum z^n\]
converge si et seulement si $\abs{z}<1$. Si tel est le cas, sa somme est
\[\sum_{n=0}^{+\infty} z^n = \frac{1}{1-z}.\]
\end{proposition}

\begin{exoUnique}
\exo Soit $(F_n)$ la suite de Fibonacci définie par
  \[F_0\defeq 0, \quad F_1\defeq 1, \et \forall n\in\N \qsep F_{n+2}\defeq F_{n+1}+F_n.\]
  Démontrer l'existence puis calculer
  \[\sum_{n=0}^{+\infty} \frac{F_n}{2^n}.\]
\begin{sol}
On trouve 2.
\end{sol}
\end{exoUnique}

% \begin{exoUnique}
% \exo Pour tout $x\in\intero{-1}{1}$, calculer
%   \[\sum (n+1)x^n.\]
%   On pourra conjecturer le résultat avant de le prouver rigoureusement.
% \end{exoUnique}

\begin{proposition}
Soit $z\in\C$. Alors la série
\[\sum \frac{z^n}{n!}\]
est convergente et
\[\sum_{n=0}^{+\infty} \frac{z^n}{n!}=\e^z.\]
\end{proposition}

\begin{exoUnique}
\exo Établir l'existence et calculer
    \[\sum_{n=0}^{+\infty} \frac{n^2+1}{n!}.\]
\begin{sol}
On trouve $3\e$.
\end{sol}
\end{exoUnique}



\subsection{Série à termes positifs}

\begin{definition}
On dit qu'une série réelle $\sum u_n$ est à termes positifs lorsque
\[\forall n\in\N\qsep u_n\geq 0.\]
\end{definition}

\begin{remarques}
\remarque La suite des sommes partielles d'une série à termes positifs est croissante.
  Réciproquement, si $(u_n)$ est une suite croissante, sa suite dérivée $(u_{n+1}-u_n)$ est
  une suite à termes positifs.
\remarque Puisque la convergence d'une série ne dépend pas de ses premiers termes, les
  théorèmes de convergence sur les séries à termes positifs s'appliquent même si la série
  est à termes positifs à partir d'un certain rang. Bien entendu, des théorèmes similaires
  aux théorèmes que nous allons énoncer existent pour les séries à termes négatifs.
  Les théorèmes que nous allons énoncer dans cette section sont donc utiles pour étudier
  les séries de signe constant à partir d'un certain rang.
\end{remarques}

\begin{proposition}
Une série à termes positifs converge si et seulement si la suite de ses sommes partielles
est majorée.
\end{proposition}

\begin{proposition}[nom={Série de \nom{Riemann}}]
Soit $\alpha\in\R$. Alors, la série
\[\sum \frac{1}{n^\alpha}\]
est convergente si et seulement si $\alpha > 1$.
\end{proposition}

\begin{remarqueUnique}
\remarque En particulier, la série harmonique $(H_n)$ définie par
  \[\forall n\in\N\qsep H_n\defeq\sum_{k=1}^n \frac{1}{k}\]
  diverge. Une comparaison série intégrale permet de montrer que
  $H_n\equi{n}{+\infty} \ln n$.
\end{remarqueUnique}

\begin{exos}
\exo Montrer que la série
  \[\sum \frac{1}{n^2}\]
  converge et donner un équivalent de son reste.
\exo Prouver la divergence et donner un équivalent des sommes partielles de la série
  \[\sum \frac{1}{n \ln n}.\]
\end{exos}

\begin{proposition}
Soit $\sum u_n$ et $\sum v_n$ deux séries à termes positifs telles que
\[\forall n\in\N\qsep u_n\leq v_n.\]
\begin{itemize}
\item Si la série $\sum v_n$ converge, alors il en est de même pour $\sum u_n$.
\item Si la série $\sum u_n$ diverge, alors il en est de même pour $\sum v_n$.
\end{itemize}
\end{proposition}

\begin{exos}
\exo Donner la nature des séries
  \[\sum\frac{\sin^2 n}{n^3},\qquad\sum\frac{\cos^2\p{\frac{n\pi}{3}}}{\sqrt{n}}.\]
\exo Soit $r\in\RP$. Déterminer la nature de la série
  \[\sum \frac{r^n}{n}\]
  en fonction de $r$.
\end{exos}

\begin{proposition}
Soit $\sum u_n$ et $\sum v_n$ deux séries à termes positifs. On suppose que
\[u_n=\grando{n}{+\infty}{v_n}\]
et que la série $\sum v_n$ est convergente. Alors la série $\sum u_n$ est convergente.
\end{proposition}

\begin{remarqueUnique}
\remarque En particulier, pour des séries à termes positifs, si
\[u_n=\petito{n}{+\infty}{v_n}\]
et si $\sum v_n$ est convergente, alors $\sum u_n$ est convergente. 
\end{remarqueUnique}

\begin{exoUnique}
\exo Déterminer la nature de la série
  \[\sum \frac{\ln n}{n\sqrt{n}}.\]
\end{exoUnique}

\begin{proposition}
Soit $\sum u_n$ et $\sum v_n$ deux séries réelles. On suppose que $\sum v_n$ est à termes
positifs et que
\[u_n\equi{n}{+\infty} v_n.\]
Alors la série $\sum u_n$ est à termes positifs à partir d'un certain rang et les deux séries
sont de même nature.
\end{proposition}

\begin{exos}
\exo Établir la nature des séries suivantes
  \[\sum \frac{1}{3n+1}, \qquad\sum \tan\p{\frac{1}{n}}, \qquad \sum \frac{1}{2^n-n}.\]
\exo Donner la nature des séries
  \[\sum \ln\p{\tan\frac{\pi n}{4n+1}},\qquad \sum \cro{\p{\tanh n}^{\frac{1}{n}}-1}.\]
\exo Montrer qu'il existe $\gamma\in\R$ tel que
  \[\sum_{k=1}^n \frac 1k = \ln n+ \gamma + \petito{n}{+\infty}{1}.\]
  La constante $\gamma\approx 0.577$ est appelée constante d'\nom{Euler}.
\end{exos}


\subsection{Série absolument convergente}

\begin{definition}
Soit $\sum u_n$ une série d'éléments de $\K$. Si la série à termes positifs
\[\sum \abs{u_n}\]
converge, alors la série $\sum u_n$ converge. On dit dans ce cas que la série
$\sum u_n$ est \emph{absolument convergente}.
\end{definition}

\begin{remarqueUnique}
\remarque Une série convergente qui n'est pas absolument convergente est appelée série
  \emph{semi-convergente}.
\end{remarqueUnique}

\begin{exoUnique}
\exo Montrer que la série
  \[\sum\frac{\sin n}{n\sqrt{n}}\]
  est convergente.
\end{exoUnique}


\begin{proposition}
Soit $\sum u_n$ une série d'éléments de $\K$ et $\sum v_n$ une série à termes positifs
telle que
\[u_n=\grando{n}{+\infty}{v_n}.\]
Si $\sum v_n$ est convergente, alors $\sum u_n$ est absolument convergente, donc convergente.
\end{proposition}

\begin{remarqueUnique}
\remarque En particulier, si $(v_n)$ est une suite positive, si
  \[u_n=\petito{n}{+\infty}{v_n}\]
  et si $\sum v_n$ est convergente, alors $\sum u_n$ est absolument convergente, donc
  convergente.
\end{remarqueUnique}

\begin{proposition}[nom={Règle de d'\nom{Alembert}}]
Soit $\sum u_n$ une série d'éléments de $\K$ ne s'annulant pas. On suppose que
\[\abs{\frac{u_{n+1}}{u_n}}\tendvers{n}{+\infty} \omega\in\RP\cup\ens{+\infty}.\]
Alors
\begin{itemize}
\item Si $\omega<1$, la série $\sum u_n$ est absolument convergente.
\item Si $\omega>1$, la série $\sum u_n$ est grossièrement divergente.
\end{itemize}
\end{proposition}

\begin{remarqueUnique}
\remarque Si $\omega=1$, la règle de d'\nom{Alembert} ne permet pas de conclure. Dans ce cas, il peut être
  intéressant d'effectuer une comparaison avec une série de \nom{Riemann}.
\end{remarqueUnique}

\begin{exos}
\exo Déterminer la nature de la série
  \[\sum \frac{n^4}{3^n}.\]
\exo Soit $z\in\C$. Retrouver le fait que la série
  \[\sum \frac{z^n}{n!}\]
  est convergente.
  \exo Soit $a,b\in\R$. Donner une condition nécessaire et suffisante sur $a$ et $b$ pour que la série
  \[\sum \cro{\frac{n^2+2}{n^2+2n+1}-\p{a+\frac{b}{n}}}\]
  soit convergente.
\end{exos}

\subsection{Série semi-convergente}

\begin{theoreme}[nom={Théorème des séries alternées}]
Soit $(u_n)$ une suite à termes positifs, décroissante et convergeant vers 0.
Alors la série
\[\sum (-1)^n u_n\]
converge. De plus, si $(R_n)$ est la suite des restes définie par
\[\forall n\in\N\qsep R_n\defeq\sum_{k=n+1}^{+\infty} (-1)^k u_k\]
alors, pour tout $n\in\N$, $R_n$ est du signe de $(-1)^{n+1}$ et $\abs{R_n}\leq u_{n+1}$.
\end{theoreme}

\begin{remarques}
\remarque La série
  \[\sum \frac{(-1)^n}{\sqrt{n}}\]
  est convergente, mais n'est pas absolument convergente.
\remarque Les séries alternées permettent de construire des suites $(u_n)$ et $(v_n)$
  qui sont équivalentes en $+\infty$ mais qui ne sont pas de même nature. Par exemple,
  si on définit les suites $(u_n)$ et $(v_n)$ par
  \[\forall n\in\Ns\qsep u_n\defeq\frac{(-1)^n}{\sqrt{n}} \et
    v_n\defeq\frac{(-1)^n}{\sqrt{n}}+\frac{1}{n}\]
  alors $(u_n)$ et $(v_n)$ sont équivalentes en $+\infty$ bien que $\sum u_n$ soit convergente
  et que $\sum v_n$ soit divergente.
\end{remarques}

\begin{exos}
\exo Soit $\alpha\in\R$. Donner une condition nécessaire et suffisante sur $\alpha$ pour que
  la série
  \[\sum \frac{(-1)^n}{n^\alpha}\]
  soit convergente.
\exo Soit $\alpha>0$. Discuter, selon $\alpha$, de la nature de la série
  \[\sum \frac{(-1)^n}{n^\alpha + (-1)^n}.\]
\end{exos}

\begin{remarqueUnique}
\remarque Il arrive que l'on doive prouver la convergence de séries de la forme
  $\sum a_n u_n$ où
  \begin{itemize}
  \item La suite $(u_n)$ est une suite positive, décroissante et convergeant vers 0.
  \item La série de terme général $(a_n)$ est bornée.
  \end{itemize}
  Pour prouver la convergence d'une telle série on applique ce qu'on appelle une
  \emph{transformée d'\nom{Abel}} qui est l'équivalent discret d'une intégration par
  parties. On définit la suite $(A_n)$ par
  \[\forall n\in\N\qsep A_n\defeq\sum_{k=0}^{n-1} a_k.\]
  Alors
  \begin{eqnarray*}
  \forall n\in\N\qsep
  \sum_{k=0}^n a_k u_k
  &=& \sum_{k=0}^n (A_{k+1} - A_k)u_k\\
  &=& \sum_{k=0}^n A_{k+1} u_k  - \sum_{k=0}^n A_k u_k\\
  &=& \sum_{k=1}^{n+1} A_k u_{k-1}  - \sum_{k=0}^n A_k u_k\\
  &=& \p{A_{n+1} u_n - A_0 u_0} - \sum_{k=1}^n A_k (u_{k}-u_{k-1}).
  \end{eqnarray*}
  On montre ensuite que $A_{n+1} u_n$ tend vers 0 lorsque $n$ tend vers $+\infty$ et
  que la série $\sum A_n(u_n - u_{n-1})$ est absolument convergente, ce qui permet de prouver
  la convergence de $\sum a_n u_n$.
\end{remarqueUnique}

\begin{exoUnique}
\exo Soit $\theta\in\R\setminus 2\pi\Z$ et $\alpha\in\RPs$. Montrer que la série
  \[\sum \frac{\cos(n\theta)}{n^\alpha}\]
  est convergente.
\end{exoUnique}


% \subsection{Développement décimal d'un réel}

% \begin{proposition}[utile=1, nom=Développement décimal propre d'un réel]
% Soit $x$ un réel positif. Alors il existe un unique $a_0\in\N$ et une unique suite
% $(a_n)_{n \geq 1}$ d'éléments de $\intere{0}{9}$, non stationnaire égale à $9$, telle que 
% $$x=\sum\limits_{n=0}^{+\infty} \frac{a_n}{10^n}.$$
% \end{proposition}


% Le but de cette partie est de donner un sens aux écritures suivantes :
% $$\pi = 3,14159265358979\dots \qquad - \frac{7}{11} = -0,6363636363 \dots $$


% \begin{center}
% \begin{tabular}{lr}
% \begin{minipage}{6cm}
% \begin{eqnarray*}
% x & = & 3,14159265358979\dots \\
% 10^3 x & = & 3141,59265358979\dots \\
% \ent{10^3 x} & = & 3141 \\
% \frac{\ent{10^3 x}}{10^3} &= & 3,141
% \end{eqnarray*}
% \end{minipage}

% &
% \begin{minipage}{6cm}
% \begin{eqnarray*}
% x & = & 3,14159265358979\dots \\
% 10^4 x & = & 31415,9265358979\dots \\
% \ent{10^4x} & = & 31415 \\
% \frac{\ent{10^4x}}{10^4} &= & 3,1415
% \end{eqnarray*}
% \end{minipage}
% \end{tabular}
% \end{center}


% On obtient donc une suite $S_n = \frac{\ent{10^n x}}{10^n}$ qui converge vers $x$ et les chiffres $a_n$ qui sont définis par 
% $$a_0 = \ent{x} \text{ et } a_n = \ent{10^n x} - 10 \ent{10^{n-1}x} \text{ pour } n \geq 1$$
% de sorte que 
% $$x = \sum\limits_{n=0}^{\infty} \frac{a_n}{10^n}$$

% Pour que ceci corresponde au développement décimal usuel, il faut prendre $x \geq 0$. Si $x<0$, $\ent{x}$ ne donne pas l'entier avant la virgule.
% La convention, pour $x<0$, est de prendre le développement décimal de $-x$ puis de mettre un signe moins devant.
% $a_0$ n'est pas un chiffre mais un entier quelconque.

% Par contre, si $n \geq 1$ alors $a_n \in \intere{0}{9}$.
% Malheureusement, il n'y a pas unicité d'une telle écriture. Par exemple,
% $$1=1,000 \dots = 0,999\dots$$

% Pour avoir l'unicité, il faut imposer que la suite $(a_n)$ n'est pas stationnaire égale à $9$ i.e. 
% $$\forall N \in \N \qsep \exists n \geq N \qsep a_n \neq 9.$$
% On dit alors que le développement décimal est propre. Montrons que l'on a bien alors l'unicité et l'existence de l'écriture décimale.

% \begin{proposition}[utile=1, nom=Développement décimal propre d'un réel]
% Soit $x$ un réel positif. Alors il existe un unique $a_0 \in \N$ et une unique suite $(a_n)_{n \geq 1}$ d'entiers entre $0$ et $9$ non stationnaire égale à $9$ telle que 
% $$x=\sum\limits_{n=0}^{\infty} \frac{a_n}{10^n}.$$

% \end{proposition}

\section{Famille sommable}

Les séries nous ont permis de donner un sens, lorsque c'est possible, à
\[\sum_{n=0}^{+\infty} u_n.\]
Cependant, de nombreux problèmes arrivent lorsque l'on souhaite sommer des familles $(u_{i,j})_{(i,j)\in\N^2}$
indexées par deux entiers. Il serait naturel de définir une telle somme, lorsque les séries en jeu
sont convergentes, par
\[\sum_{i=0}^{+\infty} \sum_{j=0}^{+\infty} u_{i,j}.\]
Mais on trouve rapidement des exemples pour lesquels
\[\sum_{i=0}^{+\infty} \sum_{j=0}^{+\infty} u_{i,j} \quad\et\quad
  \sum_{j=0}^{+\infty} \sum_{i=0}^{+\infty} u_{i,j}\]
ont toutes deux un sens et des valeurs différentes. Par exemple, si on définit la famille $(u_{i,j})_{(i,j)\in\N^2}$
par
\[\forall i,j\in\N\qsep u_{i,j}\defeq\begin{cases}
  1 & \text{si $i=j$,}\\
  -1 & \text{si $i=j+1$,}\\
  0 & \text{sinon,}\end{cases}\]
les séries doubles définies plus haut ont toutes deux un sens, mais la première est égale à 1 tandis que la seconde
vaut 0. Contrairement à ce qui se passe dans le cas des sommes finies, il arrive donc que la \og somme \fg des éléments
d'une famille infinie dépende de l'ordre de sommation. L'objet de la théorie des \emph{familles sommables} est d'avoir
un cadre dans lequel la somme de ces familles ne dépend pas de cet ordre.


\subsection{Famille sommable de réels positifs}

\begin{definition}
On pose $[0,+\infty]\defeq [0,+\infty[\cup\ens{+\infty}$.
\begin{itemize}
\item On étend la définition de $+$ sur $[0,+\infty]$ en posant
  \begin{eqnarray*}
  \forall x\in[0,+\infty],& &x + (+\infty) \defeq +\infty\\ 
                          & &(+\infty) + x \defeq +\infty
  \end{eqnarray*}
  On vérifie que $+$ est associative et commutative sur $[0,+\infty]$ et que 0 est élément
  neutre. 
\item On étend la définition de $\times$ en posant
  \begin{eqnarray*}
  \forall x\in]0,+\infty],& &x \times (+\infty) \defeq +\infty\\ 
                          & &(+\infty) \times x \defeq +\infty
  \end{eqnarray*}
  On pose enfin $0\times(+\infty)\defeq 0$ et $(+\infty)\times 0\defeq 0$. On vérifie que $\times$
  est associative et commutative sur $[0,+\infty]$ et que 1 est élément neutre. 
\item On étend la définition de $\leq$ sur $[0,+\infty]$ en posant
  \[\forall x\in[0,+\infty]\qsep x\leq +\infty.\]
  Muni de $\leq$, $[0,+\infty]$ est un ensemble totalement ordonné.
\end{itemize}
\end{definition}

\begin{remarques}
\remarque Excepté $+\infty$, tous les éléments de $[0,+\infty]$ sont réguliers pour $+$.
  Excepté 0 et $+\infty$, tous les éléments de $[0,+\infty]$ sont réguliers pour $\times$.
\remarque La relation d'ordre $\leq$ reste compatible avec l'addition et la multiplication~:
  il est toujours possible d'ajouter et de multiplier entre elles des inégalités puisque
  ces dernières sont positives.
\end{remarques}

\begin{definition}
Soit $A$ une partie de $[0,+\infty]$. On dit que $A$ admet une \emph{borne supérieure dans
$[0,+\infty]$} lorsque l'ensemble des majorants de $A$ dans $[0,+\infty]$ admet un plus
petit élément. Si tel est le cas, on le note $\supb A$.
\end{definition}

\begin{proposition}
Toute partie de $[0,+\infty]$ admet une borne supérieure dans $[0,+\infty]$.
\end{proposition}

\begin{remarques}
\remarque Soit $A$ une partie de $[0,+\infty]$. Si $+\infty\in A$, alors $\supb A=+\infty$. Sinon,
  $A$ est une partie de $[0,+\infty[$ et
  \begin{itemize}
  \item Si $A$ est vide, alors $\supb A=0$.
  \item Si $A$ n'est pas majorée, alors $\supb A=+\infty$. 
  \item Sinon, $A$ est non vide majorée. Elle admet donc une borne supérieure $\sup A$
    dans $\R$ et $\supb A=\sup A$.
  \end{itemize}
\remarque Soit $A$ une partie de $[0,+\infty]$.
  \begin{itemize}
  \item Si $B$ est une partie de $A$, alors $\supb B\leq\supb A$.
  \item Si $x\in [0,+\infty]$, on définit $x+A$ par
    \[x+A\defeq \ensim{x+a}{a\in A}.\]
    Alors, si $A$ est non vide, $\supb(x+A)=x+\supb A$.
  \item Si $\lambda\in[0,+\infty]$, on définit $\lambda A$ par
    \[\lambda A\defeq \ensim{\lambda a}{a\in A}.\]
    Alors, $\supb(\lambda A)=\lambda\supb A$.
  \end{itemize}
% L'ensemble vide admet une borne supérieure dans $[0,+\infty]$ qui est 0. Si $A$
%   est une partie de $[0,+\infty[$ qui n'est pas majorée dans $[0,+\infty[$, alors elle
%   admet une borne supérieure dans $[0,+\infty]$ qui est $+\infty$.
% \remarque Si $A$ est une partie de $[0,+\infty[$ non vide majorée, elle admet une borne
%   supérieure $\alpha\defeq\sup A\in\R$ au sens usuel. Alors $\alpha$ est la borne supérieure de $A$
%   dans $[0,+\infty]$. 
\end{remarques}

\begin{definition}
Soit $(u_i)_{i\in I}$ une famille d'éléments de $[0,+\infty]$. On appelle \emph{somme des $u_i$}
pour $i\in I$ et on note $\sum_{i\in I} u_i$ la borne supérieure de
\[A\defeq\ensim{\sum_{i\in K} u_i}{\text{$K$ est une partie finie de $I$}}\]
dans $[0,+\infty]$.
\end{definition}

\begin{remarqueUnique}
\remarque Si $I$ est fini, la somme que l'on vient de définir n'est autre que la somme
  $\sum_{i\in I} u_i$ définie de manière classique.
% \remarque Soit $(u_i)_{i\in I}$ une famille de réels positifs.
%   \begin{itemize}
%   \item Alors $\sum_{i\in I} u_i=+\infty$ si et seulement si, quel que soit $m\in\RP$, il existe une partie
%     finie $J$ de $I$ tel que
%     \[\sum_{i\in J} u_i\geq m.\]
%   \item De plus, si $l\in\RP$, alors $\sum_{i\in I} u_i=l$ si et seulement si, quel que
%   soit la partie finie $J$ de $I$, on a
%   \[\sum_{i\in J} u_i\leq l\]
%   et quel que soit $\epsilon>0$, il existe une partie finie $J$ de $I$ telle que
%   \[\sum_{i\in J} u_i\geq l-\epsilon.\] 
%   \end{itemize}
\end{remarqueUnique}

\begin{exoUnique}
\exo Montrer que
  \[\sum_{(i,j)\in\Ns^2} \frac{1}{(i+j)^2}=+\infty.\]
\end{exoUnique}

\begin{definition}
On dit qu'une famille $(u_i)_{i\in I}$ d'éléments de $[0,+\infty]$ est \emph{sommable} lorsque
\[\sum_{i\in I} u_i < +\infty.\]
\end{definition}

\begin{remarqueUnique}
\remarque Si l'un des $x_i$ est égal à $+\infty$, alors
  \[\sum_{i\in I} u_i=+\infty.\]
  En particulier, tous les éléments d'une famille sommable sont réels.
\end{remarqueUnique}


\begin{proposition}
Soit $(u_n)_{n\in\N}$ une suite réelle positive. Alors, la famille $(u_n)_{n\in\N}$ est
sommable si et seulement si la série $\sum u_n$ est convergente. De plus, si tel est le
cas
\[\sum_{n\in\N} u_n=\sum_{n=0}^{+\infty} u_n.\]
\end{proposition}

\begin{remarqueUnique}
\remarque Si la série à termes positifs $\sum u_n$ diverge, alors
  \[\sum_{n\in\N} u_n=+\infty.\]
  C'est pourquoi, certains auteurs se permettent d'écrire
  $\sum_{n=0}^{+\infty} u_n=+\infty$.
\end{remarqueUnique}

\begin{proposition}
Soit $(u_i)_{i\in I}$ une famille d'éléments de $[0,+\infty]$ et $J$ une partie de
  $I$. Alors
  \[\sum_{i\in J} u_i \leq \sum_{i\in I} u_i.\]  
\end{proposition}

\begin{remarqueUnique}
\remarque En particulier, si $(u_i)_{i\in I}$ est sommable, alors $(u_i)_{i\in J}$ est sommable.
\end{remarqueUnique}



\begin{proposition}
Soit $(u_i)_{i\in I}$ et $(v_i)_{i\in I}$ deux familles d'éléments de $[0,+\infty]$ et $\lambda,\mu\in[0,+\infty]$.
Alors
\[\sum_{i\in I} \p{\lambda u_i+\mu v_i}=\lambda \sum_{i\in I} u_i + \mu \sum_{i\in I} v_i.\]
\end{proposition}

  

\begin{proposition}
Soit $(u_i)_{i\in I}$ et $(v_i)_{i\in I}$ deux familles d'éléments de $[0,+\infty]$
telles que
\[\forall i\in I\qsep u_i\leq v_i.\]
Alors
\[\sum_{i\in I} u_i \leq \sum_{i\in I} v_i.\]
\end{proposition}

\begin{remarqueUnique}
\remarque En particulier, si $(v_i)_{i\in I}$ est sommable, alors $(u_i)_{i\in I}$ est sommable.
\end{remarqueUnique}


\begin{proposition}
Soit $(u_i)_{i\in I}$ une famille d'éléments de $[0,+\infty]$ et $\sigma:J\to I$ une bijection.
Alors
\[\sum_{i\in I} u_i=\sum_{j\in J} u_{\sigma(j)}.\]
\end{proposition}


\begin{proposition}[nom={Théorème de sommation par paquets}]
Soit $(u_i)_{i\in I}$ une famille d'éléments de $[0,+\infty]$. Si $(I_j)_{j\in J}$ est
une partition de $I$, alors
\[\sum_{j\in J} \p{\sum_{i\in I_j} u_i}=\sum_{i\in I} u_i.\]
\end{proposition}

\begin{remarqueUnique}
\remarque En particulier, si $(u_i)_{i\in I}$ une famille d'éléments de $[0,+\infty]$ et
    $I_1,I_2\in\mathcal{P}(I)$ sont tels que $I=I_1 \sqcup I_2$, alors
    \[\sum_{i\in I} u_i = \sum_{i\in I_1} u_i + \sum_{i\in I_2} u_i.\]
\end{remarqueUnique}

\begin{exoUnique}
\exo Pour quelles valeurs de $\alpha\in\R$ la famille
  \[\p{\frac{pq}{(p+q)^{\alpha}}}_{(p,q)\in\Ns^2}\]
  est-elle sommable~?
\begin{sol}
Si et seulement si $\alpha>4$. 
\end{sol}
\end{exoUnique}

\begin{proposition}[nom={Théorème de \nom{Fubini}}]
Soit $(u_{i,j})_{(i,j)\in I\times J}$ une famille d'éléments de $[0,+\infty]$. Alors
\[\sum_{j\in J} \p{\sum_{i\in I} u_{i,j}}=
  \sum_{i\in I} \p{\sum_{j\in J} u_{i,j}}=
  \sum_{(i,j)\in I\times J} u_{i,j}.\]
\end{proposition}


\begin{exos}
\exo Soit $a,b\in\RPs$. Montrer que la famille
  \[\p{\e^{-(ap+bq)}}_{(p,q)\in\N^2}\]
  est sommable et calculer sa somme.
\exo On considère la fonction $\zeta$ de \nom{Riemann} définie par
  \[\zeta(x)\defeq\sum_{n\in\Ns} \frac{1}{n^x}.\]
  \begin{questions}
  \question Montrer que $\zeta(x)$ est fini si et seulement si $x>1$.
  \question Montrer que
    \[\sum_{n\geq 2} (\zeta(n)-1)=1.\] 
  \end{questions}
\end{exos}

% \subsection{Familles sommables de réels positifs}




% \begin{proposition}
% Soit $(u_i)_{i\in I}$ et $(v_i)_{i\in I}$ deux familles d'éléments de $\RP$ telles que
% \[\forall i\in I\qsep 0\leq u_i\leq v_i.\]
% Si $(v_i)_{i\in I}$ est sommable, alors $(u_i)_{i\in I}$ est sommable et
% \[\sum_{i\in I} u_i\leq\sum_{i\in I} v_i.\] 
% \end{proposition}


% \begin{proposition}
% Soit $(u_i)_{i\in I}$ et $(v_i)_{i\in I}$ deux familles sommables d'éléments de $\RP$ et
% $\lambda,\mu\in\RP$. Alors $(\lambda u_i+\mu v_i)_{i\in I}$ est sommable et
% \[\sum_{i\in I} \p{\lambda u_i+\mu v_i} = \lambda \sum_{i\in I} u_i + \mu \sum_{i\in I} v_i.\]
% \end{proposition}


% \begin{proposition}[nom={Théorème de \nom{Fubini}}]
% Soit $(u_{i,j})_{(i,j)\in I\times J}$ une famille d'éléments de $\RP$. Alors
% \[\sum_{(i,j)\in I\times J} u_{i,j}=
%   \sum_{j\in J} \p{\sum_{i\in I} u_{i,j}}=
%   \sum_{i\in I} \p{\sum_{j\in J} u_{i,j}}.\]
% \end{proposition}


\subsection{Famille sommable d'éléments de $\K$}

\begin{definition}
On dit qu'une famille $(u_i)_{i\in I}$ d'éléments de $\K$ est \emph{sommable} lorsque
la famille des réels positifs $(\abs{u_i})_{i\in I}$ est sommable.
\end{definition}

\begin{remarques}
\remarque L'ensemble des familles sommables indexées par $I$ est noté $\ell(I,\K)$. C'est un sous-espace vectoriel de $\K^I$.
\remarque Si $(u_i)_{i\in I}$ est une famille sommable et $J$ est une partie de $I$, alors $(u_i)_{i\in J}$ est sommable.
\end{remarques}

\begin{exos}
\exo Montrer que la famille
  \[\p{\frac{\sin(p+q)}{p^2 q^2}}_{(p,q)\in\Ns^2}\]
  est sommable.
\exo Montrer que la famille
  \[\p{z^{ij}}_{(i,j)\in\Ns^2}\]
  est sommable si et seulement si $\abs{z}<1$.
\end{exos}

\begin{definition}
\begin{itemize}
\item Soit $(u_i)_{i\in I}$ une famille de réels sommable. Alors, les familles $(u_i^{+})_{i\in I}$ et
  $(u_i^{-})_{i\in I}$ sont sommables et on définit
  \[\sum_{i\in I} u_i\defeq \sum_{i\in I} u_i^{+} - \sum_{i\in I} u_i^{-}.\]
\item Soit $(u_i)_{i\in I}$ une famille de nombres complexes sommable. En décomposant $u_i = a_i + \ii b_i$ en sa
  partie réelle et sa partie imaginaire, les familles $(a_i)_{i\in I}$ et $(b_i)_{i\in I}$ sont sommables et on définit
  \[\sum_{i\in I} u_i\defeq \sum_{i\in I} a_i + \ii \sum_{i\in I} b_i.\]
\end{itemize}
\end{definition}


\begin{proposition}
Soit $(u_n)_{n\in\N}$ une suite d'éléments de $\K$. Alors, la famille $(u_n)_{n\in\N}$ est
sommable si et seulement si la série $\sum u_n$ est absolument convergente. De plus, si tel est le
cas
\[\sum_{n\in\N} u_n=\sum_{n=0}^{+\infty} u_n.\]
\end{proposition}
  

\begin{proposition}
Soit $(u_i)_{i\in I}$ une famille sommable d'éléments de $\K$ et $l\in\K$. Alors
\[\sum_{i\in I} u_i=l\]
si et seulement si, quel que soit $\epsilon>0$, il existe une partie finie $K$ de $I$ telle que
pour toute partie finie $L$ de $I$ telle que $K\subset L$, on a
\[\abs{\sum_{i\in L} u_i - l}\leq\epsilon.\]
\end{proposition}

\begin{remarqueUnique}
\remarque La définition \og historique \fg d'une famille sommable est la suivante~: on dit
  qu'une famille $(u_i)_{i\in I}$ d'éléments de $\K$ est sommable lorsqu'il existe $l\in\K$ tel que quel
  que soit $\epsilon>0$, il existe une partie finie $K$ de $I$ telle que pour toute partie finie $L$ de
  $I$ telle que $K\subset L$, on a
  \[\abs{\sum_{i\in L} u_i - l}\leq\epsilon.\]
  Si c'est le cas, $l$ est unique, et est appelé somme de la famille $(u_i)_{i\in I}$. La proposition
  précédente nous montre donc que si une famille est sommable pour le sens donné dans ce cours,
  alors elle est sommable pour le sens \og historique \fg. Réciproquement, on peut montrer que si une famille
  est sommable pour le sens \og historique \fg, elle est sommable pour le sens donné dans ce cours.
  Ceci montre l'équivalence des deux approches.
\end{remarqueUnique}

\begin{proposition}
Soit $(u_i)_{i\in I}$ et $(v_i)_{i\in I}$ deux familles sommables d'éléments de $\K$ et
$\lambda,\mu\in\K$. Alors $(\lambda u_i+\mu v_i)_{i\in I}$ est sommable et
\[\sum_{i\in I} \p{\lambda u_i+\mu v_i} = \lambda \sum_{i\in I} u_i + \mu \sum_{i\in I} v_i.\]
\end{proposition}

\begin{remarqueUnique}
\remarque Attention, il est possible que $(\lambda u_i+\mu v_i)_{i\in I}$ soit sommable sans
  que $(u_i)_{i\in I}$ et $(v_i)_{i\in I}$ le soient. Par exemple, bien que la famille
  \[\p{\frac{1}{n(n+1)}}_{n\in\Ns}\] soit sommable, on ne peut pas écrire
  \[\sum_{n\in\Ns} \frac{1}{n(n+1)}=\sum_{n\in\Ns} \p{\frac{1}{n}-\frac{1}{n+1}}=
    \sum_{n\in\Ns} \frac{1}{n} - \sum_{n\in\Ns}\frac{1}{n+1}\]
  puisque les familles $\p{\frac{1}{n}}_{n\in\Ns}$ et $\p{\frac{1}{n+1}}_{n\in\Ns}$ ne sont pas sommables.
\end{remarqueUnique}

\begin{proposition}
  Soit $(u_i)_{i\in I}$ et $(v_i)_{i\in I}$ deux familles réelles sommables telles que
  \[\forall i\in I\qsep u_i\leq v_i.\]
  Alors
  \[\sum_{i\in I} u_i \leq \sum_{i\in I} v_i.\]
  \end{proposition}

\begin{proposition}
  Soit $(u_i)_{i\in I}$ une famille sommable d'éléments de $\K$. Alors
  \[\abs{\sum_{i\in I} u_i}\leq\sum_{i\in I} \abs{u_i}.\]
  \end{proposition}

\begin{proposition}
Soit $(u_i)_{i\in I}$ une famille sommable d'éléments de $\K$ et $\sigma:J\to I$ une bijection.
Alors $\sum u_{\sigma(j)}$ est sommable et
\[\sum_{i\in I} u_i=\sum_{j\in J} u_{\sigma(j)}.\]
\end{proposition}

\begin{remarques}
\remarque Soit $\sum u_n$ une série absolument convergente. Alors, quel que soit
  la bijection $\sigma:\N\to\N$, la série $\sum u_{\sigma(n)}$ est absolument convergente et
  \[\sum_{n=0}^{+\infty} u_n=\sum_{n=0}^{+\infty} u_{\sigma(n)}.\]
\remarque Cette propriété est fausse pour les séries semi-convergentes. En effet, le théorème de réarrangement
  de \nom{Riemann} montre que si $\sum u_n$ est une série réelle semi-convergente, quel que soit
  $l\in\R$, il existe une bijection $\sigma:\N\to\N$ telle que
  \[\sum_{n=0}^{+\infty} u_{\sigma(n)} = l.\]
\end{remarques}

\begin{proposition}[nom={Théorème de sommation par paquets}]
Soit $(u_i)_{i\in I}$ une famille sommable d'éléments de $\K$. Si $(I_j)_{j\in J}$ est
une partition de $I$, alors la famille $(\sum_{i\in I_j} u_i)_{j\in J}$ est sommable et
\[\sum_{j\in J} \p{\sum_{i\in I_j} u_i}=\sum_{i\in I} u_i.\]
\end{proposition}

\begin{proposition}[nom={Théorème de \nom{Fubini}}]
Soit $(u_{i,j})_{(i,j)\in I\times J}$ une famille sommable d'éléments de $\K$. Alors, les
familles $(\sum_{i\in I} u_{i,j})_{j\in J}$ et $(\sum_{j\in J} u_{i,j})_{i\in I}$ sont
sommables et
\[\sum_{j\in J} \p{\sum_{i\in I} u_{i,j}}=
  \sum_{i\in I} \p{\sum_{j\in J} u_{i,j}}=
  \sum_{(i,j)\in I\times J} u_{i,j}.\]
\end{proposition}


\begin{remarqueUnique}
\remarque Soit $(u_{i,j})_{(i,j)\in\N^2}$ une famille sommable d'éléments de $\K$. Alors
\begin{eqnarray*}
\sum_{(i,j)\in\N^2} u_{i,j} &=&
  \sum_{i\in\N} \sum_{j\in\N} u_{i,j} =
  \sum_{j\in\N} \sum_{i\in\N} u_{i,j}\\
  &=& \sum_{i=0}^{+\infty} \sum_{j=0}^{+\infty} u_{i,j} =
  \sum_{j=0}^{+\infty} \sum_{i=0}^{+\infty} u_{i,j}.
\end{eqnarray*}
% \remarque Nous verrons qu'il existe des familles $(u_{i,j})_{(i,j)\in\N^2}$ telles que les sommes
%   \[\sum_{i=0}^{+\infty} \sum_{j=0}^{+\infty} u_{i,j} \quad\et\quad 
%     \sum_{j=0}^{+\infty} \sum_{i=0}^{+\infty} u_{i,j}\]
%   soient bien définies et distinctes. Bien entendu, de telles familles ne sont pas sommables.
\end{remarqueUnique}

\begin{exoUnique}
\exo Montrer que la famille
  \[\p{\frac{\e^{\frac{2\ii k\pi}{n}}}{2^n}}_{k\in\Ns, n> k}\]
  est sommable et calculer sa somme.
% \exo On définit la famille $(u_{i,j})_{(i,j)\in\N^2}$ par
%   \[\forall i,j\in\N\qsep u_{i,j}\defeq
%     \begin{cases}
%     1 & \text{si $i=j$,}\\
%     -\frac{1}{2^{j-i}} & \text{si $i<j$,}\\
%     0 & \text{sinon.}
%     \end{cases}\]
%   Calculer
%   \[\sum_{i=0}^{+\infty} \sum_{j=0}^{+\infty} u_{i,j} \quad\et\quad
%     \sum_{j=0}^{+\infty} \sum_{i=0}^{+\infty} u_{i,j}.\]
\end{exoUnique}

\begin{definition}
Soit $(u_n)_{n\in\N}$ et $(v_n)_{n\in\N}$ deux suites. On appelle
\emph{produit de \nom{Cauchy}} de ces suites la suite $(w_n)_{n\in\N}$
définie par
\[\forall n\in\N\qsep w_n\defeq\sum_{k=0}^n u_k v_{n-k}.\]
\end{definition}

\begin{proposition}
Soit $\sum u_n$ et $\sum v_n$ deux séries absolument convergentes. Alors, leur produit de
\nom{Cauchy} est absolument convergent et
\[\sum_{n=0}^{+\infty} \p{\sum_{k=0}^n u_k v_{n-k}}=
  \p{\sum_{n=0}^{+\infty} u_n}\p{\sum_{n=0}^{+\infty} v_n}.\]
\end{proposition}

\begin{exoUnique}
\remarque Retrouver le fait que
  \[\forall a,b\in\C\qsep \e^{a+b}=\e^a \e^b.\]
% \remarque Montrer que
%   \[\forall x\in\intero{-1}{1}\qsep \sum_{n=1}^{+\infty} n x^{n-1}=\frac{1}{(1-x)^2}.\]
\end{exoUnique}


%END_BOOK
\end{document}