\documentclass{magnolia}

\magtex{tex_driver={pdftex},
        tex_packages={epigraph,pgfplots,float,xypic},
        tex_pstricks={pstricks,pst-plot}}
\magfiche{document_nom={Cours sur la dérivation},
          auteur_nom={François Fayard},
          auteur_mail={fayard.prof@gmail.com}}
\magcours{cours_matiere={maths},
          cours_niveau={mpsi},
          cours_chapitre_numero={13},
          cours_chapitre={Dérivation}}
\magmisenpage{}
\maglieudiff{}
\magprocess

\begin{document}

%BEGIN_BOOK
\setlength\epigraphwidth{.5\textwidth}
\epigraph{\og Le chemin le plus court d'un point à un autre est la ligne droite, à condition que les deux points soient bien en face l'un de l'autre.\fg}{--- \textsc{Pierre Dac (1893--1975)}}
\bigskip
\hfill\includegraphics[width=0.4\textwidth]{../../Commun/Images/maths-cours-rolle.png}

\magtoc
\vspace{2ex}
Dans ce chapitre, $\K$ désignera l'un des corps $\R$ ou $\C$.

\section{Fonction dérivable, dérivées successives}
\subsection{Définition}

\begin{definition}[utile=-3]
On dit qu'une fonction $f:\dom\to\K$ est \emph{dérivable} en $x_0\in\mathcal{D}$ lorsque le
\emph{taux d'accroissement}
\[\frac{f(x)-f\p{x_0}}{x-x_0}\]
admet une limite finie lorsque $x$ tend vers $x_0$; si tel est le cas, on note
$f'\p{x_0}$ cette limite que l'on appelle \emph{nombre dérivé} de $f$ en $x_0$.
La propriété \og est dérivable en $x_0$ \fg est locale en $x_0$.
\end{definition}

%% Exemples :
%% 1) Si n\in\Ns, la fonction x -> x^n est dérivable en tout point de R et sa
%%    dérivée est nx^{n-1}
%% 2) La fonction f définie sur RP par f(x)=\sqrt{x} est dérivable en tout
%%    point de RPs et sa dérivée est 1/(2\sqrt{x}). De plus, f n'est pas
%%    dérivable en 0.
%%
%% Remarque : interprétation géométrique
%%   Équation d'une tangente
%%   Parler de tangente horizontale et de tangente verticale

\begin{remarques}
\remarque Un changement de variable montre que $f$ est dérivable en $x_0\in\mathcal{D}$
  si et seulement si
  \[\frac{f(x_0+h)-f(x_0)}{h}\]
  admet une limite finie lorsque $h$ tend vers $0$.
\remarque Si $f:\dom\to\R$ est dérivable en $x_0\in\dom$, son graphe admet une tangente
  d'équation
  \[y=f\p{x_0}+f'\p{x_0}\p{x-x_0}.\]
  Cette tangente est horizontale si et seulement si $f'\p{x_0}=0$.
  Lorsque le taux d'accroissement de $f$ en $x_0$ tend vers
  $\pm\infty$, le graphe de $f$ admet une tangente verticale.
  \begin{sol}
  Faire des dessins + on parle souvent de demi-tangente si $f$ est définie uniquement à droite ou à gauche du point considéré.
  \end{sol}
% \remarque Soit $f$ une fonction définie sur un intervalle $I$ 
%   qu'elle laisse stable. Étant donné $u_0\in I$, on définit la suite
%   $\p{u_n}$ par
%   \[\forall n\in\N \quad u_{n+1}=f\p{u_n}\]
%   Soit $x_0$ un point fixe de $f$ en lequel elle est dérivable.
%   \begin{itemize}
%   \item On dit que $x_0$ est \textit{attractif} lorsque $\abs{f'\p{x_0}}<1$. Dans ce
%     cas, il existe $\eta>0$ tel que, si $u_0\in I\cap\interf{x_0-\eta}{x_0+\eta}$,
%     $\p{u_n}$ converge vers $x_0$.
%   \item  Si $\abs{f'\p{x_0}}>1$, on dit que $x_0$ est \textit{répulsif}. Alors,
%     à l'exception du cas où elle est stationnaire en $x_0$, $\p{u_n}$ ne
%     converge pas vers $x_0$.
%   \end{itemize}
\end{remarques}

\begin{definition}[utile=-3]
$\quad$
\begin{itemize}
\item On dit qu'une fonction $f:\dom\to\K$ est \emph{dérivable à gauche} en $x_0\in\mathcal{D}$
  lorsque le taux d'accroissement
  \[\frac{f(x)-f\p{x_0}}{x-x_0}\]
  admet une limite finie lorsque $x$ tend vers $x_0$ par la gauche; si tel est le cas,
  on note $f_g'\p{x_0}$ cette limite que l'on appelle nombre dérivé à gauche de $f$
  en $x_0$. La propriété \og est dérivable à gauche en $x_0$ \fg est locale à
  gauche, au sens large, en $x_0$.
\item On dit qu'une fonction $f:\dom\to\K$ est \emph{dérivable à droite} en $x_0\in\mathcal{D}$
  lorsque le taux d'accroissement
  \[\frac{f(x)-f\p{x_0}}{x-x_0}\]
  admet une limite finie lorsque $x$ tend vers $x_0$ par la droite; si tel est le cas,
  on note $f_d'\p{x_0}$ cette limite que l'on appelle nombre dérivé à droite de $f$
  en $x_0$. La propriété \og est dérivable à droite en $x_0$ \fg est locale à
  droite, au sens large, en $x_0$.
\end{itemize}
\end{definition}

%% Exemple :
%% 1) La fonction x -> abs(x) est dérivable à gauche et à droite en 0

\begin{proposition}[utile=-3]
Une fonction $f:\dom\to\K$ est dérivable en $x_0\in\mathcal{D}$ si
et seulement si les objets ci-dessous susceptibles d'avoir un sens
\[f_g'\p{x_0} \et f_d'\p{x_0}\]
existent et sont égaux. Si tel est le cas, $f'\p{x_0}$ est cette valeur commune.
\end{proposition}

\begin{exoUnique}
\exo Étudier la dérivabilité des fonctions d'expression
  \[\abs{x} \et \frac{x}{1+\abs{x}}\]
  en 0.
\end{exoUnique}
\begin{sol}
On étudie pour chacune d'entre elles les taux d'accroissements en traitant les cas $h<0$ et $h>0$ à part. Ca ne marche (évidemment) pas pour la première, mais ça fonctionne pour l'autre où la dérivée en $0$ vaut $1$.
\end{sol}

\begin{proposition}[utile=-3]
Une fonction dérivable en un point est continue en ce point.
\end{proposition}

\begin{preuve}
Soit $f$ une fonction dérivable en $x_0$. Montrons qu'elle est continue en $x_0$.
$$f(x_0+h)=\underbrace{h}_{\tendversp{h}{0}0}\underbrace{\frac{f(x_0+h)-f(x_0)}{h}}_{\tendversp{h}{0}f'(x_0)}+f(x_0)\tendversp{h}{0}f(x_0).$$
Ainsi, $f(x_0+h)\tendvers{h}{0}f(x_0)$ ce qui signifie que $f$ est continue en $x_0$.
\end{preuve}

\begin{remarqueUnique}
\remarque La réciproque de cette proposition est fausse comme le montre l'exemple de la fonction
  $x\mapsto\sqrt{x}$ en 0.
\end{remarqueUnique}


\begin{proposition}[utile=-3]
Une fonction $f:\dom\to\K$ est dérivable en $x_0\in\mathcal{D}$ si et seulement si elle
y admet un développement limité à l'ordre 1. Si tel est le cas
\[f\p{x_0+h}=f\p{x_0}+f'\p{x_0}h+\petito{h}{0}{h}.\]
\end{proposition}

\begin{preuve}
Déjà vu dans le chapitre sur les DL.
\end{preuve}

\begin{definition}[utile=-3]
Soit $f:\dom\to\K$.
\begin{itemize}
\item On dit que $f$ est \emph{dérivable} lorsqu'elle est dérivable en tout point
  de $\dom$.
\item Si $A$ est une partie de $\dom$, on dit que $f$ est \emph{dérivable sur $A$} lorsque
  la restriction de $f$ à $A$ est dérivable.
\end{itemize}
\end{definition}

\begin{remarqueUnique}
\remarque Si $A$ est une partie de $\dom$ et que $f$ est dérivable en tout point de $A$,
  alors $f$ est dérivable sur $A$. Cependant la réciproque est fausse. Par exemple,
  la fonction $f$ définie sur $\R$ par $f(x)\defeq\abs{x}$ est dérivable sur $\RP$ mais elle
  n'est pas dérivable en 0.
\end{remarqueUnique}

\begin{proposition}
Soit $f:\dom\to\K$ une fonction définie sur un domaine \emph{élémentaire} et
$\dom=I_1\cup\cdots\cup I_n$ la décomposition de $\dom$ en composantes connexes. Alors
$f'$ est dérivable sur $\dom$ si et seulement si, pour tout $k\in\intere{1}{n}$, $f$
est dérivable sur $I_k$.
\end{proposition}

\begin{proposition}
Soit $f:I\to\K$ une fonction définie sur un intervalle $I$ et $a\in I$.
\begin{itemize}
\item Si $f$ est dérivable sur $I\cap\intero{a}{+\infty}$, elle est dérivable en tout
  point de $I\cap\intero{a}{+\infty}$. De plus, en notant $g$ la restriction de 
  $f$ à $I\cap\intero{a}{+\infty}$, on a
  \[\forall x\in I\cap\intero{a}{+\infty}\qsep f'(x)=g'(x).\]
\item Si $f$ est dérivable sur $I\cap\interfo{a}{+\infty}$, elle est dérivable à droite
  en $a$ et en tout point de $I\cap\intero{a}{+\infty}$. De plus, en notant $g$ la 
  restriction de $f$ à $I\cap\interfo{a}{+\infty}$, on a
  \[f_d'(a)=g'(a) \et \forall x\in I\cap\intero{a}{+\infty}\qsep f'(x)=g'(x).\]
\end{itemize}
\end{proposition}

\subsection{Théorèmes usuels}

\begin{proposition}[utile=-3]
Soit $f,g:\dom\to\K$ deux fonctions dérivables en $x_0$.
\begin{itemize}
\item Si $\lambda,\mu\in\K$, alors $\lambda f+\mu g$ est dérivable
  en $x_0$ et
  \[\p{\lambda f+\mu g}'\p{x_0}=\lambda f'\p{x_0}+\mu g'\p{x_0}.\]
\item $fg$ est dérivable en $x_0$ et
  \[\p{fg}'\p{x_0}=f'\p{x_0}g\p{x_0}+f\p{x_0}g'\p{x_0}.\]
\item Si $f\p{x_0}\neq 0$, alors $f$ ne s'annule pas au voisinage de $x_0$
  et $1/f$ est dérivable en $x_0$. De plus
  \[\p{\frac{1}{f}}'\p{x_0}=-\frac{f'\p{x_0}}{f^2\p{x_0}}.\]
\item Si $g\p{x_0}\neq 0$, alors $g$ ne s'annule pas au voisinage de $x_0$
  et $f/g$ est dérivable en $x_0$. De plus
  \[\p{\frac{f}{g}}'\p{x_0}=\frac{f'\p{x_0}g\p{x_0}-f\p{x_0}g'\p{x_0}}{
    g^2\p{x_0}}.\]
\end{itemize}
\end{proposition}

\begin{preuve}
\begin{itemize}
\item Ca passe tout seul pour la somme en écrivant le taux d'accroissement.
\item 
\begin{eqnarray*}
\frac{(fg)(x_0+h)-(fg)(x_0)}{h}&=&\frac{f(x_0+h)g(x_0+h)-f(x_0)g(x_0)}{h}\\
&=&\frac{f(x_0+h)g(x_0+h)-f(x_0)g(x_0+h)+f(x_0)g(x_0+h)-f(x_0)g(x_0)}{h}\\
&=&\underbrace{g(x_0+h)}_{\tendvers{h}{0}g(x_0) \text { \underline{c}}}\underbrace{\frac{f(x_0+h)-f(x_0)}{h}}_{\tendvers{h}{0}f'(x_0)}+f(x_0)\underbrace{\frac{g(x_0+h)-g(x_0)}{h}}_{\tendvers{h}{0}g'(x_0)}.
\end{eqnarray*}
Donc $fg$ est dérivable en $x_0$ et
  \[\p{fg}'\p{x_0}=f'\p{x_0}g\p{x_0}+f\p{x_0}g'\p{x_0}.\]
  \item Si $f\p{x_0}\neq 0$, comme $f$ est dérivable en $x_0$, elle y est continue, il existe donc $\eta>0$ tel que $\forall h\in \interf{-\eta}{\eta}$, $x+h\in \mathcal{D}_f \Longrightarrow f(x_0+h)\neq 0$. Ainsi, $\forall h \in \interf{-\eta}{eta}$, $$\frac{\frac{1}{f(x_0+h)}-\frac{1}{f(x_0)}}{h}=\frac{f(x_0)-f(x_0+h)}{hf(x_0+h)f(x_0)}=\underbrace{\frac{f(x_0)-f(x_0+h)}{h}}_{\tendvers{h}{0}-f'(x_0)}\underbrace{\frac{1}{f(x_0+h)f(x_0)}}_{\tendvers{h}{0}\frac{1}{f(x_0)^2}}\tendvers{h}{0}-\frac{f'(x_0)}{f(x_0)^2}.$$
  On a donc bien
    \[\p{\frac{1}{f}}'\p{x_0}=-\frac{f'\p{x_0}}{f^2\p{x_0}}.\]
\item Tout vient des deux points précédentes en écrivant $\dfrac{f}{g}=f\dfrac{1}{g}$.
\end{itemize}

\end{preuve}

\begin{proposition}[utile=-3]
Soit $f$ une fonction dérivable en $x_0$ et $g$ une fonction dérivable en
$f\p{x_0}$. Alors $g\circ f$ est dérivable en $x_0$ et
\[\p{g\circ f}'\p{x_0}=g'\p{f\p{x_0}}f'\p{x_0}.\]
\end{proposition}

\begin{preuve}
\begin{itemize}
\item[$\bullet$] \textbf{Preuve fausse (mais tentante):}
$$\frac{g(f(x_0+h))-g(f(x_0))}{h}=\underbrace{\frac{g(f(x_0+h))-g(f(x_0))}{f(x_0+h)-f(x_0)}}_{\tendvers{h}{0}g'(f(x_0))}\underbrace{\frac{f(x_0+h)-f(x_0)}{h}}_{\tendvers{h}{0}f'(x_0)}.$$
Problème si $f(x_0+h)-f(x_0)=0$ donc cette preuve fonctionne si $f$ est injective, mais pas dans le cas général.
\item[$\bullet$] \textbf{Vraie preuve :} On va éviter d'écrire des dénominateurs inconsidérés en recourant à des développements limités. \\
Comme $f$ et $g$ sont respectivement dérivables en $x_0$ et $f(x_0)$, il existe des fonctions $\epsilon_1$ et $\epsilon_2$ telles que pour $h$ cohérent avec les ensembles de définitions :
$$\begin{cases}
\epsilon_1(h)\tendvers{h}{0}0\\
\epsilon_2(h)\tendvers{h}{0}0\\
f(x_0+h)=f(x_0)+hf'(x_0)+h\epsilon_1(h)\\
g(f(x_0)+h)=g(f(x_0))+hg'(f(x_0))+h\epsilon_2(h)
\end{cases}$$
Ainsi, 
\begin{eqnarray*}
g(f(x_0+h))&=&g\p{f(x_0)+hf'(x_0)+h\epsilon_1(h)}\\
&=&g(f(x_0))+\p{hf'(x_0)+h\epsilon_1(h)}g'(f(x_0))+\p{hf'(x_0)+h\epsilon_1(h)}\epsilon_2(h)\\
&=&g(f(x_0))+hg'(f(x_0))f'(x_0)+h\underbrace{\epsilon_1(h)g'(f(x_0))+f'(x_0)\epsilon_2(h)+\epsilon_1(h)\epsilon_2(h)}_{=\epsilon(h)\tendvers{h}{0}0}
\end{eqnarray*}
D'où le résultat d'après la proposition sur les DL à l'ordre $1$.

\end{itemize}

\end{preuve}

% \begin{remarqueUnique}
% \remarque Soit $f\defeq g_1^{\alpha_1}g_2^{\alpha_2}\cdots g_n^{\alpha_n}$ où
%   $\alpha_1,\ldots,\alpha_n\in\R$. Sans se soucier du signe des fonctions,
%   on écrit formellement
%   $\ln f=\alpha_1 \ln g_1+\alpha_2 \ln g_2+\cdots+\alpha_n \ln g_n$.
%   En dérivant cette relation, on obtient
%   \[\frac{f'}{f}=\alpha_1 \frac{g_1'}{g_1}+\alpha_2 \frac{g_2'}{g_2}+\cdots+
%                  \alpha_n \frac{g_n'}{g_n}\]
%   ce qui permet de calculer symboliquement $f'$.
% \end{remarqueUnique}

\begin{exoUnique}
\exo Étudier les variations de la fonction $f$ définie sur $\R$ par
  \[\forall x\in\R \qsep f(x)\defeq\frac{x+2}{\sqrt[3]{1+x^2}}.\]
  \begin{sol}
  d'après les théorèmes usuels, elle est dérivable sur $\R$ et
  \[\forall x\in\R \quad f'(x)=
    \frac{\p{x-1}\p{x-3}}{3\p{1+x^2}^{\frac{4}{3}}}\]    
  \end{sol}
\end{exoUnique}

\begin{proposition}[utile=-3]
Soit $f$ une bijection continue d'un intervalle $I$ sur un intervalle $J$ et
$y_0\in J$. Si $f$ est dérivable en $x_0=f^{-1}\p{y_0}$, alors $f^{-1}$
est dérivable en $y_0$ si et seulement si $f'\p{x_0}\neq 0$. De plus, si tel
est le cas
\begin{eqnarray*}
\p{f^{-1}}'\p{y_0}&=&\frac{1}{f'\p{x_0}}\\
&=&\frac{1}{f'\p{f^{-1}(y_0)}}.
\end{eqnarray*}
\end{proposition}

\begin{preuve}
Soit $y_0\in J$ et $x_0=f^{-1}\p{y_0}$. On suppose que $f'(x_0)\neq 0$.
Pour $y\neq y_0$, par injectivité, on a aussi $f^{-1}(y)\neq f^{-1}\p{y_0}$. Ainsi, on peut écrire :
\begin{eqnarray*}
\frac{f^{-1}(y)-f^{-1}\p{y_0}}{y-y_0}&=&\frac{1}{\frac{y-y_0}{f^{-1}(y)-f^{-1}\p{y_0}}}\\
&=&\frac{1}{\frac{f(f^{-1}(y))-f(f^{-1}(y_0))}{f^{-1}(y)-f^{-1}\p{y_0}}}
\end{eqnarray*}
Or $$\frac{f(x)-f(x_0)}{x-x_0}\tendvers{x}{x_0}f'(x_0) \et f^{-1}(y)\tendvers{y}{y_0}f^{-1}(y_0)=x_0$$ car $f^{-1}$ est continue en $y$. Donc, par composition des limites, $$\frac{1}{\frac{f(f^{-1}(y))-f(f^{-1}(y_0))}{f^{-1}(y)-f^{-1}\p{y_0}}}\tendvers{y}{y_0}\frac{1}{f'\p{f^{-1}(y_0)}}=\frac{1}{f'\p{x_0}}.$$
\begin{victor}
Victor :Pour retrouver la formule : $\forall x \in I$, $f^{-1}(f(x))=x$ donc $f'(x)(f^{-1})'(f(x))=1$. Ainsi, $\forall u \in f(I)$, en appliquant cela en $x=f^{-1}(u)$ on obtient :
\[(f^{-1})'(u)=\frac{1}{f'(f^{-1}(u))}.\]
\end{victor}

\end{preuve}

% \begin{exos}
% \exo Montrer que La fonction $\arcsin$ est dérivable sur $\intero{-1}{1}$ et
%   $\arcsin' x=1/\sqrt{1-x^2}$. De plus, $\arcsin$ n'est dérivable ni en $-1$,
%   ni en $1$.
% \end{exos}


\begin{proposition}[utile=-3]
Soit $f:\dom\to\C$ et $x_0\in\mathcal{D}$.
\begin{itemize}
\item Alors $\conj{f}$ est dérivable en $x_0$ si et seulement si $f$ l'est. De
  plus, si tel est le cas
  \[\conj{f}\ '\p{x_0}=\conj{f'\p{x_0}}.\]
\item De même, $f$ est dérivable en $x_0$ si et seulement si $\Re(f)$ et
  $\Im(f)$ le sont. De plus, si tel est le cas
  \[f'\p{x_0}=\Re(f)'\p{x_0}+\ii\Im(f)'\p{x_0}.\]
\item Enfin, si $f$ est dérivable en $x_0$, il en est de même pour $\e^f$ et
  \[\p{\e^f}'\p{x_0}=f'\p{x_0}\e^{f\p{x_0}}\]
\end{itemize}
\end{proposition}

% \begin{exoUnique}
% \exo Calculer
%   \[\prim{\e^x\cos x}{x} \et \prim{(x+1)\sin x}{x}.\]
% \end{exoUnique}

%% Exemple :
%% 1) Si a\in\K, la fonction x -> exp(ax) est dérivable sur R et f'(x)=a exp(ax)


%% Remarque :
%% 1) On dit qu'une fonction f est dérivable sur A lorsque la restriction
%%    de f à A est dérivable.
%%    Par exemple la fonction f : x -> abs(x) est dérivable sur RP et RM.
%%    Pourtant f n'est pas dérivable en 0.
%% 2) Si f est dérivable sur un intervalle ouvert I, f est dérivable en tout point
%%    de I. 
%%
%% Exemple :
%% 1) Dérivabilité de la fonction f définie sur R par
%%    f(x) = 0 si x<=0 et f(x)=exp(-1/x) si x>0
%%    est dérivable sur R.

\subsection{Fonction dérivée, dérivées successives}

\begin{definition}[utile=-3]
Soit $f:\dom\to\K$ une fonction. On note $\mathcal{D}_{f'}$
l'ensemble des $x\in\mathcal{D}$ en lesquels $f$ est dérivable.
On définit la \emph{fonction dérivée} de $f$, notée
$f'$, par
\[\dspappli{f'}{\mathcal{D}_{f'}}{\K}{x}{f'(x)}.\]
\end{definition}

%% Remarques :
%% 1) Si f est T-périodique, il en est de même pour f'.
%%    Si f est paire, f' est impaire
%%    Si f est impaire, f' est paire
%% 2) Attention. Les théorèmes usuels permettent de trouver un ensemble A tel
%%    que A \subset D_f'. Mais A n'est pas forcément égal à D_f'.
%%    Par exemple, si f=sqrt(1-cos x) sur [0,Pi/2], f est dérivable sur
%%    ]0,Pi/2] d'après les théorèmes usuels. Mais f est aussi dérivable en 0

\begin{definition}[utile=-3]
Si $f:\dom\to\K$ est une fonction, on définit par récurrence
la \emph{dérivée $n$-ième} de $f$ de la manière suivante~:
\begin{itemize}
\item On pose $f^{(0)}\defeq f$.
\item Si $n\in\N$, on définit $f^{(n+1)}$ comme étant la dérivée  de $f^{(n)}$.
\end{itemize}
Si $x_0\in\mathcal{D}$, on dit que $f$ est dérivable $n$ fois en $x_0$
lorsque $f^{(n)}$ est définie en $x_0$; cette notion est locale en $x_0$.
\end{definition}

\begin{remarques}
\remarque On dit qu'une fonction est dérivable $n$ fois lorsqu'elle est
  dérivable $n$ fois en tout point de son domaine de définition.
\remarque Soit $f:\dom\to\K$, et $A\subset\dom$. On dit que $f$ est
  dérivable $n$ fois sur $A$ lorsque la restriction de $f$ à $A$ est dérivable
  $n$ fois.
% \remarque Soit $n\in\N$ et $k\in\N$. Alors, la fonction $f$ d'expression
%   $x^n$ est dérivable $k$ fois sur $\R$ et
%   \[\forall x\in\R \quad f^{(k)}(x)=
%     \begin{cases}
%     \frac{n!}{\p{n-k}!}x^{n-k} & \text{si $k\leq n$}\\
%     0 & \text{sinon}
%     \end{cases}\]
% \remarque Quel que soit $n\in\N$, la fonction $\sin$ est dérivable $n$ fois sur
%   $\R$ et
%   \[\forall n\in\N \quad \forall x\in\R \quad \sin^{(n)}(x)=
%     \sin\p{x+n\frac{\pi}{2}}\]
\end{remarques}

\begin{proposition}[utile=-3]
Soit $f, g:\dom\to\K$ deux fonctions dérivables $n$ fois.
\begin{itemize}
\item Soit $\lambda,\mu\in\K$. Alors $\lambda f+\mu g$ est dérivable $n$
  fois et
  \[\forall x\in\mathcal{D} \qsep
    \p{\lambda f+\mu g}^{(n)}(x)=\lambda f^{(n)}(x)+\mu g^{(n)}(x).\]
\item $fg$ est dérivable $n$ fois et
  \[\forall x\in \mathcal{D} \qsep
    \p{fg}^{(n)}(x)=\sum_{k=0}^n \binom{n}{k} f^{(k)}(x)g^{(n-k)}(x).\]
  Cette formule est appelée formule de \nom{Leibniz}.
\item Si $g$ ne s'annule pas, alors $f/g$ est dérivable $n$ fois.
\end{itemize}
\end{proposition}

\begin{preuve}
\begin{itemize}
\item Se montre aisément par récurrence.
\item Démonstration par récurrence similaire au binôme de Newton.
\item Soit $\mathcal{D}$ une partie de $\R$. On montre $\mathcal{H}_n$ : "Si $f$ et $g$ sont dérivables $n$ fois sur $\mathcal{D}$ et $g$ ne s'annule pas sur $\mathcal{D}$, alors $f/g$ est dérivable $n$ fois sur $\mathcal{D}$".
Pour passer à l'hérédité, on applique $\mathcal{H}_n$ à $f'g-g'f$ et $g^2$, leur rapport est dérivable $n$ fois donc $(f/g)'$ est dérivable $n$ fois.
\end{itemize}

\end{preuve}

\begin{proposition}[utile=-3]
Si $f$ et $g$ sont deux fonctions dérivables $n$ fois, alors $g\circ f$ est
dérivable $n$ fois.  
\end{proposition}

\begin{preuve}
Même méthode que pour le quotient.
\end{preuve}

\begin{remarqueUnique}
\remarque Si $a\in\R$ et $f:\R\to\K$ est dérivable $n$ fois, alors la fonction $g:x\mapsto f\p{ax}$
  est dérivable $n$ fois et
  \[\forall x\in\R \qsep g^{(n)}(x)=a^n f^{(n)}\p{ax}.\]
\end{remarqueUnique}

\begin{exos}
\exo %Soit $f$ une fonction dérivable $n$ fois sur $\R$.
  Donner la dérivée $n$-ième des fonctions $x\mapsto x^2f(x)$ et
  $x\mapsto f(x)\e^x$.
\begin{sol}
On utilise la formule de Leibniz. 
Pour la première $$g^{(n)}(x)=x^2f^{(n)}(x)+2nf^{(n-1)}(x)+n(n-1)f^{(n-2)}(x).$$
\end{sol}
\exo Calculer la dérivée $n$-ième de la fonction $x\mapsto \cos^3 x$.
  \begin{sol}
  On trouve grâce à la linéarisation du cos :
  \[\forall x\in\R \quad
    f^{(n)}(x)=\frac{3}{4}\cos\p{x+n\frac{\pi}{2}}+
                  \frac{3^n}{4}\cos\p{3x+n\frac{\pi}{2}}\]  
  \end{sol}
\exo Soit $f$ la fonction de $\R$ dans $\R$ définie par
  \[\forall x\in\R \qsep f(x)\defeq\e^{-x^2}.\]
  Montrer que pour tout $n\in\N$, $f^{(n)}$ est bornée.
  \begin{sol}
    On montre par récurrence sur $n$ que $f^{(n)}(x)=P_n(x)e^{-x^2}$, puis limites en $\pm \infty$ plus continue implique borné.
    
  Pour prouver $x^ke^{-x^2}\tendvers{x}{+\infty}0$, on se ramène au "vrai" théorème de croissance comparées en posant pour $x\geq 0$, $u=x^2$, cela donne $u^{k/2}e^{-u}\tendvers{u}{+\infty}0$. 
    \end{sol}
% \exo Soit $f$ la fonction définie sur $\intero{-1}{1}$ par
%   \[\forall x\in\intero{-1}{1} \quad f(x)=\frac{1}{\sqrt{1-x^2}}\]
%   Montrer que pour tout $n\in\N$, $f$ est dérivable $n$ fois sur
%   $\intero{-1}{1}$ et qu'il existe un unique polynôme $P_n\in\polyR$ tel que
%   \[\forall x\in\intero{-1}{1} \quad
%     f^{(n)}(x)=\frac{P_n(x)}{\p{1-x^2}^{\frac{2n+1}{2}}}\]
%   Donner une relation de récurrence liant $P_{n+1}$ et $P_n$. En déduire le
%   degré et le coefficient dominant de $P_n$.
%   \begin{sol}
%   De plus
%   \[\forall n\in\N \quad P_{n+1}=\p{1-X^2}P_n'+\p{2n+1}XP_n\]
%   En particulier, $P_n$ est de degré $n$ et son coefficient dominant est $n!$.    
%   \end{sol}
\end{exos}

\begin{proposition}[utile=-3]
Soit $n\in\Ns$ et $f$ une bijection de l'intervalle $I$ sur l'intervalle
$J$, dérivable $n$ fois sur $I$. Alors $f^{-1}$ est dérivable $n$ fois sur $J$ si et seulement
si
\[\forall x\in I \qsep f'(x)\neq 0.\]
\end{proposition}

\begin{preuve}
  \begin{itemize}
  \item[$\bullet$] Sens gauche-droite : Comme $n\geq 1$, en particulier $f^{-1}$ est dérivable une fois et en dérivant l'identité $f^{-1}(f(x))=x, \forall x \in I$, on tire $\forall x \in I, f'(x)\p{f^{-1}}'(f(x))=1$ donc $\forall x \in I, f'(x)\neq 0$.
  \item[$\bullet$] Sens droite-gauche : Par récurrence sur $n$. Pour l'hérédité, si $f$ est dérivable $n+1$ fois, on utilise la dérivée première $$\p{f^{-1}}'(y)=\frac{1}{f'\p{f^{-1}(y)}}$$ où pour le dénominateur, $f^{-1}$ est dérivable $n$ fois par H.R, $f'$ est dérivable $n$ fois par hypothèse et la composée est dérivable $n$ fois par la proposition précédente. 
  \end{itemize}
  \end{preuve}


\subsection{Fonctions de classe $\classec{n}$}

\begin{definition}[utile=-3]
Soit $n\in\N$. On dit qu'une fonction $f:\dom\to\K$ est de classe $\classec{n}$
lorsqu'elle est dérivable $n$ fois et sa dérivée $n$-ième est continue. On note
$\classec{n}\p{\mathcal{D},\K}$  l'ensemble des fonctions de $\mathcal{D}$
dans $\K$ de classe $\classec{n}$.
\end{definition}

\begin{remarques}
\remarque Les fonctions de classe $\classec{0}$ sont les fonctions continues.
\remarque Une fonction peut être dérivable sur $\R$ sans que sa dérivée soit
  continue. Par exemple la fonction $f$ définie sur $\R$ par
  \[\forall x\in\R \qsep f(x)\defeq
    \begin{cases}
    x^2\sin\p{\frac{1}{x}} & \text{si $x\neq 0$}\\
    0 & \text{si $x=0$}
    \end{cases}\]
  est dérivable sur $\R$ mais sa dérivée n'est pas continue en 0.
\remarque Si on note $\classed{n}$ l'ensemble des fonctions dérivables $n$
  fois, on a
  \[\classec{0} \supset \classed{1} \supset \classec{1} \supset \classed{2}
    \supset \classec{2} \cdots \]
  On peut montrer que toutes ces inclusions sont strictes.
\remarque Si $A$ est une partie de $\mathcal{D}$, on dit que $f$ est de
  classe $\classec{n}$ sur $A$ lorsque la restriction de $f$ à $A$
  est de classe $\classec{n}$.
\end{remarques}

\begin{exoUnique}
\exo Soit $f:\R\to\R$ la fonction définie par
  \[\forall x\in\R\qsep f(x)\defeq\begin{cases}\frac{\sin x}{x} & \text{si $x\neq 0$}\\1&\text{si $x=0$.}\end{cases}\]
  Montrer que $f$ est de classe $\classec{1}$ sur $\R$.
\end{exoUnique}

\begin{definition}[utile=-3]
On dit qu'une fonction $f:\dom\to\K$ est de classe $\classec{\infty}$ lorsqu'elle est de
classe $\classec{n}$ pour tout $n\in\N$.
\end{definition}

\begin{remarques}
\remarque Une fonction est de classe $\classec{\infty}$ si et seulement si elle
  est dérivable $n$ fois quel que soit $n\in\N$.
\remarque Les fonctions usuelles sont de classe $\classec{\infty}$
  sur le domaine sur lequel elle sont dérivables.
  % \begin{enumerate}
  % \item La fonction valeur absolue qui l'est sur $\RP$ et $\RM$ mais pas sur
  %   $\R$.
  % \item Les fonctions $x\mapsto\sqrt[n]{x}$ qui ne le sont que sur $\RPs$.
  % \item Les fonctions $\arcsin$ et $\arccos$ qui ne le sont que sur
  %   $\intero{-1}{1}$.
  % \item La fonction $\argch$ qui ne l'est que sur $\intero{1}{+\infty}$.
  % \end{enumerate}
\end{remarques}


\begin{proposition}[utile=-3]
Soit $n\in\N\cup\ens{\infty}$ et $f, g:\dom\to\K$ deux fonctions de classe $\classec{n}$.
\begin{itemize}
\item Si $\lambda,\mu\in\K$, alors $\lambda f+\mu g$ est de classe
  $\classec{n}$.
\item $fg$ est de classe $\classec{n}$.
\item Si $g$ ne s'annule pas, alors $f/g$ est de classe $\classec{n}$.
\end{itemize}
\end{proposition}

\begin{proposition}[utile=-3]
Si $n\in\N\cup\ens{\infty}$, la composée de deux fonctions de classe $\classec{n}$ est de classe
$\classec{n}$.
\end{proposition}

% \begin{exos}
% \exo Soit $f$ la fonction définie sur $\R$ par~:
%   \[\forall x\in\R \quad f(x)=
%     \begin{cases}
%     e^x & \text{si $x<0$}\\
%     ax^2+bx+c & \text{si $x\geq 0$}
%     \end{cases}\]
%     À quelles conditions sur $a,b,c\in\R$ $f$ est-elle de classe $\classec{1}$~?
%     de classe $\classec{2}$~?
% \exo Soit $f$ la fonction définie sur $\R$ par
%   \[\forall x\in\R \quad f(x)=
%     \begin{cases}
%     \frac{\sin x}{x} & \text{si $x\neq 0$}\\
%     1 & \text{si $x=0$}
%     \end{cases}\]
%    Alors $f$ est de classe $\classec{1}$ sur $\R$.
% \exo Soit $f$ la fonction définie sur $\R$ par
%   \[\forall x\in\R \quad f(x)=
%     \begin{cases}
%     0 & \text{si $x\leq 0$}\\
%     e^{-\frac{1}{x}} & \text{si $x>0$}
%     \end{cases}\]
%   Alors $f$ est de classe $\classec{\infty}$ sur $\R$.
% \exo Soit $y$ une solution de l'équation différentielle
%   \[\forall t\in\R \quad y''(t)+\sin\p{y(t)}=0\]
%   Alors $y$ est de classe $\classec{\infty}$ sur $\R$.
% \end{exos}

\begin{definition}[utile=-3]
Soit $f$ une bijection de l'intervalle $I$ sur l'intervalle $J$ et
$n\in\N\cup\ens{\infty}$. On dit que $f$ est un $\classec{n}$-\emph{difféomorphisme}
de $I$ sur $J$ lorsque $f$ et $f^{-1}$ sont de classe $\classec{n}$.
\end{definition}

% \begin{exos}
% \exo La fonction $\exp$ est un $\classec{\infty}$-difféomorphisme de
%   $\R$ sur $\RPs$.
% \end{exos}

\begin{proposition}[utile=-3]
Soit $n\in\Ns\cup\ens{\infty}$. Une bijection $f$ de classe $\classec{n}$ de
l'intervalle $I$ sur l'intervalle $J$ est un $\classec{n}$-difféomorphisme si
et seulement si $f'$ ne s'annule pas.
\end{proposition}

\begin{exos}
\exo Soit $f$ l'application de $\RP$ dans $\RP$ qui à $x$ associe $x\e^x$. Montrer que $f$ est un
  $\classec{\infty}$-difféomorphisme. Calculer un développement limité de $f^{-1}$ en 0 à l'ordre 3.
  % et
  %calculer le développement limité de $f^{-1}$ en 0 à l'ordre 3.
  \begin{sol}
  Alors $f$ est une bijection de classe $\classec{\infty}$.
  Sa dérivée ne s'annulant pas, on en déduit que $f^{-1}$ est un
  $\classec{\infty}$-difféomorphisme. En particulier,  $f^{-1}$ admet un
  développement limité à tout ordre en 0. On trouve
  \[f^{-1}(x)=x-x^2+\frac{3}{2}x^3+\petitozero{x}{x^3}\]    
  \end{sol}
\exo Montrer qu'il existe une unique fonction $f\in\classec{\infty}(\R,\R)$
  telle que
  \[\forall x\in\R \qsep f^5(x)+f(x)+x=0.\]
  \begin{sol}
    On introduit $\phi:\R\mapsto\R$ définie par $\phi(x)=-x^5-x$. On montre que $\phi$ est bijective et $\classec{\infty}$. L'équation se réécrit $\phi(f(x))=x \forall x$. $g=\phi^{-1}$ est donc l'unique solution.
    \end{sol}
\end{exos}



\section{Théorème de \nom{Rolle} et applications}
\subsection{Extrémum local}
\begin{definition}[utile=-3]
Soit $f:\dom\to\R$ une fonction réelle et $x_0\in\mathcal{D}$. On dit que
\begin{itemize}
\item $f$ présente un \emph{maximum global} en $x_0$ lorsque
  \[\forall x\in\mathcal{D} \qsep f(x)\leq f\p{x_0}.\]
\item $f$ présente un \emph{maximum local} en $x_0$ lorsque
  \[\exists \eta>0 \qsep \forall x\in\mathcal{D} \qsep
    \abs{x-x_0}\leq\eta \implique f(x)\leq f\p{x_0}.\]
\item $f$ présente un \emph{minimum global} en $x_0$ lorsque
  \[\forall x\in\mathcal{D} \qsep f(x)\geq f\p{x_0}.\]
\item $f$ présente un \emph{minimum local} en $x_0$ lorsque
  \[\exists \eta>0 \qsep \forall x\in\mathcal{D} \qsep
    \abs{x-x_0}\leq\eta \implique f(x)\geq f\p{x_0}.\]
\end{itemize}
\end{definition}

\begin{remarqueUnique}
\remarque On peut indifféremment utiliser \og maximums \fg ou \og maximas \fg
  pour le pluriel de \og maximum \fg. De même, on peut utiliser \og minimums \fg
  ou \og minimas \fg pour le pluriel de \og minimum \fg.
  %  Cependant, le pluriel d' \og extrémum \fg
  % est \og extrémums \fg.
\end{remarqueUnique}

\begin{proposition}[utile=-3]
Soit $f:\dom\to\R$ une fonction admettant un extrémum local en un point $x_0$ intérieur à
$\mathcal{D}$. Si elle est dérivable en $x_0$, alors $f'\p{x_0}=0$.
\end{proposition}

\begin{preuve}
On fait ici la preuve dans le cas d'un maximum local.\\
Comme $x_0$ est intérieur, on peut fixer $\eta_1>0$ tel que $\interf{x_0-\eta_1}{x_0+\eta_1} \in \mathcal{D}$. Comme $f$ admet un maximum local en $x_0$, il existe $\eta_2>0$ tel que $\forall x\in \mathcal{D}, |x-x_0|\leq \eta_2 \Longrightarrow f(x)\leq f(x_0)$.\\
Avec $\eta=\min(\eta_1,\eta_2)$, on a donc : $\forall x \in \interf{x_0-\eta}{x_0+\eta}, f(x)\leq f(x_0)$.\\

Ainsi, $\forall x \in \interof{x_0}{x_0+\eta}, \dfrac{f(x)-f(x_0)}{x-x_0}\leq 0$ d'où par passage à la limite $f'(x_0)=f_d'(x_0)\leq 0$.\\
Le même raisonnement à gauche de $x_0$ donne cette fois $f'(x_0)\geq 0$.

Ainsi, $f'(x_0)=0$.
\end{preuve}

\begin{remarques}
\remarque Attention, ce n'est pas parce que $f'$ s'annule en $x_0$ que $f$
  y admet un extrémum local. Par exemple la fonction $x\mapsto x^3$
  a une dérivée qui s'annule en 0 mais n'admet pas d'extrémum local en ce point.
\remarque Les extrémums locaux d'une fonction $f$ définie sur $\mathcal{D}$ sont
  donc à chercher parmi les bornes de $\mathcal{D}$, les points où $f$ n'est
  pas dérivable et ceux où la dérivée de $f$ est nulle.
\remarque Si $f'(x_0)=0$ et $f$ est assez régulière, un développement limité permet généralement de déterminer
  si $f$ admet un maximum ou un minimum local en $x_0$. En effet, supposons que
  \[f(x_0+h)=f(x_0)+\alpha h^\omega+\petitozero{h}{h^\omega}\]
  avec $\alpha\in\Rs$ et $\omega\geq 2$.
  \begin{itemize}
  \item Supposons que $\omega$ est pair. Si $\alpha>0$, alors $f$ admet un minimum local en $x_0$. Si $\alpha<0$, alors $f$ admet un maximum local en $x_0$.
\medskip
\pgfplotsset{
    standard/.style={
        axis x line=middle,
        axis y line=middle,
        enlarge x limits=0.15,
        enlarge y limits=0.15,
        every axis x label/.style={at={(current axis.right of origin)},anchor=north west},
        every axis y label/.style={at={(current axis.above origin)},anchor=north east}
    }
}
\begin{center}
\begin{figure}[H]
\begin{center}
\begin{tikzpicture}[>=latex,scale=1.0]
%  \draw[->] (-0.2,0) -- (2,0) node[below] {$x$};
%  \draw[->] (0.0,-0.2) -- (0.0,2.5) node[left] {$y$};
%  \draw[-] (0.1,0.1) -- (2,2);
\begin{axis}[standard,mark=none,
      xmin=-0.1,ymin=-0.1,
      xmax=2.0,ymax=2.0,
      axis lines*=middle,
      axis line style={->},
      xlabel=$x$,
      ylabel=$y$,
      xtick={1.0},xticklabels={$x_0$},
      ytick={1.0},yticklabels={$f(x_0)$},
      enlargelimits]
\addplot[no marks,domain=0.1:2.0,samples=301] {1.0};
\addplot[line width=2pt,no marks,domain=0.1:2.0,samples=301] {1.0+(x - 1.0)^2};
 \draw[shift={(0,0)}] (1.0,2pt) -- (1.0,-2pt) node[below] {$a$};
 %{1.0*(x - 1.0)^2};
\end{axis}
\end{tikzpicture}
$\qquad$
\begin{tikzpicture}[>=latex,scale=1.0]
%  \draw[->] (-0.2,0) -- (2,0) node[below] {$x$};
%  \draw[->] (0.0,-0.2) -- (0.0,2.5) node[left] {$y$};
%  \draw[-] (0.1,0.1) -- (2,2);
\begin{axis}[standard,mark=none,
      xmin=-0.1,ymin=-0.1,
      xmax=2.0,ymax=2.0,
      axis lines*=middle,
      axis line style={->},
      xlabel=$x$,
      ylabel=$y$,
      xtick={1.0},xticklabels={$x_0$},
      ytick={1.0},yticklabels={$f(x_0)$},
      enlargelimits]
\addplot[no marks,domain=0.1:2.0,samples=301] {1.0};
\addplot[line width=2pt,no marks,domain=0.1:2.0,samples=301] {1.0-(x - 1.0)^2};
 %{+1.0*(x - 1.0)^2};
\end{axis}
\end{tikzpicture}
\end{center}
\captionsetup{labelformat=empty}
\caption{$\alpha>0$ et $\omega$ pair $\qquad\qquad\qquad\ \qquad\qquad\qquad\qquad\qquad$ $\alpha<0$ et $\omega$ pair}
\end{figure}
 \begin{figure}[H]
 \centering

 \captionsetup{labelformat=empty}
 \caption{$\alpha<0$ et $\omega$ pair}
 \end{figure}
\end{center}
\medskip
  \item Si $\omega$ est impair et que $x_0$ est intérieur à $\mathcal{D}$, $f$ n'admet pas d'extrémum local en $x_0$.
\medskip
\begin{center}
\begin{figure}[H]
\centering
\begin{tikzpicture}[>=latex,scale=1.0]
%  \draw[->] (-0.2,0) -- (2,0) node[below] {$x$};
%  \draw[->] (0.0,-0.2) -- (0.0,2.5) node[left] {$y$};
%  \draw[-] (0.1,0.1) -- (2,2);
\begin{axis}[standard,mark=none,
      xmin=-0.1,ymin=-0.1,
      xmax=2.0,ymax=2.0,
      axis lines*=middle,
      axis line style={->},
      xlabel=$x$,
      ylabel=$y$,
      xtick={1.0},xticklabels={$x_0$},
      ytick={1.0},yticklabels={$f(x_0)$},
      enlargelimits]
\addplot[no marks,domain=0.1:2.0,samples=301] {1.0};
\addplot[line width=2pt,no marks,domain=0.1:2.0,samples=301] {1.0+(x - 1.0)^3};
% +1.0*(x - 1.0)^2};
\end{axis}
\end{tikzpicture}
\captionsetup{labelformat=empty}
\caption{$\alpha>0$ et $\omega$ impair}
\end{figure}
\end{center}
  \end{itemize}
\end{remarques}

\begin{exoUnique}
\exo Rechercher les extrémums de la fonction d'expression $\abs{x\p{x-1}}$
  sur $\interf{0}{2}$.
  \begin{sol}
  On étudie la fonction sans les valeurs absolues, on dresse son tableau de variations et on observe dessus son signe pour passer au tableau de variations de sa valeur absolue. On y voit donc un minimum global en $0$ et en $1$ ainsi qu'un maximum global en $2$ et un maximum local en $1/2$.
  \end{sol}
\end{exoUnique}

\subsection{Théorème de \nom{Rolle}, accroissements finis}

Dans cette section, lorsqu'on considèrera un segment $\interf{a}{b}$, on supposera que $a<b$.

\begin{theoreme}[nom={Théorème de \nom{Rolle}}]
Soit $f$ une fonction réelle continue sur $\interf{a}{b}$, dérivable sur
$\intero{a}{b}$, telle que $f(a)=f(b)$. Alors il existe $c\in\intero{a}{b}$
tel que $f'(c)=0$.
\end{theoreme}

\begin{preuve}
Il nous faut chercher un extremum local atteint en un point de $[a,b]$. Comme $f$ y est continue, elle est bornée et atteint ses bornes. On peut donc fixer $(\alpha,\beta) \in [a,b]^2$ tels que $\forall x \in \interf{a}{b}, f(\alpha)\leq f(x)\leq f(\beta)$.
\begin{itemize}
\item[$\bullet$] 1er cas : $f(\alpha)=f(\beta)$ : $f$ est alors constante et on a l'embarras du choix.
\item[$\bullet$] 2ème cas : $f(\alpha)\neq f(\beta)$. Comme $f(a)=f(b)$, l'un au moins des points $\alpha$ et $\beta$ est élément de $\intero{a}{b}$. Notons $c$ un tel point. On a alors d'après la proposition précédente $f'(c)=0$.
\end{itemize}
\end{preuve}

\smallskip
\begin{center}
\begin{pdfpic}
\readdata{\listeP}{graph/graphe_rolle.txt}
\psset{xunit=3cm,yunit=2.5cm}
\begin{pspicture}(-0.2,-0.2)(3,2)
\psaxes[labels=none,ticks=none]{->}(0,0)(-0.2,-0.2)(3,2)
\dataplot[plotstyle=curve,linewidth=2pt]{\listeP}
\uput[r](3,0){$x$}
\uput[l](0,2){$y$}
\uput[d](0.5,0){$a$}
\uput[d](2.5,0){$b$}
\uput[d](2.2746,0){$c$}
\psline[linestyle=dashed](0.5,0)(0.5,1)
\psline[linestyle=dashed](2.5,0)(2.5,1)
\psline[linestyle=dashed](0,1)(2.5,1)
\psline[linestyle=dashed](2.2746,0)(2.2746,1.5707)
\psline{<->}(1.9746,1.5707)(2.5746,1.5707)
\end{pspicture}
\end{pdfpic}
\end{center}

\begin{exoUnique}
\exo Soit $f$ une fonction dérivable $n$ fois sur l'intervalle $I$
  admettant $n+1$ zéros. Montrer que $f^{(n)}$ s'annule au moins une fois sur
  $I$. Retrouver le fait qu'un polynôme $P\in\polyR$ de degré $n$ admet au plus
  $n$ racines réelles.
 % \exo Soit $f$ une fonction dérivable sur $\R$ telle que
%   \[f(x)\tendvers{x}{-\infty}0 \et f(x)\tendvers{x}{+\infty}0\]
%   Alors il existe $c\in\R$ tel que $f'(c)=0$.
\end{exoUnique}

\begin{remarqueUnique}
\remarque Cette proposition est fausse si $f$ est à valeurs complexes.
  Par exemple, si $f$ est la fonction définie sur $\R$ par
  \[\forall x\in\R \qsep f(x)\defeq\e^{\ii x}\]
  alors $f$ est dérivable sur $\R$, $f(0)=f(2\pi)$ mais $f'$ ne s'annule
  pas.
\end{remarqueUnique}

\begin{theoreme}[nom={Théorème des accroissements finis}]
Soit $f$ une fonction réelle continue sur $\interf{a}{b}$ et dérivable sur
$\intero{a}{b}$. Alors, il existe $c\in\intero{a}{b}$ tel que
\[\frac{f(b)-f(a)}{b-a}=f'(c).\]
\end{theoreme}

\begin{preuve}
On applique Rolle à $h(x)=f(x)-\p{f(a)+\frac{f(b)-f(a)}{b-a}(x-a)}$. On a enlevé à $f$ la fonction affine coïncidant avec $f$ en $a$ et en $b$.
\end{preuve}

\smallskip
\begin{center}
\begin{pdfpic}
\readdata{\listeP}{graph/graphe_af.txt}
\psset{xunit=1.1cm,yunit=0.65cm}
\begin{pspicture}(-1.5,-2)(7,7)
\psaxes[labels=none,ticks=none]{->}(-1,-1)(-1.5,-2)(7,7)
\dataplot[plotstyle=curve,linewidth=2pt]{\listeP}
\uput[r](7,-1){$x$}
\uput[l](-1,7){$y$}
\uput[d](0,-1){$a$}
\uput[d](6.2831,-1){$b$}
\uput[d](1.5708,-1){$c$}
\psline[linestyle=dashed](1.5708,-1)(1.5708,4.5708)
\psline[linestyle=dashed](0,-1)(0,0)
\psline[linestyle=dashed](6.2831,-1)(6.2831,6.2831)
\psline(0,0)(6.2831,6.2831)
\psline{<->}(0.8708,3.8708)(2.2708,5.2708)
\end{pspicture}
\end{pdfpic}
\end{center}

\begin{remarqueUnique}
\remarque Remarquons que le taux d'accroissement $(f(b)-f(a))/(b-a)$ est une grandeur invariante par échange de $a$ et $b$.
  Par conséquent, si $f$ est dérivable sur $I$, quels que soient $a,b\in I$ tels que $a\neq b$, il existe
  $c$ strictement compris entre $a$ et $b$ tel que
  \[\frac{f(b)-f(a)}{b-a}=f'(c).\]
\remarque Puisque $c$ est strictement compris entre $a$ et $b$, il arrive qu'on l'écrive sous la forme
  \[c=\theta a+(1-\theta) b\]
  avec $\theta\in\intero{0}{1}$.
\end{remarqueUnique}

\begin{proposition}[utile=-3,nom={Inégalité des accroissements finis}]
Soit $f$ une fonction réelle continue sur $\interf{a}{b}$ et
dérivable sur $\intero{a}{b}$. On suppose qu'il existe $m,M\in\R$ tels que
\[\forall x\in\intero{a}{b} \qsep m\leq f'(x)\leq M.\]
 Alors
\[m\p{b-a}\leq f(b)-f(a)\leq M\p{b-a}.\]
\end{proposition}

\begin{preuve}
On applique le TAF.
\end{preuve}

\begin{proposition}[nom={Inégalité des accroissements finis}]
Soit $f:I\to\K$ une fonction dérivable sur un intervalle $I$.
On suppose qu'il existe $M\in\RP$ tel que
\[\forall x\in I \qsep \abs{f'(x)}\leq M.\]
Alors
\[\forall x, y\in I\qsep \abs{f(x)-f(y)}\leq M\abs{x-y}.\]
Autrement dit, $f$ est $M$-lipschitzienne.
\end{proposition}

\begin{preuve}
  Si $x=y$, ok. Sinon, on applique le TAF sur $"[x,y]"$.
  
  Pour le cas complexe, on peut le faire dans le cas où $f$ est $\classec{1}$ avec le théorème fondamental de l'analyse. Le cas général est hors de porté.
  \end{preuve}

\begin{remarqueUnique}
\remarque Une fonction de classe $\classec{1}$ sur un segment est
  lipschitzienne.
% \remarque On dit qu'une fonction $f$ est contractante lorsqu'elle est
%   $M$-Lipschitzienne avec $M<1$. Si $f$ est contractante sur un intervalle $I$
%   qu'elle laisse stable et si elle admet un point fixe $x_0\in I$, alors,
%   quel que soit $u_0\in I$, la suite $\p{u_n}$ définie par
%   \[\forall n\in\N \qsep u_{n+1}\defeq f\p{u_n}\]
%   converge vers $x_0$.
\end{remarqueUnique}

% \begin{exoUnique}
% \exo Soit $\alpha\in\R$. Démontrer que la suite $(u_n)$ définie par
%   \[u_0=\alpha \et \forall n\in\N \qsep u_{n+1}\defeq\cos(u_n)\]
%   est convergente.
% \end{exoUnique}

\subsection{Dérivation et monotonie}

\begin{proposition}[utile=-3]
Soit $f:I\to\R$ une fonction réelle, dérivable sur un intervalle $I$. Alors
\begin{itemize}
\item $f$ est croissante si et seulement si
  \[\forall x\in I \qsep f'(x)\geq 0.\]
\item $f$ est décroissante si et seulement si
  \[\forall x\in I \qsep f'(x)\leq 0.\]
\end{itemize}
\end{proposition}

\begin{preuve}
\begin{itemize}
\item 
\begin{itemize}
\item[$\bullet$] On suppose que $f$ est croissante. Soit $x\in I$. Si $x$ n'est pas la borne supérieure de $I$, on regarde le taux d'accroissement pour $y$ à droite de $I$. Grâce à la croissance de $f$, on a $\dfrac{f(y)-f(x)}{y-x}\geq 0$ d'où le résultat par passage à la limite.
Si $x$ est la borne supérieure de $I$, alors on regarde le taux d'accroissement en venant de la gauche pour obtenir le même résultat.
\item[$\bullet$] On suppose que 
  \[\forall x\in I \qsep f'(x)\geq 0.\]
  Soient $x$ et $y$ dans $I$ tels que $x<y$. D'après le TAF, il existe $c\in \intero{x}{y}$ tel que $$f(y)-f(x)=\underbrace{(y-x)}_{>0}\underbrace{f'(c)}_{\geq 0}\geq 0.$$
\end{itemize}
\item $f$ est décroissante si et seulement si $-f$ est croissante si et seulement si \[\forall x\in I \qsep (-f)'(x)\geq 0,\] si et seulement si
  \[\forall x\in I \qsep f'(x)\leq 0.\]
\item $f$ est constante si et seulement si $f$ est croissante et décroissante si et seulement si \[\forall x\in I \qsep 0\leq f'(x)\leq 0\] si et seulement si
  \[\forall x\in I \qsep f'(x)=0.\]
\end{itemize}
\end{preuve}

\begin{remarqueUnique}
\remarque Ce théorème reste vrai lorsque $f$ est continue sur $I$, dérivable sur $I$
  et que sa dérivée est de signe constant sur l'intérieur de $I$.
% \remarque Pour montrer que $f$ est croissante sur $I$, il suffit de montrer
%   qu'elle est continue sur $I$, dérivable sur l'intérieur de $I$ et que
%   $f'$ est positif sur cet intérieur. Par exemple, la fonction $f$
%   définie sur $\RP$ par
%   \[\forall x\geq 0 \qsep f(x)\defeq\sqrt{x}\e^{-x}\]
%   est croissante sur $\interf{0}{1/2}$.
%   \begin{sol}
%   D'après les TOC, $f$ est dérivable sur $\RPs$ et $\forall x>0$ :
%   $$f'(x)=\dfrac{1-2x}{2x}\sqrt{x}\e^{-x}.$$
%   $f'(x)\geq 0 \Longleftrightarrow x\leq 1/2$
%   En revanche, $f$ n'est pas dérivable en $0$ (pas parce-que les TOC le disent) parce-que $$\frac{f(x)-f(0)}{x}=\frac{\e^{-x}}{\sqrt{x}}\tendvers{x}{0}+\infty.$$
%   Le théorème précédent nous dit que $f$ est croissante sur $\interof{0}{1/2}$ mais grâce à la continuité en $0$, $f$ est croissante sur $\interf{0}{1/2}$.
%   \end{sol}
% \remarque Le dernier point de cette proposition reste vrai lorsque $f$ est
%   une fonction à valeurs complexes.
%   \begin{sol}
%   Soit $f:I\mapsto \K$ une fonction dérivable.
%   \begin{eqnarray*}
%   f \text{ est constante } &\Longleftrightarrow & \Re(f) \et \Im(f) \text{ le sont}\\
%   &\Longleftrightarrow & \forall x \in I, (\Re(f))'(x)=0 \et \forall x \in I, (\Im(f))'(x)=0\\
%   &\Longleftrightarrow & \forall x \in I, f'(x)=0.
%   \end{eqnarray*}
%   \end{sol}
\end{remarqueUnique}

%% Remarque :
%% 1) Ce théorème n'est vrai que sur un intervalle. Par exemple :
%%    - La fonction f définie sur Rs par f(x)=0 si x<0 et f(x)=1 si x>0
%%      est dérivable sur Rs de dérivée nulle. Pourtant f n'est pas constante
%%    - La fonction f définie sur Rs par f(x)=1/x est dérivable de dérivée
%%      négative mais n'est pas décroissante sur Rs. Pourtant, elle est
%%      décroissante sur RPs et sur RMs
%% 2) On remarque que dans la démonstration, il suffit que f soit continue
%%    sur [a,b] et dérivable sur ]a,b[ de dérivée positive pour en déduire que
%%    f est croissante sur [a,b].

\begin{exos}
\exo Étudier les variations de la fonction $f$ définie sur $\RP$ par
  \[\forall x\geq 0 \qsep f(x)\defeq\sqrt{x}\e^{-x}.\]
\exo Calculer
  \[\inf_{x,y>0} \frac{\sqrt{x+y}}{\sqrt{x}+\sqrt{y}}.\]
  \begin{sol}
  $$\frac{\sqrt{x+y}}{\sqrt{x}+\sqrt{y}}=\frac{\sqrt{1+\frac{y}{x}}}{1+\sqrt{\frac{y}{x}}}.$$
On introduit sur $\RP$ $\phi(u)=\frac{\sqrt{1+u}}{1+\sqrt{u}}$. $\phi$ est continue sur $\RP$ et dérivable sur $\RPs$ par TOC et on a $\forall u>0$ :
$$\phi'(u)=\ldots=\frac{\sqrt{u}-1}{\text{den. positif}}.$$
On fait le tableau de variations, $\phi$ est minimal en $1$ de valeur $\dfrac{1}{\sqrt{2}}$. Ainsi, $$\frac{\sqrt{x+y}}{\sqrt{x}+\sqrt{y}}\geq \dfrac{1}{\sqrt{2}}$$ donc l'inf est bien définie et ce minorant est atteint pour $x=y=1$ donc c'est un min. 
  \end{sol}
\end{exos}

\begin{proposition}
  Soit $f:I\to\K$, dérivable sur un intervalle $I$. Alors $f$ est constante si et seulement si
    \[\forall x\in I \qsep f'(x)=0.\]
\end{proposition}

\begin{exoUnique}
  \exo Soit $\alpha>0$. On dit qu'une fonction $f:I\to\K$ définie sur un intervalle
  $I$ est $\alpha$-Hölderienne lorsqu'il existe $C\geq 0$ tel que
  \[\forall x,y\in I \qsep \abs{f(x)-f(y)}\leq C\abs{x-y}^\alpha.\]
  Montrer que si $\alpha>1$, alors $f$ est constante.
  \begin{sol}
  On divise pour faire apparaitre un taux d'accroissement à gauche, puis on "gendarmise".
  \end{sol}
\end{exoUnique}

\begin{proposition}[utile=-3]
Soit $f:I\to\R$ une fonction réelle, dérivable sur un intervalle $I$. Alors $f$ est
strictement croissante si et seulement si
\begin{itemize}
\item $\forall x\in I \qsep f'(x)\geq 0$,
\item Il n'existe pas d'intervalle non trivial sur lequel $f'$ est nulle.
\end{itemize}
\end{proposition}

\begin{preuve}
\begin{itemize}
\item[$\bullet$] On suppose que $f$ est strictement croissante. Alors, comme en particulier $f$ est croissante, $f'\geq 0$. Soient $x,y \in I$ tels que $x<y$. Si jamais $f'$ était nulle sur $[x,y]$, elle y serait constante et en particulier $f(x)=f(y)$ ce qui contredirait la stricte croissance de $f$.
\item[$\bullet$] Sens droite-gauche : Déjà, $f$ est croissante. Soient $x,y \in I$ tels que $x<y$. On a déjà $f(x)\leq f(y)$. Si jamais $f(x)=f(y)$ alors comme $f$ croît, $f$ serait constante sur $[x,y]$ ce qui serait un intervalle non trivial sur lequel $f'$ s'annule, d'où la contradiction.
\end{itemize}

\end{preuve}

\begin{remarques}
\remarque On rappelle qu'un intervalle non trivial est un intervalle qui contient au moins deux points.
\remarque Rappelons au passage qu'une fonction croissante qui n'est pas
  strictement croissante est constante sur un intervalle non trivial.
\end{remarques}

%% Remarque :
%% 1) En particulier si le nombre de zéros de f est fini (ou plus généralement
%%    fini sur tout segment) et que f' >=0, on en déduit que f est strictement
%%    croissante.

\subsection{Théorème de la limite de la dérivée}

\begin{proposition}[utile=-3]
$\quad$
\begin{itemize}
\item Soit $f:I\to\K$ une fonction définie sur
  un intervalle $I$ et $x_0\in I$. On suppose que $f$ est continue sur $I$,
  dérivable sur $I\setminus\ens{x_0}$ et que
  \[f'(x)\tendversp{x}{x_0} l\in\K.\]
  Alors $f$ est dérivable en $x_0$ et $f'\p{x_0}=l$.
\item Soit $f:I\to\R$ une fonction réelle définie sur un intervalle
  $I$ et $x_0\in I$. On suppose que $f$ est continue sur $I$, dérivable sur
  $I\setminus\ens{x_0}$
  et que
  \[f'(x)\tendversp{x}{x_0} \pm\infty.\]
  Alors
  \[\frac{f(x)-f\p{x_0}}{x-x_0} \tendvers{x}{x_0} \pm\infty.\]
  Autrement dit, le graphe de $f$ admet une demi-tangente verticale en $x_0$.
  En particulier, la fonction $f$ n'est pas dérivable en $x_0$.
\end{itemize}
\end{proposition}

\begin{preuve}
  \begin{itemize}
  \item Cas réel : Soit $\epsilon>0$. Comme $f'(x)\tendversp{x}{x_0} \ell\in\R,$ on peut fixer $\eta>0$ tel que $\forall x \in I$ (le $0<$ vient du fait que $f'$ n'est pas définie en $x_0$), $$0<|x-a|\leq \eta \Longrightarrow |f'(x)-\ell|\leq \epsilon.$$
  Fixons $x\in I$ tel que $0<|x-x_0|\leq \eta$. $f$ est continue sur $"\interf{x_0}{x}"$ et dérivable sur $"\intero{x_0}{x}"$ donc d'après le TAF, il existe $c_x\in "\intero{x_0}{x}"$ tel que $f'(c_x)=\dfrac{f(x)-f(x_0)}{x-x_0}$. On a alors $0<|c_x-x_0|\leq |x-x_0|\leq\eta$ d'où $|\dfrac{f(x)-f(x_0)}{x-x_0}-\ell|=|f'(c_x)-\ell|\leq \epsilon$. Si on observe maintenant que l'inégalité $|\dfrac{f(x)-f(x_0)}{x-x_0}-\ell|\leq \epsilon$ a été obtenu pour tout $x\in I$ vérifiant $0<|x-x_0|\leq \eta$ cela signifie bien qu'il existe $\underset{x\underset{\neq}{\to}x_0}{\lim}\dfrac{f(x)-f(x_0)}{x-x_0}=\ell$ et donc que $f$ est dérivable en $x_0$ et que $f'(x_0)=\ell$.
  
  Cas complexe : On applique le résultat aux parties réelles et imaginaires.
    
  \item Preuve similaire.
  \end{itemize}
  
  \end{preuve}


\begin{exos}
% \exo Soit $f$ la fonction définie sur $\intero{-\pi/2}{\pi/2}$ par
%   \[\forall x\in\intero{-\frac{\pi}{2}}{\frac{\pi}{2}} \qsep
%     f(x)\defeq
%     \begin{cases}
%     \frac{1}{x}-\frac{1}{\sin x} & \text{si $x\neq 0$}\\
%     0 & \text{si $x=0$.} 
%     \end{cases}\]
%   Montrer que $f$ est de classe $\classec{1}$ sur $\intero{-\pi/2}{\pi/2}$.  
%  \begin{sol}
%  $\forall x \neq 0$, on calcule $f'(x)=\dfrac{-\sin^2(x)+x^2\cos(x)}{x^2\sin^2(x)}$. Le dénominateur étant équivalent à $x^4$ en $0$, on effectue un DL(0,4) du numérateur :
%  $$-\sin^2(x)+x^2\cos(x)=-\p{x-\frac{x^3}{6}}^2+x^2\p{1-\frac{x^2}{2}}+\petitozero{x}{x^4}=-\frac{1}{6}x^4+\petitozero{x}{x^4}$$ donc $f'(x)\tendversp{x}{0} -\frac{1}{6}$. Alors d'après le théorème de la limite de la dérivée, $f$ est dérivable en $0$ et $f'(0)=-\frac{1}{6}$. Et on obtient bien (toujours lorsqu'on applique ce théorème en fait) une fonction $\classec{1}$ sur $\intero{-\pi/2}{\pi/2}$.
%  \end{sol}
\exo Montrer que
  \[\frac{\arcsin\p{-1+h}+\frac{\pi}{2}}{h}\tendvers{h}{0}+\infty\]
  % \begin{sol}
  % Application directe de la deuxième partie du théorème avec $f=\arcsin$ et $x_0=-1$.
  % \end{sol}
  \begin{sol}
    Application directe de la deuxième partie du théorème avec $f=\arcsin$ et $x_0=-1$.
    $f$ est continue sur $[-1;1]$, dérivable sur l'ouvert et $f'(x)=\dfrac{1}{\sqrt{1-x^2}}\tendversp{x}{-1}+\infty$ donc $\frac{f(-1+h)-f(-1)}{h}\tendvers{h}{0}+\infty$.
    \end{sol}
\exo Soit $f$ la fonction définie sur $\R$ par
  \[\forall x\in\R \qsep f(x)\defeq
    \begin{cases}
    \e^{-\frac{1}{x}} & \text{si $x> 0$}\\
    0 & \text{si $x\leq 0$.}
    \end{cases}\]
  \begin{questions}
\question Montrer que $f$ est de classe $\classec{\infty}$ sur $\RPs$ et que, pour tout $n\in\N$, existe
  $P_n\in\polyR$ tel que
  \[\forall x>0\qsep f^{(n)}(x)=P_n\p{\frac{1}{x}}\e^{-\frac{1}{x}}.\]
\question Montrer que
  \[\forall n\in\N\qsep f^{(n)}(x)\tendversdp{x}{0}0.\]
\question En déduire que $f$ est de classe $\classec{\infty}$ sur $\R$.
  \end{questions}
  \begin{sol}
    On montre que $f$ est continue, que pour tout $n\in \N$, pour tout $x>0$, $$f^{(n)}(x)=P_n\p{\frac{1}{x}}\e^{-\frac{1}{x}}\tendversdp{x}{0}0.$$ On complète avec la limite à gauche pour obtenir $f^{(n)}(x)\tendversp{x}{0}0$ et on applique la proposition précédente.
    \end{sol}
\end{exos}




%% Exemples :
%% 1)
%%
%% 2) On en déduit que le graphe de la fonction arcsin admet une demi-tangente
%%    verticale en 1. En effet arcsin'(x) -> +infini (lorsque x tend vers 1)
%%    donc (arcsin(x)-arcsin(1))/(x-1) -> + infini.

%%%%%%% u_(n+1)=f(u_n) %%%%%%%
%%
%% Application :
%% 1) Soit f:I->I une fonction et x_0 \in I un point fixe de f (f(x_0)=x_0) en
%%    lequel elle est dérivable.
%%    - On dit que le point fixe est attractif lorsque |f'(x_0)|< 1
%%      Si tel est le cas, il existe eta>0 tel que pour tout x\in [x_0-eta,x_0+eta]
%%      la suite définie par u_0=x et u_(n+1)=f(u_n) converge vers x_0.
%%    - On dit que le point fixe est répulsif lorsque |f'(x_0)|>1
%%      Si tel est le cas, si la suite définie par u_0=x et u_(n+1)=f(u_n)
%%      converge vers x_0 elle est égale à x_0 à partir d'un certain rang
%%
%%%%%%%% Méthode de Newton %%%%%%%%
%%
%% Application :

\section{Convexité}

\subsection{Définition, propriétés élémentaires}

\begin{proposition}
Soit $x_1,x_2\in\R$.
\begin{itemize}
\item On a
\[\text{\og}\interf{x_1}{x_2}\text{\fg}=\ensim{tx_1+(1-t)x_2}{t\in[0,1]}.\]
\item Soit $y_1,y_2\in\R$. Alors il existe
une unique fonction affine $\phi$ sur $\intergf{x_1}{x_2}$ telle que
$\phi\p{x_1}=y_1$ et $\phi\p{x_2}=y_2$. De plus
\[\forall t\in\interf{0}{1}\qsep \phi\p{tx_1+\p{1-t}x_2}=
  t \phi\p{x_1}+\p{1-t}\phi\p{x_2}.\]
\end{itemize}
\end{proposition}


% \begin{proposition}[utile=-3]
% Soit $x_1,x_2\in\R$ avec $x_1\neq x_2$ et $y_1,y_2\in\R$. Alors il existe
% une unique fonction affine $\phi$ sur $\intergf{x_1}{x_2}$ telle que
% $\phi\p{x_1}=y_1$ et $\phi\p{x_2}=y_2$. De plus
% \[\forall t\in\interf{0}{1}\qsep \phi\p{tx_1+\p{1-t}x_2}=
%   t \phi\p{x_1}+\p{1-t}\phi\p{x_2}\]
% \end{proposition}

% \begin{preuve}
% $\quad$
% \begin{itemize}
% \item Soit $x_1,x_2\in\R$ et $x\in\R$.\\
%   Supposons que $x\in\intergf{x_1}{x_2}$ et montrons qu'il existe
%   $t\in\interf{0}{1}$ tel que $x=tx_1+\p{1-t}x_2$. Si $x_1=x_2$, alors $x$ est
%   cette valeur commune et $t=0$ convient. Sinon, supposons d'abord que
%   $x_1<x_2$. Alors
%   \begin{eqnarray*}
%   \forall t\in\R \quad tx_1+\p{1-t}x_2=x
%   &\ssi& t\p{x_1-x_2}=x-x_2\\
%   &\ssi& t=\frac{x_2-x}{x_2-x_1}
%   \end{eqnarray*}
%   On pose donc $t=\p{x_2-x}/\p{x_2-x_1}$. Comme $x_1\leq x\leq x_2$, on en
%   déduit que $-x_2\leq -x\leq -x_1$ puis que $0\leq x_2-x\leq x_2-x_1$. Donc
%   $0\leq t\leq 1$. Si $x_2<x_1$, alors d'après ce qui précède, il existe
%   $u\in\interf{0}{1}$ tel que $x=ux_2+\p{1-u}x_1$. On pose alors
%   $t=1-u\in\interf{0}{1}$ et on a bien $x=tx_1+\p{1-t}x_2$.\\
%   Réciproquement, on suppose qu'il existe $t\in\interf{0}{1}$ tel que
%   $x=tx_1+\p{1-t}x_2$. Montrons que $x\in\intergf{x_1}{x_2}$. On suppose
%   d'abord que $x_1\leq x_2$. Alors $tx_1\leq t x_2$ car $t\geq 0$ donc
%   $x=tx_1+\p{1-t}x_2\leq tx_2+\p{1-t}x_2=x_2$. De même,
%   $\p{1-t}x_1\leq\p{1-t}x_2$ donc $x_1=tx_1+\p{1-t}x_1\leq tx_1+\p{1-t}x_2=x$.
%   En conclusion $x\in\interf{x_1}{x_2}$. De même, si $x_2\leq x_1$, on montre
%   que $x\in\interf{x_2}{x_1}$.
% \item Soit $x_1,x_1\in\R$ avec $x_1\neq x_2$ et $y_1,y_2\in\R$.
%   Soit $\alpha,\beta\in\R$ et $f$ la fonction définie sur $\intergf{x_1}{x_2}$
%   par
%   \[\forall x\in\intergf{x_1}{x_2} \quad f(x)=\alpha x+\beta\]
%   Alors
%   \begin{eqnarray*}
%   \syslin{f\p{x_1}&=&y_1\cr
%           f\p{x_2}&=&y_2}
%   &\ssi& \syslin{\alpha x_1+\beta&=&y_1\cr
%                  \alpha x_2+\beta&=&y_2}
%   \end{eqnarray*}
%   Le déterminant de ce système linéaire en $\alpha,\beta$ est $x_1-x_2\neq 0$.
%   Il admet donc une unique solution, d'où l'existence et l'unicité d'une telle
%   fonction affine. De plus, si $t\in\interf{0}{1}$
%   \begin{eqnarray*}
%   tf\p{x_1}+\p{1-t}f\p{x_2}
%   &=& t\p{\alpha x_1+\beta}+\p{1-t}\p{\alpha x_2+\beta}\\
%   &=& \alpha\p{t x_1+\p{1-t}x_2}+\beta\\
%   &=& f\p{t x_1+\p{1-t}x_2}
%   \end{eqnarray*}
% \end{itemize}
% \end{preuve}

\begin{definition}[utile=-3]
Soit $f:I\to\R$ une fonction réelle définie sur un intervalle $I$. On dit que $f$ est
\emph{convexe} lorsque
\[\forall x_1,x_2\in I \qsep \forall t\in\interf{0}{1} \qsep
  f\p{tx_1+\p{1-t}x_2}\leq tf\p{x_1}+\p{1-t}f\p{x_2}.\]
\end{definition}

\medskip
\begin{center}
\begin{pdfpic}
\readdata{\listeP}{graph/graphe_convexite.txt}
\psset{xunit=2.5cm,yunit=1.4cm}
\begin{pspicture}(-1.2,-0.5)(4,5)
\psaxes[labels=none,ticks=none]{->}(0.5,0)(0,-0.5)(3.5,5)
\dataplot[plotstyle=curve,linewidth=2pt]{\listeP}
\uput[r](3.5,0){$x$}
\uput[l](0.5,5){$y$}
\psline[linestyle=dashed](1,0)(1,2)
\psline[linestyle=dashed](1,2)(0.5,2)
\psline[linestyle=dashed](3,0)(3,4)
\psline[linestyle=dashed](3,4)(0.5,4)
\psline[linestyle=dashed](2,0)(2,3)
\psline[linestyle=dashed](2,3)(0.5,3)
\psline[linestyle=dashed](2,1)(0.5,1)
\psline(1,2)(3,4)
\uput[d](1,0){$x_1$}
\uput[d](2,0){$tx_1+(1-t)x_2$}
\uput[d](3,0){$x_2$}
\uput[l](0.5,2){$f(x_1)$}
\uput[l](0.5,4){$f(x_2)$}
\uput[l](0.5,3){$tf(x_1)+(1-t)f(x_2)$}
\uput[l](0.5,1){$f(tx_1+(1-t)x_2)$}
\end{pspicture}
\end{pdfpic}
\end{center}


\begin{remarques}
\remarque Les fonctions affines sont convexes sur $\R$.
\remarque Une combinaison linéaire positive de fonctions convexes est convexe.
  Cependant, si $f$ est une fonction convexe, en général, $-f$ n'est pas convexe.
  % qui n'est pas affine, $-f$ n'est
  % pas convexe.
\end{remarques}

\begin{exos}
\exo Montrer que les fonctions $x\mapsto\abs{x}$ et $x\mapsto x^2$ sont
  convexes sur $\R$.
\end{exos}

\begin{proposition}[nom={Inégalité de \nom{Jensen}}, utile=-3]
Soit $f:I\to\R$ une fonction convexe sur un intervalle $I$. Alors
\[\forall x_1,\ldots,x_n\in I \qsep \forall t_1,\ldots,t_n\in\interf{0}{1},\]
\[t_1+\cdots+t_n=1 \quad\implique\quad f\p{t_1 x_1+\cdots+t_n x_n} \leq
  t_1 f\p{x_1}+\cdots+t_n f\p{x_n}.\]
\end{proposition}

\subsection{Convexité et dérivation}

% \begin{proposition}[nom={Lemme des 3 pentes}]
% Soit $f:I\to\R$ une fonction réelle. Alors $f$ est convexe si et seulement si, quels que
% soient $x_1,x_2,x_3\in I$ tels que $x_1<x_2<x_3$, on a
% \[\frac{f\p{x_2}-f\p{x_1}}{x_2-x_1}\leq\frac{f\p{x_3}-f\p{x_1}}{x_3-x_1}\leq
%   \frac{f\p{x_3}-f\p{x_2}}{x_3-x_2}.\]
% \end{proposition}



\begin{proposition}
Soit $f:I\to\R$ une fonction réelle. Alors $f$ est convexe si et seulement si, pour
tout $x_0\in I$, la fonction
\[\dspappli{\tau_{x_0}}{I\setminus\ens{x_0}}{\R}{x}{\frac{f(x)-f(x_0)}{x-x_0}}\]
est croissante.
\end{proposition}



\begin{remarqueUnique}
% \remarque Soit $f$ une fonction convexe sur $I$ et $x_0\in I$. Alors la fonction
%   $\tau_{x_0}$ définie sur $I\setminus\ens{x_0}$ par
%   \[\forall x\in I\setminus\ens{x_0} \qsep
%     \tau_{x_0}(x)\defeq\frac{f(x)-f\p{x_0}}{x-x_0}\]
%   est croissante. % Réciproquement, une fonction telle que
%   $\tau_{x_0}$ est croissante quel que soit $x_0\in I$, est convexe.
\remarque Soit $f:I\to\R$ une fonction convexe. Alors, quels que
soient $x_1,x_2,x_3\in I$ tels que $x_1<x_2<x_3$, on a
\[\frac{f\p{x_2}-f\p{x_1}}{x_2-x_1}\leq\frac{f\p{x_3}-f\p{x_1}}{x_3-x_1}\leq
  \frac{f\p{x_3}-f\p{x_2}}{x_3-x_2}.\]
  \begin{center}
    \begin{pdfpic}
    \readdata{\listeP}{graph/graphe_convexite.txt}
    \psset{xunit=3cm,yunit=1.4cm}
    \begin{pspicture}(0,-0.5)(3.7,5)
    \psaxes[labels=none,ticks=none]{->}(0.5,0)(0,-0.5)(3.5,5)
    \dataplot[plotstyle=curve,linewidth=2pt]{\listeP}
    \uput[r](3.5,0){$x$}
    \uput[l](0.5,5){$y$}
    \psline[linestyle=dashed](1,0)(1,2)
    \psline[linestyle=dashed](3,0)(3,4)
    \psline[linestyle=dashed](2,0)(2,1)
    \psline[linestyle=dashed](1,2)(0.5,2)
    \psline[linestyle=dashed](3,4)(0.5,4)
    \psline[linestyle=dashed](2,1)(0.5,1)
    \psline(1,2)(3,4)
    \psline(1,2)(2,1)
    \psline(2,1)(3,4)
    \uput[d](1,0){$x_1$}
    \uput[d](2,0){$x_2$}
    \uput[d](3,0){$x_3$}
    \uput[l](0.5,2){$f(x_1)$}
    \uput[l](0.5,4){$f(x_3)$}
    \uput[l](0.5,1){$f(x_2)$}
    \end{pspicture}
    \end{pdfpic}
    \end{center}
\end{remarqueUnique}

\begin{exos}
\exo Montrer que sur $\R$, une fonction convexe majorée est constante.
\end{exos}

\begin{proposition}
Soit $f:I\to\R$ une fonction convexe.
\begin{itemize}
\item Alors $f$ est dérivable à gauche et à droite en tout point intérieur à $I$.
\item En particulier, $f$ est continue en tout point intérieur à $I$.
\end{itemize}
\end{proposition}

\begin{remarqueUnique}
  \remarque 
  Remarquons qu'une fonction convexe peut
  très bien ne pas être dérivable en un point intérieur à $I$ comme le montre
  l'exemple de la valeur absolue en 0. De même, une fonction convexe peut
  être discontinue en les bornes de son intervalle de définition comme le
  montre l'exemple de la fonction
  \[\dspappli{f}{\interf{0}{1}}{\R}{x}{
    \begin{cases}
    1 & \text{si $x=0$ ou $x=1$,}\\
    0 & \text{sinon.}  
    \end{cases}}\]
\end{remarqueUnique}

\begin{proposition}[utile=-3]
Soit $f:I\to\R$ une fonction convexe sur un intervalle $I$. Si $x_0\in I$ est un point
en lequel $f$ est dérivable, alors
\[\forall x\in I\qsep f(x)\geq f\p{x_0}+f'\p{x_0}(x-x_0).\]
Autrement dit, le graphe de $f$ est au-dessus de ses tangentes.
\end{proposition}



\begin{proposition}[utile=-3]
Soit $f:I\to\R$ une fonction réelle, dérivable sur un intervalle $I$. Alors
$f$ est convexe si et seulement si $f'$ est croissante.
\end{proposition}
\begin{preuve}
Se ramener au cas où $f(0)=f(1)=0$ et montrer que pour tout
$x\in\interf{0}{1}$, $f(x)\leq 0$. Utiliser Rolle pour montrer que $f'$ s'annule
et conclure par un tableau de variations.
\end{preuve}

\begin{proposition}[utile=-3]
Soit $f:I\to\R$ une fonction réelle deux fois dérivable sur un intervalle $I$. Alors
$f$ est convexe si et seulement si
\[\forall x\in I \qsep f''(x)\geq 0.\]
\end{proposition}

\begin{remarques}
% \remarque Plus généralement, on a montré qu'une fonction dérivable sur un intervalle
%   $I$ est convexe si et seulement si sa dérivée est croissante.
\remarque La fonction $\exp$ est convexe sur $\R$. La fonction $x\mapsto 1/x$
  est convexe sur $\RPs$.
\remarque On dit qu'une fonction réelle définie sur un intervalle $I$ est
  concave lorsque
  \[\forall x_1,x_2\in I \qsep \forall t\in\interf{0}{1} \qsep
    f\p{tx_1+\p{1-t}x_2}\geq tf\p{x_1}+\p{1-t}f\p{x_2}.\]
  Une fonction $f$ est concave si et seulement si $-f$ est convexe. On en déduit
  que toutes les propositions énoncées pour les fonctions convexes ont leur
  équivalent pour les fonctions concaves. En particulier, les fonctions
  concaves sont en dessous de leur tangentes et une fonction deux fois
  dérivable sur un intervalle est concave si et seulement si sa dérivée
  seconde est négative.
\end{remarques}

\begin{exos}
\exo Montrer que
  \[\forall x\in\R \qsep \e^x \geq 1+x, \qquad \forall x\in\intero{-1}{+\infty}
    \qsep \ln\p{1+x}\leq x\]
  \[\forall x\in\interf{0}{\frac{\pi}{2}} \qsep
    \frac{2}{\pi}\cdot x\leq\sin x\leq x.\]
% \exo Montrer que la suite de terme général
%   \[\p{1+\frac{1}{n}}^n\]
%   est croissante.
\exo Soit $x_1,\ldots,x_n\in\RP$. Montrer que
  \[\sqrt[n]{x_1 x_2\cdots x_n}\leq\frac{x_1+x_2+\cdots+x_n}{n}.\]
  C'est l'inégalité arithmético-géométrique qui se démontre facilement pour
  $n=2$.
\end{exos}

% \begin{remarques}
% \remarque \textbf{Convergence de la méthode de Newton dans le cas d'une fonction convexe~:}\\
%   Soit $f:\interf{a}{b}\to\R$ une fonction de classe
%   $\classec{\infty}$ telle que~:
%   \[f(a) < 0 < f(b) \qquad f'>0\text{ et }f''>0\text{ sur }\interf{a}{b}\]
%   Alors la fonction $f$ admet un unique racine sur $\intero{a}{b}$ notée
%   $\alpha$. De plus, la suite définie par~:
%   \[u_0=b \et \forall n\in\N \quad u_{n+1}=u_n-\frac{f(u_n)}{f'(u_n)}\]
%   est bien définie, décroissante et converge vers alpha en moyenne quadratique, c'est à dire
%   que si $(\epsilon_n)$ est la suite des erreurs définie par
%   $\epsilon_n=\abs{u_n-\alpha}$, on a
%   \[\frac{\epsilon_{n+1}}{\epsilon_n^2}\tendvers{n}{+\infty}
%     \frac{f''(\alpha)}{2f'(\alpha)}\]
%   En particulier, si $(p_n)$ est la suite définie par $p_n=-\log_{10} \epsilon_n$
%   (intuitivement, $p_n$ est le nombre de chiffres après la virgule exacts
%   lorsque l'on considère $u_n$ comme valeur approchée de $\alpha$), on a~:
%   \[p_{n+1} \equi{n}{+\infty} 2 p_n\]
%   Autrement dit, le nombre de chiffre exacts donnés par la méthode de Newton
%   double à chaque itération. Rappelons qu'avec la méthode de dichotomie,
%   le nombre de chiffres exacts augmente de 1 toutes les 3 itérations.
%   \begin{sol}
%   En effet~:
%   \begin{itemize}
%   \item Existence et unicité de $\alpha$.
%   \item Soit $\phi$ la fonction définie sur [a,b] par
%     $\phi(x)=x-f(x)/f'(x)$
%     $\phi$ est dérivable sur [a,b] et $\phi'=f.f''/(f'^2)$
%     en particulier, $\phi'$ est du signe de f, donc croissante sur $[\alpha,b]$
%     Comme $\phi(\alpha)=\alpha$ et $\phi(b)=b-f(b)/f'(b)<=b$, on en déduit que
%     $[\alpha,b]$ est stable par $\phi$.
%   \item donc $(u_n)$ est bien définie et $\alpha <= u_n <= b$
%     De plus, sur $[\alpha,b] \phi(x)<=x$ donc $(u_n)$ est décroissante.
%     Comme elle est décroissante minorée, elle converge vers un réel
%     $l$. On a : $\alpha <= l <= b$. Comme $\phi(l)=l$, on a $l=l-f(l)/f'(l)$, donc
%     $f(l)=0$, donc $l=\alpha$.
%   \item Reste à montrer que $(u_n)$ converge vers $\alpha$ en moyenne quadratique
%     \begin{itemize}
%     \item Montrons que $u_{n+1} < u_n$. Par récurrence sur n
%       $n=0$ : En effet $u_1=u_0-f(u_0)/f'(u_0) < u_0$
%       $n-> n+1$ : car $\phi$ est strictement croissante sur $[\alpha,b]$
%       Donc $(u_n)$ est strictement décroissante. En particulier $\alpha < u_n$
%       pour tout $n$.
%     \item Comme $\phi$ est de classe $\classec{\infty}$, elle admet un
%       développement limité à l'ordre 2 en $\alpha$ et~:
%       \[\phi(\alpha+h)=\alpha+(f''(\alpha)/(2 f'(\alpha)) h^2+o(h^2)\]
%       En particulier~:
%       \[(u_{n+1}-\alpha)/(u_n-\alpha)^2 -> (f''(\alpha)/(2 f'(\alpha)))\]
%     \item Enfin $p_{n+1} ~ 2 p_n$.
%     \end{itemize}
%   \end{itemize}
%   \end{sol}
% \end{remarques}

%END_BOOK

\end{document}


