\documentclass{magnolia}


\magtex{tex_driver={pdftex},
        tex_packages={xypic}}
\magfiche{document_nom={Cours sur les fonctions de plusieurs variables},
          auteur_nom={François Fayard},
          auteur_mail={fayard.prof@gmail.com}}
\magcours{cours_matiere={maths},
          cours_niveau={mpsi},
          cours_chapitre_numero={24},
          cours_chapitre={Fonctions de deux variables}}
\magmisenpage{}
\maglieudiff{}
\magprocess

\begin{document}

%BEGIN_BOOK
\magtoc

\section{Limite, continuité}



\subsection{Notion d'ouvert}

\begin{definition}
On note $\norme{.}$ la norme dérivant du produit scalaire usuel sur $\R^2$.
On rappelle que
\begin{eqnarray*}
\forall x\in\R^2, & & \norme{x}\geq 0\\
\forall x\in\R^2, & & \norme{x}=0 \ssi x=\p{0,0}\\
\forall x\in\R^2 \qsep \forall \lambda\in\R, & &
        \norme{\lambda x}=\abs{\lambda}\norme{x}\\
\forall x,y\in\R^2, & & \norme{x+y}\leq\norme{x}+\norme{y}\\
\forall x,y\in\R^2, & & \abs{\norme{x}-\norme{y}}\leq\norme{x-y}
\end{eqnarray*}
\end{definition}

\begin{proposition}
\begin{eqnarray*}
\forall \p{x_1,x_2}\in\R^2, & & \abs{x_1}\leq\norme{\p{x_1,x_2}} \et 
       \abs{x_2}\leq\norme{\p{x_1,x_2}}\\
\forall \p{x_1,x_2}\in\R^2, & & \norme{\p{x_1,x_2}}\leq\abs{x_1}+\abs{x_2}
\end{eqnarray*}
\end{proposition}

\begin{definition}
Soit $x_0\in\R^2$ et $r>0$.
\begin{itemize}
\item On appelle \emph{boule ouverte} de centre $x_0$ et de rayon $r$ l'ensemble
  \[B_O\p{x_0,r}\defeq\enstq{x\in\R^2}{\norme{x-x_0}<r}.\]
\item On appelle \emph{boule fermée} de centre $x_0$ et de rayon $r$ l'ensemble
  \[B_F\p{x_0,r}\defeq\enstq{x\in\R^2}{\norme{x-x_0}\leq r}.\]
\end{itemize}
\end{definition}

\begin{definition}
On dit qu'une partie $\mathcal{U}$ de $\R^2$ est un \emph{ouvert} lorsque
\[\forall x\in\mathcal{U} \qsep \exists \eta>0 \qsep
  B_F\p{x,\eta} \subset \mathcal{U}.\]
\end{definition}

\begin{proposition}
$\quad$
\begin{itemize}
\item $\emptyset$ et $\R^2$ sont des ouverts.
\item Une union d'ouverts est un ouvert.
\item Une intersection finie d'ouverts est un ouvert.
\end{itemize}
\end{proposition}

\begin{proposition}
$\quad$
\begin{itemize}
\item Les boules ouvertes sont des ouverts.
\item Si $ax+by+c=0$ est l'équation d'une droite, le demi-plan d'équation
  \[ax+by+c>0\]
  est un ouvert.
\item Si on enlève un nombre fini de points à un ouvert, il reste ouvert.
\end{itemize}
\end{proposition}

\begin{exoUnique}
\exo Soit $x_0\in\R^2$. Montrer que
  \[\bigcap_{n\in\Ns} B_O\p{x_0,\frac{1}{n}}\]
  n'est pas un ouvert.
\end{exoUnique}

\begin{definition}
Soit $\mathcal{U}$ une partie de $\R^2$ et $a\in\R^2$. On dit que $a$ est
\emph{adhérent} à $\mathcal{U}$ lorsque
\[\forall\epsilon>0 \qsep B_F\p{a,\epsilon}\cap\mathcal{U}\neq\emptyset.\]
\end{definition}

\begin{exoUnique}
\exemple Soit $x_0\in\R^2$ et $r>0$. Montrer que l'ensemble des points adhérents à
  la boule ouverte de centre $x_0$ et de rayon $r$ est la boule fermée de
  même centre et de même rayon.
\end{exoUnique}


\begin{definition}
On appelle \emph{fonction réelle de deux variables} toute fonction définie sur une
partie ouverte $\mathcal{U}$ de $\R^2$ à valeurs dans $\R$.
\end{definition}

\subsection{Limite}



\begin{definition}
Soit $f:\mathcal{U}\subset\R^2\to\R$ et $a\in\R^2$ un point adhérent à $\mathcal{U}$.
\begin{itemize}
\item Soit $l\in\R$. On dit que $f(x)$ \emph{tend} vers $l$ lorsque $x$ tend
  vers $a$ lorsque
  \[\forall \epsilon>0 \qsep \exists \eta>0 \qsep \forall x\in\mathcal{U}
    \qsep \norme{x-a}\leq\eta \implique \abs{f(x)-l}\leq\epsilon.\]
\item On dit que $f(x)$ \emph{tend} vers $+\infty$ lorsque $x$ tend vers $a$
  lorsque
  \[\forall m\in\R \qsep \exists \eta>0 \qsep \forall x\in\mathcal{U}
    \qsep \norme{x-a}\leq\eta \implique f(x)\geq m.\]
\item On dit que $f(x)$ \emph{tend} vers $-\infty$ lorsque $x$ tend vers $a$
  lorsque
  \[\forall M\in\R \qsep \exists \eta>0 \qsep \forall x\in\mathcal{U}
    \qsep \norme{x-a}\leq\eta \implique f(x)\leq M.\]
\end{itemize}
\end{definition}

\begin{remarqueUnique}
\remarque Soit $\p{x_0,y_0}\in\R^2$. Alors
  \[x\tendvers{\p{x,y}}{\p{x_0,y_0}} x_0, \qquad
    y\tendvers{\p{x,y}}{\p{x_0,y_0}} y_0 \et
    \norme{\p{x,y}} \tendvers{\p{x,y}}{\p{x_0,y_0}} \norme{\p{x_0,y_0}}.\]
\end{remarqueUnique}

\begin{proposition}
Soit $f$ et $g:\mathcal{U}\subset\R^2\to\R$ 
 et $a\in\R^2$ un point adhérent à $\mathcal{U}$. On suppose que $f(x)$
tend vers $l_1\in\R$ et $g(x)$ tend vers $l_2\in\R$ lorsque $x$ tend vers $a$. Alors
\begin{itemize}
\item Si $\lambda,\mu\in\R$
  \[\lambda f(x)+\mu g(x)\tendvers{x}{a} \lambda l_1+\mu l_2.\]
\item On a
  \[f(x)g(x)\tendvers{x}{a} l_1 l_2.\]
\item Si $l_2\neq 0$, alors il existe une boule de centre $a$ sur
  laquelle $g$ ne s'annule pas. De plus
  \[\frac{f(x)}{g(x)}\tendvers{x}{a}\frac{l_1}{l_2}.\]
\end{itemize}
\end{proposition}

\begin{proposition}
Soit $f:\mathcal{U}\subset\R^2\to\R$, $a\in\R^2$ un
point adhérent à $\mathcal{U}$ et $g:\mathcal{D}_g\subset\R\to\R$ une fonction telle que
$f\p{\mathcal{U}}\subset\mathcal{D}_g$. On
suppose que
\[f(x)\tendvers{x}{a} l_f\in\Rbar \et
  g(y)\tendvers{y}{l_f} l_g\in\Rbar.\]
Alors
\[g\p{f(x)}\tendvers{x}{a} l_g.\]
\end{proposition}

\begin{proposition}
Soit $f$, $g$ et $h:\mathcal{U}\subset\R^2\to\R$ et $a\in\R^2$ un point adhérent à $\mathcal{U}$.
\begin{itemize}
\item On suppose que
  \[\forall x\in\mathcal{U} \qsep f(x)\leq g(x)\leq h(x).\]
  et que $f(x)$ et $h(x)$ admettent la même limite finie $l\in\R$ lorsque
  $x$ tend vers $a$. Alors
  \[f(x)\tendvers{x}{a} l.\]
\item On suppose qu'il existe $l\in\R$ tel que
  \[\forall x\in\mathcal{U} \qsep \abs{f(x)-l} \leq g(x).\]
  et que $g(x)$ tend vers 0 lorsque $x$ tend vers $a$. Alors
  \[f(x)\tendvers{x}{a} l.\]
\item On suppose que
  \[\forall x\in\mathcal{U} \qsep f(x)\leq g(x).\]
  \begin{itemize}
  \item Si $f(x)\tendvers{x}{a} +\infty$, alors $g(x)\tendvers{x}{a} +\infty$.
  \item Si $g(x)\tendvers{x}{a} -\infty$, alors $f(x)\tendvers{x}{a} -\infty$.
  \end{itemize}
\end{itemize}
\end{proposition}

\begin{remarqueUnique}
\remarque Si il existe une fonction $\phi:\RP\to\RP$ telle que
  \[\phi(r)\tendvers{r}{0}0 \et
    \forall r\geq 0 \qsep \forall \theta\in\R \qsep 
    \abs{f\p{a_1+r\cos\theta,a_2+r\sin\theta}-l}\leq \phi\p{r}\]
  alors $f(x)\tendvers{x}{a}l$.
\end{remarqueUnique}

\begin{exoUnique}
\exo Soit $f$ la fonction définie par
  \[\forall\p{x,y}\in\R^2\setminus\ens{\p{0,0}} \qsep
    f\p{x,y}\defeq\frac{x^2 y}{x^2+y^2}.\]
  Montrer que $f$ admet une limite en $\p{0,0}$.
\end{exoUnique}

\subsection{Continuité}

\begin{definition}
On dit qu'une fonction $f:\mathcal{U}\subset\R^2\to\R$ est \emph{continue} en $x_0\in\mathcal{U}$ lorsque
\[f(x)\tendvers{x}{x_0} f\p{x_0}.\]
\end{definition}

\begin{remarques}
\remarque On dit qu'une fonction $f:\mathcal{U}\subset\R^2\to\R$ est continue lorsqu'elle est continue en tout
  point de $\mathcal{U}$.
% \remarque Les fonctions $(x,y)\mapsto x$ et $(x,y)\mapsto y$ sont continues.
\remarque Les fonctions
  \[\dspappli{f}{\R^2}{\R}{(x,y)}{x} \quad\et\quad
    \dspappli{g}{\R^2}{\R}{(x,y)}{y}\]
  sont continues.
\end{remarques}

\begin{proposition}
Soit $f$ et $g:\mathcal{U}\subset\R^2\to\R$ deux fonctions continues en $x_0\in\mathcal{U}$.  Alors
\begin{itemize}
\item Si $\lambda,\mu\in\R$, $\lambda f+\mu g$ est continue en $x_0$.
\item $fg$ est continue en $x_0$.
\item Si $g\p{x_0}\neq 0$, il existe une boule de centre $x_0$ sur
  laquelle $g$ ne s'annule pas et $f/g$ est continue en $x_0$.
\end{itemize}
\end{proposition}

% \begin{exempleUnique}
% \remarque D'après les théorèmes usuels, la fonction $f$ définie sur $\R^2$ par
%   \[\forall \p{x,y}\in\R^2 \qsep f\p{x,y}\defeq\frac{xy}{1+x^2+y^2}\]
%   est continue sur $\R^2$.
% \end{exempleUnique}

\begin{proposition}
Soit $f:\mathcal{U}\subset\R^2\to\R$ et $g:\mathcal{V}\subset\R\to\R$ deux fonctions
telles que $f\p{\mathcal{U}}\subset\mathcal{V}$. Si
$f$ est continue en $x_0\in\mathcal{U}$ et $g$ est continue en $f\p{x_0}$, alors $g\circ f$
est continue en $x_0$.
\end{proposition}

\subsection{Application partielle}

\begin{definition}
Soit $f:\mathcal{U}\subset\R^2\to\R$ et $x\defeq\p{x_1,x_2}\in\mathcal{U}$. On
définit les fonctions réelles d'une variable réelle $f_{p_1}$ et $f_{p_2}$ par
\[f_{p_1}(t)\defeq f\p{t,x_2} \et f_{p_2}(t)\defeq f\p{x_1,t}.\]
$f_{p_1}$ et $f_{p_2}$ sont appelées \emph{applications partielles} de $f$ au point
$x=\p{x_1,x_2}$.
\end{definition}

\begin{exoUnique}
\exemple Soit $f$ la fonction définie sur $\R^2$ par
  \[\forall\p{x,y}\in\R^2 \qsep f\p{x,y}\defeq\frac{\p{2-y}\cos\p{xy}}{1+x^2}.\]
  Déterminer les applications partielles de $f$ en $(0,0)$.
  \begin{sol}
  les applications partielles en $\p{0,0}$ sont respectivement
  $f_1(t)=2/\p{1+x^2}$ et $f_2(t)=2-y$.    
  \end{sol}
\end{exoUnique}

\begin{proposition}
\begin{itemize}
\item Si $f(x)$ tend vers $l\in\Rbar$ lorsque $x$ tend vers $a\defeq\p{a_1,a_2}$,
  $f_{p_1}(t)$ tend vers $l$ lorsque $t$ tend vers $a_1$ et $f_{p_2}(t)$
  tend vers $l$ lorsque $t$ tend vers $a_2$.
\item Si $f$ est continue en $x\defeq\p{x_1,x_2}$, $f_{p_1}$ est continue en $x_1$ et
  $f_{p_2}$ est continue en $x_2$.
\end{itemize}
\end{proposition}

\begin{remarqueUnique}
\remarque Nous verrons que les réciproques de ces théorèmes sont fausses. Par
  exemple, $f_{p_1}$ peut être continue en $x_1$ et $f_{p_2}$ peut être continue
  en $x_2$ sans que $f$ soit continue en $\p{x_1,x_2}$.
\end{remarqueUnique}

\begin{exoUnique}
\exo Montrer que la fonction $f$ définie sur $\R^2\setminus\ens{\p{0,0}}$
  par
  \[\forall\p{x,y}\in\R^2\setminus\ens{\p{0,0}} \qsep
    f\p{x,y}\defeq\frac{x^2-y^2}{x^2+y^2}\]
  n'a pas de limite en $\p{0,0}$.
  \begin{sol}
  La limite selon $y=0$ est 1. Celle selon $x=0$ est $-1$.
  \end{sol}
\end{exoUnique}

\subsection{Extension aux fonctions à valeurs dans $\R^2$}

\begin{definition}
Soit $f:\mathcal{U}\subset\R^2\to\R^2$. Les fonctions $f_{1}$ et $f_{2}:\mathcal{U}\subset\R^2\to\R$ définies par
\[\forall x\in\mathcal{U} \qsep f(x)=\p{f_{1}(x),f_{2}(x)}.\]
sont appelées \emph{fonctions coordonnées} de $f$.
\end{definition}

\begin{remarqueUnique}
\remarque On étend les notions de limite et de continuité aux fonctions à valeurs dans $\R^2$ en remplaçant
  les valeurs absolues par des normes dans les définitions données plus haut.
  Par exemple, si $f:\mathcal{U}\subset\R^2\to\R^2$, $a\in\R^2$ est un
  point adhérent à $\mathcal{U}$ et $l\in\R^2$, on dit que $f(x)$ tend vers $l$ lorsque $x$ tend vers $a$
  lorsque
  \[\forall \epsilon>0\qsep \exists \eta>0\qsep \forall x\in\mathcal{U}\qsep
    \norme{x-a}\leq\eta \implique \norme{f(x)-l}\leq\epsilon.\]
\end{remarqueUnique}

\begin{proposition}
Soit $f:\mathcal{U}\subset\R^2\to\R^2$.
\begin{itemize}
\item Si $a$ est un point adhérent à $\mathcal{U}$ et $l\defeq\p{l_1,l_2}\in\R^2$
  alors $f(x)$ tend vers $l$ lorsque $x$ tend vers $a$ si et seulement si
  \[f_{1}(x) \tendvers{x}{a} l_1 \et f_{2}(x) \tendvers{x}{a} l_2.\]
\item Si $x_0\in\mathcal{U}$, $f$ est continue en $x_0$ si et seulement si
  $f_{1}$ et $f_{2}$ le sont.
\end{itemize}
\end{proposition}

\begin{proposition}
Soit $f:\mathcal{U}\subset\R^p\to\R^q$ et $g:\mathcal{V}\subset\R^q\to\R^r$
avec $p,q,r\in\intere{1}{2}$. On suppose que
$f\p{\mathcal{U}}\subset\mathcal{V}$.
\begin{itemize}
\item Soit $a\in\R^p$ un point adhérent à $\mathcal{U}$. Si $f(x)$ tend
  vers $b\in\R^q$ lorsque $x$ tend vers $a$ et $g(y)$ tend vers $l$ lorsque
  $y$ tend vers $b$, alors
  \[g\p{f(x)}\tendvers{x}{a} l.\]
\item Soit $x_0\in\mathcal{U}$. Si $f$ est continue en $x_0$ et si $g$ est
  continue en $f\p{x_0}$, alors $g\circ f$ est continue en $x_0$.
\end{itemize}
\end{proposition}

\begin{remarqueUnique}
\remarque Soit $f:\mathcal{U}\subset\R^2\to\R$ et $a$ un point adhérent à
  $\mathcal{U}$ en lequel $f$ admet une limite $l\in\Rbar$. Si
  $g:I\to\mathcal{U}\subset\R^2$ où $I$ est un intervalle de $\R$, $b$ est une
  borne de $I$ et $g(t)$ tend vers $a$ lorsque $t$ tend vers $b$, alors
  \[f\p{g(t)}\tendvers{t}{b}  l.\]
  On peut ainsi, en choisissant différentes fonctions $g$, se faire une idée de
  la limite éventuelle de $f$ en $a$, ou prouver que $f$ n'admet pas de limite
  en $a$.
\end{remarqueUnique}


\begin{exos}
\exo Soit $f$ la fonction définie par
  \[\forall\p{x,y}\in\R^2\setminus\ens{\p{0,0}} \qsep f\p{x,y}\defeq
    \frac{xy}{x^2+y^2}.\]
  Étudier la limite éventuelle de $f$ en $\p{0,0}$.
  \begin{sol}
  Passer en coordonnées polaires.
  \end{sol}
\exo Soit $f$ la fonction définie sur ${\RPs}^2$ par
  \[\forall (x,y)\in{\RPs}^2 \qsep f\p{x,y}\defeq x^y.\]
  Étudier la limite éventuelle de $f$ en $\p{0,0}$.
  \begin{sol}
  Si on arrive en diagonale, la limite est $1$. Cependant, si $x=t$ et
  $y=-1/\p{\ln t}$, la limite est $e^{-1}$.
  \end{sol}
\end{exos}

\section{Dérivation}



\subsection{Dérivée partielle}

\begin{definition}
Soit $f:\mathcal{U}\subset\R^2\to\R$ et $x\defeq\p{x_1,x_2}\in\mathcal{U}$.
\begin{itemize}
\item On dit que $f$ admet \emph{une dérivée partielle par rapport à la première variable} en
  $x$ lorsque
  \[\frac{f\p{x_1+t,x_2}-f\p{x_1,x_2}}{t}\]
  admet une limite finie lorsque $t$ tend vers 0. Si tel est le cas, cette
  limite est notée
  \[\parfrac{f}{x_1}(x).\]
\item On dit que $f$ admet \emph{une dérivée partielle par rapport à la seconde variable} en
  $x$ lorsque
  \[\frac{f\p{x_1,x_2+t}-f\p{x_1,x_2}}{t}\]
  admet une limite finie lorsque $t$ tend vers 0. Si tel est le cas, cette
  limite est notée
  \[\parfrac{f}{x_2}(x).\]
\end{itemize}
\end{definition}


\begin{remarqueUnique}
\remarque Soit $f:\R^2\to\R$ une fonction admettant des dérivées partielles en tout point
  par rapport à la première et à la seconde variable, $x\defeq (x_1,x_2)$
  et $f_{p_1}$, $f_{p_2}:\R\to\R$ les fonctions partielles définies par
  \[\forall t\in\R\qsep f_{p_1}(t)\defeq f(t, x_2)
    \quad\et\quad
    f_{p_2}(t)\defeq (x_1,t).\]
  Alors $f_{p_1}$ et $f_{p_2}$ sont dérivables sur $\R$ et
  \[\forall t\in\R\qsep f_{p_1}'(t)=\parfrac{f}{x_1}(t,x_2) \quad\et\quad
    f_{p_2}'(t)=\parfrac{f}{x_2}(x_1,t).\]
\end{remarqueUnique}

\begin{definition}
Soit $f:\mathcal{U}\subset\R^2\to\R$, $x\in\mathcal{U}$ et $h\in\R^2$. On dit
que $f$ admet une \emph{dérivée en $x$ selon le vecteur $h$} lorsque
\[\frac{f\p{x+th}-f(x)}{t}\]
admet une limite finie lorsque $t$ tend vers 0. Si tel est le cas, cette limite
est notée
\[{\rm D}f_x(h).\]
\end{definition}

\begin{remarques}
\remarque Si $e_1\defeq(1,0)$, $f$ admet une dérivée en $x$
  selon le vecteur $e_1$ si et seulement si elle admet une dérivée partielle par rapport à la
  première variable en $x$. De plus, si tel est le cas
  \[{\rm D}f_{x}(e_1)=\parfrac{f}{x_1}(x).\]
  De même, $f$ admet une dérivée en $x$ selon le vecteur $e_2\defeq(0,1)$ si et seulement
  si elle est admet une dérivée partielle par rapport à la seconde variable en $x$.
\remarque Soit $f:\mathcal{U}\subset\R^2\to\R$. Alors $f$ admet une dérivée en tout
  $x\in\mathcal{U}$ selon le vecteur nul et ${\rm D}f_x(0)=0$.
  De plus, si $f$ admet une dérivée en $x$ selon le vecteur $h\in\R^2$ alors, pour tout
  $\lambda\in\R$, elle admet une dérivée en $x$ selon le vecteur $\lambda h$
  et
  \[{\rm D}f_x(\lambda h)=\lambda {\rm D}f_x(h).\]
\end{remarques}

\begin{exoUnique}
\exo Soit $f$ la fonction définie sur $\R^2$ par
  \[\forall\p{x,y}\in\R^2 \qsep f\p{x,y}\defeq
    \begin{cases}
    \frac{x^2 y}{x^4+y^2} & \text{si $\p{x,y}\neq\p{0,0}$}\\
    0 & \text{si $\p{x,y}=0$.}
    \end{cases}\]
  Montrer que $f$ est dérivable en $\p{0,0}$ selon tout vecteur mais n'est pas
  continue en $\p{0,0}$.
  \begin{sol}
  Calculer la dérivée selon le vecteur $\p{\cos\theta,\sin\theta}$. On trouve
  0 si $\sin\theta=0$ et $\cos^2 \theta/\sin\theta$ sinon. Cependant
  $f\p{x,x^2}=1/2$ qui ne tend pas vers 0 en 0, donc $f$ n'est pas continue
  en 0.
  \end{sol}
\end{exoUnique}

\begin{definition}
On dit qu'une fonction $f:\mathcal{U}\subset\R^2\to\R$ est de \emph{classe
$\classec{1}$} lorsque
\[\parfrac{f}{x_1} \et \parfrac{f}{x_2}\]
sont définies et continues sur $\mathcal{U}$.
\end{definition}

\begin{remarques}
\remarque Si $f$ est de classe $\classec{1}$, les applications partielles
  sont de classe $\classec{1}$.
\remarque Les fonctions
  \[\dspappli{f}{\R^2}{\R}{(x,y)}{x} \quad\et\quad
    \dspappli{g}{\R^2}{\R}{(x,y)}{y}\]
  sont de classe $\classec{1}$.
\end{remarques}

\begin{proposition}
Soit $f$ et $g:\mathcal{U}\subset\R^2\to\R$ deux fonctions de classe $\classec{1}$.
Alors
\begin{itemize}
\item Si $\lambda,\mu\in\R$, $\lambda f+\mu g$ est de classe $\classec{1}$ et
  \[\forall i\in\intere{1}{2} \qsep \forall x\in\mathcal{U} \qsep
    \parfrac{\p{\lambda f+\mu g}}{x_i}(x)=\lambda\parfrac{f}{x_i}(x)+\mu
    \parfrac{g}{x_i}(x).\]
\item $fg$ est de classe $\classec{1}$ et
  \[\forall i\in\intere{1}{2} \qsep \forall x\in\mathcal{U} \qsep
    \parfrac{\p{fg}}{x_i}(x)=\parfrac{f}{x_i}(x)g(x)+
    f(x)\parfrac{g}{x_i}(x).\]
\item Si $g$ ne s'annule pas sur $\mathcal{U}$, $f/g$ est de classe
  $\classec{1}$ et
  \[\forall i\in\intere{1}{2} \qsep \forall x\in\mathcal{U} \qsep
    \parfrac{\p{\dsp\frac{f}{g}}}{x_i}(x)=
    \frac{\dsp\parfrac{f}{x_i}(x)g(x)-
    f(x)\dsp\parfrac{g}{x_i}(x)}{g(x)^2}.\]
\end{itemize}
\end{proposition}

\begin{proposition}
Soit $f:\mathcal{U}\subset\R^2\to\R$ et $g:\mathcal{V}\subset\R\to\R$ deux fonctions de
classe $\classec{1}$ telles que $f(\mathcal{U})\subset\mathcal{V}$. Alors
$g\circ f$ est de classe $\classec{1}$ et
\[\forall i\in\intere{1}{2}\qsep \forall x\in\mathcal{U}\qsep
  \parfrac{(g\circ f)}{x_i}(x)=g'(f(x))\parfrac{f}{x_i}(x).\]
\end{proposition}

\begin{exoUnique}
\exo Trouver l'ensemble des fonctions $f:\R^2\to\R$ de classe $\classec{1}$ telles que
  \[\forall\p{x,y}\in\R^2 \qsep \parfrac{f}{x} \p{x,y}=0.\]
\exo Déterminer l'ensemble des fonctions $f:\R^2\to\R$ de classe $\classec{1}$ telles que
  \[\forall\p{x,y}\in\R^2 \qsep \parfrac{f}{y} \p{x,y}= x+y.\]
  \begin{sol}
  On trouve $f\p{x,y}=y^2/2+xy+\phi(x)$ où $\phi$ est une fonction de classe
  $\classec{1}$ quelconque sur $\R$.
  \end{sol}
\end{exoUnique}

\subsection{Développement limité, gradient}

\begin{definition}
Soit $f:\mathcal{U}\subset\R^2\to\R$ une fonction définie en $(0,0)$. On dit que $f(h)$ est \emph{négligeable} devant $\norme{h}$ en $\p{0,0}$ lorsque
\[\forall\epsilon>0 \qsep \exists \eta>0 \qsep \forall h\in\mathcal{U}
  \qsep \norme{h}\leq\eta \implique \abs{f(h)}\leq\epsilon\norme{h}.\]
Si tel est le cas, on note
\[f(h)=\petito{h}{\p{0,0}}{\norme{h}}.\]
\end{definition}

% \begin{remarqueUnique}
% \remarque Si $f(h)$ est négligeable devant $\norme{h}$ en $(0,0)$, alors $f(h)$ tend vers 0 lorsque $h$ tend vers
%   $(0,0)$ et $f$ admet une dérivée nulle selon tout vecteur en $(0,0)$.
% \end{remarqueUnique}

\begin{proposition}
Soit $f:\mathcal{U}\subset\R^2\to\R$ une fonction de classe $\classec{1}$ et
$x\in\mathcal{U}$. Alors, en notant $h=\p{h_1,h_2}$
\[f\p{x+h}=f(x)+\parfrac{f}{x_1}(x)h_1+\parfrac{f}{x_2}(x)h_2+
  \petito{h}{\p{0,0}}{\norme{h}}.\]
\end{proposition}

\begin{remarqueUnique}
\remarque Si $f:\mathcal{U}\subset\R^2\to\R$ est une fonction de classe $\classec{1}$ et $(x_0,y_0)\in\mathcal{U}$ alors,
  le plan d'équation
  \[z=f(x_0,y_0)+\parfrac{f}{x}(x_0,y_0)(x-x_0)+\parfrac{f}{y}(x_0,y_0)(y-y_0)\]
  est appelé \emph{plan tangent} en $(x_0,y_0)$ à la surface d'équation $z=f(x,y)$.
\end{remarqueUnique}

\begin{proposition}
Soit $f:\mathcal{U}\subset\R^2\to\R$ une fonction de classe $\classec{1}$.
Alors $f$ est continue.
\end{proposition}

\begin{proposition}
Soit $f:\mathcal{U}\subset\R^2\to\R$ une fonction de classe $\classec{1}$ et
$x\in\mathcal{U}$. Alors $f$ admet une dérivée en $x$ selon tout vecteur
$h\defeq\p{h_1,h_2}$ et
\[{\rm D}f_x(h)=\parfrac{f}{x_1}(x)h_1 
  +\parfrac{f}{x_2}(x)h_2.\]
Autrement dit, ${\rm D}f_x$ est une forme linéaire sur $\R^2$.
\end{proposition}



\begin{exoUnique}
\exo Soit $f$ la fonction définie sur $\R^2$ par
  \[\forall\p{x,y}\in\R^2 \qsep f\p{x,y}\defeq\frac{xy}{1+x^2}.\]
  Montrer que $f$ est de classe $\classec{1}$ et calculer
  $\text{D}f_{\p{x,y}}\p{h_1,h_2}$.
  \begin{sol}
  On trouve
  \[\text{d}f_{\p{x,y}}\p{h_1,h_2}=y\cdot\frac{1-x^2}{\p{1+x^2}^2}h_1+
    \frac{x}{1+x^2}h_2\]
  ce que l'on écrit aussi
  \[\text{d}f=y\cdot\frac{1-x^2}{\p{1+x^2}^2}\text{d}x+
    \frac{x}{1+x^2}\text{d}y\]
  \end{sol}
\end{exoUnique}

\begin{definition}
Soit $f:\mathcal{U}\subset\R^2\to\R$ une fonction de classe $\classec{1}$ et
$x\in\mathcal{U}$. On appelle \emph{gradient} de $f$ en $x$ l'unique vecteur
$\nabla f(x)\in\R^2$ tel que
\[\forall h\in\R^2 \qsep {\rm D}f_{x}(h)=\ps{\nabla f(x)}{h}.\]
Autrement dit
\[\nabla f(x)=\p{\parfrac{f}{x_1}(x),\parfrac{f}{x_2}(x)}.\]
\end{definition}

\begin{remarques}
\remarque Le symbole $\nabla$ est un delta majuscule inversé. Il se prononce \og nabla \fg en référence au nom grec désignant une harpe phénicienne.
\remarque Pour tout $\theta\in\R$, on pose $u_\theta\defeq(\cos\theta,\sin\theta)$. Alors,
d'après Cauchy-Schwarz
\[\forall \theta\in\R\qsep {\rm D}f_x(u_\theta)=\ps{\nabla f(x)}{u_\theta} \leq \norme{\nabla f(x)}\norme{u_\theta}=\norme{\nabla f(x)}\]
cette inégalité étant une égalité si et seulement si $\nabla f(x)$ et $u_\theta$ sont
positivement liés. On en déduit que $\nabla f(x)$ donne la direction dans laquelle
$f$ croît le plus vite.
\end{remarques}


\subsection{Dérivation des fonctions composées}

\begin{proposition}
Soit $f:\mathcal{U}\subset\R^2\to\R$ et $g:\mathcal{V}\subset\R\to\R$ deux
  fonctions de classe $\classec{1}$ telles que
  $f\p{\mathcal{U}}\subset\mathcal{V}$. Alors $g\circ f$ est de classe
  $\classec{1}$ et
  \[\forall x\in\mathcal{U} \qsep
    \cro{\nabla (g\circ f)} (x)=g'\p{f(x)}\nabla f(x).\]
  On écrit aussi, de manière abusive
  \[\nabla\p{g\circ f}=\frac{{\rm d}g}{{\rm d}f}\nabla f.\]
\end{proposition}


\begin{exoUnique}
\exo Calculer $\nabla f$, où $f$ est définie sur
  $\R^2\setminus\ens{\p{0,0}}$ par
  \[f\p{x,y}\defeq\frac{1}{\sqrt{x^2+y^2}}.\]
  \begin{sol}
  On trouve
  \[\nabla\p{\frac{1}{r}}=-\frac{1}{r^2}\ve{u_\theta}\]
  \end{sol}
\end{exoUnique}

% \begin{definition}
% Soit $p,q\in\intere{1}{2}$ et $f:\mathcal{U}\subset\R^p\to\R^q$. On dit que $f$
% est de classe $\classec{1}$ sur $\mathcal{U}$ lorsque quels que soient
% $j\in\intere{1}{p}$ et $i\in\intere{1}{q}$, l'application
% \[\parfrac{f_i}{x_j}\]
% est définie et continue sur $\mathcal{U}$. Si tel est la cas, on définit pour
% tout $x\in\mathcal{U}$ la \emph{matrice jacobienne} de $f$ en $x$ par
% \[J_f(x)\defeq \p{\parfrac{f_i}{x_j}}_{\substack{1\leq i\leq q\\1\leq j\leq p}}
%   \in\mat{q,p}{\R}\]
% \end{definition}

% \begin{exoUnique}
% \exo Soit $f$ la fonction définie sur $\R^2$ par
%   \[\forall\p{r,\theta}\in\R^2 \qsep f\p{r,\theta}\defeq \p{r\cos\theta,r\sin\theta}.\]
%   Calculer la jacobienne de $f$ en $\p{r,\theta}$.
% \end{exoUnique}

% \begin{proposition}
% Soit $p,q,r\in\intere{1}{2}$, $f:\mathcal{U}\subset\R^p\to\R^q$ et
% $g:\mathcal{V}\subset\R^q\to\R^r$ tels que $f\p{\mathcal{U}}\subset\mathcal{V}$.
% Si $f$ et $g$ sont $\classec{1}$, alors $g\circ f$ est $\classec{1}$ et
% \[J_{g\circ f}(x)=J_g\p{f(x)}J_f(x).\]
% \end{proposition}

\begin{proposition}
Soit $f:\mathcal{U}\subset\R\to\R^2$ et $g:\mathcal{V}\subset\R^2\to\R$ deux
fonctions de classe $\classec{1}$ telles que
$f\p{\mathcal{U}}\subset\mathcal{V}$. Alors $g\circ f$ est de classe
$\classec{1}$ et
\begin{eqnarray*}
\forall t\in\mathcal{U} \qsep \p{g\circ f}'(t)
&=&\parfrac{g}{x_1}\p{f(t)}f_1'(t)+\parfrac{g}{x_2}\p{f(t)}f_2'(t)\\
&=&\ps{\nabla g \p{f(t)}}{f'(t)}
\end{eqnarray*}
On écrit aussi, de manière abusive
\begin{eqnarray*}
\frac{{\rm d}\p{g\p{f_1,f_2}}}{{\rm d} t}
&=& \parfrac{g}{f_1}\frac{{\rm d}f_1}{{\rm d} t}+
    \parfrac{g}{f_2}\frac{{\rm d}f_2}{{\rm d} t}\\
&=& \ps{\nabla g}{\frac{{\rm d}f}{{\rm d}t}}
\end{eqnarray*}
\end{proposition}

\begin{exoUnique}
\exo Soit $f:\R^2\to\R$ une fonction de classe $\classec{1}$ et
  $\phi$ la fonction définie sur $\R$ par
  \[\forall t\in\R \qsep \phi(t)\defeq f\p{t^2,t^3}.\]
  Montrer que $\phi$ est de classe $\classec{1}$ et calculer $\phi'(t)$.
\end{exoUnique}

\begin{proposition}
Soit $f:\mathcal{U}\subset\R^2\to\R$ une fonction de classe $\classec{1}$ et
$M:I\to\R^2$ une fonction de classe $\classec{1}$ à valeurs dans 
$\mathcal{U}$. On suppose qu'il existe $c\in\R$ tel que
\[\forall t\in I \qsep f\p{M(t)}=c.\]
Alors, pour tout $t\in I$, $M'(t)$ est orthogonal au gradient
de $f$ en $M(t)$.
\end{proposition}

\begin{remarqueUnique}
\remarque Si $f:\mathcal{U}\subset\R^2\to\R$, on appelle \emph{ligne de niveau} de hauteur $c\in\R$
  l'ensemble
  \[\mathcal{L}_c\defeq\enstq{(x,y)\in\mathcal{U}}{f(x,y)=c}.\]
  La proposition précédente nous assure que le gradient de $f$ est orthogonal à ses lignes de niveau.
\end{remarqueUnique}

\begin{proposition}
Soit $f:\mathcal{U}\subset\R^2\to\R^2$ et $g:\mathcal{V}\subset\R^2\to\R$ deux
fonctions de classe $\classec{1}$ telles que
$f\p{\mathcal{U}}\subset\mathcal{V}$. Alors $g\circ f$ est de classe
$\classec{1}$ et
\[\forall x\in\mathcal{U} \qsep
  \begin{cases}
  \dsp\parfrac{\p{g\circ f}}{x_1}(x) = \dsp\parfrac{g}{y_1}\p{f(x)}
  \dsp\parfrac{f_1}{x_1}(x)+\dsp\parfrac{g}{y_2}\p{f(x)}
  \dsp\parfrac{f_2}{x_1}(x) &\\
  \dsp\parfrac{\p{g\circ f}}{x_2}(x) = \dsp\parfrac{g}{y_1}\p{f(x)}
  \dsp\parfrac{f_1}{x_2}(x)+\dsp\parfrac{g}{y_2}\p{f(x)}
  \dsp\parfrac{f_2}{x_2}(x) &
  \end{cases}\]
  On écrit aussi, de manière abusive
  \[\dsp\parfrac{\p{g\p{f_1,f_2}}}{x_1}=\parfrac{g}{f_1}\parfrac{f_1}{x_1}
    +\parfrac{g}{f_2}\parfrac{f_2}{x_1} \et
    \dsp\parfrac{\p{g\p{f_1,f_2}}}{x_2}=\parfrac{g}{f_1}\parfrac{f_1}{x_2}
    +\parfrac{g}{f_2}\parfrac{f_2}{x_2}.\]
\end{proposition}

\begin{exos}
\exo Soit $f:\R^2\to\R$ une fonction de classe $\classec{1}$. Montrer que la fonction
  $h:\R^2\to\R$ définie par
  \[\forall (x,y)\in\R^2\qsep h\p{x,y}\defeq f\p{2xy,x}\]
  est de classe $\classec{1}$ et calculer ses dérivées partielles.
  \begin{sol}
  On trouve
   \[\parfrac{h}{x}\p{x,y}=2y\parfrac{f}{x}\p{2xy,x}+\parfrac{f}{y}\p{2xy,x}
     \et \parfrac{h}{y}\p{x,y}=2x\parfrac{f}{x}\p{2xy,x}\]
  \end{sol}
% \exo Calcul en coordonnées polaires. On trouve
%   \[\syslin{\parfrac{}{r}&=&\cos\theta\parfrac{}{x}+&\sin\theta\parfrac{}{y}\cr
%     \frac{1}{r}\parfrac{}{\theta}&=&-\sin\theta\parfrac{}{x}+&
%     \cos\theta\parfrac{}{y}}\]
%   Donc
%   \[\syslin{\parfrac{}{x}&=&\cos\theta\parfrac{}{r}-&\sin\theta\frac{1}{r}
%     \parfrac{}{\theta}\cr
%     \parfrac{}{y}&=&\sin\theta\parfrac{}{r}+&
%     \cos\theta\frac{1}{r}\parfrac{}{\theta}}\]
%   Donc
%   \[\nabla f=\parfrac{f}{r}\ve{u_\theta}+\frac{1}{r}\parfrac{f}{\theta}
%     \ve{v_\theta}\]
%   En particulier
%   \[\nabla\p{\frac{1}{r}}=-\frac{1}{r^2}\ve{u_\theta}\]
\exo Soit $\p{a,b}\in\R^2\setminus\ens{\p{0,0}}$. On souhaite déterminer les fonctions
  $f:\R^2\to\R$ de classe $\classec{1}$ telles que
  \[(E) \quad \forall\p{x,y}\in\R^2 \qsep
    a\parfrac{f}{x}\p{x,y}+b\parfrac{f}{y}\p{x,y}=0.\]
  \begin{questions}
  \question Montrer que l'application
    \[\dspappli{\phi}{\R^2}{\R^2}{(u,v)}{(au-bv,bu+av)}\]
    est bijective et calculer $\phi^{-1}$.
  \question Soit $f:\R^2\to\R$ une fonction de classe $\classec{1}$. On définit
    la fonction $g:\R^2\to\R$ par
    \[\forall(u,v)\in\R^2\qsep g(u,v)\defeq f(au-bv,bu+av).\]
    Montrer que $f$ est solution de $(E)$ si et seulement si
    \[\forall(u,v)\in\R^2\qsep \parfrac{g}{u}(u,v)=0.\]
  \question Conclure.
  \end{questions}
  % On pourra introduire la fonction $g:\R^2\to\R$ définie par
  \begin{sol}%
  On fait le changement de variable
  $g\p{u,v}=f\p{au-bv,bu+av}$. On trouve $\parfrac{g}{u}=0$, donc
  $g=\phi\p{v}$ donc $f=\psi\p{bx-ay}$.    
  \end{sol}
\end{exos}



% \begin{exemples}
% \exemple Soit $\mathcal{E}$ l'ellipse de foyer $F_1$ et $F_2$ et de demi-grand
%   axe $a$. Si $M$ est un point de $\mathcal{E}$ la tangente en $M$ est
%   la bissectrice extérieure des demi-droites $[F_1M)$ et $[F_2M)$.
% \end{exemples}

\subsection{Extrémum d'une fonction de deux variables}

\begin{definition}
Soit $f:\mathcal{U}\subset\R^2\to\R$ et $x_0\in\mathcal{U}$. On dit que
\begin{itemize}
\item $f$ présente un \emph{maximum global} en $x_0$ lorsque
  \[\forall x\in\mathcal{U} \qsep f(x)\leq f\p{x_0}.\]
\item $f$ présente un \emph{maximum local} en $x_0$ lorsque
  \[\exists r>0\qsep \forall x\in\mathcal{U} \qsep \norme{x-x_0}\leq r \implique
    f(x)\leq f\p{x_0}.\]
\item $f$ présente un \emph{minimum global} en $x_0$ lorsque
  \[\forall x\in\mathcal{U} \qsep f(x)\geq f\p{x_0}.\]
\item $f$ présente un \emph{minimum local} en $x_0$ lorsque
  \[\exists r>0\qsep \forall x\in\mathcal{U} \qsep \norme{x-x_0}\leq r \implique
    f(x)\geq f\p{x_0}.\]
\end{itemize}
\end{definition}

\begin{remarques}
\remarque Un extrémum global est un extrémum local, la réciproque étant fausse en général.
\remarque Si $f:\R^2\to\R$ admet un minimum local en
  $(x_1,x_2)\in\R^2$, alors les fonctions partielles
  \[\dspappli{f_{p_1}}{\R}{\R}{t}{f(t,x_2)} \quad\et\quad
    \dspappli{f_{p_2}}{\R}{\R}{t}{f(x_1,t)}\]
  admettent un minimum local, respectivement en $x_1$ et $x_2$.
  Cependant, la réciproque est fausse comme le montre l'exemple de la fonction
  $f:\R^2\to\R$ définie par
  \[\forall (x,y)\in\R^2\qsep f(x,y)=x^2-3xy+y^2.\]
\remarque Plus généralement, si $f:\R^2\to\R$ admet un minimul local en $x_0\in\R^2$ alors,
  quel que soit $h\in\R^2$, la fonction $\phi:\R\to\R$ définie par
  \[\forall t\in\R\qsep \phi(t)\defeq f(x_0+th)\]
  admet un minimum local en $0$. 
\end{remarques}

\begin{definition}
Soit $f:\mathcal{U}\subset\R^2\to\R$ une fonction de classe $\classec{1}$. On dit que $x\in\mathcal{U}$
est un point \emph{critique} de $f$ lorsque
\[\parfrac{f}{x_1}\p{x}=0 \et \parfrac{f}{x_2}\p{x}=0.\]
\end{definition}

\begin{proposition}
Soit $f:\mathcal{U}\subset\R^2\to\R$ une fonction de classe $\classec{1}$ définie sur un ouvert $\mathcal{U}$.
Si $f$ présente un extrémum local en $x_0$ alors $x_0$ est un point critique pour $f$.
\end{proposition}

\begin{remarques}
\remarque Les extrémums locaux d'une fonction $f:\mathcal{U}\subset\R^2\to\R$ de classe $\classec{1}$ définie
  sur un ouvert $\mathcal{U}$ sont donc à chercher parmi les points critiques de $f$.
\remarque La réciproque est fausse. Par exemple la fonction $f:\R^2\to\R$ définie par
  \[\forall (x,y)\in\R^2\qsep f(x,y)\defeq x^2-y^2\]
  admet $(0,0)$ pour point critique alors qu'elle ne présente pas d'extrémum local en $(0,0)$.
\end{remarques}

\begin{exoUnique}
\exo Soit $f$ la fonction définie sur $\R^2$ par
  \[\forall\p{x,y}\in\R^2 \qsep f\p{x,y}\defeq -x^2-6xy+9y^2+18x-18y+1.\]
  Déterminer les extrémums éventuels de $f$.
  \begin{sol}
  On trouve un unique point critique $x=3$ et $y=2$. On pose $x=3+u$ et
  $y=2+v$. On a $f\p{3+u,2+v}=10+9v^2-6uv-u^2=\p{3v-u}^2-2u^2$. On est en
  un point selle.
  \end{sol}
\end{exoUnique}



%END_BOOK

\end{document}