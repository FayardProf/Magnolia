\documentclass{magnolia}

\magtex{tex_driver={pdftex},
        tex_packages={epigraph,xypic}}
\magfiche{document_nom={Continuité, limites},
          auteur_nom={François Fayard},
          auteur_mail={fayard.prof@gmail.com}}
\magcours{cours_matiere={maths},
          cours_niveau={mpsi},
          cours_chapitre_numero={10},
          cours_chapitre={Continuité, limites}}
\magmisenpage{}
\maglieudiff{}
\magprocess

\begin{document}

%BEGIN_BOOK
\setlength\epigraphwidth{.75\textwidth}
\epigraph{\og Un mathématicien est une machine à transformer le café en théorèmes.\fg}{--- \textsc{Paul Erdös (1913-1996)}}
\setlength\epigraphwidth{.65\textwidth}
\epigraph{\og Si les gens ne croient pas que les mathématiques sont simples, c’est seulement parce qu’ils ne réalisent pas combien la vie est compliquée !\fg}{--- \textsc{John Von Neumann (1903--1957)}}

\magtoc

\section{Fonction numérique, topologie élémentaire}

\begin{definition}[utile=-3]
On appelle \emph{fonction numérique} toute fonction définie sur une partie $\mathcal{D}$
de $\R$, à valeurs dans $\R$ ou $\C$.
\end{definition}

\begin{remarques}
\remarque Dans la suite de ce chapitre, $\K$ désignera le corps $\R$ ou $\C$.
\remarque Il arrive que l'on définisse une fonction par son expression \og $f(x)$ \fg. C'est
  alors au lecteur de déterminer son domaine de définition, c'est-à-dire l'ensemble des
  $x\in\R$ pour lesquels \og $f(x)$ \fg a un sens.
\end{remarques}

\begin{definition}[utile=-3]
Soit $f:\mathcal{D}\to\K$ et $A$ une partie de $\mathcal{D}$. On dit que $f$
\emph{vérifie la propriété $\mathcal{P}$ sur $A$} lorsque la restriction de $f$ à $A$ vérifie
la propriété $\mathcal{P}$.
\end{definition}

% \subsection{Symétries, fonctions lipschitziennes}

% \begin{definition}[utile=-3]
% Soit $f$ une fonction dont le domaine de définition est symétrique par rapport
% à 0, c'est-à-dire tel que
% \[\forall x\in\mathcal{D} \qsep -x\in\mathcal{D}.\]
% On dit que $f$ est
% \begin{itemize}
% \item \emph{paire} lorsque
%   \[\forall x\in\mathcal{D} \qsep f\p{-x}=f(x).\]
% \item \emph{impaire} lorsque
%   \[\forall x\in\mathcal{D} \qsep f\p{-x}=-f(x).\]
% \end{itemize}
% \end{definition}

% % \begin{remarques}
% % \remarque Si $f$ est une fonction dérivable paire, alors $f'$ est impaire. De
% %   même, si $f$ est une fonction dérivable  impaire, $f'$ est paire.
% % \remarque Si $f$ est une fonction impaire (resp. paire) continue sur $\R$. Que
% %   peut-on dire de ses primitives~?
% % \end{remarques}

% %% Si f est impaire, F est paire.
% %% Si f est paire, seule la primitive s'annulant en 0 est impaire.

% \begin{definition}[utile=-3]
% Soit $T\in\R$ et $f$ une fonction dont le domaine de définition vérifie
% \[\forall x\in\mathcal{D} \qsep \forall k\in\Z \qsep x+kT\in\mathcal{D}\]
% On dit que $f$ est \emph{$T$-périodique}, ou que $T$ est une \emph{période} de $f$, lorsque
% \[\forall x\in\mathcal{D} \qsep f\p{x+T}=f(x).\]
% Lorsque $f$ admet une période non nulle, on dit que $f$ est \emph{périodique}.
% \end{definition}

% % \begin{remarques}
% % \remarque L'ensemble $G$ des périodes d'une fonction $f$ est un sous-groupe de
% %   $\p{\R,+}$. En particulier, si $T$ est une période de $f$ et si $k\in\Z$,
% %   alors $kT$ est une période de $f$.
% %   Lorsque $f$ admet une plus petite période strictement positive, notée
% %   $T_0$, ont dit que $T_0$ est la période de $f$. On montre dans ce cas
% %   que $G=T_0\Z$.
% % \remarque Si $f$ est dérivable et $T$-périodique, il en est de même pour $f'$.
% %   Cependant, il est possible que $f'$ soit $T$-périodique sans que $f$ le soit.
% % \end{remarques}

% % \begin{exos}
% % \exo Les périodes de $\tan$ sont les $k\pi$, pour $k\in\Z$. Donc $\tan$
% %   est périodique et sa période est $\pi$.
% % \end{exos}

% \begin{definition}[utile=-3]
% Soit $k\in\RP$. On dit qu'une fonction $f$ est \emph{$k$-lipschitzienne} lorsque
% \[\forall x,y\in \mathcal{D} \qsep \abs{f(x)-f(y)}\leq k\abs{x-y}.\] 
% \end{definition}

% \begin{exoUnique}
% \exo Montrer que les fonctions \og valeur absolue\fg et \og sinus\fg sont
%   1-lipschitziennes.
% % \exo La fonction valeur absolue est $1$-lipschitzienne. En effet
% %   \[\forall x,y\in\R \qsep \abs{\abs{x}-\abs{y}}\leq\abs{x-y}\]
% % \exo La fonction $\sin$ est $1$-lipschitzienne.
% \end{exoUnique}

% % \begin{remarques}
% % \remarque Si $f$ est une fonction dérivable et $k$-lipschitzienne,
% %   alors
% %   \[\forall x\in\mathcal{D} \qsep \abs{f'(x)}\leq k\]
% %   Réciproquement, nous verrons que si $f$ est dérivable sur un intervalle $I$
% %   et que
% %   \[\forall x\in I \qsep \abs{f'(x)}\leq k\]
% %   alors $f$ est $k$-lipschitzienne. Parmi les fonctions dérivable sur un
% %   intervalle $I$, les fonctions lipschitziennes sont donc les fonctions dont
% %   la dérivée est bornée.
% % \remarque L'ensemble des fonctions lipschitziennes est un sous-espace vectoriel
% %   de $\mathcal{F}\p{\R,\R}$, mais ce n'est pas une sous-algèbre. En effet, la
% %   fonction $f:x\mapsto x$ est lipschitzienne, mais $f^2$ ne l'est pas.
% % \end{remarques}




\subsection{Propriété locale}

% \begin{definition}[utile=-3]
% \begin{itemize}
% \item Soit $a\in\R$. On dit qu'une partie $\mathcal{V}$ de $\R$ est un \emph{voisinage
%   fondamental} de $a$ lorsqu'il existe $\epsilon>0$ tel que 
%   \[\mathcal{V}=\enstq{x\in\R}{\abs{x-a}\leq\epsilon} 
%     =\interf{a-\epsilon}{a+\epsilon}.\]
% \item On dit qu'une partie $\mathcal{V}$ de $\R$ est un \emph{voisinage fondamental} de $+\infty$
%   lorsqu'il existe $m\in\R$ tel que
%   \[\mathcal{V}=\interfo{m}{+\infty}.\]
% \item On dit qu'une partie $\mathcal{V}$ de $\R$ est un \emph{voisinage fondamental} de $-\infty$
%   lorsqu'il existe $M\in\R$ tel que
%   \[\mathcal{V}=\interof{-\infty}{M}.\]
% \item Soit $a\in\C$. On dit qu'une partie $\mathcal{V}$ de $\C$ est un \emph{voisinage
%   fondamental} de $a$ lorsqu'il existe $\epsilon>0$ tel que
%   \[\mathcal{V}=\enstq{z\in\C}{\abs{z-a}\leq\epsilon}.\]
% \end{itemize}
% \end{definition}

% \begin{remarques}
% \remarque Une intersection de deux voisinages fondamentaux d'un même élément est un voisinage fondamental. Plus
%   généralement, une intersection d'un nombre fini de voisinages fondamentaux est un
%   voisinage fondamental.
% \begin{sol}
% Attention, c'est faux pour une infinité de voisinages : $\displaystyle \bigcap_{n\in \Ns}\intero{a-\frac{1}{n}}{a+\frac{1}{n}}$.
% \end{sol}
% \remarque Une suite réelle (ou complexe) $\p{u_n}$ converge vers
%   $l\in\RbarouC$ si
%   et seulement si pour tout voisinage fondamental $\mathcal{V}$ de $l$, il existe un
%   voisinage fondamental $\mathcal{W}$ de $+\infty$ tel que
%   \[\forall n\in\N \qsep n\in\mathcal{W} \implique u_n\in\mathcal{V}.\]
%   \begin{sol}
%   C'est la définition de la limite pour chacun d'entre eux.
%   \end{sol}
% \remarque Une partie $A$ de $\R$ est dense dans $\R$ si et seulement si pour
%   tout $x\in\R$ et tout voisinage $\mathcal{V}$ de $x$,
%   $A\cap\mathcal{V}\neq\emptyset$.
% \remarque En seconde année, vous verrez une autre définition des voisinages
%   d'un élément $a\in\Rbar$ (ou $\C$). Les voisinages, tels qu'ils sont définis
%   dans ce cours, seront toujours des voisinages l'année prochaine, mais la
%   réciproque est fausse.
% sont des voisinages tels qu'ils seront définis l'année prochaine,
%   mais la réciproque est fausse.
% \end{remarques}

% \begin{definition}[utile=-3]
% Soit $a\in\R$. On dit qu'une partie $\mathcal{V}$ de $\R$ est~:
% \begin{itemize}
% \item un voisinage à droite de $a$ lorsqu'il existe $\epsilon>0$ tel que~:
%   \[\interof{a}{a+\epsilon} \subset \mathcal{V}\]
% \item un voisinage à gauche de $a$ lorsqu'il existe $\epsilon>0$ tel que~:
%   \[\interfo{a-\epsilon}{a} \subset \mathcal{V}\]
% \end{itemize}
% \end{definition}

% \begin{exos}
% \exo $\RPs$ est un voisinage à droite de 0.
% \end{exos}

% \begin{proposition}
% L'intersection d'un nombre fini de voisinages fondamentaux de $a\in\RbarouC$ est un voisinage fondamental.
% \end{proposition}

\begin{definition}[utile=-3]
On dit que $f:\mathcal{D}\to\K$ est \emph{définie au voisinage}
de $a\in\Rbar$ lorsque, pour tout voisinage $\mathcal{V}$ de $a$, $\mathcal{D}\cap\mathcal{V}\neq\emptyset$.
\end{definition}

\begin{proposition}[utile=-3]
Une fonction $f:\mathcal{D}\to\K$ est définie au voisinage de $a\in\Rbar$ si et seulement si il existe une
suite d'éléments de $\mathcal{D}$ qui tend vers $a$.
\end{proposition}

\begin{preuve}
On va faire le cas où $a\in \R$, les autres se traitent de la même façon.
\begin{itemize}
\item[$\bullet$] Si $f$ est définie au voisinage de $a$, alors pour tout voisinage $\mathcal{V}$ de $a$, $\mathcal{D}\cap\mathcal{V}\neq\emptyset$. En particulier, pour chaque $n\in \N$, on peut construire $u_n \in \mathcal{D}\cap \interf{a-\frac{1}{n}}{a+\frac{1}{n}}$. $(u_n)$ est une suite d'éléments de $\mathcal{D}$ qui converge vers $a$.
\item[$\bullet$] Réciproquement, s'il existe une suite $(u_n) \in \mathcal{D}^\N$ qui converge vers $a$, alors pour chaque voisinage $\mathcal{V}$ de $a$, il existe $n\in N$ tel que $\forall n \geq N$, $u_n \in \mathcal{V}$. En particulier $u_N \in \mathcal{D}\cap\mathcal{V}$ qui est donc différent du vide.
\end{itemize}
\end{preuve}

\begin{definition}[utile=-3]
On dit que $f:\mathcal{D}\to\K$ vérifie la propriété $\mathcal{P}$ \emph{au voisinage de $a\in\Rbar$}
lorsqu'il existe un voisinage $\mathcal{V}$ de $a$ tel que $f$ vérifie la propriété $\mathcal{P}$ sur $\mathcal{D}\cap\mathcal{V}$.
% \begin{itemize}
% \item On dit que $f$ est définie au voisinage de $a$ lorsque tout voisinage
%   de $a$ admet une intersection non vide avec $\mathcal{D}$. 
% \item On dit que $f$ vérifie la propriété $\mathcal{P}$ au voisinage de $a$
%   lorsqu'il existe un voisinage $\mathcal{V}$ de $a$ tel que la restriction
%   de $f$ à $\mathcal{D}\cap\mathcal{V}$ vérifie la propriété $\mathcal{P}$.
% \end{itemize}
\end{definition}

%% Remarque :
%% 1) Si f vérifie une propriété P_1  au voisinage de a et vérifie une
%%    propriété P_2 au voisinage de a, alors f vérifie la propriété P_1 et P_2
%%    au voisinage de a.
%% 2) De même, si a\in\R, on dit que f est définie au voisinage à gauche
%%    (resp. à droite) de a lorsque tout voisinage à gauche (resp. à droite) de
%%    a rencontre \mathcal{D}.
%%


\begin{remarques}
\remarque La fonction $\sin$ est croissante sur $\interf{-\pi/2}{\pi/2}$. Elle
  est donc croissante au voisinage de 0.
\remarque Une fonction $f:\mathcal{D}\to\K$ est bornée au voisinage de $a\in\R$ si et seulement si
   \[\exists \eta>0 \qsep \exists M\in\RP \qsep \forall x\in\mathcal{D}
     \qsep \abs{x-a}\leq\eta \implique \abs{f(x)}\leq M.\]
  Une fonction $f:\mathcal{D}\to\K$ est bornée au voisinage de $+\infty$ si et seulement si
   \[\exists m\in\R \qsep \exists M\in\RP \qsep \forall x\in\mathcal{D}
     \qsep x\geq m \implique \abs{f(x)}\leq M.\]
% \exo La fonction $\appli{f}{\RPs}{\R}{x}{1/x}$ est bornée au voisinage de
%   $+\infty$ mais n'est pas bornée sur son ensemble de définition.
\end{remarques}


\begin{definition}[utile=-3]
On dit qu'une propriété $\mathcal{P}$ est \emph{locale en $a\in\Rbar$} lorsque, quelles que
soient les fonctions $f:\dom\to\R$ et $g:\dom\to\R$ définies au voisinage de $a$, si il
existe un voisinage $\mathcal{V}$ de $a$ tel que
\[\forall x\in\mathcal{D}\cap\mathcal{V} \qsep g(x)=f(x)\]
alors $\mathcal{P}(f)$ est vrai si et seulement si $\mathcal{P}(g)$ est vrai.
\end{definition}


\begin{definition}
On dit qu'un élément $x\in \mathcal{D}$ est \emph{intérieur} à $\mathcal{D}$ lorsqu'il existe un voisinage
$\mathcal{V}$ de $a$ tel que $\mathcal{V}\subset\mathcal{D}$. 
\end{definition}

\begin{remarqueUnique}
\remarque Si $I$ est un intervalle, un élément $a\in I$ est intérieur à $I$ si et seulement si ce n'est pas
  une de ses extrémités.
\end{remarqueUnique}
% \begin{remarques}
% \remarque Toute personne comprenant les subtilités de cette définition gagnera
%   une sucette géante.
% \end{remarques}
%% Remarque :
%% 1) Locale à gauche, à gauche au sens large (dérivation)

% \subsection{Ouvert}

% \begin{preuve}
% \begin{itemize}
% \item[$\bullet$] On suppose que $\alpha=\sup A$. Pour chaque $n\geq 1$, $s-1/n$ ne majore pas $A$, on peut donc fixer $u_n\in A$ tel que $s-1/n<u_n(\leq s)$. Ainsi, $\forall n\in \Ns, |u_n-s|\leq 1/n$. La suite $u$ ainsi construite converge donc vers $s$.
% \item[$\bullet$] Réciproquement, si $\alpha$ est un majorant de $A$. Soit $\epsilon>0$. Il existe $N\in \N$ tel que $\forall n\geq N$, $|u_n-\alpha|\leq \epsilon$. En particulier $\alpha-\epsilon \leq u_N$. Donc $\alpha$ est bien la borne supérieure de $A$ d'après la caractérisation de la borne supérieure.
% \end{itemize}
% \end{preuve}

% \begin{remarqueUnique}
% \remarque On a bien évidemment une proposition similaire avec la borne inférieure.
% \end{remarqueUnique}


% \begin{definition}[utile=-3]
% On dit qu'une partie $O$ de $\R$ est un \emph{ouvert} lorsque c'est une réunion
% d'intervalles ouverts.
% \end{definition}

% \begin{remarqueUnique}
% \remarque Un intervalle $I$ est ouvert si et seulement si ses extremités ne sont pas des
%   éléments de $I$.
% \end{remarqueUnique}

% \begin{definition}[utile=-3]
% On dit qu'un réel $x_0$ est \emph{intérieur} à une partie $A$ de $\R$ lorsqu'il existe
% $\eta>0$ tel que $[x_0-\eta,x_0+\eta]\subset A$.
% \end{definition}

% \begin{remarques}
% \remarque Si $x_0$ est intérieur à $A$, alors $x_0\in A$. Cependant, la
%   réciproque est fausse. Par exemple $0\in\ens{0}$ mais 0 n'est pas intérieur
%   à $\ens{0}$.
% \remarque Si $A$ est un ouvert, tous les points de $A$ sont intérieurs à $A$.
% \end{remarques}

% \begin{remarques}
% \remarque Si la fonction $f$ est définie sur un intervalle $I$, $f$ est définie
%   au voisinage de tous les points de $I$ et de ses bornes.
% \end{remarques}

% \subsection{Théorème de \nom{Bolzano-Weierstrass}}




\section{Limite}

\subsection{Définition, propriétés élémentaires}

% \begin{definition}[utile=-3]
% Soit $f$ une fonction définie au voisinage de $a\in\Rbar$ et $l\in\RbarouC$. On
% dit que $f(x)$ tend vers $l$ lorsque $x$ tend vers $a$ lorsque pour tout
% voisinage $\mathcal{W}$ de $l$, il existe un voisinage $\mathcal{V}$ de $a$ tel
% que~:
% \[\forall x\in\mathcal{D}\cap\mathcal{V} \qsep f(x)\in\mathcal{W}\]
% Si tel est le cas, on note~:
% \[f(x)\tendvers{x}{a} l\]
% La propriété \og tend vers $l$ lorsque $x$ tend vers $a$ \fg est locale en $a$.
% \end{definition}

\begin{definition}[utile=-3]
Soit $f:\mathcal{D}\to\K$ une fonction définie au voisinage de $a\in\Rbar$ et $l\in\K$.
On dit que $f(x)$ \emph{tend} vers $l$ lorsque $x$ tend vers $a$ et on note
\[f(x)\tendvers{x}{a} l\]
lorsque, quel que soit le voisinage $\mathcal{W}$ de $l$, il existe un voisinage
$\mathcal{V}$ de $a$ tel que
\[\forall x\in\dom\qsep x\in\mathcal{V}\implique f(x)\in\mathcal{W}.\]
La propriété \og tend vers $l$ lorsque $x$ tend vers $a$ \fg est locale en $a$.
% lorsque pour tout
% voisinage $\mathcal{W}$ de $l$, il existe un voisinage $\mathcal{V}$ de $a$ tel
% que~:
% \[\forall x\in\mathcal{D}\cap\mathcal{V} \qsep f(x)\in\mathcal{W}\]
% Si tel est le cas, on note~:

% La propriété \og tend vers $l$ lorsque $x$ tend vers $a$ \fg est locale en $a$.


% Alors~:
% \[f(x)\tendvers{x}{a}l\]
% si et seulement si~:

\end{definition}

\begin{remarqueUnique}
\remarque En pratique, on utilisera les caractérisations suivantes~:
\begin{itemize}
  \item Pour $a\in\R$ et $l\in\K$
    \[\forall \epsilon>0 \qsep \exists \eta>0 \qsep \forall x\in\mathcal{D}
      \qsep \abs{x-a}\leq\eta \implique \abs{f(x)-l}\leq\epsilon.\]
  \item Pour $a\in\R$ et $l=-\infty$
    \[\forall M\in\R \qsep \exists \eta>0 \qsep \forall x\in\mathcal{D}
      \qsep \abs{x-a}\leq\eta \implique f(x)\leq M.\]
  \item Pour $a\in\R$ et $l=+\infty$
    \[\forall m\in\R \qsep \exists \eta>0 \qsep \forall x\in\mathcal{D}
      \qsep \abs{x-a}\leq\eta \implique f(x)\geq m.\]
  \item Pour $a=-\infty$ et $l\in\K$
    \[\forall \epsilon>0 \qsep \exists A\in\R \qsep \forall x\in\mathcal{D}
      \qsep x\leq A \implique \abs{f(x)-l}\leq\epsilon.\]
  \item Pour $a=-\infty$ et $l=-\infty$
    \[\forall M\in\R \qsep \exists A\in\R \qsep \forall x\in\mathcal{D}
      \qsep x\leq A \implique f(x)\leq M.\]
  \item Pour $a=-\infty$ et $l=+\infty$
    \[\forall m\in\R \qsep \exists A\in\R \qsep \forall x\in\mathcal{D}
      \qsep x\leq A \implique f(x)\geq m.\]
  \item Pour $a=+\infty$ et $l\in\K$
    \[\forall \epsilon>0 \qsep \exists B\in\R \qsep \forall x\in\mathcal{D}
      \qsep x\geq B \implique \abs{f(x)-l}\leq\epsilon.\]
  \item Pour $a=+\infty$ et $l=-\infty$
    \[\forall M\in\R \qsep \exists B\in\R \qsep \forall x\in\mathcal{D}
      \qsep x\geq B \implique f(x)\leq M.\]
  \item Pour $a=+\infty$ et $l=+\infty$
    \[\forall m\in\R \qsep \exists B\in\R \qsep \forall x\in\mathcal{D}
      \qsep x\geq B \implique f(x)\geq m.\]
  \end{itemize}
% \remarque On montre que $f(x)$ tend vers $l\in\RbarouC$ lorsque $x$ tend
%   vers $a\in\Rbar$ si et seulement si pour tout voisinage $\mathcal{W}$ de $l$,
%   il existe un voisinage $\mathcal{V}$ de $a$ tel que
%   \[\forall x\in\mathcal{D}\cap\mathcal{V} \qsep f(x)\in\mathcal{W}.\]
\end{remarqueUnique}

\begin{exoUnique}
\exo Soit $f:\R\to\R$ une fonction croissante telle que
  \[f(n)\tendvers{n}{+\infty}+\infty.\]
  Montrer que $f(x)$ tend vers $+\infty$ lorsque $x$ tend vers $+\infty$.
  Que dire si $f$ n'est pas croissante~?
% \exo En remarquant que $(\sin x)/x$ tend vers 1 lorsque $x$ tend vers
%   0, donner la limite, si elle existe, de
%   \[\sum_{k=0}^n \sin\frac{k}{n^2}\]
%   lorsque $n$ tend vers $+\infty$.
  %% Utiliser le fait que (sin x-x)/x^2 tend vers 0 lorsque x tend vers 0
\end{exoUnique}


\begin{proposition}[utile=-3]
Soit $f:\mathcal{D}\to\K$ une fonction définie au voisinage de $a\in\Rbar$ et $l\in\K$.
Alors
\[f(x)\tendvers{x}{a} l\]
si et seulement si, pour toute suite $\p{u_n}$ d'éléments de $\mathcal{D}$
tendant vers $a$, la suite $\p{f\p{u_n}}$ tend vers $l$.
\end{proposition}

\begin{preuve}
C'est la CARACTERISATION SEQUENTIELLE DE LA LIMITE :
On va faire la preuve avec $a\in \R$ et $\ell \in \R$.
\begin{itemize}
\item[$\bullet$] Soit $\epsilon >0$. Si $f(x)\tendvers{x}{a} \ell$, il existe $\eta>0$ tel que $\forall x \in \mathcal{D}_f, \abs{x-a}\leq \eta \Longrightarrow \abs{f(x)-\ell}\leq \epsilon$. Considérons alors $(u_n)$ une suite d'éléments de $\mathcal{D}_f$ qui tend vers $a$. Il existe donc $N\in \N$ tel que $\forall n \geq N$, $\abs{u_n-a}\leq \eta$. Mais alors, $\abs{f(u_n)-\ell}\leq \epsilon$, ce qui prouve le sens gauche-droite.
\item[$\bullet$] Réciproquement, on suppose que pour toute suite $\p{u_n}$ d'éléments de $\mathcal{D}_f$
tendant vers $a$, la suite $\p{f\p{u_n}}$ tend vers $\ell$. On raisonne par l'absurde, c'est-à-dire qu'on suppose qu'il existe $\epsilon>0$ tel que $\forall \eta>0, \exists x \in \mathcal{D}_f$, on ait $\abs{x-a}\leq \eta$ et $\abs{f(x)-\ell}>\epsilon$.\\
Soit $n\in \N$, avec $\eta=\frac{1}{n}$, il existe $x_n\in \mathcal{D}_f$ tel que $\abs{x_n-a}\leq \dfrac{1}{n}$ et $\abs{f(x_n)-\ell}>\epsilon$. $(x_n)$ converge vers $a$ donc par hypothèse, $(f(x_n))$ converge vers $\ell$, d'où $0\geq \epsilon$ par passage à la limite, ce qui fournit une contradiction.
\end{itemize}

\end{preuve}

\begin{remarques}
\remarque Si $f$ est définie en $a\in\R$ et admet une limite en $a$, cette
  limite est $f(a)$.
\exo Cette proposition est utile pour prouver qu'une fonction $f$ n'a pas
  de limite en $a$. Pour cela, il suffit de trouver deux suites
  $\p{u_n}$ et $\p{v_n}$ telles que
  \[u_n\tendvers{n}{+\infty} a \et v_n\tendvers{n}{+\infty} a.\]
  et telles que les suites de terme général $f\p{u_n}$ et $f\p{v_n}$ aient
  des limites distinctes.
%   Par exemple $x\mapsto 1/x$ n'a pas de limite en 0, et $x\mapsto \sin x$ n'a
%   pas de limite en $+\infty$.
\end{remarques}

\begin{exos}
\exo Soit $f:\R\to\R$ une fonction périodique admettant une limite finie en
  $+\infty$. Montrer que $f$ est constante.
  \begin{sol}
  $(f(x+nT))$ tend vers $\ell$, mais cette suite est aussi égale à $f(x)$ donc $\ell=f(x)$ et ce peu importe $x$.
  \end{sol}
\exo Montrer que la fonction d'expression $\sin\p{\frac{1}{x}}$ n'a pas de
  limite en 0.
  \begin{sol}
  Prendre $u_n=\dfrac{1}{\frac{\pi}{2}+2n\pi}$ et $v_n=\dfrac{1}{-\frac{\pi}{2}+2n\pi}$.
  \end{sol}
\end{exos}

\begin{proposition}[utile=-3]
Si $f$ admet une limite $l\in\RbarouC$ en $a\in\Rbar$, alors cette limite
est unique. Si tel est le cas, on écrit
\[\lim_{x\to a} f(x)=l.\]
\end{proposition}

\begin{preuve}
\'Etant donné qu'on a démontré les résultats sur les suites, on va utiliser la caractérisation séquentielle de la limite pour obtenir chacun des résultats ci-dessous. Par exemple, pour celui-ci :
On considère $a\in \Rbar$. On suppose qu'il existe $\ell_1, \ell_2$ tels que $f(x)\tendvers{x}{a} \ell_1$ et $f(x)\tendvers{x}{a} \ell_2$. Puisque $f$ est définie au voisinage de $a$, il existe une suite $(u_n)$ d'éléments de $\mathcal{D}_f$ telle que $u_n\tendvers{n}{+\infty}{a}$ donc $f(u_n)\tendvers{n}{+\infty}{\ell_1}$ et $f(u_n)\tendvers{n}{+\infty}{\ell_2}$. D'où : $\ell_1=\ell_2$ par unicité de la limite sur les suites.
\end{preuve}

\begin{proposition}[utile=-3]
Soit $f$ une fonction définie au voisinage de $a\in\Rbar$
et $l\in\K$. On suppose que
\[f(x)\tendvers{x}{a} l.\]
Alors
\[\abs{f(x)}\tendvers{x}{a} \abs{l} \et
  \conj{f}(x)\tendvers{x}{a} \conj{l}.\]
\end{proposition}

\begin{proposition}[utile=-3]
Soit $f$ une fonction complexe définie au voisinage de $a\in\Rbar$ et $l\in\C$.
Alors
\[f(x)\tendvers{x}{a} l \quad\ssi\quad \cro{\Re\cro{f(x)}\tendvers{x}{a}\Re(l) \et
  \Im\cro{f(x)}\tendvers{x}{a}\Im(l)}.\]
\end{proposition}

\begin{proposition}[utile=-3]
\begin{itemize}
\item Les théorèmes usuels portant sur les combinaisons linéaires, les produits
  et les quotients de limites de suites restent vrais pour les fonctions.
\item Soit $f$ une fonction définie au voisinage de $a\in\Rbar$ tendant vers $l_1\in\Rbar$
  en $a$ et $g$ une fonction définie au voisinage de $l_1$ tendant vers
  $l_2\in\RbarouC$ en $l_1$. Si $g\circ f$ est définie au voisinage de $a$, alors
  \[\p{g\circ f}(x) \tendvers{x}{a} l_2.\]
\end{itemize}
\end{proposition}

\begin{preuve}
Pour la composition  :
Soit $(u_n)$ une suite d'éléments de $\mathcal{D}_f$ telle que $u_n\tendvers{n}{+\infty}a$, on a $f(x)\tendvers{x}{a}\ell_1$ donc $f(u_n)\tendvers{n}{+\infty}\ell_1$. De plus, $g(x)\tendvers{x}{\ell_1}\ell_2$ donc $g(f(u_n))\tendvers{n}{+\infty}\ell_2$ et ce peu importe la suite qui tend vers $a$ donc $g(f(x))\tendvers{x}{a} \ell_2$.
\end{preuve}

\begin{remarques}
\remarque Comme pour les suites, la somme d'une fonction admettant une limite
  finie en $a$ et d'une fonction n'admettant pas de limite en $a$ n'admet pas de
  limite en $a$.
  Les autres théorèmes de ce type sont souvent faux;
  par exemple, il est possible qu'une fonction $f$ n'admette pas de limite en
  $a$ bien que $g\circ f$ admette une limite en $a$.
\remarque Soit $f$ et $g$ sont deux fonctions réelles définies sur
  $\mathcal{D}$. On définit les fonctions $\sup\p{f,g}$ et $\inf\p{f,g}$ par
  \[\forall x\in\mathcal{D} \qsep \sup\p{f,g}(x)\defeq\max\p{f(x),g(x)} \et
    \inf\p{f,g}(x)\defeq\min\p{f(x),g(x)}.\]
  Si $f$ et $g$ admettent pour limites respectives $l_f$ et $l_g\in\R$ en
  $a\in\Rbar$, alors
  \[\sup\p{f,g}(x)\tendvers{x}{a}\max\p{l_f,l_g} \et 
    \inf\p{f,g}(x)\tendvers{x}{a}\min\p{l_f,l_g}.\]
\end{remarques}

% \begin{exoUnique}
% \exo Déterminer la limite, si elle existe de
%   \[\frac{\sqrt{1+\sin x}-\sqrt{1-\sin x}}{x}\] en 0.
% \end{exoUnique}
% \begin{sol}
% On passe à la quantité conjuguée et le tout tend vers $1$.
% \end{sol}

\subsection{Limite et ordre sur $\R$}

\begin{proposition}[utile=-3]
Soit $f:\mathcal{D}\to\K$ une fonction admettant une limite finie en $a$.
Alors $f$ est bornée au voisinage de $a$.  
\end{proposition}

\begin{preuve}
On écrit la définition de la limite avec $\epsilon=1$ puis I.T.
\end{preuve}

\begin{proposition}[utile=-3]
Soit $f:\mathcal{D}\to\R$ une fonction telle que $f(x)$ tend vers $l\in\R$ lorsque
$x$ tend vers $a\in\Rbar$.
\begin{itemize}
\item Si $f$ est majorée par $M\in\R$, alors $l\leq M$.
\item Si $f$ est minorée par $m\in\R$, alors $l\geq m$.
\end{itemize}
\end{proposition}

\begin{preuve}
Cas majoré : On prend une suite $(u_n)$ d'éléments de $\mathcal{D}_f$ qui converge vers $a$. Alors $\forall n \in \N$, $f(u_n)\leq M$ d'où par passage à la limite (SUR LES SUITES), $\ell\leq M$.

\end{preuve}

%% Remarque :
%% 1) Si f est croissante sur R et f(x) -> l_1 (en  -infini) et
%%    f(x) -> l_2 (en +infini), alors l_1 <= f(x) <= l_2
%% 2) Si f est strictement croissante et f(x) -> l_1 (en  -infini) et
%%    f(x) -> l_2 (en +infini), alors l_1 < f(x) < l_2

\begin{proposition}[utile=-3]
Soit $f:\mathcal{D}\to\R$ une fonction telle que $f(x)$ tend vers $l\in\Rbar$ lorsque
$x$ tend vers $a\in\Rbar$.
\begin{itemize}
\item Si $M$ est un réel tel que $l<M$, il existe un voisinage $\mathcal{V}$
  de $a$ tel que
  \[\forall x\in\mathcal{D} \qsep x\in\mathcal{V} \implique f(x)\leq M.\]
\item Si $m$ est un réel tel que $l>m$, il existe un voisinage $\mathcal{V}$
  de $a$ tel que
  \[\forall x\in\mathcal{D} \qsep x\in\mathcal{V} \implique f(x)\geq m.\]
\end{itemize}
\end{proposition}

\begin{preuve}
Pour changer, faisons cette preuve avec $a=+\infty$ et $\ell \in \R$.
On pose $\epsilon=M-\ell >0$. Alors, il existe $m\in \R$ tel que $\forall x\geq m, \abs{f(x)-\ell}\leq \epsilon$, i.e. $f(x)\leq \ell+\epsilon=M$.
\end{preuve}

\begin{remarques}
\remarque En pratique, il conviendra d'expliciter les voisinages. Par
  exemple, si $f(x)$ tend vers $l<M$ lorsque $x$ tend vers $a\in\R$, il
  existe $\epsilon>0$ tel que
  \[\forall x\in\mathcal{D} \qsep \abs{x-a}\leq\epsilon \implique
    f(x)\leq M.\]
\remarque Si une fonction complexe $f$ admet une limite $l\in\C$
  non nulle en $+\infty$, il existe $m\in\R$ tel que
  \[\forall x\in\mathcal{D} \qsep x\geq m \implique f(x)\neq 0.\] 
\end{remarques}

\begin{theoreme}[utile=-3, nom={Théorème des gendarmes}]
Soit $f$, $g$ et $h:\mathcal{D}\to\R$ telles que
\[\forall x\in\mathcal{D} \qsep f(x)\leq g(x)\leq h(x).\]
On suppose que $f$ et $h$ admettent la même limite finie $l\in\R$ en $a\in\Rbar$.
Alors
\[g(x) \tendvers{x}{a} l.\]
\end{theoreme}

\begin{preuve}
Soit $(u_n)$ une suite d'éléments de $\mathcal{D}$ qui tend vers $a$. On a d'une part $\forall n \in \N, f\p{u_n}\leq g\p{u_n}\leq h\p{u_n}$ et d'autre part par caractérisation séquentielle de la limite, $(f(u_n))$ et $(h(u_n))$ tendent vers $\ell$. Donc $(g(u_n))$ tend vers $\ell$ par théorème des gendarmes sur les suites, et ceci étant valable pour n'importe quelle suite $(u_n)$ convergeant vers $a$, on en déduit que $g(x) \tendvers{x}{a} \ell$ par caractérisation séquentielle de la limite.
\end{preuve}


\begin{proposition}[utile=-3]
Soit $f$ et $g:\mathcal{D}\to\R$ telles que
\[\forall x\in\mathcal{D} \qsep f(x)\leq g(x)\]
et $a\in\Rbar$.
\begin{itemize}
\item Si $f(x)\tendvers{x}{a} +\infty$, alors $g(x)\tendvers{x}{a}+\infty$.
\item Si $g(x)\tendvers{x}{a} -\infty$, alors $f(x)\tendvers{x}{a}-\infty$.
\end{itemize}
\end{proposition}

\begin{preuve}
Par caractérisation séquentielle.
\end{preuve}

\begin{proposition}[utile=-3]
Soit $f:\mathcal{D}\to\K$ une fonction, $a\in\Rbar$, $l\in\K$ et $g$ une
fonction réelle positive telle que
\begin{itemize}
\item $\forall x\in\mathcal{D} \qsep \abs{f(x)-l}\leq g(x)$,
\item $g(x)\tendvers{x}{a} 0$.
\end{itemize}
Alors
\[f(x)\tendvers{x}{a} l.\]
\end{proposition}

\begin{preuve}
Par caractérisation séquentielle.
\end{preuve}

\begin{exoUnique}
\exo Déterminer la limite, si elle existe, de la fonction d'expression
  $\frac{x}{2+\sin\p{\frac{1}{x}}}$ en 0.
\begin{sol}
On le coince entre $x/3$ et $x$ si $x\geq 0$ et entre $x$ et $x/3$ si $x\leq 0$. Donc $\forall x \in \Rs, -\abs{x}\leq \frac{x}{2+\sin\p{\frac{1}{x}}} \leq \abs{x}$ + thm des gendarmes.
\end{sol}
% \exo On a
%   \[\integ{0}{1}{\frac{e^{-xt}}{1+t^2}}{t}\tendvers{x}{+\infty}0\]
\end{exoUnique}

\subsection{Limite à gauche, à droite}

\begin{definition}[utile=-3]

\begin{itemize}
\item Soit $f$ une fonction définie au voisinage à gauche de $a\in\R$ et
  $l\in\RbarouC$. On dit que $f$ admet $l$ pour \emph{limite à gauche} en $a$ lorsque
  la restriction de $f$ à $\mathcal{D}\cap\intero{-\infty}{a}$ admet $l$ pour
  limite en $a$. Si tel est le cas, on note
  \[f(x)\tendversgp{x}{a} l.\]
  La propriété \og tend vers $l$ lorsque $x$ tend vers $a$ par la gauche \fg est
  locale à gauche en $a$.
\item Soit $f$ une fonction définie au voisinage à droite de $a\in\R$ et
  $l\in\RbarouC$. On dit que $f$ admet $l$ pour \emph{limite à droite} en $a$ lorsque
  la restriction de $f$ à $\mathcal{D}\cap\intero{a}{+\infty}$ admet $l$ pour
  limite en $a$. Si tel est le cas, on note
  \[f(x)\tendversdp{x}{a} l.\]
  La propriété \og tend vers $l$ lorsque $x$ tend vers $a$ par la droite \fg est
  locale à droite en $a$.
\end{itemize}
\end{definition}

\begin{remarqueUnique}
\remarque On définit de même les notions de limite à gauche au sens large, de limite à
  droite et de limite épointée (pour $x\neq a$). On écrit alors respectivement
  \[f(x)\tendversg{x}{a} l, \qquad
    f(x)\tendversd{x}{a} l, \qquad
    f(x)\tendversp{x}{a} l.\]
\end{remarqueUnique}


\begin{proposition}[utile=-3]
Soit $f$ une fonction définie au voisinage de $a$ et $l\in\RbarouC$. Alors
$f(x)$ tend vers $l$ lorsque $x$ tend vers $a$ si et seulement si, les objets
ci-dessous susceptibles d'avoir un sens
\[\lim_{\substack{x\to a\\x<a}} f(x), \quad f(a) \et
  \lim_{\substack{x\to a\\x>a}} f(x)\]
existent et sont égaux à $l$.
\end{proposition}

\begin{preuve}
Sens gauche-droite immédiat.
Pour le sens droite-gauche, écrire les deux limites avec $\eta_1$ et $\eta_2$ puis prendre le $\min$.
\end{preuve}

\begin{remarqueUnique}
\remarque On a bien entendu des théorèmes similaires pour les limites au sens large. Par
  exemple, $f(x)$ tend vers $l$ lorsque $x$ tend vers $a$ si et seulement si, les  objets
  ci-dessous susceptibles d'avoir un sens
  \[\lim_{\substack{x\to a\\x\leq a}} f(x) \et
    \lim_{\substack{x\to a\\x>a}} f(x)\]
  existent et sont égaux à $l$.
\end{remarqueUnique}

\begin{exoUnique}
\exo Soit $f$ la fonction définie sur $\R$ par
  \[\forall x\in\R \qsep f(x)=
    \begin{cases}
    0 & \text{si $x\leq 0$}\\
    \e^{-1/x} & \text{si $x>0$}
    \end{cases}\]
  Montrer que $f(x)\tendvers{x}{0} 0$.
\end{exoUnique}



\begin{theoreme}[nom={Théorème de la limite monotone}]
Soit $f:I\to\R$ une fonction croissante sur un intervalle $I$.
\begin{itemize}
\item Si $a\in I$ n'est pas une borne de $I$, $f$ admet une limite finie 
  à gauche et une limite finie à droite en $a$. De plus
  \[\lim_{\substack{x\to a\\x<a}} f(x) \leq f(a) \leq
    \lim_{\substack{x\to a\\x>a}} f(x).\]
\item Si $a$ est la borne supérieure de $I$, $f$ admet une limite en $a$.
  Cette limite est finie si $f$ est majorée, et est égale à $+\infty$ sinon.
\item Si $a$ est la borne inférieure de $I$, $f$ admet une limite en $a$.
  Cette limite est finie si $f$ est minorée, et est égale à $-\infty$ sinon.
\end{itemize}
\end{theoreme}

\begin{preuve}
Nous allons traiter seulement certains cas (plus simples) pour s'éviter des détails techniques superflus :
\begin{itemize}
\item[$\bullet$] Traitons le cas où $f:\R\mapsto \R$ est une fonction croissante et $a\in \R$. On veut montrer que $f$ admet une limite à gauche en $a$.
\begin{itemize}
\item Posons $A=\set{f(x), x<a}$. $A$ est non vide ($f(a-1)\in A$) et majorée par $f(a)$ par croissance. $A$ admet donc une borne supérieure que l'on note $\ell_1$. $f(a)$ étant un majorant, on a $\ell_1\leq f(a)$.
\item Soit $\epsilon>0$. Alors, il existe $y_0\in A$ tel que $y_0\geq \ell_1-\epsilon$, i.e. il existe $x_0\in ]-\infty;a[$ tel que $f(x_0)\geq \ell_1-\epsilon$. Posons alors $\eta=a-x_0>0$. Pour tout $x\in [a-\eta;a[, a-\eta=x_0\leq x$ donc $f(x_0)\leq f(x)$ par croissance de $f$ donc $\ell_1-\epsilon\leq f(x)$. Et $f(x)\in A$ donc $f(x)\leq \ell_1$. Finalement, pour tout $x\in [a-\eta;a[, \abs{f(x)-\ell_1}\leq \epsilon$.\\
\item $\ell_1$ est donc bien la limite à gauche de $f$ en $a$ et $\ell_1\leq f(a)$.
\item On montre de même qu'il existe une limite à droite en $a$ et qu'elle est supérieure à $f(a)$.
\end{itemize}
\item[$\bullet$] Traitons le cas où $f:\RPs\mapsto \R$ est une fonction croissante et où $a=0$. Montrons que $f$ admet une limite à droite en $0$ qui soit ou bien finie (si elle est minorée) ou bien $-\infty$.\\
On pose $A=\set{f(x), x\in \RPs}$.
\begin{itemize}
\item Si $A$ n'est pas minorée, montrons qu'alors la limite à droite vaut $-\infty$. Soit $M\in \R$. Puisque $A$ n'est pas minorée, il existe $x_0\in \RPs$ tel que $f(x_0)<M$. Mais alors, par croissance, pour tout $x\in ]0;x_0], f(x)(\leq f(x_0))\leq M$, ce qui prouve le résultat souhaité.
\item Si $A$ est minorée, comme $A$ est non vide, il existe une borne inférieure qu'on note $\ell$. Soit $\epsilon>0$, il existe $a_0\in A$ tel que $ a_0 \leq \ell+\epsilon$. Mais $\exists x_0\in \RPs$ tel que $a_0=f(x_0)$ et pour tout $x\in ]0;x_0]$ par croissance et parce-que $f(x)\in A$, on a $\ell \leq f(x) \leq \ell+\epsilon$ d'où $\abs{f(x)-\ell}\leq \epsilon$.
\end{itemize}
\end{itemize}
\end{preuve}

\begin{remarques}
\remarque Une proposition similaire existe pour les fonctions décroissantes.
\remarque Si $f:\intero{-\infty}{a}\to\R$ est une fonction croissante admettant une limite $l\in\R$ en $a\in\R\cup\ens{+\infty}$, alors
  \[\forall x\in\intero{-\infty}{a} \qsep f(x)\leq l.\]
  De plus, si $f$ est strictement croissante, alors
  \[\forall x\in\intero{-\infty}{a} \qsep f(x)< l.\]
\end{remarques}

\section{Continuité}

\subsection{Continuité ponctuelle}

\begin{definition}[utile=-3]
On dit qu'une fonction $f:\mathcal{D}\to\K$ est \emph{continue} en $x_0\in\mathcal{D}$ lorsque
\[f(x)\tendvers{x}{x_0} f\p{x_0}.\]
La propriété \og est continue en $x_0$ \fg est locale en $x_0$. On appelle
domaine de continuité de $f$ l'ensemble des $x_0\in\mathcal{D}$ en lesquels
$f$ est continue.
\end{definition}


\begin{remarques}
\remarque La continuité de $f$ en $x_0$ s'écrit 
  \[\forall \epsilon>0 \qsep \exists \eta>0 \qsep \forall x\in\mathcal{D}
    \qsep \abs{x-x_0}\leq\eta \implique \abs{f(x)-f\p{x_0}}\leq\epsilon.\]
\remarque Une fonction $f:\mathcal{D}\to\K$ est continue en $x_0\in\mathcal{D}$ si et seulement si
  elle admet une limite en $x_0$.
\begin{sol}
Sens gauche-droite évident. Si maintenant, il existe $\ell \in \Rbar$ tel que $f(x)\tendvers{x}{x_0} \ell$ alors $f$ étant définie en $x_0$, on obtient d'après la définition de la limite $\forall \epsilon>0, |f(x_0)-\ell|\leq \epsilon$ (étant donné que pour tout $\eta>0$, $|x_0-x_0|\leq \eta$) d'où $\ell=f(x_0)$.
\end{sol}
\remarque On dit qu'une fonction $f$ est \emph{continue à droite} en $x_0$ lorsque
  \[f(x)\tendversdp{x}{x_0} f\p{x_0}.\]
  De même, on définit la notion de \emph{continuité à gauche}. Une fonction est
  continue en $x_0$ si et seulement si elle est continue à droite et à gauche
  en $x_0$.
\remarque On dit qu'une fonction $f$ admet une \emph{discontinuité de première
  espèce} en $x_0$ lorsqu'elle admet des limites à droite et à gauche
  et lorsque l'une de ces limites est différente de $f(x_0)$.
  Par exemple, la fonction partie entière admet une discontinuité de première
  espèce en tout point $x_0\in\Z$.
% \remarque Toute fonction lipschitzienne est continue en tout point de son
%   domaine de définition.
%   \begin{sol}
%   Soit $k>0$. On rappelle qu'une fonction est dite $k$-lipschitzienne si $\forall (x,y)\in \mathcal{D}^2, \abs{f(x)-f(y)}\leq k\abs{x-y}$. Pour voir qu'elle est continue, pour $\epsilon>0$, il suffit de prendre $\eta=\frac{\epsilon}{k}$.
%   \end{sol}
% \remarque La fonction $f$ définie sur $\R$ par $f(x)\defeq\sin\p{1/x}$ si
%   $x\neq 0$ et $f\p{0}\defeq 0$ est discontinue en 0. Cette discontinuité n'est pas
%   une discontinuité de première espèce.
% \begin{sol}
% On parle alors de discontinuité essentielle ou discontinuité de deuxième espèce lorsque au moins une des deux limites à gauche et à droite n'existe pas ou est infinie.
% \end{sol}
\remarque Les fonctions valeur absolue, puissance (en particulier les puissances
  entières et les racines $n$-ièmes), $\ln$, $\exp$, les fonctions
  trigonométriques circulaires et hyperboliques, directes et réciproques sont
  continues en tout point de leur domaine de définition.
  \begin{sol}
La seule fonction "qui a un nom" qui admet des discontinuités (elles sont de première espèce) est la fonction partie entière.
\end{sol} 
\remarque Une fonction continue en $x_0$ est bornée au voisinage de $x_0$.
\end{remarques}

\begin{exoUnique}
\exo Soit $f:\R\to\R$ la fonction définie par
  \[\forall x\in\R\qsep f(x)\defeq\begin{cases}\sin\p{\frac{1}{x}} & \text{si $x\neq 0$}\\
    0 & \text{si $x=0$.}\end{cases}\]
  Montrer que $f$ est discontinue en 0 et que cette discontinuité n'est pas de première
  espèce.
\end{exoUnique}

\begin{definition}[utile=-3]
Soit $f:\mathcal{D}\to\K$ une fonction définie au voisinage d'un point $a\in\R$ n'appartenant pas
à $\mathcal{D}$. Lorsque $f(x)$ admet une limite finie $l\in\K$ lorsque $x$
tend vers $a$, on dit que $f$ est \emph{prolongeable par continuité} en $a$. La fonction
\[\dspappli{\hat{f}}{\mathcal{D} \cup\ens{a}}{\K}{x}{
  \begin{cases}
  f(x) & \text{si $x\neq a$}\\
  l      & \text{si $x=a$}  
  \end{cases}}\]
est alors appelée \emph{prolongement par continuité} de $f$ en $a$. C'est une fonction
continue en $a$.
\end{definition}
  

% \begin{applications}
% \application Soit $f:\RPs\to\R$ une fonction croissante telle que
%   $g:x\mapsto f(x)/x$ soit décroissante. Alors $f$ est continue sur $\RPs$.
% \end{applications}
\begin{proposition}[utile=-3]
Soit $f:\mathcal{D}\to\K$ une fonction et $x_0\in\mathcal{D}$. Alors $f$ est continue en $x_0$
si et seulement si pour toute suite $\p{u_n}$ d'éléments de $\mathcal{D}$
convergeant vers $x_0$
\[f\p{u_n} \tendvers{n}{+\infty} f\p{x_0}.\]
\end{proposition}

\begin{preuve}
C'est la caractérisation séquentielle de la limite.
\end{preuve}


\begin{remarques}
\remarque Soit $f$ une fonction continue sur l'intervalle $I$ et $\p{u_n}$
  une suite telle que
  \[\forall n\in\N \qsep u_{n+1}=f\p{u_n}\]
  Alors, si $\p{u_n}$ admet une limite $l\in\Rbar$, c'est une borne de
  $I$ ou un point fixe de $f$.
  \begin{sol}
  En effet, supposons que $\ell$ ne soit pas une borne de $I$.
  Alors $\ell \in I$. Puisque $f$ est continue en $I$, $f(u_n)\tendvers{n}{+\infty}f(\ell)$. Or $u_{n+1}\tendvers{n}{+\infty}\ell$. Par unicité de la limite $f(\ell)=\ell$.
  \end{sol}
\remarque Soit $f$ et $g$ deux fonctions continues ent tout point de $\R$. Si elles coïncident
  sur $\Q$, alors $f=g$.
  \begin{sol}
  Soit $x\in \R$, $\Q$ étant dense dans $\R$, il existe une suite $(u_n)$ de rationnels tendant vers $x$. Donc $\forall n\in \N, f(u_n)=g(u_n)$ et par passage à la limite $f(x)=g(x)$.
  \end{sol}
\remarque Cette proposition est utile pour prouver qu'une fonction $f$ n'est
  pas continue en $x_0$. Pour cela, il suffit de trouver une suite $\p{u_n}$
  convergeant vers $x_0$ telle que la suite de terme général $f\p{u_n}$ ait
  une limite différente de $f\p{x_0}$.
\end{remarques}

\begin{exos}
\exo Montrer que la fonction caractéristique de $\Q$ est discontinue en
  tout point.
  \begin{sol}
  Soit $x\in \Q$, $\R\setminus\Q$ étant dense dans $\R$, il existe une suite $(u_n)$ d'irrationnels tendant vers $x$. Alors $\forall n \in N, f(u_n)=0\tendvers{n}{+\infty}0\neq f(x)=1$, donc $f$ est discontinue en $x$.\\
  Soit $x\in \R\setminus\Q$, $Q$ étant dense dans $\R$, il existe une suite $(u_n)$ de rationnels tendant vers $x$. Alors $\forall n \in N, f(u_n)=1\tendvers{n}{+\infty}1\neq f(x)=0$, donc $f$ est discontinue en $x$.
  \end{sol}
\exo Quelles sont les fonctions $f:\R\to\R$, continues en tout point de $\R$, telles que
  \[\forall x,y\in\R \qsep f\p{x+y}=f(x)+f(y).\]
  \begin{sol}
  \begin{itemize}
  \item[$\bullet$] \underline{\textbf{Analyse :}} Soit $f$ continue de $\R$ dans $\R$ vérifiant $f(x+y)=f(x)+f(y), \forall (x,y)\in \R^2$.
  \begin{itemize}
  \item On montre d'abord par récurrence sur $n\in \N$ $H_n : "f(nx)=nf(x)"$.
  \item Soit $n\in \Z$ tel que $n\leq 0$. Alors $-n\in \N$. D'où $f(nx)+f(-nx)=f(0)=0$ donc $f(nx)=-f(-nx)=-(-nf(x))=nf(x)$. On a donc montré que $\forall n \in \Z, f(nx)=nf(x)$.
  \item Soit $r\in \Q$. Il existe $p\in \Z$ et $q\in \Ns$ tels que $r=p/q$. On a $pf(x)=f(px)=f(qrx)=qf(rx)$ d'où $f(rx)=p/qf(x)=rf(x)$.
  Finalement, $\forall r\in \Q, f(r)=rf(1)$.
  \item Soit $x\in \R$, il existe une suite $(r_n)$ de rationnels qui tend vers $x$ (par densité). Par continuité de $f$ en $x$, $r_nf(1)=f(r_n)\tendvers{n}{+\infty}f(x)$. Donc par unicité de la limite $f(x)=xf(1)$.
  \item En posant $\alpha=f(1)$ on a donc $f(x)=\alpha x$.
  \end{itemize}
  \item[$\bullet$] \underline{\textbf{Synthèse :}} Soit $\alpha \in \R$. On définit $f$ sur $\R$ par $f(x)=\alpha x$. Alors $f$ est continue sur $\R$ et $\forall (x,y)\in \R^2$, on a $f(x+y)=\alpha(x+y)=\alpha x +\alpha y=f(x)+f(y)$.
  \end{itemize}
  \end{sol}
\end{exos}

\begin{proposition}[utile=-3, nom={Théorèmes usuels}]
Soit $f$ et $g:\mathcal{D}\to\K$ deux fonctions continues en $x_0\in\mathcal{D}$. Alors
\begin{itemize}
\item Si $\lambda$, $\mu\in\K$, $\lambda f+\mu g$ est continue en $x_0$.
\item $fg$ est continue en $x_0$.
\item Si $g\p{x_0}\neq 0$, $g$ ne s'annule pas au voisinage de $x_0$ et
  $f/g$ est continue en $x_0$.
\end{itemize}
\end{proposition}

% \begin{remarques}
% \remarque Si $\mathcal{D}$ est une partie de $\R$, l'ensemble
%   $\mathcal{C}^0\p{\mathcal{D},\R}$ des fonctions continues de $\mathcal{D}$
%   dans $\R$ est une $\R$-algèbre.
% \end{remarques}

\begin{proposition}[utile=-3,nom={Théorèmes usuels}]
Soit $f:\mathcal{D}_f\to\R$ et $g:\mathcal{D}_g\to\K$ deux fonctions telles que
$g\circ f$ est défini au voisinage de $x_0\in\mathcal{D}_f$. Si $f$ est continue en
$x_0$ et $g$ est continue en $f\p{x_0}$, alors $g\circ f$ est continue en $x_0$.
\end{proposition}

\begin{preuve}
Les deux résultats précédents viennent directement des propriétés d'opérations sur les limites.
\end{preuve}

\begin{remarqueUnique}
% \remarque Les deux propositions précédentes sont regroupées sous la
%   dénomination de \og théorèmes usuels \fg.
\remarque La somme d'une fonction continue en $x_0$ et d'une fonction
  discontinue en $x_0$ est discontinue en $x_0$. Les autres propositions de ce
  type peuvent être fausses. Par exemple, si $f$ et $g$ sont les fonctions
  définies sur $\R$ par
  \[\forall x\in\R \qsep f(x)\defeq x \et g(x)\defeq
    \begin{cases}
    \sin\p{\frac{1}{x}} & \text{si $x\neq 0$}\\
    0 & \text{si $x=0$,}
    \end{cases}\]
  alors $f$ est continue en 0 et $g$ ne l'est pas. Pourtant $f\cdot g$ l'est.
  On retiendra que les réciproques des théorèmes usuels peuvent être
  fausses.
\end{remarqueUnique}

\begin{sol}
Pour voir que $fg$ est continue en $0$, $\abs{(fg)(x)}\leq |x|$.
\end{sol}

\begin{proposition}[utile=-3]
Soit $f:\mathcal{D}\to\K$ une fonction continue en $x_0\in\mathcal{D}$. Alors $\conj{f}$ et $\abs{f}$ sont
continues en $x_0$.  
\end{proposition}

%% Remarque
%% 1) Si f et g sont continues sup(f,g) et inf(f,g) le sont.

\begin{proposition}[utile=-3]
Soit $f:\mathcal{D}\to\C$ une fonction et $x_0\in\mathcal{D}$. Alors
\[\cro{\text{$f$ est continue en $x_0$}} \quad\ssi\quad
  \cro{\text{$\Re(f)$ et $\Im(f)$ sont continues en $x_0$}}.\]
\end{proposition}

\begin{remarqueUnique}
\remarque Si $f$ est une fonction à valeurs dans $\C$ continue en $x_0$, alors
  $\e^f$ est continue en $x_0$.
\end{remarqueUnique}
\begin{sol}
$\e^{f(x)}=\e^{\Re(f)(x)+\ii\Im(f(x))}=\cos(\Im(f(x)))\e^{\Re(f)(x)}+\ii\cos(\Im(f(x)))\e^{\Re(f)(x)}$ dont les parties réelle et imaginaire sont continues par TOC
\end{sol}

\begin{exos}
\exo Soit $f$ la fonction définie sur $\R$ par
  \[\forall x\in\R \qsep f(x)=
    \begin{cases}
    \frac{\e^{\ii x}-1}{x} & \text{si $x\neq 0$}\\
    \ii & \text{si $x=0$.}
    \end{cases}\]
  Montrer que $f$ est continue en tout point de $\R$.
\end{exos}

\begin{sol}
Taux d'accroissement.
Sinon, pour voir la continuité en $0$, écrire pour $x\neq 0$, $$\frac{\e^{\ii x}-1}{x}=\frac{\cos(x)-1}{x}+i\frac{\sin(x)}{x}$$ donc $f(x)\tendversp{x}{0}f(0)$ d'où $f(x)\tendvers{x}{0}f(0)$ donc $f$ est continue en $0$.
\end{sol}

%% Exemple :
%% 1) Prolongement par continuité en 0 de sin x/x. On obtient le sinus cardinal.



\subsection{Continuité sur une partie}

\begin{definition}[utile=-3]
Soit $f:\dom\to\K$.
\begin{itemize}
\item On dit que $f$ est \emph{continue} lorsqu'elle est continue en tout point de
  $\dom$.
\item Si $A$ est une partie de $\dom$, on dit que $f$ est continue sur $A$ lorsque la
 restriction de $f$ à $A$ est continue.
\end{itemize}
\end{definition}

\begin{remarqueUnique}
\remarque Soit $f:\mathcal{D}\to\K$ et $A$ une partie de $\mathcal{D}$.
  Si $f$ est continue en tout point de $A$, alors $f$ est continue sur $A$. Cependant,
  la réciproque est fausse. En effet, si $f$ est la fonction de $\R$ dans $\R$ définie
  par
  \[\forall x\in\R\qsep f(x)\defeq\begin{cases}
    1 & \text{si $x\geq 0$}\\
    0 & \text{sinon}
  \end{cases}\]
  alors $f$ est continue sur $\RP$ mais n'est pas continue en 0.
\end{remarqueUnique}

\begin{definition}[utile=-3]
Soit $k\in\RP$. On dit qu'une fonction $f:\mathcal{D}\to\K$ est \emph{$k$-lipschitzienne} lorsque
\[\forall x,y\in \mathcal{D} \qsep \abs{f(x)-f(y)}\leq k\abs{x-y}.\] 
\end{definition}

\begin{exoUnique}
\exo Montrer que les fonctions \og valeur absolue\fg et \og sinus\fg sont
  1-lipschitziennes.
% \exo La fonction valeur absolue est $1$-lipschitzienne. En effet
%   \[\forall x,y\in\R \qsep \abs{\abs{x}-\abs{y}}\leq\abs{x-y}\]
% \exo La fonction $\sin$ est $1$-lipschitzienne.
\end{exoUnique}

% \begin{definition}
% On dit qu'une fonction $f:\mathcal{D}\to\RouC$ est \emph{lipschitzienne} lorsqu'il existe
% $M\geq 0$ tel que
% \[\forall x,y\in\mathcal{D}\qsep \abs{f(x)-f(y)}\leq M\abs{x-y}.\]
% \end{definition}

\begin{proposition}
Une fonction lipschitzienne est continue.
\end{proposition}

Dans la suite du cours d'analyse, un s'intéressera le plus souvent à des fonctions dont
le domaine de définition est une partie relativement simple de $\R$. Afin de formaliser cela,
une partie de $\R$ sera dite \emph{élémentaire} lorsque c'est une réunion d'un nombre fini
d'intervalles. Par exemple $\R$ et $\Rs=\intero{-\infty}{0}\cup\intero{0}{+\infty}$ sont des parties élémentaires
de $\R$ alors que $\Q$ n'en est pas une. Notons que cette définition est propre à ce cours.
Les parties élémentaires de $\R$ jouissent de nombreuses propriétés~:
$\emptyset$ et $\R$ sont des parties élémentaires,
une union finie de parties élémentaires est une partie élémentaire,
une intersection finie de parties élémentaires est une partie élémentaire,
le complémentaire d'une partie élémentaire est une partie élémentaire.
Autrement dit, tout ensemble construit à partir de parties élémentaires à l'aide d'un nombre
fini d'opérations est élémentaire.\\

On dit que deux intervalles $I$ et $J$ sont \emph{bien disjoints} lorsque $I\cup J$ n'est
pas un intervalle. En particulier deux intervalles bien disjoints sont disjoints. Mais
la réciproque est fausse comme le montre l'exemple de $I=\interf{0}{1}$ et
$J=\interof{1}{2}$ qui sont disjoints mais qui ne sont pas bien disjoints.
On montre
que si $A$ est une partie élémentaire de $\R$, à réordonnement près, il existe un unique
$n$-uplet $(I_1,\ldots,I_n)$ d'intervalles deux à deux biens disjoints tels que
$A=I_1\cup I_2\cup\cdots\cup I_n$. On dit que les $I_k$ sont les
\emph{composantes connexes} de $A$. Enfin, on dit que $a\in\Rbar$ est
une \emph{extrémité} de $A$ lorsque c'est l'extrémité d'un des $I_k$.\\

Une fonction $f:\mathcal{D}\to\K$ définie sur une partie élémentaire $\mathcal{D}$
  est définie au voisinage de $a\in\Rbar$ si et seulement si $a$ est une extrémité de $A$.
  De plus $a\in\mathcal{D}$ est intérieur à $\mathcal{D}$ si et seulement si ce n'est pas
  une extrémité de $\mathcal{D}$.\\
  
% \remarque Une partie élémentaire $\mathcal{D}$ est un ouvert si et seulement si
%   elle ne contient pas ses extrémités.
% \remarque Si $\mathcal{D}$ est une partie élémentaire de $\R$, les points intérieurs à
%   $\mathcal{D}$ sont les éléments de $\mathcal{D}$ qui n'en sont pas des extrémités.



\begin{proposition}
Soit $f:\dom\to\K$ une fonction définie sur un domaine \emph{élémentaire} et
$\dom=I_1\cup\cdots\cup I_n$ la décomposition de $\dom$ en composantes connexes. Alors $f$
est continue si et seulement si, pour tout $k\in\intere{1}{n}$, $f$ est continue
sur $I_k$.
\end{proposition}

\begin{proposition}
Soit $f:I\to\K$ une fonction définie sur un intervalle $I$ et $a\in I$.
\begin{itemize}
\item Si $f$ est continue sur $I\cap\intero{a}{+\infty}$, elle est continue en tout
  point de $I\cap\intero{a}{+\infty}$.
\item Si $f$ est continue sur $I\cap\interfo{a}{+\infty}$, elle est continue à droite
  en $a$ et en tout point de $I\cap\intero{a}{+\infty}$.
\end{itemize}
\end{proposition}


\subsection{Théorème des valeurs intermédiaires}

\begin{theoreme}[nom={Théorème des valeurs intermédiaires}]
Soit $f$ une fonction réelle continue sur $\interf{a}{b}$. Si
  $y_0\in\intergf{f(a)}{f(b)}$, il existe $x_0\in\interf{a}{b}$ tel
  que $f(x_0)=y_0$.
\end{theoreme}

\begin{preuve}
On définit la fonction $g$ sur $I$ par
\[\forall x\in I \qsep g(x)=f(x)-y_0\]
Alors, puisque $f$ est continue sur $I$, il en est de même pour $g$. Comme
de plus $y_0\in\intergf{f(a)}{f(b)}$, on en déduit que $g(a)$ et
$g(b)$ sont de signes distincts, c'est-à-dire que $g(a)g(b)\leq 0$.
Quitte à les échanger, on peut supposer que $a\leq b$. Nous souhaitons
montrer que $g$ s'annule en un point de $\interf{a}{b}$. Pour cela, nous
allons utiliser le procédé dit de dichotomie et construire deux suites
$\p{a_n}$ et $\p{b_n}$ telles que~:
\begin{itemize}
\item[$\bullet$] Pour tout $n\in\N$, $a_n\leq b_n$.
\item[$\bullet$] La suite $\p{a_n}$ est croissante et la suite $\p{b_n}$ est décroissante.
\item[$\bullet$] Pour tout $n\in\N$
  \[b_{n+1}-a_{n+1}=\frac{b_n-a_n}{2}\]
\item[$\bullet$] Pour tout $n\in\N$, $g\p{a_n}g\p{b_n}\leq 0$.
\end{itemize}
On construit ces deux suites de la manière suivante~:
\begin{itemize}
\item On pose $a_0=a$ et $b_0=b$. On a bien $a_0\leq b_0$ et
  $g\p{a_0}g\p{b_0}\leq 0$.
\item On suppose que les suites sont construites jusqu'au rang $n$. Construisons
  $a_{n+1}$ et $b_{n+1}$. On pose $c=\p{a_n+b_n}/2$. Alors
  $g\p{a_n}g(c)\leq 0$ ou $g(c)g\p{b_n}\leq 0$ (Autrement dit, soit
  $g\p{a_n}$ et $g(c)$ sont de signes distincts, soit $g(c)$ et $g\p{b_n}$
  sont de signes distincts). En effet, supposons que
  $g\p{a_n}g(c)> 0$ et $g(c)g\p{b_n}> 0$. Alors
  $g\p{a_n}g\p{b_n}g(c)^2>0$, donc $g\p{a_n}g\p{b_n}>0$ ce qui est absurde.
  \begin{itemize}
  \item Si $g\p{a_n}g(c)\leq 0$, on pose $a_{n+1}=a_n$ et $b_{n+1}=c$.
    On vérifie bien que $a_n\leq a_{n+1}$ et que $b_{n+1}\leq b_n$, que
    $a_{n+1}\leq b_{n+1}$, que $b_{n+1}-a_{n+1}=\p{b_n-a_n}/2$ et enfin
    que $g\p{a_{n+1}}g\p{b_{n+1}}\leq 0$.
  \item Sinon, $g(c)g\p{b_n}\leq 0$ et on pose $a_{n+1}=c$ et $b_{n+1}=b_n$.
  \end{itemize}
\end{itemize}
On a donc ainsi construit deux suites vérifiant les quatre assertions données
plus haut. Puisque
\[\forall n\in\N \qsep b_{n+1}-a_{n+1}=\frac{b_n-a_n}{2}\]
une récurrence immédiate nous donne
\[b_n-a_n=\frac{b-a}{2^n}\tendvers{n}{+\infty} 0\]
Comme de plus $a_n\leq b_n$ et que les suites $\p{a_n}$ et $\p{b_n}$ sont
respectivement croissantes et décroissantes, elles sont adjacentes.
Elles convergent donc
vers la même limite $x_0$. De plus $x_0\in\interf{a_0}{b_0}=\interf{a}{b}$.
Comme $g$ est continue en $x_0$, on en déduit que
\[g\p{a_n}g\p{b_n}\tendvers{n}{+\infty} g\p{x_0}^2\]
Or
\[\forall n\in\N \qsep g\p{a_n}g\p{b_n}\leq 0\]
Par passage à la limite $g\p{x_0}^2\leq 0$. On en déduit que $g\p{x_0}=0$
donc que $f\p{x_0}=y_0$.
\end{preuve}

\begin{remarques}
\remarque Une fonction réelle continue ne s'annulant pas sur un intervalle $I$ est
  de signe constant.
  \begin{sol}
  En effet, procédons par l'absurde et supposons qu'il existe
  $x_0\in I$ tel que $f\p{x_0}\geq 0$ et $x_1\in I$ tel que $f\p{x_1}\leq 0$.
  Puisque $f$ est continue sur $I$, d'après le théorème des valeurs intermédiaires,
  il existe $x\in I$ tel que $f(x)=0$. C'est absurde, donc $f$ est de signe
  constant sur $I$.    
  \end{sol}
\remarque Soit $f$ une fonction continue sur un intervalle $I$ telle que
  \[\forall x\in I \qsep \cro{f(x)=0 \ou f(x)=1}.\]
  Alors $f$ est constante. Plus généralement, si sur un intervalle, une fonction
  continue prend un nombre fini de valeurs, alors elle est constante.
\end{remarques}

\begin{exoUnique}
\exo Soit $f$ une fonction continue de $\interf{0}{1}$ dans $\interf{0}{1}$.
  Montrer que $f$ admet un point fixe.
\begin{sol}
$g(x)=f(x)-x$ puis $g(0)\geq 0$, $g(1)\leq 0$. Or $g$ continue donc d'après le TVI, il existe $x_0\in \interf{0}{1}$ tel que $g(x_0)=0$, d'où $f(x_0)=x_0$.
\end{sol}
% \exo Sur un intervalle, une fonction continue injective est strictement
%   monotone.
\end{exoUnique}

\begin{proposition}[utile=-3]
Soit $f$ une fonction réelle continue sur $\intero{a}{b}$ admettant
  respectivement pour limite $l_a$ et $l_b\in\Rbar$ en $a$ et $b\in\Rbar$. 
  Si $y_0\in\intergo{l_a}{l_b}$, il existe $x_0\in\intero{a}{b}$ tel que
  $f(x_0)=y_0$.
\end{proposition}

\begin{preuve}
Nous allons traiter le cas où $a=-\infty$, $b=+\infty$ et $l_a<l_b$, les autres
cas se traitant de manière similaire. Soit $y_0\in\intero{l_a}{l_b}$. Puisque
\[f(x)\tendvers{x}{-\infty} l_a<y_0\]
il existe $A\in\R$ tel que
\[\forall x\in\R \qsep x\leq A \implique f(x)\leq y_0\]
En particulier $f(A)\leq y_0$. De même, puisque
\[f(x)\tendvers{x}{+\infty} l_b>y_0\]
il existe $B\in\R$ tel que
\[\forall x\in\R \qsep x\geq B \implique f(x)\geq y_0\]
En particulier $f(B)\geq y_0$. Puisque $f$ est continue sur $\R$, on en
déduit, d'après le théorème des valeurs intermédiaires, qu'il existe
$x_0\in\intergf{A}{B}$ tel que $f\p{x_0}=y_0$.
\end{preuve}


\begin{exoUnique}
\exo Montrer que tout polynôme réel de degré impair admet au moins une racine réelle.
\end{exoUnique}

\begin{proposition}[utile=-3]
L'image d'un intervalle par une fonction réelle continue est un intervalle.
\end{proposition}

\begin{preuve}
Soit $f:I\mapsto\R$ une fonction continue sur un intervalle $I$. Montrons que $J=f(I)$ est un intervalle. Soit alors $y_0,y_1 \in J$ tels que $y_0\leq y_1$ et considérons $y\in [y_0,y_1]$. L'objectif est de montrer que $y\in J$.\\
Comme $J=f(I)$, il existe $x_0$ et $x_1$ dans $I$ tels que $f(x_0)=y_0$ et $f(x_1)=y_1$. Ainsi, $y\in \interf{f(x_0)}{f(y_0)}$ et $f$ est continue sur $I$, donc d'après le TVI, il existe $x\in "\interf{x_0}{x_1}"$ tel que $f(x)=y$. Or, $I$ est un intervalle donc $x\in I$ et finalement $y\in f(I)=J$. $J$ est donc bien un intervalle.
\end{preuve}

\begin{remarques}
\remarque Cette proposition est une reformulation du
  théorème des valeurs intermédiaires.
\remarque Il est possible que les intervalles $I$ et
  $f(I)$ ne soient pas de même nature (ouvert, fermé, ouvert à gauche et
  fermé à droite).
  Par exemple, si $f$ est la fonction définie sur
  $\R$ par \[\forall x\in\R \qsep f(x)\defeq \frac{1}{1+x^2}\]
  on a $f\p{\intero{-\infty}{+\infty}}=\interof{0}{1}$.
  \begin{sol}
  En dérivant $f$, on a $f$ croissante jusqu'en $0$ de valeurs $1$ et décroissante ensuite.
    \end{sol}
\end{remarques}


\begin{theoreme}[nom={Théorème de la bijection}]
\begin{itemize}
\item Soit $f$ une fonction continue, strictement croissante sur $[a,b]$. Alors elle réalise une bijection de $[a,b]$ sur \[f([a,b])=[f(a),f(b)].\]
\item Soit $a,b\in\Rbar$ et $f$ une fonction continue, strictement croissante sur $]a,b[$. On pose
\[l_a\defeq\lim_{x\to a} f(x) \et l_b\defeq\lim_{x\to b} f(x).\]
Alors $f$ réalise une bijection de $]a,b[$ sur \[f\p{\intero{a}{b}}=\intero{l_a}{l_b}.\]
\end{itemize}
\end{theoreme}

\begin{proposition}[utile=-3]
Soit $I$ un intervalle de $\R$ et $f$ une fonction réelle continue strictement
monotone sur $I$. Alors $f$ induit une bijection de l'intervalle $I$ sur
l'intervalle $J\defeq f(I)$ et sa bijection réciproque $f^{-1}:J\to I$ est
continue sur $J$.
\end{proposition}

\begin{preuve}
On suppose que $f$ est strictement croissante sur $I$, la démonstration dans le
cas où $f$ est strictement décroissante se faisant de même. Puisque $f$ est
continue sur l'intervalle $I$, d'après la proposition précédente, $J=f(I)$
est un intervalle. Puisque $f$ est strictement croissante, on en déduit
qu'elle est injective. Sa corestriction à $J$ (que l'on notera encore $f$)
est donc une bijection de $I$ dans $J$.\\
Montrons que $f^{-1}$ est continue. On se donne $y_0\in J$ et on souhaite
montrer que $f^{-1}$ est continue en $y_0$. Nous supposerons dans la suite que
$y_0$ est intérieur à $J$, la démonstration dans le cas où $y_0$ est une borne
de $J$ se faisant de même. Puisque $f$ est strictement croissante, il en est
de même pour $f^{-1}$, donc $x_0=f^{-1}\p{y_0}$ n'est pas une borne de $I$. On
se donne donc $\epsilon>0$ tel que
$\interf{x_0-\epsilon}{x_0+\epsilon}\subset I$. Par stricte croissance de $f$,
on a
\[f\p{x_0-\epsilon}<f\p{x_0}=y_0<f\p{x_0+\epsilon}\]
Il existe donc $\eta>0$ tel que $\interf{y_0-\eta}{y_0+\eta}\subset
\interf{f\p{x_0-\epsilon}}{f\p{x_0+\epsilon}}$. Soit
$y\in\interf{y_0-\eta}{y_0+\eta}$. Alors
\begin{eqnarray*}
y_0-\eta\leq y\leq y_0+\eta
&\donc& f\p{x_0-\epsilon}\leq y\leq f\p{x_0+\epsilon}\\
&\donc& x_0-\epsilon \leq f^{-1}(y) \leq x_0+\epsilon\\
&     & \text{par croissance de $f^{-1}$}\\
&\donc& \abs{f^{-1}(y)-x_0}\leq\epsilon\\
&\donc& \abs{f^{-1}(y)-f^{-1}\p{y_0}}\leq\epsilon
\end{eqnarray*}
Donc $f^{-1}$ est continue en $y_0$.
\end{preuve}

\begin{remarqueUnique}
\remarque La fonction $\sin$ est strictement croissante sur $\interf{-\pi/2}{\pi/2}$.
  Comme $\sin\p{-\pi/2}=-1$ et $\sin\p{\pi/2}=1$, elle réalise une bijection
  de $\interf{-\pi/2}{\pi/2}$ sur $\interf{-1}{1}$. Sa bijection réciproque,
  la fonction $\arcsin$ est donc continue sur $\interf{-1}{1}$.
\end{remarqueUnique}

\begin{exoUnique}
\exo Soit $f$ la fonction définie sur $\RP$ par
  \[\forall x\in\RP \qsep f(x)\defeq x \e^x\]
  Montrer que $f$ réalise une bijection continue de $\RP$ sur $\RP$, que
  $f^{-1}$ est continue et que \[f^{-1}(x)\tendvers{x}{+\infty}+\infty.\]
%   Alors $f$ est continue et strictement croissante sur $\interfo{-1}{+\infty}$.
%   Comme $f\p{-1}=-1/e$ et
%   \[f(x)\tendvers{x}{+\infty}+\infty\]
%   on en déduit que $f$ réalise une bijection de $\interfo{-1}{+\infty}$ sur
%   $\interfo{-1/e}{+\infty}$. Sa bijection réciproque, appelée fonction de Lambert,
%   est donc continue sur $\interfo{-1/e}{+\infty}$.
\end{exoUnique}

\begin{proposition}
Soit $f$ une fonction réelle, continue et injective sur un intervalle $I$. Alors $f$ est strictement monotone.
\end{proposition}

\begin{preuve}
On procède par l'absurde et on suppose qu'il existe $(x_1,y_1) \in I^2, x_1<y_1 \et f(x_1)\geq f(y_1)$ et $(x_2,y_2) \in I^2, x_2<y_2 \et f(x_2)\leq f(y_2)$.

Définissons alors $\phi : \interf{0}{1}
\mapsto \R$ définie par $$\phi(t)=f((1-t)x_1+tx_2)-f((1-t)y_1+ty_2).$$
$\phi$ est continue. $\phi(0)=f(x_1)-f(y_1)\geq 0$ et $\phi(1)=f(x_2)-f(y_2)\leq 0$ donc par le TVI, il existe $t\in [0,1]$ tel que $\phi(t)=0$. En posant $x_0=(1-t)x_1+tx_2$ et $y_0=(1-t)y_1+ty_2$, on a $x_0<y_0$ et $f(x_0)=f(y_0)$ ce qui contredit l'injectivité de $f$.
\end{preuve}

\subsection{Théorème de compacité}

\begin{definition}[utile=-3]
Soit $f$ une fonction réelle définie sur un ensemble non vide $X$. Si $f$ est
majorée sur $X$, $\ensim{f(x)}{x\in X}$ est une partie non vide majorée de
$\R$. Elle admet donc une borne supérieure notée
\[\sup_{x\in X} f(x).\]
On dit que cette borne est atteinte lorsqu'il existe $x_0\in X$ tel que
\[f\p{x_0}=\sup_{x\in X} f(x)\]
c'est-à-dire lorsque l'ensemble $\ensim{f(x)}{x\in X}$ admet un plus grand
élément; si tel est le cas, la borne supérieure est notée
\[\max_{x\in X} f(x).\]
\end{definition}

\begin{remarques}
\remarque On définit de même la notion de borne inférieure.
\remarque Soit $f$ la fonction définie sur $\R$  par
  \[\forall x\in\R \qsep f(x)\defeq x\p{1-x}.\]
  Alors $f$ est bornée et atteint ses bornes sur $\interf{0}{1}$.
  \[\sup_{x\in\interf{0}{1}} x\p{1-x}=f\p{\frac{1}{2}}=\frac{1}{4} \et
    \inf_{x\in\interf{0}{1}} x\p{1-x}=f\p{0}=f\p{1}=0.\]
\remarque Soit $f$ la fonction définie sur $\RPs$ par
  \[\forall x\in\RPs \qsep f(x)\defeq \frac{1}{x}.\]
  Alors $f$ n'est pas majorée sur $\RPs$. De plus, elle est minorée mais
  n'atteint pas sa borne inférieure sur $\RPs$.
  \[\inf_{x\in\RPs} \frac{1}{x}=0.\]
\end{remarques}

\begin{theoreme}[nom={Théorème de compacité}]
Sur un segment, une fonction réelle continue est bornée et atteint ses bornes.
\end{theoreme}

\begin{preuve}
Soit $f$ une fonction continue sur le segment $\interf{a}{b}$.
\begin{itemize}
\item Montrons que $f$ est bornée.\\
  Puisque $f$ est continue, il en est de même pour $\abs{f}$. Montrons que
  $\abs{f}$ est majorée. On raisonne par l'absurde et on suppose que
  \[\non\cro{\exists M\in\R \qsep \forall x\in\interf{a}{b} \qsep
           \abs{f(x)}\leq M}\]
   donc
  \[\forall M\in\R \qsep \exists x\in\interf{a}{b} \qsep \abs{f(x)}>M\]
   Étant donné $n\in\N$, on pose $M=n$. Il existe donc $x_n\in\interf{a}{b}$
   tel que $\abs{f\p{x_n}}\geq n$. On construit ainsi une suite
   $\p{x_n}$ d'éléments de $\interf{a}{b}$. Cette suite étant bornée, d'après
   le théorème de Bolzano-Weierstrass, il existe une extractrice $\phi$ et un
   réel $l$ tel que
   \[x_{\phi(n)}\tendvers{n}{+\infty} l\]
   Comme
   \[\forall n\in\N \qsep a\leq x_{\phi(n)}\leq b\]
   on en déduit que $l\in\interf{a}{b}$. En utilisant la continuité de
   $\abs{f}$ en $l$, il vient
   \[\abs{f\p{x_{\phi(n)}}}\tendvers{n}{+\infty} \abs{f(l)}\]
   C'est contradictoire avec le fait que
   \[\forall n\in\N \qsep \abs{f\p{x_{\phi(n)}}}\geq \phi(n)\geq n
     \tendvers{n}{+\infty} +\infty\]
   Donc $\abs{f}$ est majorée, donc $f$ est bornée.
\item Montrons que $f$ atteint ses bornes.\\
   Montrons d'abord que $f$ atteint sa borne supérieure. On pose
   \[M=\sup_{x\in\interf{a}{b}} f(x)\]
   Par définition de la borne supérieure
   \[\forall \epsilon>0 \qsep \exists x\in\interf{a}{b} \qsep
     f(x)\geq M-\epsilon\]
   Étant donné $n\in\N$, en posant $\epsilon=1/2^n>0$, il existe donc
   $x_n\in\interf{a}{b}$ tel que $f\p{x_n}\geq M-1/2^n$. 
   On construit ainsi une suite $\p{x_n}$ d'éléments de $\interf{a}{b}$. Cette
   suite étant bornée, d'après le théorème de Bolzano-Weierstrass, il existe
   une extractrice $\phi$ et un réel $l$ tel que
   \[x_{\phi(n)}\tendvers{n}{+\infty} l\]
   Comme
   \[\forall n\in\N \qsep a\leq x_{\phi(n)}\leq b\]
   on en déduit que $l\in\interf{a}{b}$. Comme de plus
   \[\forall n\in\N \qsep
     \underbrace{M-\frac{1}{2^{\phi(n)}}}_{\tendvers{n}{+\infty} M}
     \leq f\p{x_{\phi(n)}}\leq M\]
   on en déduit, d'après le théorème des gendarmes, que
   \[f\p{x_{\phi(n)}}\tendvers{n}{+\infty} M\]
   Or, par continuité de $f$ en $l$
   \[f\p{x_{\phi(n)}}\tendvers{n}{+\infty} f(l)\]
   donc $M=f(l)$. La borne supérieure de $f$ est donc atteinte.\\
   Comme $f$ est
   continue, $-f$ est continue donc atteint sa borne supérieure. Donc $f$
   atteint sa borne inférieure.
\end{itemize}
\end{preuve}

\begin{remarqueUnique}
\remarque Si $f:[a,b]\to\C$ est une fonction continue, on applique souvent ce théorème à la fonction $\abs{f}$.
  On en déduit notamment que $f$ est bornée.
\end{remarqueUnique}

\begin{exoUnique}
\exo Soit $f$ une fonction continue sur un segment $\interf{a}{b}$ telle
  que $\forall x\in\interf{a}{b} \qsep 0\leq f(x)<1$. Montrer que si
  $(u_n)$ est une suite d'éléments de $[a,b]$, alors
  \[f(u_n)^n\tendvers{n}{+\infty} 0.\]
  \begin{sol}
  $f$ étant continue sur $\interf{a}{b}$, elle est majorée et atteint ses bornes. Il existe donc $\alpha\in \interf{a}{b}$ tel que $\forall x \in \interf{a}{b}, 0\leq f(x)\leq f(\alpha)<1$. D'où le résultat en élevant à la puissance $n$ et d'après le théorème des gendarmes.
  \end{sol}
\end{exoUnique}


 
\begin{proposition}[utile=-3]
L'image d'un segment par une fonction réelle continue est un segment.
\end{proposition}

\begin{preuve}
Soit $f$ un fonction continue sur un segment $S$. Puisque $S$ est un
intervalle, d'après le théorème des valeurs intermédiaires, $f(S)$ est
un intervalle. D'après la proposition précédente, $f(S)$ est borné et contient
ses bornes. C'est donc un segment.
\end{preuve}

\subsection{Continuité uniforme}

\begin{definition}[utile=-3]
On dit qu'une fonction $f:\mathcal{D}\to\K$ est \emph{uniformément continue} lorsque
\[\forall \epsilon>0 \qsep \exists \eta>0 \qsep \forall x,y\in\mathcal{D}\qsep
  \abs{x-y}\leq\eta \implique \abs{f(x)-f(y)}\leq\epsilon.\]
\end{definition}

\begin{remarqueUnique}
\remarque Une fonction lipschitzienne est uniformément continue.
% \remarque Soit $\alpha\in\RPs$. On dit qu'une fonction $f$ est
%   $\alpha$-Hölderienne lorsqu'il existe $M\in\RP$ tel que
%   \[\forall x\in\mathcal{D} \qsep \abs{f(x)-f(y)}\leq M\abs{x-y}^\alpha\]
%   Une fonction Hölderienne est uniformément continue.
\end{remarqueUnique}

\begin{exoUnique}
\exo Montrer que la fonction $x\mapsto\sqrt{x}$ est uniformément continue
  mais n'est pas lipschitzienne.
\end{exoUnique}

\begin{proposition}[utile=-3]
Si $f$ est uniformément continue, alors elle est continue.
\end{proposition}

\begin{remarqueUnique}
\remarque Soit $f$ une fonction continue. Alors
  \[\forall x\in\mathcal{D} \qsep \forall \epsilon>0 \qsep \exists \eta>0
    \qsep \forall y\in\mathcal{D} \qsep \abs{x-y}\leq\eta \implique
    \abs{f(x)-f(y)}\leq\epsilon.\]
  Les deux premiers quantificateurs étant de même nature, on peut les échanger,
  donc
  \[\forall \epsilon>0 \qsep \forall x\in\mathcal{D} \qsep \exists \eta>0
    \qsep \forall y\in\mathcal{D} \qsep \abs{x-y}\leq\eta \implique
    \abs{f(x)-f(y)}\leq\epsilon.\]
  Une fonction est donc uniformément continue lorsqu'on peut échanger les
  quantificateurs portant sur $x$ et $\eta$, c'est-à-dire lorsqu'il est
  possible de choisir $\eta$ indépendamment de $x$.
% \remarque Supposons que $f$ soit uniformément continue. Alors, si $\p{u_n}$
%   et $\p{v_n}$ sont deux suites d'éléments de $\mathcal{D}$ telles que
%   \[v_n-u_n\tendvers{n}{+\infty}0\]
%   on a $f\p{u_n}-f\p{v_n}\tendvers{n}{+\infty}0$. En effet, soit $\epsilon>0$.
%   Puisque $f$ est uniformément continue, il existe $\eta>0$ tel que
%   \[\forall x,y\in\mathcal{D}\qsep
%     \abs{x-y}\leq\eta \implique \abs{f(x)-f(y)}\leq\epsilon\]
%   Puisque $u_n-v_n$ tend vers 0, il existe $N\in\N$ tel que
%   \[\forall n\geq N \qsep \abs{u_n-v_n}\leq\eta\]
%   Donc, pour $n\geq N$, on a $\abs{f\p{u_n}-f\p{v_n}}\leq\epsilon$. En
%   conclusion
%   \[f\p{u_n}-f\p{v_n}\tendvers{n}{+\infty}0\]
%   En particulier, si on trouve deux suites $\p{u_n}$ et $\p{v_n}$ d'élements
%   de $\mathcal{D}$ telles que $u_n-v_n$ converge vers 0 et $f\p{u_n}-f\p{v_n}$
%   ne converge pas vers 0, on peut en conclure que $f$ n'est pas uniformément
%   continue.\\
%   On démontre ainsi que $x\mapsto x^2$ n'est pas uniformément continue.
%   En effet si $\p{u_n}$ et $\p{v_n}$ sont définies par
%   \[\forall n\in\Ns \qsep u_n=n+\frac{1}{n} \et v_n=n\]
%   alors $v_n-u_n=1/n$ tend vers 0, et
%   \[u_n^2-v_n^2=\p{n+\frac{1}{n}}^2-n^2=2+\frac{1}{n^2}\tendvers{n}{+\infty}
%     2\neq 0\]
%   Donc $x\mapsto x^2$ n'est pas uniformément continue.
\end{remarqueUnique}

\begin{exoUnique}
\exo Montrer que la fonction $f$ définie sur $\R$ par $f(x)\defeq x^2$
  n'est pas uniformément continue.
  \begin{sol}
  On souhaite montrer que
  \[\non\cro{\forall \epsilon>0 \qsep \exists \eta>0 \qsep
           \forall x,y\in\R\qsep
           \abs{x-y}\leq\eta \implique \abs{x^2-y^2}\leq\epsilon}\]
  c'est-à-dire que
  \[\exists \epsilon>0 \qsep \forall \eta>0 \qsep \exists x,y\in\R
    \qsep
    \begin{cases}
    \abs{x-y}\leq\eta \\
    \abs{x^2-y^2}>\epsilon
    \end{cases}\]
  On pose $\epsilon=1$. Soit $\eta>0$. Alors, il existe $n\in\Ns$ tel que
  $1/n\leq\eta$. On pose alors $x=n+1/n$ et $y=n$. Alors $\abs{x-y}=1/n\leq\eta$
  et $\abs{x^2-y^2}=2+1/n^2>1=\epsilon$. Donc $f$ est continue mais n'est pas
  uniformément continue.      
  \end{sol}
\end{exoUnique}

\begin{theoreme}[nom=Théorème de \nom{Heine}]
Sur un segment, une fonction continue est uniformément continue.
\end{theoreme}

\begin{preuve}
Soit $f$ une fonction continue sur $\interf{a}{b}$. Montrons qu'elle y est
uniformément continue. On raisonne par l'absurde et on suppose que
\[\non\cro{\forall \epsilon>0 \qsep \exists \eta>0 \qsep
           \forall x,y\in\interf{a}{b}\qsep
           \abs{x-y}\leq\eta \implique \abs{f(x)-f(y)}\leq\epsilon}\]
donc
\[\exists \epsilon>0 \qsep \forall \eta>0 \qsep \exists x,y\in\interf{a}{b}
  \qsep
  \begin{cases}
  \abs{x-y}\leq\eta \\
  \abs{f(x)-f(y)}>\epsilon
  \end{cases}\]
Soit un tel $\epsilon$. Étant donné $n\in\N$, en posant $\eta=1/2^n>0$, il
existe $x_n,y_n\in\interf{a}{b}$ tels que $\abs{x_n-y_n}\leq 1/2^n$
et $\abs{f\p{x_n}-f\p{y_n}}\geq \epsilon$. On construit ainsi deux suites
$\p{x_n}$ et $\p{y_n}$ d'éléments de $\interf{a}{b}$. Comme $\p{x_n}$ est
bornée, d'après le théorème de Bolzano-Weierstrass, il existe une extractrice
$\phi$ et un réel $l$ tel que
\[x_{\phi(n)}\tendvers{n}{+\infty} l\]
Comme
\[\forall n\in\N \qsep a\leq x_{\phi(n)}\leq b\]
on en déduit que $l\in\interf{a}{b}$. De plus
\begin{eqnarray*}
\forall n\in\N \qsep \abs{y_{\phi(n)}-l}
&=& \abs{y_{\phi(n)}-x_{\phi(n)}+x_{\phi(n)}-l}\\
&\leq& \abs{y_{\phi(n)}-x_{\phi(n)}}+\abs{x_{\phi(n)}-l}\\
&\leq& \frac{1}{2^{\phi(n)}}+\abs{x_{\phi(n)}-l}\tendvers{n}{+\infty} 0
\end{eqnarray*}
Donc 
\[y_{\phi(n)}\tendvers{n}{+\infty} l\]
Par continuité de $f$ en $l$, on en déduit que
\[\abs{f\p{x_{\phi(n)}}-f\p{y_{\phi(n)}}}\tendvers{n}{+\infty}
  \abs{f(l)-f(l)}=0\]
C'est contradictoire avec le fait que
\[\forall n\in\N \qsep \abs{f\p{x_{\phi(n)}}-f\p{y_{\phi(n)}}}\geq \epsilon\]
Donc $f$ est uniformément continue.
\end{preuve}
%END_BOOK

\end{document}


