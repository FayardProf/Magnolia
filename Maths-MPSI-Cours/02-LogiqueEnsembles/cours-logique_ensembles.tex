\documentclass{magnolia}

\magtex{tex_driver={pdftex},
        tex_packages={slashbox,epigraph,xypic}}
\magfiche{document_nom={Cours sur les ensembles},
          auteur_nom={François Fayard},
          auteur_mail={fayard.prof@gmail.com}}
\magcours{cours_matiere={maths},
          cours_niveau={mpsi},
          cours_chapitre_numero={2},
          cours_chapitre={Logique, Ensembles}}
\magmisenpage{}
\maglieudiff{}
\magprocess

\begin{document}

%BEGIN_BOOK
\setlength\epigraphwidth{.7\textwidth}
\epigraph{\og Si la logique est l'hygiène du mathématicien, ce n'est pas elle qui lui fournit sa
  nourriture; le pain quotidien dont il vit, ce sont les grands problèmes.\fg}{--- \textsc{André Weil (1906-1998)}}
\setlength\epigraphwidth{.6\textwidth}
\epigraph{\og  Sur l’enseigne du barbier du village, on peut lire~:~Je rase tous les hommes du
  village qui ne se rasent pas eux-mêmes, et seulement ceux-là. Savez-vous qui rase
  le barbier ? \fg}{--- \textsc{Bertrand Russel (1872--1970)}}
\setlength\epigraphwidth{.5\textwidth}
\epigraph{\og J'aimais et j'aime encore les mathématiques pour elles-mêmes comme n'admettant
  pas l'hypocrisie et le vague, mes deux bêtes d'aversion.\fg}{--- \textsc{Stendhal (1783-1842)}}

\magtoc

\section{Éléments de logique}

\subsection{Assertion, prédicat}

\begin{definition}[utile=-3]
\begin{itemize}
\item On appelle \emph{assertion} toute phrase mathématique à laquelle
  on peut attribuer une et une seule valeur de vérité~: vrai ou faux.
\item Soit $E$ un ensemble. On appelle \emph{prédicat} sur $E$ toute phrase mathématique
  dont la valeur de vérité dépend d'un élément $x\in E$.
\end{itemize}
\end{definition}

\begin{exemples}
\exemple \og 7 est un nombre premier \fg est une assertion vraie.
  L'assertion \og 7 est divisible par 3 \fg est fausse.
\exemple $P(x)\defeq$ \og $x$ est rationnel \fg est un prédicat sur $\R$. $P\p{3/4}$
  est vrai alors que $P(\sqrt{2})$ est faux.
\exemple $P\p{a,b,c}\defeq$ \og $a^2+b^2=c^2$ \fg est un prédicat sur $\N^3$.
\exemple \og L'ensemble des nombres premiers est infini \fg est une assertion
  vraie. L'assertion \og Il existe une infinité de nombres premiers $p$ tels
  que $p+2$ est premier \fg est une assertion dont on pense qu'elle est vraie. Mais aujourd'hui, personne n'en a fait la preuve.
\end{exemples}

\begin{remarques}
\remarque Deux principes fondamentaux gouvernent les valeurs de vérité des assertions.
  \begin{itemize}
  \item Le \emph{principe de non-contradiction}~: Une assertion ne peut être à la fois vraie et fausse.
  \item Le \emph{principe du tiers exclu}~: Une assertion qui n'est pas vraie est fausse.
  \end{itemize}
\remarque Si $P$ est un prédicat, on dit que $P$ est vrai lorsque, quel que
  soit $x\in E$, $P(x)$ est vraie. Dire que $P$ n'est pas vrai
  signifie qu'il existe $x\in E$ tel que $P(x)$ est faux.
\end{remarques}

\begin{definition}[utile=-3]
$\quad$
\begin{itemize}
\item Le \emph{quantificateur universel} $\forall$ signifie \og pour tout \fg
\item Le \emph{quantificateur existentiel} $\exists$ signifie \og il existe (au moins) un \fg
\end{itemize}
\end{definition}

\begin{remarqueUnique}
\remarque On trouve parfois le quantificateur $\exists !$ qui signifie
  \og il existe un unique \fg. 
\end{remarqueUnique}

\begin{exos}
\exo Les assertions suivantes sont-elles vraies~?
  \begin{questions}
  \question $\forall y\in\R \qsep \exists x\in\R \qsep x+y\geq 0$.
  \question $\exists x\in\R \qsep \forall y\in\R \qsep x+y\geq 0$.
  \question $\exists x\in\R \qsep \forall y\in\R \qsep y^2\geq x$.
  \end{questions}
\exo Déterminer les $x\in\R$ tels que
  \[\forall n\in\N \qsep x^{n+2}\leq x^{n+1}+x^n.\]
\end{exos}

\begin{sol}
$x^n(x^2-x-1)\leq 0$. Si $x\geq 0$, ok.
Si $x<0$, on doit avoir $x^2-x-1=0$, d'où $S= \interf{0}{\frac{1+\sqrt{5}}{2}}\cup\ens{\frac{1-\sqrt{5}}{2}}$.
\end{sol}

\begin{definition}[utile=-3]
Soit $P$ et $Q$ deux assertions.
\begin{itemize}
\item On définit l'assertion $\p{\non\  P}$ comme étant vraie
  lorsque $P$ est fausse et fausse lorsque $P$ est vraie.
\item On définit l'assertion $\cro{P \et Q}$ comme étant vraie
  lorsque $P$ et $Q$ sont vraies et fausse sinon.
\item On définit l'assertion $\cro{P \ou Q}$ comme étant vraie
  lorsqu'au moins l'une des deux assertions est vraie, et fausse sinon.
\end{itemize}
\end{definition}

\begin{remarques}
\remarque Les valeurs de vérité de ces assertions sont données
  par les tables suivantes.
  \begin{center}
  \begin{minipage}{0.3\linewidth}
  \begin{center}
  \begin{tabular}{|c|c|c|}
  \hline
  P & V & F \\
  \hline
  \non\ P &  F & V \\
  \hline
  \end{tabular}
  \[\non\ P\]
  \end{center}
  \end{minipage}
  \begin{minipage}{0.3\linewidth}
  \begin{center}
  \begin{tabular}{|c|c|c|}
  \hline
  \backslashbox{P}{Q} & V & F \\
  \hline
  V & V & F\\
  \hline
  F & F & F\\
  \hline
  \end{tabular}
  \[P \et Q\]
  \end{center}
  \end{minipage}
  \begin{minipage}{0.3\linewidth}
  \begin{center}
  \begin{tabular}{|c|c|c|}
  \hline
  \backslashbox{P}{Q} & V & F \\
  \hline
  V & V & V\\
  \hline
  F & V & F\\
  \hline
  \end{tabular}
  \[P \ou Q\]
  \end{center}
  \end{minipage}

  \end{center}
\remarque Lorsque le menu d'un restaurant vous propose \og fromage ou dessert
  \fg, le \og ou \fg est employé au sens strict (on dit aussi exclusif); il
  n'est pas possible d'avoir les deux. En mathématiques, le \og ou \fg est
  employé au sens large (on dit aussi inclusif). Lorsqu'on dit qu'un
  entier naturel $n$ est divisible par 2 ou par 3, il peut très bien être divisible par 2 et par 3.
\end{remarques}

\subsection{Implication, équivalence}

\begin{definition}[utile=-3]
Soit $P$ et $Q$ deux assertions. On définit l'assertion $P\implique Q$ comme
étant fausse lorsque $P$ est vraie et $Q$ est fausse, et vraie sinon.
\end{definition}

\begin{remarques}
\remarque Montrer $P\implique Q$ revient à prouver que si $P$ est
  vraie, alors $Q$ est vraie.
\remarque Si $P$ et $Q$ sont deux prédicats sur $E$, $P\implique Q$
  signifie que $Q(x)$ est vraie dès que $P(x)$ est vraie.
  Si c'est le cas, on écrit
  \[\forall x\in E \qsep P(x) \implique Q(x)\]
  et on dit que $P$ est une condition suffisante pour $Q$ ou que $Q$ est une
  condition nécessaire pour $P$.
\end{remarques}

\begin{exos}
\exo Dans les exemples suivants, dites si le prédicat $P$ est une
  condition nécessaire ou une condition suffisante pour $Q$.
  \begin{itemize}
  \item $E=\R$, $P(x)\defeq$ \og $x\in\Q$ \fg et $Q(x)\defeq$ \og $x^2\in\Q$ \fg. 
  \item $E$ est l'ensemble des triangles du plan, $P(T)\defeq$ \og
    $T$ est isocèle \fg et $Q(T)\defeq$ \og $T$ est équilatéral \fg.
  \item $E=\R^2$, $P\p{x,y}\defeq$ \og $x\equiv y\ [2\pi]$ \fg et $Q\p{x,y}\defeq$
    \og $x\equiv y\ [\pi]$ \fg.
  \end{itemize}
\exo Montrer que
  \[\forall x,y\in\R \qsep\cro{xy>0 \et x+y>0}\quad\implique\quad
    \cro{x>0 \et y>0}.\]
\exo Montrer que
  \[\forall x\in\R \qsep
    \cro{\forall \epsilon\in\RPs \qsep\abs{x}\leq\epsilon} \quad\implique\quad
    x=0.\]
\end{exos}

\begin{proposition}[utile=-3, nom={Modus Ponens}]
Soit $P$ et $Q$ deux assertions. Si $P$ et $P\implique Q$ sont vraies, alors
$Q$ est vraie.   
\end{proposition}

\begin{remarqueUnique}
\remarque En pratique, on utilise cette proposition lorsque $P$ et $Q$ sont des
  prédicats. Si $P\implique Q$ est vrai et $x$ est un élément de $E$ tel que
  $P(x)$ est vrai, alors $Q(x)$ est vrai. Dans ce cadre, on dit que
  $P\implique Q$ est un théorème. Vérifier les hypothèses
  du théorème revient à vérifier que $P(x)$ est vrai et appliquer le
  théorème nous permet de conclure que $Q(x)$ est vrai.
  Traduisons mathématiquement le raisonnement suivant~: \og Socrate
  est un homme. Puisque tous les hommes sont mortels, alors Socrate est
  mortel \fg. Si $P(x)\defeq$ \og $x$ est un homme \fg et $Q(x)\defeq$ \og $x$ est
  mortel \fg, alors l'énoncé \og Tous les hommes sont mortels \fg s'écrit
  \[\forall x\in U \qsep P(x) \implique Q(x).\]
  Puisque Socrate est un homme ($P\p{\text{Socrate}}$ est vrai), on en déduit
  que Socrate est mortel ($Q\p{\text{Socrate}}$ est vrai).
\end{remarqueUnique}

\begin{exoUnique}
\exo Soit $a,b\in\R$ tels que
  \[\forall x\in\R \qsep x<a \quad\implique\quad x\leq b.\]
  Montrer que $a\leq b$.
\end{exoUnique}

\begin{sol}
Absurde + on cale un $c$ strictement entre $a$ et $b$.
\end{sol}

\begin{definition}[utile=-3]
Soit $P$ et $Q$ deux assertions. On définit l'assertion $P\ssi Q$ comme
étant vraie lorsque $P$ et $Q$ ont même valeur de vérité, et fausse sinon.
\end{definition}

\begin{remarques}
\remarque Les valeurs de vérité des assertions $P \implique Q$ et
  $P \ssi Q$ sont regroupées dans les tableaux suivants.
  \begin{center}
  \begin{minipage}{0.3\linewidth}
  \begin{center}
  \begin{tabular}{|c|c|c|}
  \hline
  \backslashbox{P}{Q} & V & F \\
  \hline
  V &  V & F \\
  \hline
  F &  V & V \\
  \hline
  \end{tabular}
  \[P \implique Q\]
  \end{center}
  \end{minipage}
  \begin{minipage}{0.3\linewidth}
  \begin{center}
  \begin{tabular}{|c|c|c|}
  \hline
  \backslashbox{P}{Q} & V & F \\
  \hline
  V & V & F \\
  \hline
  F &  F & V\\
  \hline
  \end{tabular}
  \[P \ssi Q\]
  \end{center}
  \end{minipage}
  \end{center}
\remarque Les assertions $P\ssi Q$ et $Q\ssi P$ ont même valeur de vérité; on
  dit que la relation d'équivalence est symétrique.
\remarque Si $P$ et $Q$ sont deux prédicats sur $E$, dire que $P\ssi Q$
  est vrai signifie que $Q(x)$ et $P(x)$ ont même valeur de vérité
  quel que soit $x\in E$. Si c'est le cas, on écrit
  \[\forall x\in E \qsep P(x) \ssi Q(x)\]
  et on dit que $P$ est une condition nécessaire et suffisante pour $Q$.
\end{remarques}

\begin{proposition}[utile=-3]
Soit $P$ et $Q$ deux assertions. Alors $P\ssi Q$ et $[\p{P\implique Q}$ et
$\p{Q \implique P}]$ ont même valeur de vérité.
\end{proposition}

\begin{remarqueUnique}
\remarque Pour démontrer que $P\ssi Q$, on pourra démontrer que
  $P\implique Q$, puis que $Q\implique P$; on dit qu'on raisonne par
  double implication.
\end{remarqueUnique}

\begin{exoUnique}
\exo Soit $\lambda\in\R$ et $f$ la fonction définie sur $\R$ par
  \[\forall x\in\R \qsep f(x)\defeq \sin(\lambda x).\]
  Donner une condition nécessaire et suffisante sur $\lambda$ pour que
  $f$ soit $2\pi$-périodique.
% \exo Soit $a,b,c\in\R$ avec $a\neq 0$. On note $P\defeq aX^2+bX+c$. Donner une
%   condition nécessaire et suffisante sur $a,b,c$ pour qu'il existe $x,y\in\R$
%   tels que $P(x)=y$ et $P(y)=x$.
\end{exoUnique}

\begin{sol}
Ok pour $\lambda=0$
Si $f$ est $2\pi$-per, prendre $x=\pi/2\lambda$ et on obtient $\lambda\in\Z$. Réciproque ok.
\end{sol}

\begin{proposition}[utile=-3]
Soit $P$, $Q$, $R$ trois assertions. Alors
\begin{eqnarray*}
\cro{P \et \p{Q \ou R}} &\ssi& \cro{\p{P \et Q} \ou \p{P \et R}},\\
\cro{P \ou \p{Q \et R}} &\ssi& \cro{\p{P \ou Q} \et \p{P \ou R}}.
\end{eqnarray*}
\end{proposition}

\begin{preuve}

~~
\begin{center}
  \begin{tabular}{|c|c|c|c|c|c|c|c|}
  \hline
  P & Q & R & Q$\lor$ R & P$\wedge$ (Q$\lor$ R) & P$\wedge$ Q & P $\wedge$ R & (P$\wedge$ Q)$\lor$(P$\wedge$ R) \\
  \hline
  V &V & V &V &V &V &V &V \\
  \hline
  V & V & F & V &V &V &F & V \\
  \hline
  V& F & F & F &F &F &F & F \\
  \hline
  V & F & V & V &V &F &V & V \\
  \hline
  F & V & V & V &F &F &F & F \\
  \hline
  F & V & F & V &F &F &F & F \\
  \hline
  F &F &F & F &F &F &F & F \\
  \hline
  F &F &V & V &F &F &F & F \\
  \hline
  \end{tabular}
  Preuve de la première équivalence
  
\end{center}
\end{preuve}

\begin{proposition}[utile=-3, nom={Lois de \nom{Morgan}}]
Soit $P$ et $Q$ deux assertions. Alors
\begin{eqnarray*}
\non\p{P \et Q} &\ssi& \cro{\p{\non\  P} \ou \p{\non\  Q}},\\
\non\p{P \ou Q} &\ssi& \cro{\p{\non\  P} \et \p{\non\  Q}},\\
\non\p{\non\  P} &\ssi& P.
\end{eqnarray*}
\end{proposition}

\begin{proposition}[utile=-3, nom={Raisonnement par contraposée}]
Soit $P$ et $Q$ deux assertions. Alors
\[\cro{P \implique Q} \quad\ssi\quad \cro{\non\  Q \implique \non\  P}.\]
\end{proposition}

\begin{remarqueUnique}
\remarque Lorsque l'on démontre $\cro{\p{\non\  Q} \implique \p{\non\  P}}$ pour montrer
  que $\cro{P\implique Q}$, on dit que l'on raisonne par contraposée.
\end{remarqueUnique}

\begin{exoUnique}
\exo Supposons que l'on ait montré que $\pi^2$ est irrationnel. Peut-on en
  déduire que $\pi$ est irrationnel~?
\end{exoUnique}

\begin{proposition}[utile=-3]
Soit $P$ et $Q$ deux assertions. Alors
\[\cro{\non\p{P \implique Q}} \quad\ssi\quad \cro{P \et \p{\non\ Q}}.\]  
\end{proposition}


\begin{proposition}[utile=-3]
Soit $P$ un prédicat sur l'ensemble $E$. Alors
\begin{eqnarray*}
\non\cro{\forall x\in E \qsep P(x)} &\ssi&
\cro{\exists x\in E \qsep \non\p{P(x)}},\\
\non\cro{\exists x\in E \qsep P(x)} &\ssi&
\cro{\forall x\in E \qsep \non\p{P(x)}}.
\end{eqnarray*}
\end{proposition}

\begin{exoUnique}
\exo Soit $f:\R\to\R$. Écrire les phrases suivantes avec des
  quantificateurs. En déduire leur négation. \begin{center}
  \og $f$ est majorée \fg,
  \og $f$ est croissante \fg, \og $f$ est décroissante \fg. \end{center}
\end{exoUnique}
% \subsection{Principes de démonstration}

% Utiliser dans une preuve une hypothèse de la forme $\cro{P \et Q}$.



% Prouver une proposition de la forme $\cro{P \et Q}$.



% Utiliser dans un raisonnement une proposition de la forme $\cro{P \ou Q}$.

% \begin{exos}
% \exo Soit $a,b\in\R$. Montrer que
%    \[\max\p{a,b}=\frac{a+b+\abs{a-b}}{2} \et
%      \min\p{a,b}=\frac{a+b-\abs{a-b}}{2}\]
% \exo Si $n$ est la somme de deux carrés d'entiers, montrer que le reste
%   de la division euclidienne de $n$ par 4 n'est jamais égal à 3.
%   \begin{sol}
%   Traiter les cas où~:
%   \begin{itemize}
%   \item $a$ et $b$ sont pairs~: on trouve 0.
%   \item $a$ est pair, $b$ est impair~: on trouve 1. Le cas où $a$ est impair
%     et $b$ est pair se montre de même.
%   \item $a$ et $b$ sont impairs.
%   \end{itemize}
%   \end{sol}
% \end{exos}

% Prouver une proposition de la forme $P \ou Q$.

% Le principe à retenir est le suivant~:
% \begin{itemize}
% \item Si $P$ est vraie, alors $\cro{P \ou Q}$ est vraie.
% \item Sinon, on peut rajouter $\p{\non\ P}$ aux hypothèses et on cherche à
%   montrer que $Q$ est vraie.
% \end{itemize}

% \begin{exos}
% \exo Montrer qu'il existe deux irrationnels strictement positifs $x$ et $y$
%   tels que $x^y$ soit rationnel.
% \end{exos}


\section{Ensemble}
\subsection{Ensemble, élément}

\begin{definition}[utile=-3]
Les notions d'\emph{ensemble}, d'\emph{élément} et d'\emph{appartenance} sont des notions
premières en mathématiques que l'on ne définit pas. Intuitivement, un ensemble
est une collection d'objets mathématiques appelés éléments. La notation $x\in E$
signifie que l'élément $x$ appartient à l'ensemble $E$.
\end{definition}

\begin{remarqueUnique}
\remarque Si $x_1,\ldots,x_n$ sont des objets mathématiques, l'ensemble
  constitué de ces éléments est noté $\ens{x_1,\ldots,x_n}$.
\end{remarqueUnique}

\begin{definition}[utile=-3]
Soit $A$ et $B$ deux ensembles. On dit que $A$ est \emph{inclus} dans $B$ et on note
$A\subset B$ lorsque
\[\forall x\in A \qsep x\in B.\]  
\end{definition}

\begin{proposition}[utile=-3]
Deux ensembles $A$ et $B$ sont égaux lorsqu'ils possèdent les mêmes
éléments, c'est-à-dire lorsque
\[A\subset B \et B\subset A.\]
\end{proposition}

\begin{remarqueUnique}
\remarque En particulier $\ens{0,1}=\ens{1,0}$ et $\ens{0,0,1}=\ens{0,1}$.
\end{remarqueUnique}

\begin{definition}[utile=-3]
Soit $E$ un ensemble. On appelle partie de $E$ tout ensemble $A$ inclus dans $E$.
L'ensemble des parties de $E$ est noté $\mathcal{P}(E)$.
\end{definition}

\begin{remarqueUnique}
\remarque Un même objet mathématique peut très bien, selon le contexte, être un élément
  ou un ensemble. Par exemple, l'ensemble $\N$ est un élément de $\mathcal{P}(\R)$.
\end{remarqueUnique}

\begin{exoUnique}
\exo Déterminer $\parties{\ens{1,2}}$,
  $\parties{\parties{\emptyset}}$ et $\parties{\parties{\parties{\emptyset}}}$.
\end{exoUnique}

\begin{sol}
$\parties{\ens{1,2}}=\ens{\emptyset,\ens{1},\ens{2},\ens{1,2}}$. $\parties{\emptyset}=\ens{\emptyset}$ donc $\parties{\parties{\emptyset}}=\ens{\emptyset, \ens{\emptyset}}$. Pour finir, $\parties{\parties{\emptyset}}=\ens{\emptyset,\ens{\emptyset},\ens{\emptyset,\ens{\emptyset}},\ens{\ens{\emptyset}}}$.

\end{sol}

\subsection{Opérations élémentaires}

\begin{definition}[utile=-3]
Soit $E$ un ensemble et $P$ un prédicat sur $E$. On définit
\[\enstq{x\in E}{P(x)}\]
comme l'ensemble des éléments $x$ de $E$ tels que $P(x)$ est vrai. C'est une partie
de $E$.
\end{definition}

\begin{definition}[utile=-3]
Soit $A$ et $B$ deux parties de $E$. On définit
\[A\cap B\defeq\enstq{x\in E}{x\in A \et x\in B}, \qquad
  A\cup B\defeq\enstq{x\in E}{x\in A \ou x\in B},\]
\[\overline{A}\defeq\enstq{x\in E}{x\not\in A}.\]
\end{definition}

\begin{remarques}
\remarque Le complémentaire de $A$ dans $E$ est aussi noté $A^c$.
\remarque On dit que deux ensembles $A$ et $B$ sont disjoints lorsque
  $A\cap B=\emptyset$.
\end{remarques}

\begin{proposition}[utile=-3]
Soit $A$ et $B$ deux parties de $E$. Alors
\begin{eqnarray*}
A\cap\p{B\cup C}&=&\p{A\cap B}\cup\p{A\cap C},\\
A\cup\p{B\cap C}&=&\p{A\cup B}\cap\p{A\cup C}.
\end{eqnarray*}
\end{proposition}

\begin{preuve}
Avec la proposition 1.3, on raisonne par équivalence avec $x\in A\cap\p{B\cup C}$.
\end{preuve}

\begin{proposition}[utile=-3, nom={Lois de \nom{Morgan}}]
Soit $A$ et $B$ deux parties de $E$. Alors
\begin{eqnarray*}
\overline{A\cap B}&=&\overline{A}\cup \overline{B},\\
\overline{A\cup B}&=&\overline{A}\cap \overline{B},\\
\overline{\overline{A}}&=&A.
\end{eqnarray*}
\end{proposition}

\begin{preuve}
Avec la proposition 1.4, on raisonne par équivalence avec $x\in \overline{A\cap B}$.
\end{preuve}

\begin{exos}
\exo Soit $A$ et $B$ deux parties d'un même ensemble. Montrer que
  $\parties{A\cap B}=\parties{A}\cap\parties{B}$.
\exo Soit $A$, $B$, $C$ trois parties d'un ensemble $E$ non vide.
  \begin{questions}
  \question Si $A\cup B=A\cup C$, a-t-on $B=C$~?
  \question Si $A\cup B=A\cap B$, a-t-on $A=B$~?
  \question Montrer que si $A\cup B=A\cup C$ et $A\cap B=A\cap C$, alors $B=C$.
  \question Montrer que si $A\cup B=E$ et $A\cap B=\emptyset$, alors $A=\overline{B}$ et
  $B=\overline{A}$.
  \end{questions}
\end{exos}

\begin{definition}[utile=-3]
Soit $A$ et $B$ deux parties de $E$. On définit
\[A\setminus B\defeq\enstq{x\in E}{x\in A \et x\not\in B}.\]
\end{definition}

\begin{remarqueUnique}
\remarque En particulier, si $A$ est une partie de $E$, $\bar{A}=E\setminus A$.
\end{remarqueUnique}

\begin{definition}
Soit $A$ et $B$ deux ensembles. On définit $A\times B$ comme l'ensemble
des \emph{couples} $\p{a,b}$ avec $a\in A$ et $b\in B$. Par définition, deux couples
$\p{a_1,b_1}, \p{a_2,b_2}\in A\times B$ sont égaux lorsque $a_1=a_2$ et $b_1=b_2$.
\end{definition}

\begin{definition}[utile=-3]
\begin{itemize}
\item Si $A_1,\ldots,A_n$ sont $n$ ensembles, on définit
  $A_1\times\cdots\times A_n$ comme l'ensemble des $n$-uplets
  $\p{a_1,\ldots,a_n}$ avec $a_1\in A_1,\ldots,a_n\in A_n$. Par définition, deux $n$-uplets
  $\p{a_1,\ldots,a_n}, \p{b_1,\ldots,b_n}\in A_1\times\cdots\times A_n$ sont égaux lorsque
  \[\forall k\in\intere{1}{n} \qsep a_k=b_k.\]
\item Si A est un ensemble et $n\in\N$, on définit $A^n$ comme
  \[A^n\defeq\underbrace{A\times A\times\cdots\times A}_{\text{$n$ fois $A$}}.\]
\end{itemize}
\end{definition}

\begin{remarques}
\remarque $A^1$ est l'ensemble des 1-uplets $(a)$, pour $a\in A$; on confondra cet
  ensemble avec $A$. Quant à $A^0$, c'est l'ensemble qui contient un unique élément,
  le $0$-uplet $()$.
\remarque Pour énoncer qu'un prédicat portant sur deux variables est vrai, on peut écrire
  \og $\forall x\in A\qsep \forall y\in A\qsep P(x,y)$ \fg.
  On condense cependant souvent cette phrase en \og $\forall (x,y)\in A^2\qsep P(x,y)$ \fg ou
  en \og $\forall x,y\in A\qsep P(x,y)$\fg.
  % Daniel Perrin le fait
\end{remarques}

\section{Application}

\subsection{Définition, exemples}
\begin{definition}[utile=-3]
Soit $E$ et $F$ deux ensembles. Une \emph{application} $f$ de $E$ dans $F$ associe
à tout élément $x\in E$ un unique élément $f(x)\in F$, appelé image de $x$ par
$f$. On note
\[\dspappli{f}{E}{F}{x}{f(x).}\]
On dit que $E$ est le \emph{domaine} de $f$ et que $F$ est son \emph{codomaine}. L'ensemble des
applications de $E$ dans $F$ est noté $\mathcal{F}\p{E,F}$.
\end{definition}


\begin{remarques}
\remarque Deux applications sont égales lorsqu'elles ont même domaine et codomaine et qu'elles prennent la
  même valeur en chaque point de ce domaine.
\remarque On utilise aussi les expressions \og ensemble de départ \fg et \og ensemble d'arrivée \fg
  d'une application pour désigner respectivement son domaine et son codomaine.
\remarque \og application \fg et \og fonction \fg sont synonymes. L'usage veut cependant que l'on
  réserve le mot \og fonction \fg aux applications dont le domaine et le codomaine sont des parties de $\C$.
\remarque L'ensemble $\mathcal{F}\p{E,F}$ est aussi noté $F^E$.
\remarque Pour les fonctions usuelles, il arrive qu'on omette les parenthèses et
  qu'on écrive $\sin x$ au lieu de $\sin(x)$. Cependant, on ne se permettra pas de faire cela avec les
  autres fonctions.
\end{remarques}

\begin{definition}
Soit $A$ une partie de $E$. On appelle \emph{fonction caractéristique} de $A$ et on note $\mathds{1}_A$ l'application de $E$ dans $\ens{0,1}$ définie
par
  \[\forall x\in E \qsep \mathds{1}_A(x)\defeq
    \begin{cases}
    1 & \text{si $x\in A$}\\
    0 & \text{sinon.}
    \end{cases}\]
\end{definition}

\begin{remarques}
\remarque Deux parties $A$ et $B$ de $E$ sont égales si et seulement si $\mathds{1}_{A}=\mathds{1}_{B}$.
\remarque Si $A$ et $B$ sont deux parties de $E$, alors
  \[\forall x\in E\qsep \mathds{1}_{A\cap B}(x)=\mathds{1}_{A}(x)\mathds{1}_{B}(x) \et \mathds{1}_{A\cup B}(x)=\max\p{\mathds{1}_{A}(x), \mathds{1}_{B}(x)}.\]
\end{remarques}

%% Remarques :
%% 1) 
%% 2) 
%% 3) Lorsqu'on définit une fonction f:E->F, il faudra toujours bien veiller à
%%    ce que :
%%    - f(x) a un sens pour tout x dans E
%%    - f(x) est dans F
%%     -f(x) ne dépend que de x : (problème avec exp(i theta)->theta
%%
%% Exemples :
%% 1) f : R -> R   x -> 1/(1+x^2)
%% 2) Si A est une partie de E, on définit
%%    f: P(E) -> P(E)   X -> X \cap A
%% 3) f : F(R,R) -> F(RPs,RPs)  g -> exp o g o ln

\begin{definition}[utile=-3]
Si $f$ est une application de $E$ dans $F$, on appelle \emph{graphe} de $f$ l'ensemble
\[\enstq{\p{x,y}\in E\times F}{f(x)=y}.\]
\end{definition}

\begin{definition}[utile=-3]
Soit $f:E\to F$ et $y\in F$. On appelle \emph{antécédent} de $y$ tout élément $x\in E$
tel que $f(x)=y$.  
\end{definition}

%% Exemples :
%% 1) Soit f : R -> R  x -> sin(x)
%%    Calcul des antécédents du réel y

\begin{exoUnique}
\exo Soit $f$ l'application de $\R^2$ dans $\R^2$ qui au couple $(x,y)$
  associe le couple $(x+2y,xy)$. Déterminer les antécédents de $(3,1)$.
  \begin{sol}
  Ce sont $(1,1)$ et $(2,1/2)$.
  \end{sol}
\end{exoUnique}

\begin{definition}[utile=-3]
Soit $f:E\to F$.
\begin{itemize}
\item Soit $B$ une partie de $F$. On appelle image réciproque de $B$
  et on note $f^\applirec (B)$ l'ensemble des éléments de $E$
  dont l'image par $f$ est dans $B$.
  \[f^\applirec (B)\defeq\enstq{x\in E}{f(x)\in B}\]
\item Soit $A$ une partie de $E$. On appelle image directe de $A$ 
  et on note $f(A)$ l'ensemble des éléments de $F$ qui sont
  image d'un élément de $A$ par $f$.
  \[f(A)\defeq\enstq{y\in F}{\exists x\in A \qsep f(x)=y}\]  
  L'ensemble $f(E)$ est appelé image de $f$ et noté $\im f$.
\end{itemize}
\end{definition}

\begin{remarqueUnique}
\remarque L'ensemble image $f(A)$ est aussi noté
  \[\ensim{f(x)}{x\in A}.\]
\end{remarqueUnique}

\begin{exos}
\exo Soit $f$ la fonction
  \[\dspappli{f}{\R}{\R}{x}{\sin x}\]
  Calculer $f^{-1}(f(\ens{\pi/2}))$ et $f(f^{-1}(\ens{0, 2}))$.
\exo Soit $f$ la fonction de $\C\setminus\ens{\ii}$ dans $\C$ qui à $z$
  associe $\frac{z+\ii}{z-\ii}$. Calculer %$f\p{\U\setminus\ens{i}}$ et  
  $f^\applirec \p{\U}$.
  \begin{sol}
  On a
  \[f\p{e^{i\theta}}=-i\cotan\p{\frac{\theta}{2}-\frac{\pi}{4}}\]
  donc $f\p{\U\setminus\ens{i}}=i\R$.) On trouve $f^\applirec \p{\U}=\R$
  (utiliser que la norme au carré fait $1$ et passer par le conjugué).
  \end{sol}
\exo Soit $f$ la fonction de $\R$ dans $\R$ définie par
  \[\forall x\in\R \qsep f(x)\defeq\frac{x}{1+x^2}.\]
  En lisant le tableau de variations de $f$, intuiter $f(\R)$, puis
  prouver rigoureusement ce résultat.
  \begin{sol}
  On trouve $\interf{-1/2}{1/2}$ (étude de fonctions avec dérivée et tableau de variations).
  \end{sol}
% \exo Soit $f$ l'application de $\R$ dans $\R$ définie par
%   \[\forall x\in\R \qsep f(x)=\sqrt{3}\sin x+\cos x\]
%   Déterminer $f^\applirec\p{\interf{0}{1}}$.
\exo Soit $f$ une application de $E$ dans $F$. Si $A$ est une partie de $E$,
  comparer $f^{-1}(f(A))$ et $A$. De même, si $B$ est une partie de $F$,
  comparer $f(f^{-1}(B))$ et $B$.  
 \begin{sol}
 $f^{-1}(f(A))\supset A$ et $f(f^{-1}(B))\subset B$.
 \end{sol}
\end{exos}

%% Exemples :
%% 1) Étudier l'image réciproque par f(x)=x^2 de [-1,1] puis l'image directe par
%%    f de [-1,1]

\begin{definition}[utile=-3]
Soit $f$ une application de $E$ dans $F$.
\begin{itemize}
\item Si $A$ est une partie de $E$, l'application
  \[\dspappli{\restri{f}{A}}{A}{F}{x}{f(x)}\]
  est appelée \emph{restriction} de $f$ à $A$.
\item On dit qu'une application $g$ est un
  \emph{prolongement} de $f$ lorsque $f$ est une restriction de $g$.
\item Si $B$ est une partie de $F$ telle que
  \[\forall x\in E \qsep f(x)\in B\]
  l'application
  \[\dspappli{\corestri{f}{B}}{E}{B}{x}{f(x)}\]
  est appelée \emph{corestriction} de $f$ à $B$.
\end{itemize}
\end{definition}

\begin{remarqueUnique}
\remarque Soit $f:E\to F$, $A$ une partie de $E$ et $B$ une partie de $F$
  telles que
  \[\forall x\in A\qsep f(x)\in B.\]
  Alors, on peut définir l'application
  \[\dspappli{\restricorestri{f}{A}{B}}{A}{B}{x}{f(x)}\]
  appelée restriction de $f$ à $A$, corestreinte à $B$.
\end{remarqueUnique}


%% Exemples :
%% 1) l'application f : [-Pi/2,Pi/2] -> [-1,1]  x -> sin(x) est la restriction
%%    de l'application de g à [-Pi/2,Pi/2] corestreinte à [-1,1] où g est
%%    l'application R -> R  x -> sin(x)
%% 2) l'application f : R -> R  x -> (sin x)/x si x \neq 0 et x-> 1 si x=0
%%    est un prolongement de x->sin x/x. C'est d'ailleurs le seul prolongement
%%    continu en 0.

\begin{definition}[utile=-3]
Soit $f:E\to F$ et $g:F\to G$. On définit l'application $g\circ f$ de $E$ dans $G$ par
\[\forall x\in E\qsep (g\circ f)(x)\defeq g\p{f(x)}.\]
\end{definition}

\begin{remarqueUnique}
\remarque Si $A$ est une partie de $E$, alors $(g\circ f)(A)=g(f(A))$. De même, si
  $B$ est une partie de $F$, alors \[(g\circ f)^{-1}(B)=f^{-1}(g^{-1}(B)).\]
\end{remarqueUnique}

%% Exemple :
%% 1) Si f : R -> RP x -> x^2 et g : RP -> R x -> sqrt(x)
%%    alors (f o g) : RP -> RP  x -> x
%%    et    (g o f) : R  -> R   x -> abs(x) 
%%
%% Remarque :
%% 1) si f : E -> B et g : F -> G, et que B\subset F, on peut composer gof
%%    il suffit de restreindre g

\begin{proposition}[utile=-3]
Soit $f:E\to F$, $g:F\to G$ et $h:G\to H$. Alors
\[\p{h\circ g}\circ f=h\circ\p{g\circ f}.\]
On note cette application $h\circ g\circ f$.
\end{proposition}

\subsection{Application injective, surjective, bijective}

\begin{definition}[utile=-3]
Soit $f:E\to F$. On dit que $f$ est \emph{injective} lorsque
\[\forall x_1,x_2\in E \qsep f\p{x_1}=f\p{x_2} \implique x_1=x_2\]
c'est-à-dire lorsque tout élément de $F$ a au plus un antécédent.
\end{definition}

\begin{exos}
\exo Soit $f$ une fonction de $\R$ dans $\R$. Montrer que si $f$ est
  strictement monotone alors elle est injective. La réciproque est-elle vraie~?
\exo Soit $f$ une fonction de $\R$ dans $\R$ telle que :
  $\forall x,y\in\R \qsep \abs{f(x)-f(y)}\geq\abs{x-y}$. Montrer qu'elle est
  injective.
\exo Soit $\phi$ l'application qui à la fonction $f$ de
  $[-1,1]$ dans $\R$ associe la fonction $\phi(f)$ de $\R$ dans $\R$
  définie par
  \[\forall x\in\R \qsep \cro{\phi(f)}(x)\defeq f\p{\sin x}\]
  Montrer que $\phi$ est injective.
  \begin{sol}
  $\phi(f)=\phi(g)$ implique $f\p{\sin x}=g\p{\sin x}, \forall x\in\R$ donc $f$ et $g$ sont égales étant donné les valeurs prises par $\sin$.
  \end{sol}
\exo Soit $E$ un ensemble et $A$ une partie de $E$. Donner une condition
  nécessaire et suffisante sur $A$ pour que
  \[\dspappli{\phi}{\parties{E}}{\parties{E}}{X}{X\cap A}\]
  soit injective.
  \begin{sol}
  $A=E$, sinon on raisonne par l'absurde en prenant un élément $b$ de $E\setminus A$ et $X=A$, $Y=A\cup \ens{b}$.
  \end{sol}
\end{exos}

%% Exemple :
%% 1) f : RP -> R x->x^2, alors f est injective
%% 2) f : R -> R^2  x -> (x,0) est injective
%% 3) l'application phi qui a une fonction f de [-1,1] dans R associe le fonction
%%    g de R dans R définie par g(x)=f(sin(x)) est une application injective.  
%%
%% En pratique :
%% Pour montrer qu'une application est injective :
%% 1) On montre que l'équation f(x)=y a au plus une solution : pour cela, on
%%    peut raisonner par implication (ou par analyse)
%% 2) On montre que f(x_1)=f(x_2) implique x_1=x_2

\begin{definition}[utile=-3]
Soit $f:E\to F$. On dit que $f$ est \emph{surjective} lorsque
\[\forall y\in F \qsep \exists x\in E \qsep f(x)=y\]
c'est-à-dire lorsque tout élément de $F$ a au moins un antécédent.
\end{definition}

%% Exemple :
%% 1) f : R -> RP   x -> x^2 est surjective
%% 2) f : R^2 -> R   (x,y) -> x est surjective
%% 3) Si D est une droite du plan P, l'application f de P dans D qui au point M
%%    associe son projeté orthogonal sur D est surjective 
%%
%% En pratique :
%% 1) On montre que l'équation f(x)=y a au moins une solution
%%
%% Remarque :
%% 1) f est surjective si et seulement si f(E)=F

\begin{proposition}[utile=-3]
Une application $f:E\to F$ est surjective si et seulement si $\im f=F$.
\end{proposition}

\begin{exos}
% \exo Soit $f$ le fonction de $\R$ dans $\R$ définie par~:
%   $\forall x\in\R \qsep f(x)=x^3-x$. $f$ est-elle injective~? surjective~?
%   Calculer $f\p{\RP}$ et $f^\applirec \p{\interf{-6}{6}}$.
\exo L'application
  \[\dspappli{\phi}{\mathcal{F}\p{\R,\R}}{\R}{f}{f\p{0}}\]
  est-elle injective~? surjective~?
% \exo Soit $f:E\to F$ et $g:E\to G$. On définit l'application $\phi$ de
%   $E$ dans $F\times G$ par \[\forall x\in E \qsep \phi(x)\defeq \p{f(x),g(x)}.\]
%   Que dire des assertions \og $\phi$ est injective si et seulement si $f$
%   et $g$ le sont \fg et \og $\phi$ est surjective si et seulement si $f$ et
%   $g$ le sont \fg~?
%   \begin{sol}
%   Faux. Si $f=\id$ et $g=1$, $\phi$ est injective sans que $g$ le soit. Il suffit en fait que l'une des deux soit injective pour que $\phi$ le soit.
%   Pour la deuxième, c'est l'inverse, si $\phi$ est surjective, alors $f$ et $g$ le sont. Mais si $f=\id=g$, $\phi$ n'est pas surjective.
%   \end{sol}
\end{exos}

\begin{definition}[utile=-3]
On dit qu'une application $f:E\to F$ est \emph{bijective} lorsqu'elle est injective et
surjective, c'est-à-dire lorsque tout élément de $F$ possède un unique antécédent.
\end{definition}

\begin{exos}
\exo Montrer que la fonction $f$ qui à $x$ associe $\frac{1+\ii x}{1-\ii x}$
  réalise une bijection de $\R$ dans $\U\setminus\ens{-1}$.
  \begin{sol}
  $\displaystyle z=\frac{1+ix}{1-ix}\Longleftrightarrow x=-i \frac{z-1}{z+1}$, ce qui est bien dans $\R$, on le constate en prenant le conjugué et en utilisant que $\conj{z}=1/z$.
  \end{sol}
\exo Montrer que l'application
  \[\dspappli{f}{\N^2}{\N}{\p{a,b}}{2^a\p{2b+1}-1}\]
  est bijective.
\exo Soit $X$ un ensemble et $f:X^2\to X$ une bijection. Montrer que
  \[\dspappli{g}{X^3}{X}{(x,y,z)}{f\p{x,f(y,z)}}\]
  est bijective.
\end{exos}


%% Exemples :
%% 1) Si n>=1, f : RP -> RP  x -> x^n  est bijective
%% 2) f : RPs -> R  x -> ln(x) est bijective
%% 3) f : R -> RPs  x -> exp(x) est bijective
%% 4) f : [-Pi/2,Pi/2] -> [-1,1] -> sin(x) est bijective
%% 5) L'application de P(E) dans P(E) qui a X associe son complémentaire est
%%    bijective.
%% 6) L'apllication phi de F(RP,RP) dans F(R,R) qui à la fonction f associe la
%%    fonction g définie par g(x)=ln(f(exp(x))) est bijective.
%%
%% En pratique :
%% 1) On montre que l'équation f(x)=y a une et une seule solution
%% 2) On montre que f est injective, puis que f est surjective

\begin{proposition}[utile=-3]
\begin{itemize}
\item La composée de deux applications injectives est injective.
\item La composée de deux applications surjectives est surjective.
\item La composée de deux applications bijectives est bijective.
\end{itemize}
\end{proposition}

\begin{exos}
\exo Soit $f:E\to F$ et $g:F\to G$. Montrer que si $g\circ f$ est injective,
  alors $f$ est injective. De même, montrer que si $g\circ f$ est surjective,
  alors $g$ est surjective.
\exo Est-il vrai que si $g\circ f$ est bijective, $f$ et $g$ le sont~?
\begin{sol}
Contre-exemple : $f:\ens{0,1}\to\R$ telle que $f(x)=x$, $g:\R\to\ens{0,1}$ telle que $g(x)=x$ si $x\in \ens{0,1}$ et $g(x)=0$ sinon.
\end{sol}
\end{exos}

\begin{definition}[utile=-3]
Soit $E$ un ensemble. On appelle \emph{identité} et on note $\id_E$
l'application de $E$ dans $E$ définie par
\[\forall x\in E \qsep \id_E(x)\defeq x.\]
Si $f$ est une application de $E$ dans $F$
\[f\circ \id_E=f \et \id_F\circ f=f.\]
\end{definition}

\begin{proposition}[utile=-3]
Soit $f$ une application de $E$ dans $F$.
\begin{itemize}
\item L'application $f$ est bijective si et seulement si il existe une
  application $g:F\to E$ telle que
  \[g\circ f=\id_E \et f\circ g=\id_F.\]
  Si tel est le cas, $g$ est unique; on l'appelle \emph{bijection réciproque} de $f$
  et on la note $f^{-1}$.
\item Si $f:E\to F$ est bijective, $f^{-1}$ est bijective et $\p{f^{-1}}^{-1}=f$.
\end{itemize}
\end{proposition}

\begin{preuve}
On raisonne par double implication. 
\begin{itemize}
\item [$\bullet$]
Tout d'abord, si il existe une application $g:F\to E$ telle que \[g\circ f=\id_E \et f\circ g=\id_F.\]
Comme $\id_E$ est injective, $g\circ f$ est injective donc $f$ est injective (cf. ex. précédent). De même, comme $\id_F$ est surjective, $f\circ g $ est surjective donc $f$ est surjective.
Finalement, $f$ est bijective.
\item [$\bullet$] Si $f$ est bijective. Fixons $y\in F$, il existe un unique $x$ dans $E$ tel que $f(x)=y$. Notons-le $g(y)$. On définit ainsi une fonction $g:F\to E$ vérifiant $f(g(y))=y, \forall y \in F$. $g$ associe donc à tout élément de $F$ son antécédent par $f$. Ainsi, pour tout $x\in E$, $g$ associe à $f(x)$ son antécédent par $f$, c'est-à-dire $x$, donc $g(f(x))=x$.
\end{itemize}
Pour l'unicité, supposons qu'il existe $h$ qui réalise la même chose, on a donc $h\circ f=\id_E$, on compose par $g$ à droite et on obtient $h=g$.

Enfin, si $f$ est bijective, il existe donc $f^{-1} : F\to E$ telle que \[f^{-1}\circ f=\id_E \et f\circ f^{-1}=\id_F.\] donc en appliquant la première partie du théorème à $f^{-1}$ avec $f$ qui joue le rôle de $g$, on obtient bien que $f^{-1}$ est bijective et $\p{f^{-1}}^{-1}=f$.

\end{preuve}

\begin{remarques}
\remarque Soit $f$ une bijection de $E$ dans $F$. Quel que soit $y\in F$, si $x\in E$
  est tel que $f(x)=y$, alors $f^{-1}(y)=x$.
\remarque La fonction $\ln$ de $\RPs$ dans $\R$ est une bijection et sa
  bijection réciproque est la fonction $\exp$ de $\R$ dans $\RPs$.
\end{remarques}

\begin{exos}
\exo Montrer que l'application
  \[\dspappli{f}{\Z^2}{\Z^2}{(x,y)}{(2x+y,5x+3y)}\]
  est bijective et calculer $f^{-1}$.
	\begin{sol}
$f^{-1}(a, b)=(3a-b,-5a+2b)$
	\end{sol}
\exo Soit $f$ une bijection de $\R$ dans $\R$. Montrer que si $f$ est
  strictement croissante, il en est de même pour $f^{-1}$. Que dire si $f$
  est impaire~?
  \begin{sol}
  Si $f$ est impaire, on regarde $f(f^{-1}(-x))=-x$ et $f(-f^{-1}(x))=-f(f^{-1}(x))=-x$ donc $f^{-1}(-x)=-f^{-1}(x)$ par injectivité.
  
  Une fonction paire ne peut pas être bijective.
  \end{sol}
% \exo Soit $f:F\to E$ et $g:G\to E$ deux applications. Montrer qu'il existe
%   une application $h:G\to F$ telle que $g=f\circ h$ si et seulement si
%   $g(G)\subset f(F)$.
\end{exos}

%% Exemples : calcul de la bijection réciproque de
%% 1) Si n>=1, f : RP -> RP  x -> x^n  est bijective
%% 2) f : RPs -> R  x -> ln(x) est bijective
%% 3) f : R -> RPs  x -> exp(x) est bijective
%% 4) f : [-Pi/2,Pi/2] -> [-1,1] -> sin(x) est bijective
%% 5) L'application de P(E) dans P(E) qui a X associe son complémentaire est
%%    bijective.
%% 6) L'apllication phi de F(RP,RP) dans F(R,R) qui à la fonction f associe la
%%    fonction g définie par g(x)=ln(f(exp(x))) est bijective.
%%
%% En pratique :
%% 1) On résout l'équation f(x)=y. L'unique solution en x est f^{-1}(y)
%%
%% Applications :
%% 1) Changement de variable
%%    \forall x\in E_1 f_1(x)=f_2(x) \ssi
%%      \forall t\in E_2 f_1\p{x(t)}=f_2\p{x(t)}
%%    pour cela, il suffit que x soit surjective (exemple avec sinus).
%%    Le plus souvent le changement de variable est bijectif (exemple avec
%%    l'exponentielle)

\begin{proposition}[utile=-3]
Soit $f:E\to F$ et $g:F\to G$ deux applications bijectives. Alors $g\circ f$
est bijective et
\[\p{g\circ f}^{-1}=f^{-1} \circ g^{-1}.\]
\end{proposition}

\subsection{Famille}

Si $E$ est un ensemble, il est courant de se donner $n$ éléments
$f_1,\ldots,f_n$ de $E$. Cela revient à définir une application
\[\dspappli{f}{\intere{1}{n}}{E}{i}{f(i)}\]
où l'on pose $f(i)\defeq f_i$ pour tout $i\in\intere{1}{n}$. Nous dirons que $f$ est une
famille d'éléments de $E$ indexée par $\intere{1}{n}$. On peut généraliser ce principe
et construire des familles indexées par un ensemble quelconque. Par exemple, on peut considérer
l'application $f$ de $\R$ dans $\mathcal{F}(\R, \R)$, qui à $\lambda\in\R$ associe la
fonction $f_\lambda:\R\to\R$ définie par
\[\forall x\in\R\qsep f_\lambda(x)\defeq\e^{\lambda x}.\]
On a ainsi défini une famille d'éléments de $\mathcal{F}(\R, \R)$ indexée par $\R$.

\begin{definition}[utile=-3]
Soit $E$ un ensemble et $I$ un ensemble, appelé ensemble d'indices. On appelle
\emph{famille d'éléments de $E$ indexée par $I$} toute application
\[\dspappli{f}{I}{E}{i}{f_i.}\]
Cette application est notée $\p{f_i}_{i\in I}$. L'ensemble des familles
d'éléments de $E$ indexées par $I$ est noté $E^I$.
\end{definition}

\begin{remarques}
\remarque Une famille d'éléments de $E$ indexée par $\N$ est une suite d'éléments de $E$.
\remarque On appelle sous-famille d'une famille $\p{f_i}_{i\in I}$ toute famille
  de la forme $\p{f_i}_{i\in J}$ où $J$ est une partie de $I$.
\remarque Si $A$ est un ensemble, on dit qu'une famille $(f_i)_{i\in I}$ est la famille des éléments de $A$ lorsque $f$ est une bijection de $I$ dans $A$. Le fait de parler de \og la\fg famille des éléments de $A$ est un abus de langage, car cette famille n'est pas unique.
% Par abus de langage, bien qu'elle ne soit pas unique, on parlera souvent de la famille des éléments d'un ensemble pour désigner une telle famille.
% Si $A$ est un ensemble, on appelle famille des éléments de $A$
%   l'application
%   \[\dspappli{f}{A}{A}{a}{f_a=a}\]
%   que l'on note $(f_a)_{a\in A}$ ou plus simplement $(f_i)_{i\in I}$ (où $I=A$, ce
%   que l'on s'empresse d'oublier).
\end{remarques}

%% Exemples :
%% 1) une suite  d'éléments de E est une famille indexée par N
%% 2) (f_a)_(a\in\R) la famille indexée par \R des fonctions de \R dans \C définies
%%    par f_a(x)=exp(iax)
%% 3) La famille ([-1/n,1/n])_(n\in\Ns)
%% Exemples :
%% 1) la famille (f_n)_(n\in\N) définie par f_n(x)=exp(inx) est une sous-famille
%%    de la famille définie plus haut.

\begin{definition}[utile=-3]
Soit $E$ un ensemble et $\p{A_i}_{i\in I}$ une famille de parties de $E$. On
définit alors
\[\bigcap\limits_{i\in I} A_i\defeq \enstq{x\in E}{\forall i\in I \qsep x\in A_i},\]
\[\bigcup\limits_{i\in I} A_i\defeq \enstq{x\in E}{\exists i\in I \qsep x\in A_i}.\]
\end{definition}

%% Exemples :
%% 1) intersection des [-1/n,1/n]

\begin{exoUnique}
% \exo Soit $f:E\to F$ et $\p{A_i}_{i\in I}$ une famille de parties de
%   $E$. Comparer
%   \[f\p{\bigcap\limits_{i\in I} A_i} \et \bigcap\limits_{i\in I} f\p{A_i}\]
\exo Soit $f:E\to E$. On définit $f^n$ pour tout $n\in\N$ par
  \[f^0\defeq \id_E \et \cro{\forall n\in\N \qsep f^{n+1}\defeq f\circ f^n}\]
  Soit $A$ une partie de $E$. Pour tout $n\in\N$, on pose $A_n\defeq f^n(A)$.
  Enfin, on pose $B\defeq \cup_{n\in\N} A_n$. Montrer que $A\subset B$ et que
  $f(B)\subset B$.
%   De plus, montrer que si $C$ est une partie de $E$ telle que
%   $A\subset C$ et $f(C)\subset C$, alors $B\subset C$.
\end{exoUnique}

\begin{proposition}[utile=-3]
Soit $E$ un ensemble et $\p{A_i}_{i\in I}$ une famille de parties de $E$. Alors
\[\overline{\bigcap\limits_{i\in I} A_i}=\bigcup\limits_{i\in I} \overline{A_i} \et
  \overline{\bigcup\limits_{i\in I} A_i}=\bigcap\limits_{i\in I} \overline{A_i}.\]
\end{proposition}

\begin{definition}[utile=-3, nom=Partition]
Soit $E$ un ensemble et $\p{A_i}_{i\in I}$ une famille de parties de $E$.
On dit que $\p{A_i}_{i\in I}$ est une \emph{partition} de $E$ lorsque
\[E = \bigcup_{i\in I} A_i \qquad\et\qquad \cro{\forall i,j\in I\qsep i\neq j\implique
  A_i\cap A_j=\emptyset}.\]
\end{definition}

\begin{remarques}
\remarque La définition de partition peut varier d'un cours à l'autre.
  Dans certains cours, on demande en plus que les $A_i$ soient non vides; on appelle
  alors \emph{recouvrement disjoint} ce que nous appelons ici partition.
\remarque La notion de partition a été définie à l'aide de familles. Mais on peut aussi
  la définir de manière ensembliste; on dit qu'une partie $\mathcal{R}$ de
  $\mathcal{P}(E)$ est une \emph{partition (au sens ensembliste)} de $E$ lorsque  
  \begin{itemize}
  \item $\forall x\in E\qsep \exists A\in\mathcal{R}\qsep x\in A$.
  \item $\forall A_1, A_2\in\mathcal{R}\qsep A_1\neq A_2\implique A_1\cap A_2=\emptyset$.
  \item $\forall A\in\mathcal{R}\qsep A\neq\emptyset$. 
  \end{itemize}
  Remarquons que dans la définition ensembliste, on demande à ce que les ensembles appartenant à $\mathcal{R}$
  soient non vides.
\end{remarques}

\begin{exoUnique}
\exo Déterminer les partitions (au sens ensembliste) de $E\defeq\ens{1,2,3}$.
\end{exoUnique}

\section{Relation binaire}

\begin{definition}[utile=-3]
Soit $E$ un ensemble. On appelle \emph{relation binaire} sur $E$ tout prédicat
$\mathcal{R}$ défini sur $E\times E$. Si $x$ et $y$ sont deux
éléments de $E$ et $\mathcal{R}\p{x,y}$ est vrai, on écrit $x\mathcal{R} y$.
\end{definition}

\begin{definition}[utile=-3]
On dit qu'une relation binaire $\mathcal{R}$ sur $E$ est
\begin{itemize}
\item \emph{réflexive} lorsque
  \[\forall x\in E \qsep x\mathcal{R} x.\]
\item \emph{transitive} lorsque
  \[\forall x,y,z\in E \qsep \cro{x\mathcal{R} y \et y\mathcal{R} z} \implique
    x\mathcal{R} z.\]
\item \emph{symétrique} lorsque
  \[\forall x,y\in E \qsep x\mathcal{R} y \implique y\mathcal{R} x.\]
\item \emph{antisymétrique} lorsque
  \[\forall x,y\in E \qsep \cro{x \mathcal{R}y \et y\mathcal{R} x} \implique
    x=y.\]
\end{itemize}
\end{definition}

\subsection{Relation d'ordre}

\begin{definition}[utile=-3]
On dit qu'une relation binaire $\preceq$ est une
\emph{relation d'ordre} lorsqu'elle est
\begin{itemize}
\item réflexive~: $\forall x\in E \qsep x\preceq x.$
\item transitive~: $\forall x,y,z\in E \qsep \cro{x\preceq y \et y\preceq z}
  \implique x\preceq z.$
\item antisymétrique~: $\forall x,y\in E \qsep \cro{x\preceq y \et y\preceq x}
  \implique x=y.$
\end{itemize}
On appelle \emph{ensemble ordonné} tout ensemble muni d'une relation d'ordre.
\end{definition}

\begin{remarques}
\remarque La relation $\leq$ est une relation d'ordre sur $\R$.  La relation
  $\leq$ définie sur $\mathcal{F}\p{\R,\R}$ par
  \[\forall f,g\in\mathcal{F}\p{\R,\R} \qsep f\leq g \quad\ssi\quad
    \cro{\forall x\in\R \qsep f(x)\leq g(x)}\]
  est une relation d'ordre sur $\mathcal{F}\p{\R,\R}$.
 \remarque Si $E$ est un ensemble,
  la relation d'inclusion est une relation d'ordre sur $\parties{E}$. 
\remarque Si $\preceq$ est une relation d'ordre sur $E$, la relation
  $\succeq$ définie par
  \[\forall x,y\in E \qsep x\succeq y \quad\ssi\quad y\preceq x\]
  est une relation d'ordre appelée relation d'ordre opposée à la première.
\remarque La relation $<$ n'est pas une relation d'ordre sur $\R$ car elle
  n'est pas réflexive.
\end{remarques}

\begin{exoUnique}
\exo Montrer que la relation | définie sur $\N$ par
  \[\forall a,b\in\N \qsep a|b \quad\ssi\quad \cro{\exists k\in\N \qsep b=ka}\]
  est une relation d'ordre sur $\N$.
\end{exoUnique}

%% Exemples :
%% 1) relation d'ordre <= sur R
%% 2) relation d'ordre <= sur F(R,R)
%% 3) inclusion sur P(E)
%% 4) divisibilité sur N
%%
%% Remarques :
%% 1) si <= est une relation d'ordre, >= est aussi une relation d'ordre appelée
%%    relation d'ordre opposée
%% 2) La relation < n'est pas une relation d'ordre sur R car elle n'est pas
%%    réflexive

\begin{definition}[utile=-3]
On dit qu'une relation d'ordre $\preceq$ est totale lorsque
\[\forall x,y\in E \qsep x\preceq y \ou y\preceq x.\]
\end{definition}

\begin{remarqueUnique}
\remarque La relation d'ordre $\leq$ est totale sur $\R$. Par contre, les
  relations $\leq$ sur $\mathcal{F}\p{\R,\R}$, $\subset$ sur $\parties{E}$ et
  | sur $\N$ ne sont pas totales.
\end{remarqueUnique}

%% Exemples :
%% Voir exemples précédents
%% 1) est totale
%% 2) n'est pas totale : f(x)=0 et g(x)=x
%% 3) n'est pas totale dès que E possède plus de deux éléments
%% 4) n'est pas totale : 2 et 3

\begin{definition}[utile=-3]
Soit $\p{E,\preceq}$ un ensemble ordonné et $A$ une partie de $E$.
\begin{itemize}
\item On dit que $M\in E$ est un \emph{majorant} de $A$ lorsque
  \[\forall a\in A \qsep a\preceq M.\]
\item On dit que $m\in E$ est un \emph{minorant} de $A$ lorsque
  \[\forall a\in A \qsep m\preceq a.\]
\end{itemize}
\end{definition}

\begin{exoUnique}
\exo Soit $c>0$. On définit la relation $\preceq$ sur $\R^2$ par
  \[\forall (x,t),(x',t')\in\R^2 \qsep (x,t)\preceq (x',t')\quad\ssi\quad \abs{x'-x}\leq c\cdot (t'-t).\]
  Vérifier que c'est une relation d'ordre. Dessiner l'ensemble des majorants et
  des minorants d'un couple $(x_0,t_0)$. L'ordre est-il total~?
  \begin{sol}
  L'ensemble des majorants et des minorants forme un cône (on peut le tracer dans un repère $(x,t)$ pour une valeur donnée de $t_0$). L'ordre n'est pas
  total car $(0,0)$ et $(1,0)$ ne sont pas comparables.
  \end{sol}
\end{exoUnique}

%% Exemples :
%% 1) Si A=[0,1[, l'ensemble des majorants de A est [1,+infty[
%% 2) Z n'admet ni majorant ni minorant dans R
%% 3) Si (A_i)_(i\in I), un majorant de {A_i : i\in I} est l'union, et un minorant
%%    de cet ensemble est l'intersection.
%% 4) Si a,b sont des entiers naturels, A={a,b} est majoré par ab. ppcm(a,b) est
%%    un autre majorant (pour la relation d'ordre de divisibilité)

\begin{definition}[utile=-3]
Soit $\p{E,\preceq}$ un ensemble ordonné et $A$ une partie de $E$.
\begin{itemize}
\item On dit que $A$ admet un \emph{plus grand élément} lorsqu'il existe un majorant
  de $A$ appartenant à $A$. Si un tel élément existe, il est unique et on
  l'appelle plus grand élément de $A$.
\item On dit que $A$ admet un \emph{plus petit élément} lorsqu'il existe un minorant
  de $A$ appartenant à $A$. Si un tel élément existe, il est unique et on
  l'appelle plus petit élément de $A$.
\end{itemize}
\end{definition}

\begin{remarques}
\remarque Muni de l'ordre usuel, $\interfo{0}{1}$ admet un plus petit élément 0
  mais n'admet pas de plus grand élément. Muni de la relation de divisibilité,
  $\ens{2,3}$ n'admet ni de plus grand ni de plus petit élément.
\remarque Un ensemble admettant un plus petit ou un plus grand élément est non
   vide.
\remarque Si $E$ est totalement ordonné et $A$ est une partie finie non vide de
  $E$, alors il admet un plus petit et un plus grand élément.
\end{remarques}

\subsection{Relation d'équivalence}

\begin{definition}[utile=-3]
On dit qu'une relation binaire $\mathcal{R}$ sur $E$ est une
\emph{relation d'équivalence} lorsqu'elle est
\begin{itemize}
\item réflexive~: $\forall x\in E \qsep x\mathcal{R} x.$
\item transitive~: $\forall x,y,z\in E \qsep \cro{x\mathcal{R} y \et
  y\mathcal{R} z} \implique x\mathcal{R} z.$
\item symétrique~: $\forall x,y\in E \qsep x\mathcal{R} y \implique
  y\mathcal{R} x.$
\end{itemize}
\end{definition}

\begin{remarqueUnique}
\remarque Si $E$ est un ensemble quelconque, le relation d'égalité est une
  relation d'équivalence. Si $n\in\N$, la relation $\mathcal{R}$ définie sur $\Z$ par
  \og $\forall a,b\in\Z \qsep a\mathcal{R} b \quad\ssi\quad a\equiv b\ [n]$ \fg
  est une relation d'équivalence. De même, si $f$ est une application
  de $E$ dans $F$, la relation $\mathcal{R}$ définie sur $E$ par
  \og $\forall x,y\in E \qsep x\mathcal{R}y \quad\ssi\quad f(x)=f(y)$ \fg
  est une relation d'équivalence.
\end{remarqueUnique}

\begin{exoUnique}
\exo Soit $E$ un ensemble. Montrer que la relation $\mathcal{R}$ définie
  sur $\parties{E}$ par
  \[\forall A,B\in\parties{E} \qsep A\mathcal{R} B \quad\ssi\quad
    \text{\og Il existe une bijection de $A$ dans $B$. \fg}\]
  est une relation d'équivalence.
\end{exoUnique}

%% Exemples :
%% 1) l'égalité sur E
%% 2) Si n\in\N la relation aRb \ssi a=b+kn est une relation d'équivalence sur
%%    \Z
%% 3) Si f : E -> F, la relation f(x)=f(y) est une relation d'équivalence sur E.
%% 4) La relation << est dans la même classe que >> est une relation
%%    d'équivalence sur l'ensemble des élèves de Janson de Sailly.

\begin{definition}[utile=-3]
Soit $\mathcal{R}$ une relation d'équivalence sur $E$ et $x\in E$. On appelle
\emph{classe d'équivalence de $x$} et on note ${\rm Cl}(x)$ l'ensemble des
éléments de $E$ en relation avec $x$
\[{\rm Cl}(x)\defeq\enstq{y\in E}{x\mathcal{R}y}.\]
On dit qu'une partie $A$ de $E$ est \emph{une classe d'équivalence} lorsqu'il existe
$x\in E$ tel que $A={\rm Cl}(x)$.
\end{definition}

% Soit $E$ un ensemble et $\p{A_i}_{i\in I}$ une famille de parties de $E$.
% \begin{itemize}
% \item On dit que $\p{A_i}_{i\in I}$ est un \emph{recouvrement disjoint} de $E$ lorsque
% \[\cro{\forall i,j\in I \qsep A_i\cap A_j\neq\emptyset
%   \implique i=j}  \et \bigcup\limits_{i\in I} A_i=E.\]
% \item On dit que $\p{A_i}_{i\in I}$ est une \emph{partition} de $E$ lorsque c'est un recouvrement disjoint et
%   que
%   \[\forall i\in I\qsep A_i\neq\emptyset.\]
% \end{itemize}

\begin{proposition}[utile=-3]
Soit $\mathcal{R}$ une relation d'équivalence sur $E$. Alors, la famille des
classes d'équivalence est une partition de $E$.
\end{proposition}

\begin{exoUnique}
\exo Soit $n\in\Ns$. Déterminer le nombre de classes d'équivalence sur $\Z$
  pour la relation de congruence modulo $n$.
\end{exoUnique}
  
  

\section{L'ensemble des entiers naturels}

Dans ce cours, nous ne chercherons pas à construire l'ensemble des entiers naturels.
Nous nous limiterons à la définition intuitive suivante.
\[\N\defeq\ens{0,1,2,3,\ldots}\]
Nous supposerons aussi définies les opérations usuelles $+$ et $\times$ ainsi que la
relation d'ordre totale $\leq$. Nous admettrons enfin la proposition suivante.

\begin{proposition}[utile=-3]
Toute partie non vide de $\N$ admet un plus petit élément.
\end{proposition}
  
\begin{proposition}[utile=-3]
Toute partie non vide majorée de $\N$ admet un plus grand élément.
\end{proposition}

\subsection{Récurrence}


\begin{preuve}
Admettons le premier point pour démontrer le second. Soit $A$ une partie non vide majorée de $\N$ et $B$ l'ensemble de ses majorants ($B\neq \emptyset$). $B$ admet donc un ppe. Notons-le $m$. Montrons que $m\in A$.

Supposons un instant que $m\notin A$. Comme $m$ est un majorant de $A$, on a $\forall x \in A, x<m$. Ainsi, $m>0$, sinon $A$ serait vide. Mais alors, $\forall x \in A, x\leq m-1$ donc $m-1$ est un majorant de $A$. CONTRADICTION. Donc $m$ est le pge de $A$.
\end{preuve}

\begin{proposition}[utile=-3, nom={Principe de récurrence}]
Soit $A$ une partie de $\N$ telle que
\begin{itemize}
\item $0\in A$,
\item $\forall n\in\N\qsep n\in A \implique n+1 \in A.$
\end{itemize}
Alors $A=\N$.
\end{proposition}

\begin{preuve}
Procédons par l'absurde. Supposons le contraire : $\exists n\in \N, n\notin A$.
Posons $B=\ens{n\in \N | n\notin A}$. $B$ est non vide donc admet un ppe. Notons-le $p$. $p\neq 0$ car $0\notin B$. Posons $q=p-1 \in \N$. $q\in A$ car il ne peut pas être dans $B$. Mais alors, on conclut d'après la deuxième hypothèse.

\end{preuve}

\begin{remarques}
\remarque Cette proposition est au coeur du principe de récurrence. Si $\mathcal{H}$ est un prédicat sur $\N$ tel que
\begin{itemize}
\item $\mathcal{H}_0$ est vraie,
\item $\forall n\in\N\qsep \mathcal{H}_n \implique \mathcal{H}_{n+1}$,
\end{itemize}
alors $\mathcal{H}_n$ est vraie pour tout $n\in\N$. Il suffit pour démontrer cela d'appliquer la proposition précédente à \[A\defeq\enstq{n\in\N}{\text{$\mathcal{H}_n$ est vraie}}.\]
\remarque Le principe de récurrence double est une conséquence du principe de récurrence. En effet, si $\mathcal{H}$ est un prédicat sur $\N$ tel que
\begin{itemize}
\item $\mathcal{H}_0$ et $\mathcal{H}_1$ sont vraies,
\item $\forall n\in\N\qsep \cro{\mathcal{H}_n \et \mathcal{H}_{n+1}}\implique \mathcal{H}_{n+2}$,
\end{itemize}
alors $\mathcal{H}_n$ est vraie pour tout $n\in\N$. Il suffit pour cela de remarquer que le prédicat $\mathcal{P}$ défini sur $\N$ par \[\forall n\in\N\qsep \mathcal{P}_n\defeq\text{\og} \mathcal{H}_n\text{ et }\mathcal{H}_{n+1}\text{ sont vraies} \text{\fg}\]
vérifie le principe de récurrence.
\remarque De même, le principe de récurrence forte est une conséquence du principe de récurrence.
  En effet, si $\mathcal{H}$ est un prédicat sur $\N$ tel que
\begin{itemize}
\item $\mathcal{H}_0$ est vraie,
\item $\forall n\in\N\qsep \cro{\mathcal{H}_0 \et\ldots\et \mathcal{H}_n}\implique \mathcal{H}_{n+1}$,
\end{itemize}
  alors $\mathcal{H}_n$ est vraie pour tout $n\in\N$. Il suffit pour cela de remarquer que le prédicat $\mathcal{P}$ défini sur $\N$ par 
\[\forall n\in\N\qsep \mathcal{P}_n\defeq\text{\og} \mathcal{H}_0, \mathcal{H}_1, \ldots, \mathcal{H}_{n}\text{ sont vraies} \text{\fg}.\]
vérifie le principe de récurrence.
\end{remarques}

\begin{exos}
\exo Montrer que pour tout $n\in\N$, $4^n+2$ est un multiple de 3.
\exo Soit $(u_n)$ une suite telle que
  \[u_0=1\qsep u_1=1, \et\cro{\forall n\in\N\qsep u_{n+2}=u_{n+1}+\frac{2}{n+2}u_n}.\]
  Montrer que pour tout $n\in\Ns$, $u_n\leq n^2$.
\exo Soit $f:\N\to\N$ une fonction surjective telle que
  \[\forall n\in\N\qsep f(n)\geq n.\]
  Montrer que $f=\id$.
\end{exos}

% \begin{exoUnique}
% \exo Montrer que
%   \[\forall n\geq 2\qsep \frac{1}{\sqrt{4n+1}}\leq\frac{1\times 3\times 5\times\cdots\times(2n-1)}{2\times 4\times 6\times\cdots\times (2n)}\leq \frac{1}{\sqrt{3n+1}}.\]
% \end{exoUnique}

% \begin{sol}
% Récurrence simple et on veut vérifier si $\dfrac{1}{\sqrt{4n+5}}\leq \dfrac{1}{\sqrt{4n+1}}\dfrac{2n+1}{2n+2}$. On passe tout au numérateur, on élève au carré et on développe et ça marche. Même chose de l'autre côté.
% \end{sol}

\subsection{Définition par récurrence}

\begin{proposition}
Soit $E$ un ensemble, $f\in\mathcal{F}(E,E)$  et $x\in E$. Alors, il existe une unique
suite $(u_n)$ d'éléments de $E$ telle que
\[u_0=x \quad\et\quad \forall n\in\N\qsep u_{n+1}=f(u_n).\]
\end{proposition}

\begin{definition}
Soit $E$ un ensemble, $A$ une partie de $E$ et $f\in\mathcal{F}(A, E)$. On dit qu'une
partie $B$ de $A$ est \emph{stable} par $f$ lorsque
\[\forall x\in B\qsep f(x)\in B.\]
\end{definition}

\begin{remarques}
\remarque Si $B$ est stable par $f$, il est possible de considérer la restriction de
  $f$ à $B$, corestreinte à $B$. On parle alors d'application \emph{induite} à $B$.
\remarque Il arrive souvent que l'on ait un ensemble $E$,
  $f\in\mathcal{F}(A, E)$ où $A$ est une partie de $E$, $x\in A$, et que l'on souhaite prouver l'existence d'une
  unique $(u_n)$ telle que
  \[u_0=x \quad\et\quad \forall n\in\N\qsep u_{n+1}=f(u_n).\]
  Si on trouve une partie $B$ de $A$ contenant $x$ et stable par $f$, il suffit
  d'appliquer la proposition précédente à l'application $f$ induite à $B$.
  Mais sans une telle partie $B$, il est impossible de prouver l'existence d'une telle suite.
  Supposons par exemple, qu'on se pose la question de l'existence d'une unique  suite
  de réels $(u_n)$ telle que
  \[u_0=\frac{1}{2} \quad\et\quad \forall n\in\N\qsep u_{n+1}=\frac{3 u_n-2}{u_n-1}.\]
  Nous voyons rapidement qu'une telle suite n'existe pas. En effet, si c'était le cas,
  on aurait $u_1=1$ et $(3u_1-2)/(u_1-1)$ ne serait pas défini. On ne peut tout
  simplement pas appliquer la proposition précédente, car la fonction
  \[\dspappli{f}{\R\setminus\ens{1}}{\R}{x}{\frac{3x-2}{x-1}}\]
  n'est pas définie sur $\R$ mais sur $\R\setminus\ens{1}$.
  % Par contre, si $x\in\interfo{2}{+\infty}$, il existe
  % bien une et une seule suite $(u_n)$ telle que
  % \[u_0=x \quad\et\quad \forall n\in\N\qsep u_{n+1}=\frac{3x-2}{x-1}\]
  % car~: $\forall x\geq 2\qsep f(x)\geq 2$. Autrement dit, $B\defeq \interfo{2}{+\infty}$ est stable par $f$.
\end{remarques}

\begin{exoUnique}
\exo Soit $x\in\interfo{2}{+\infty}$. Montrer qu'il existe une unique suite $(u_n)$ telle que
  \[u_0=x \quad\et\quad \forall n\in\N\qsep u_{n+1}=\frac{3 u_n-2}{u_n-1}.\]
\end{exoUnique}



% \subsection{Ensembles finis}

% \begin{definition}[utile=-3]
% On dit qu'un ensemble $E$ est \emph{fini} lorsqu'il existe $n\in\N$ tel que $\intere{1}{n}$ et $E$ sont en bijection. Sinon, on dit qu'il est \emph{infini}.
% \end{definition}

% \begin{proposition}[utile=-3]
% Si $E$ est un ensemble fini alors il existe un unique $n\in\N$ pour lequel il existe une bijection de $\intere{1}{n}$ dans $E$. Cet entier est appelé \emph{cardinal} de $E$ et noté  $\card(E)$, $|E|$, ou $\# E$.
% \end{proposition}

% \begin{preuve}
% S'il en existe un autre, cela donnera une bijection de $\intere{1}{n}$ dans $\intere{1}{m}$. Et si $n\neq m$, c'est impossible. En effet, démontrons cela par récurrence sur $n\in \N$ avec $\mathcal{H}n$ : "$\forall m\in \Ns$, s'il y a une bijection de $\intere{1}{n}$ dans $\intere{1}{m}$ alors $m=n$."

% On initialise à $1$ et pour l'hérédité, on considère la restriction où on enlève $n+1$ au départ et $f(n+1)$ à l'arrivée.
% \end{preuve}

% \begin{remarqueUnique}
% \remarque Soit $a,b\in\Z$ tels que $a\leq b$. Alors $\intere{a}{b}$ est fini et $\card(\intere{a}{b})=b-a+1$.
% \end{remarqueUnique}

% \begin{proposition}[utile=3] Soient $E$ et $F$ deux ensembles finis.
% \begin{itemize}
% \item Il existe une injection de $E$ dans $F$ si et seulement si $\card (E)\leq\card (F)$.
% \item Il existe une surjection de $E$ dans $F$ si et seulement si $\card (E)\geq\card (F)$.
% \item Il existe une bijection de $E$ dans $F$ si et seulement si $\card (E)=\card (F)$.
% \end{itemize}
% \end{proposition}

% \begin{preuve}
% Il suffit de considérer $f(E)$ dans un sens et de la construire dans l'autre.
% \end{preuve}


% \begin{remarqueUnique}
% \remarque Du premier point découle le principe des tiroirs~: si $f$ est une application d'un ensemble fini $E$ dans un ensemble fini $F$ de cardinal strictement inférieur, alors $f$ n'est pas injective. Par exemple, si on possède 4 T-shirts que l'on a rangés dans une commode possédant 3 tiroirs, alors deux T-shirts se trouvent dans un même tiroir.
% \end{remarqueUnique}

% \begin{exoUnique}
% \exo On estime que le nombre de cheveux qu'une personne a sur la tête est inférieur à $1\ 000\ 000$. Montrer qu'il existe deux français qui ont autant de cheveux sur la tête.
% \end{exoUnique}

% \begin{proposition}[utile=3]
% Soit $E$ un ensemble fini et $F \subset E$. Alors $F$ est fini et $\card (F)
% \leq \card (E)$. De plus, il y a égalité si et seulement si $E = F$.
% \end{proposition}

% \begin{preuve}
% On peut le faire par récurrence sur le cardinal de $E$. Dans l'hérédité, ou bien $F=E$ ou bien on prend $a\in E\setminus F$ et on raisonne sur $E\setminus\ens{a}$.
% \end{preuve}

% \begin{remarqueUnique}
% \remarque En particulier, un ensemble fini ne peut pas être en bijection avec une partie stricte de lui-même. Remarquons que c'est faux pour les ensembles infinis. Par exemple
% \[\dspappli{f}{\N}{\Ns}{n}{n+1}\]
% est une bijection alors que $\Ns$ est une partie stricte de $\N$.
% \end{remarqueUnique}

% \begin{proposition}[utile=3]
% Soit $E$ et $F$ deux ensembles finis et $f$ une application de $E$ dans $F$.
% \begin{itemize}
% \item Si $f$ est injective et $\card(E)=\card(F)$, alors $f$ est bijective.
% \item Si $f$ est surjective et $\card(E)=\card(F)$, alors $f$ est bijective.
% \end{itemize}
% Autrement dit, si $\card(E)=\card(F)$
% \[\text{$f$ est injective} \quad\ssi\quad \text{$f$ est bijective}
%    \quad\ssi\quad \text{$f$ est surjective}.\]
% \end{proposition}

% \begin{preuve}
% On utilise $f$ injective ssi $|f(E)|=|E|$ et $f$ surjective ssi $|f(E)|=|F|$.
% \end{preuve}

%END_BOOK

\end{document}