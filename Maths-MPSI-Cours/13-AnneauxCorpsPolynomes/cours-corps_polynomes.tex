\documentclass{magnolia}

\magtex{tex_driver={pdftex},
        tex_packages={xypic}}
\magfiche{document_nom={Cours sur les structures algebriques},
          auteur_nom={François Fayard},
          auteur_mail={fayard.prof@gmail.com}}
\magcours{cours_matiere={maths},
          cours_niveau={mpsi},
          cours_chapitre_numero={12},
          cours_chapitre={Anneaux, Corps, Polynômes}}
\magmisenpage{}
\maglieudiff{}
\magprocess


\begin{document}

%BEGIN_BOOK
\magtoc


\section{Anneau, corps}
\subsection{Anneau}

\begin{definition}
Soit $\p{A,+}$ un groupe commutatif (d'élément neutre $0_A$) et $\times$ une
loi de composition interne sur $A$. On dit que $\p{A,+,\times}$ est un \emph{anneau}
lorsque
\begin{itemize}
\item $\times$ est associatif,
\item $\times$ admet un élément neutre $1_A$,
\item $\times$ est distributive par rapport à $+$
  \begin{eqnarray*}
  \forall a,b,c\in A, & & a\times\p{b+c}=a\times b+a\times c,\\    
                     & & \p{a+b}\times c=a\times c+b\times c.
  \end{eqnarray*}
\end{itemize}
Un élément $a\in A$ est dit \emph{inversible} lorsqu'il est inversible pour la loi
$\times$. Un anneau $\p{A,+,\times}$ est dit \emph{commutatif} lorsque $\times$ est
commutative.
\end{definition}

\begin{exemples}
\exemple $(\C,+,\times)$ et $(\R,+,\times)$ sont des anneaux commutatifs.
\exemple Si $(A,+,\times)$ est un anneau et $X$ est un ensemble,
  l'ensemble $\mathcal{F}(X,A)$ des fonctions de $X$ dans $A$, muni des lois $+$ et $\times$ définies par
  \begin{eqnarray*}
  \forall f,g\in\mathcal{F}(X,A) \qsep \forall x\in X \qsep (f+g)(x)&\defeq&f(x)+g(x),\\
                                              (f\times g)(x)&\defeq&f(x)\times g(x)
  \end{eqnarray*}
  est un anneau. En particulier, $(\mathcal{F}(\R,\R),+,\times)$ et
  $(\R^\N,+,\times)$ sont des anneaux commutatifs.
\exemple Si $n\in\N$, alors $(\mat{n}{\R},+,\times)$ est un anneau dont l'élément neutre
  pour la multiplication est la matrice $I_n$. Il est non commutatif dès que
  $n\geq 2$.
\end{exemples}

\begin{remarqueUnique}
\remarque Si $a,b\in A$, on dit que $a$ et $b$ \emph{commutent} lorsque $a\times b=b\times a$.
\end{remarqueUnique}

%% Exemple :
%% 1) Z, R, C sont des anneaux
%% 2) Si A est un anneau et X est en ensemble non vide, F(X,A) est un anneau


\begin{proposition}
Soit $\p{A,+,\times}$ un anneau. Alors
\begin{eqnarray*}
\forall a\in A, & & 0_A \times a=0_A \quad\text{et}\quad a \times 0_A = 0_A\\
\forall a,b\in A, & & a\times\p{-b}=\p{-a}\times b=-\p{a\times b}\\
\forall a,b\in A \qsep \forall n\in\Z, & &
  \p{n\cdot a}\times b=a\times\p{n\cdot b}=n\cdot\p{a\times b}.
\end{eqnarray*}
\end{proposition}

\begin{preuve}
Clés : 
$$0\times a=(0+0)\times a=0\times a + 0\times a.$$
$$a(-b)+ab=a(-b+b)=0 \text{ donc } a(-b)=-(ab)$$
et $$(-a)b+ab=(-a+a)b=0 \text{ donc } (-a)b=-(ab)=a(-b).$$
\end{preuve}

\begin{remarques}
\remarque Soit $\p{A,+,\times}$ un anneau dans lequel $0_A=1_A$. Alors $A=\ens{0_A}$. Réciproquement, si $A$ est un ensemble contenant un unique élément muni des seules lois de composition interne $+$ et $\times$ que l'on peut définir sur cet ensemble, alors $A=\ens{0_A}$ et $(A,+,\times)$ est un anneau. On dit que cet anneau est l'anneau \emph{trivial}.
\remarque Dans la suite du cours, lorsqu'il n'y a pas de confusion possible,
  les éléments $0_A$ et $1_A$ seront respectivement notés $0$ et $1$.
\end{remarques}

\begin{proposition}
Soit $\p{A,+,\times}$ un anneau et $a,b\in A$ tels que $a\times b=b\times a$.
Alors, pour tout $n\in\N$
\[\p{a+b}^n = \sum_{k=0}^n \binom{n}{k} \cdot \p{a^{n-k} \times b^k} \quad\et\quad
  a^n-b^n=\p{a-b}\times\cro{\sum_{k=0}^{n-1} a^{\p{n-1}-k}\times b^k}.\]
\end{proposition}

\begin{remarques}
\remarque Ces relations peuvent être fausses lorsque $a$ et $b$ ne commutent
  pas. Par exemple, si $a$ et $b$ sont deux éléments d'un anneau, alors
  \[(a+b)^2=a^2+2\cdot a\times b+b^2 \quad\ssi\quad a\times b=b\times a.\]
\remarque Remarquons que si $a\in A$, alors $a$ commute
  avec $1$, donc ces formules sont valables pour développer $(1+a)^n$ et
  factoriser $a^n-1$.
\end{remarques}

\begin{definition}
On dit qu'un élément $a\in A$ est nilpotent lorsqu'il existe $n\in\N$ tel que $a^n=0$.
\end{definition}

\begin{exoUnique}
\exo Montrer que si $x$ est nilpotent, alors $1-x$ est inversible.
\end{exoUnique}
\begin{sol}
On vérifie que son inverse à gauche et à droite est $\displaystyle\sum_{k=0}^{n-1}x^k$.
\end{sol}

\begin{definition}
Soit $\p{A,+,\times}$ un anneau. L'ensemble $U_A$ des éléments inversibles de $A$
est un groupe pour la multiplication.
\end{definition}

\begin{remarqueUnique}
\remarque L'ensemble $U_A$ des inversibles de $A$ est parfois noté $A^\times$. Il est important de ne pas
  confondre cet ensemble avec $A^*\defeq A\setminus\ens{0}$.
\end{remarqueUnique}

\begin{exemples}
\exemple Le groupe des inversibles de $(\Z,+,\times)$ est $\ens{-1,1}$.
\exemple Le groupe des inversibles de $(\mat{n}{\R},+,\times)$ est $\gl{n}{\R}$.
\end{exemples}

\begin{sol}
Exemple de $(\Z,+,.)$ pour qui ses éléments inversibles $\set{-1;1}$ forment un groupe pour la multiplication.
\end{sol}


\begin{definition}
On dit qu'un anneau $\p{A,+,\times}$ est \emph{intègre} lorsque
\begin{itemize}
\item $1\neq 0$
\item $\times$ est commutative
\item $\forall a,b\in A \qsep a\times b=0 \implique \cro{a=0 \ou b=0}.$
\end{itemize}
\end{definition}

\begin{exoUnique}
\exo L'anneau $(\mathcal{F}(\R,\R),+,\times)$ est-il intègre~?
\end{exoUnique}

\begin{sol}
Non, évidemment. On peut prendre les indicatrices de deux parties disjointes. Le produit est nul sans que l'une des deux ne soient nulle.
\end{sol}


\begin{definition}
Soit $\p{A,+,\times}$ un anneau et $B$ une partie de $A$. On dit que $B$ est un
\emph{sous-anneau} de $A$ lorsque
\begin{itemize}
\item $0\in B \et 1\in B$
\item $\forall b_1,b_2\in B \qsep b_1+b_2\in B \qsep -b_1\in B \et b_1\times b_2\in B.$
\end{itemize}
Si tel est le cas $\p{B,+,\times}$ est un anneau.
\end{definition}

\begin{remarques}
\remarque Si $B$ est un sous-anneau de $(A,+,\times)$, $B$ est un
  sous-groupe de $(A,+)$.
\remarque Si $B$ est un sous-anneau de $\C$, alors $\Z\subset B$.
\end{remarques}

\begin{exoUnique}
\exo Montrer que $\Z[\ii]=\ensim{a+\ii b}{a,b\in\Z}$ est un sous-anneau de $\C$.
\end{exoUnique}

\begin{definition}
Soit $\p{A,+,\times}$ et $\p{B,+,\times}$ deux anneaux. On dit qu'une
application $\phi$ de $A$ dans $B$ est un \emph{morphisme d'anneau} lorsque
\begin{eqnarray*}
\forall a_1,a_2\in A, & & \phi\p{a_1+ a_2}=\phi\p{a_1}+\phi\p{a_2}\\
\forall a_1,a_2\in A, & & \phi\p{a_1\times a_2}=\phi\p{a_1}\times\phi\p{a_2}\\
& & \phi\p{1_A}=1_B.
\end{eqnarray*}
\end{definition}

\begin{proposition}
Soit $\phi$ un morphisme d'anneau de $\p{A,+,\times}$ dans $\p{B,+,\times}$.
Alors
\begin{eqnarray*}
\forall a\in A \qsep \forall n\in\Z, & & \phi\p{n\cdot a}=n\cdot\phi(a)\\
\forall a\in A \qsep \forall n\in\N, & & \phi\p{a^n}=\cro{\phi(a)}^n.
\end{eqnarray*}
De plus, si $a\in A$ est inversible, il en est de même pour $\phi(a)$ et
\[\forall n\in\Z \qsep \phi\p{a^n}=\cro{\phi(a)}^n.\]
\end{proposition}

% \begin{remarques}
% \remarque Soit $\phi$ un morphisme d'anneau de $(A,+,\cdot)$ dans
%   $(B,+,\cdot)$.
% \end{remarques}

\begin{proposition}
\begin{itemize}
\item La composée de deux morphismes d'anneaux est un morphisme d'anneau.
\item La bijection réciproque d'un isomorphisme est un isomorphisme.
\end{itemize}
\end{proposition}

\begin{proposition}
Soit $\phi$ un isomorphisme de l'anneau $(A,+,\times)$ dans l'anneau $(B,+,\times)$. Alors
\[\forall x\in A\qsep x\in U_A \ssi \phi(x)\in U_B.\]
De plus $\phi$ induit un isomorphisme du groupe $(U_A,\times)$ dans le groupe $(U_B,\times)$.
\end{proposition}

\subsection{Corps}

\begin{definition}
On dit qu'un anneau $\p{\K,+,\times}$ est un \emph{corps} lorsque
\begin{itemize}
\item $1\neq 0$
\item $\times$ est commutative
\item Tout élément non nul de $\K$ admet un inverse pour la loi $\times$.
\end{itemize}
\end{definition}

\begin{exemples}
\exemple Muni des lois usuelles d'addition et de multiplication, $\C$ est un corps.
\exemple Soit $\mathbb{F}_2$ l'ensemble à deux éléments $\ens{\bar{0},\bar{1}}$. On
  définit sur $\mathbb{F}_2$ les lois $+$ et $\times$ par
  \begin{center}
  \begin{tabular}{|c|c|c|}
  \hline
  $+$\vphantom{\Large{A}} & $\bar{0}$ & $\bar{1}$\\
  \hline
  $\bar{0}$\vphantom{\Large{A}} & $\bar{0}$ & $\bar{1}$\\
  \hline
  $\bar{1}$\vphantom{\Large{A}} & $\bar{1}$ & $\bar{0}$\\
  \hline
  \end{tabular}
  \hspace{2cm}
  \begin{tabular}{|c|c|c|}
  \hline
  $\times$\vphantom{\Large{A}} & $\bar{0}$ & $\bar{1}$\\
  \hline
  $\bar{0}$\vphantom{\Large{A}} & $\bar{0}$ & $\bar{0}$\\
  \hline
  $\bar{1}$\vphantom{\Large{A}} & $\bar{0}$ & $\bar{1}$\\
  \hline
  \end{tabular}
  \end{center}
  Alors $(\mathbb{F}_2,+,\times)$ est un corps.
\end{exemples}

\begin{remarqueUnique}
\remarque Si $\K$ est un corps, l'ensemble des
  inversibles de $\K$ est $\Ks\defeq\K\setminus\ens{0}$.
\end{remarqueUnique}

\begin{proposition}
Un corps est intègre.  
\end{proposition}



\begin{definition}
Soit $\p{\KL,+,\times}$ un corps et $\K$ une partie de $\KL$. On dit que $\K$
est un \emph{sous-corps} de $\KL$ lorsque
\begin{itemize}
\item $\K$ est un sous-anneau de $\KL$
\item $\forall x\in\K\setminus\ens{0} \qsep x^{-1}\in\K$.
\end{itemize}
Si tel est le cas, $\p{\K,+,\times}$ est un corps.
\end{definition}

\begin{remarqueUnique}
\remarque $\Q$ et $\R$ sont des sous-corps de $\C$.
\remarque Si $\K$ est un sous-corps de $\C$, alors $\Q\subset \K$.    
\end{remarqueUnique}

\begin{definition}
Si $\p{\K,+,\times}$ et $\p{\KL,+,\times}$ sont deux corps, on appelle morphisme
de corps de $\K$ dans $\KL$ tout morphisme d'anneau pour les structures
sous-jacentes.
\end{definition}

\begin{remarques}
\remarque Si $\phi$ est un morphisme d'un sous-corps $\K$ de $\C$ dans un sous-corps
  $\KL$ de $\C$, alors
  \[\forall r\in\Q\qsep \forall x\in\K\qsep \phi(rx)=r\phi(x).\]
\remarque Si $\phi$ est un morphisme de corps, alors $\phi$ est injective.
\end{remarques}
\begin{sol}
On considère $\phi:\K\to \KL$ un morphisme de corps. En particulier, c'est un morphisme de groupes. Raisonnons par l'absurde et supposons qu'il existe $x\neq 0$ tel que $\phi(x)=0_{\KL}$. Comme $x$ est différent de $0$ et que $\K$ est un corps, alors $xx^{-1}=1_{\K}$ donc $$0_{\KL}=0_{\KL}\phi(x^{-1})=\phi(x)\phi(x^{-1})=\phi(xx^{-1})=\phi(1_{\K})=1_{\KL}$$ ce qui est absurde car $\KL$ est un corps et donc $0_{\KL}\neq 1_{\KL}$.
\end{sol}


\begin{exoUnique}
\exo Déterminer les morphismes de corps $\phi$ de $\C$ dans $\C$ tels
  que~: $\forall x\in\R \qsep \phi(x)=x$.
\end{exoUnique}

\begin{sol}
On va montrer que $\phi$ est ou bien l'identité ou bien la conjugaison. On raisonne par analyse-synthèse.
\begin{itemize}
\item[$\bullet$] \textbf{Analyse :} On a $\phi(i)^2=\phi(i^2)=\phi(-1)=-\phi(1)=-1$ donc $\phi(i)=i$ ou $\phi(i)=-i$. On montre alors que si $\phi(i)=i$ alors $\phi$ est l'identité et si $\phi(i)=-i$, c'est la conjugaison.
\item[$\bullet$] \textbf{Synthèse :} On vérifie que ce sont bien des morphismes de corps qui vérifie l'hypothèse.
\end{itemize}
\end{sol}

\begin{definition}
Soit $\K$ un corps. Alors, l'application
\[\dspappli{\phi}{\Z}{\K}{k}{k\cdot 1_{\K}}\]
est un morphisme du groupe $(\Z,+)$ dans $(\K,+)$. Il existe donc un unique $p\in\N$ tel que $\ker\phi=p\Z$.
L'entier $p$ est soit nul soit un nombre premier et est appelé \emph{caractéristique} de $\K$.
\end{definition}

\begin{remarques}
\remarque Les sous-corps de $\C$ sont de caractéristique nulle. Le corps $\mathbb{F}_2$ est de caractéristique 2.
\remarque Lorsque $\K$ est un sous-corps de $\C$, pour tout $k\in\Z$, on a $k\cdot 1_{\K}=k$. Si $\K$ est un corps
  quelconque, on confondra le plus souvent $k\cdot 1_{\K}$ et $k$.
\end{remarques}

\section{Espace vectoriel, algèbre}
\subsection{Espace vectoriel}


\begin{definition}[utile=-3]
Soit $\K$ un corps, $\p{E,+}$ un groupe commutatif d'élément neutre $0_E$ et
$\cdot$ une loi de composition externe.
\[\dspappli{\cdot}{\K\times E}{E}{\p{\lambda,x}}{\lambda\cdot x}\]
On dit que $\p{E,+,\cdot}$ est un \emph{\Kev} lorsque
\begin{eqnarray*}
\forall x,y\in E \qsep \forall \lambda\in\K, & &
  \lambda\cdot\p{x+y}=\lambda\cdot x+\lambda\cdot y\\
\forall x\in E \qsep \forall \lambda,\mu\in\K, & &
  \p{\lambda+\mu}\cdot x=\lambda\cdot x+\mu\cdot x\\
\forall x\in E \qsep \forall \lambda,\mu\in\K, & &
  \lambda\cdot\p{\mu\cdot x}=\p{\lambda\mu}\cdot x\\
\forall x\in E, & & 1\cdot x=x.
\end{eqnarray*}
Les éléments de $\K$ sont appelés \emph{scalaires}, ceux de $E$, \emph{vecteurs}.
\end{definition}

\begin{remarqueUnique}
\remarque L'ensemble du cours sur les espaces vectoriels reste valide pour un corps quelconque, excepté le paragraphe
  sur les symétries qui n'est vrai que pour les corps de caractéristique différente de 2.
\end{remarqueUnique}

\begin{proposition}[utile=1]
Soit $\p{E,+,\cdot}$ un \Lev et $\K$ un sous-corps de $\KL$. Alors
$\p{E,+,\cdot}$ est un \Kev. En particulier $\KL$ est un \Kev. 
\end{proposition}

\begin{remarques}
\remarque[utile=-2] $\C$ est un \Rev.
\remarque[utile=-1] Muni des lois usuelles, $\mathcal{F}(\R,\C)$ est un \Cev.
  Comme $\R$ est un sous-corps de $\C$, $\mathcal{F}(\R,\C)$ est aussi un \Rev.
\end{remarques}


\subsection{Algèbre}

\begin{definition}
On dit qu'un anneau $\p{A,+,\times}$ muni d'une loi de composition externe
$\cdot$ sur un corps $\K$ est une $\K$-algèbre lorsque
\begin{itemize}
\item $\p{A,+,\cdot}$ est un \Kev
\item $\times$ est compatible avec la loi de composition externe
  \[\forall x,y\in A \qsep \forall \lambda\in\K \qsep
    \p{\lambda\cdot x}\times y=x\times\p{\lambda\cdot y}=
     \lambda\cdot\p{x\times y}\]
\end{itemize}
On dit que l'algèbre $\p{A,+,\cdot,\times}$ est commutative lorsque $\times$
est commutatif.
\end{definition}

\begin{exemples}
\exemple $\K$ est une $\K$-algèbre.
\exemple Soit $X$ un ensemble. Alors $(\mathcal{F}(X,\K),+,\cdot,\times)$ est une $\K$-algèbre. En particulier, l'ensemble des fonctions de $\R$ dans $\R$ est une $\R$-algèbre et l'ensemble des suites réelles est une $\R$-algèbre.
\end{exemples}
% Exemples :
% 1) La R-algèbre des fonctions de R dans R

\begin{definition}
Soit $\p{A,+,\cdot,\times}$ et $\p{B,+,\cdot,\times}$ deux $\K$-algèbre. On dit qu'une application $\phi$ de $A$ dans $B$ est un morphisme d'algèbre lorsque c'est un morphisme
d'anneau et une application linéaire, c'est-à-dire lorsque
\begin{eqnarray*}
  \forall x,y\in A \qsep \forall \lambda,\mu\in\K, & &
    \phi\p{\lambda x+\mu y}=\lambda \phi(x)+\mu \phi(y)\\
  \forall x,y\in A, & &
    \phi\p{xy}=\phi(x)\phi(y)\\
  & & \phi(1_A)=1_B.
  \end{eqnarray*}
\end{definition}


\begin{proposition}
Soit $E$ un \Kev. Alors $(\Endo{E},+,\cdot,\circ)$ est une $\K$-algèbre.
\end{proposition}

\begin{remarqueUnique}
\remarque Le groupe des inversibles de $\Endo{E}$ est $(\gl{}{E},\circ)$.
\end{remarqueUnique}

\begin{proposition}
Soit $\K$ un corps et $n\in\N$. Alors $(\mat{n}{\K},+,\cdot,\times)$ est une $\K$-algèbre.
\end{proposition}

\begin{remarqueUnique}
\remarque Le groupe des inversibles de $\mat{n}{\K}$ est $(\gl{n}{\K},\times)$.
\end{remarqueUnique}

\section{L'algèbre $\polyK$}

\subsection{Définition}
\begin{definition}
Soit $\K$ un corps. Alors il existe une unique algèbre commutative $\polyK$ ainsi
qu'un élément $X\in\polyK$, appelé \emph{indéterminée}, tels que
\begin{itemize}
\item Pour tout $P\in\polyK$, il existe $n\in\N$ et $a_0,\ldots,a_n\in\K$ tels
  que
  \[P=a_0+a_1 X+\cdots+a_n X^n\]
  où, par abus de notation, $a_0=a_0\cdot 1_{\polyK}=a_0 X^0$.
\item Pour tout $n\in\N$ et $a_0,\ldots,a_n\in\K$
  \[a_0+a_1X+\cdots+a_n X^n=0 \quad\implique\quad  a_0=\cdots=a_n=0.\]
\end{itemize}
On l'appelle \emph{algèbre des polynômes à coefficients dans $\K$}.
\end{definition}

\begin{remarques}
\remarque Soit $P\in\polyK$ et $a_0,\ldots,a_n\in\K$ tels que
  $P=a_0+a_1 X+\cdots+a_n X^n$.
  Il est d'usage de définir $a_k$ pour tout $k>n$ en posant $a_k\defeq 0$. Ainsi,
  quel que soit $m\geq n$
  \[P=\sum_{k=0}^m a_kX^k.\]
\remarque Soit $P\in\polyK$, $a_0,\ldots,a_n\in\K$ et $b_0,\ldots,b_m\in\K$ tels
  que $P=a_0+a_1X+\cdots+a_n X^n$ et $P=b_0+b_1X+\cdots+b_m X^m$. Alors,
  en prolongeant la définition des $a_k$ et $b_k$ comme dans la remarque précédente, on a
  \[\forall k\in\N\qsep a_k=b_k.\]
  La suite $(a_k)_{k\in\N}$ est nulle à partir d'un certain rang et est appelée suite des \emph{coefficients de $P$}.
  Deux polynômes sont égaux si
  et seulement si ils ont les même coefficients.
  % Si on prolonge
  % les définitions des suites $a$ et $b$ en posant $a_k=0$ pour $k>n$ et $b_k=0$
  % pour $k>m$, alors les suites $\p{a_k}$ et $\p{b_k}$ sont égales.
  % On dit que les $a_k$ sont les coefficients du polynôme $P$.
% \remarque Deux polynômes sont égaux si et seulement si ils ont les mêmes
%   coefficients.
% \remarque L'application de $\K$ dans $\polyK$ qui à $\lambda\in\K$ associe
%   $\lambda\cdot 1_{\polyK}\in\polyK$ est un morphisme injectif d'algèbre. Il
%   permet
%   d'identifier $\K$ à une sous-algèbre de $\polyK$. Par abus de langage, on
%   dit que $\K$ est une sous-algèbre de $\polyK$.
\remarque Les coefficients d'un produit de deux polynômes se calculent par
  la formule
  \[\p{\sum_{k=0}^n a_k X^k}\p{\sum_{k=0}^m b_k X^k}=
    \sum_{k=0}^{n+m} \p{\sum_{l=0}^k a_l b_{k-l}}X^k\]
\end{remarques}

\begin{exoUnique}
\exo Soit $n\in\N$. Montrer que
  \[\sum_{k=0}^{n} \binom{n}{k}^2=\binom{2n}{n}\]
  en calculant $(1+X)^{2n}$ de deux manières différentes.
  \begin{sol}
  Il suffit de développer la relation $\p{1+X}^{2n}=\p{1+X}^n\p{1+X}^n$ avec
  le binôme de Newton.
    \end{sol}
  
\end{exoUnique}

\subsection{Substitution}

\begin{definition}
Soit $\mathcal{A}$ une $\K$-algèbre, $x\in\mathcal{A}$ et
$P=a_0+a_1X+\cdots+a_nX^n\in\polyK$. On définit $P(x)$ par
\[P(x)\defeq a_0 1_{\mathcal{A}}+a_1 x+\cdots+a_n x^n\in\mathcal{A}.\]
On dit que l'on a substitué l'élément $x\in\mathcal{A}$ à l'indéterminée $X$.
\end{definition}

\begin{remarques}
\remarque Si $\mathcal{A}$ une $\K$-algèbre et $x\in\mathcal{A}$,
  l'application $\phi$ de $\polyK$ dans $\mathcal{A}$ qui à $P$ associe
  $P(x)$ est un morphisme d'algèbre. Autrement dit
  \begin{eqnarray*}
  \forall P,Q\in\polyK \qsep \forall \lambda,\mu\in\K, & &
    \p{\lambda P+\mu Q}(x)=\lambda P(x)+\mu Q(x)\\
  \forall P,Q\in\polyK, & &
    \p{PQ}(x)=P(x)Q(x)\\
  & & 1_{\polyK}(x)=1_{\mathcal{A}}.
  \end{eqnarray*}
% \remarque Si $x\in\mathcal{A}$ et $n\in\Ns$, le calcul naïf de $x^n$ nécessite
%   $n-1$ multiplications dans $\mathcal{A}$. Si
%   $P=a_0+a_1X+\cdots+a_nX^n\in\polyK$, le calcul de $P(x)$ nécessite
%   donc $n\p{n-1}/2$ multiplications dans $\mathcal{A}$. Cependant, si on
%   écrit
%   \[P(x)=\p{\p{\cdots\p{\p{a_n x+a_{n-1}}x+a_{n-2}}x+\cdots+a_2}x+a_1}x+a_0\]
%   le calcul de $P(x)$ nécessite $n-1$ multiplications dans $\mathcal{A}$.
%   Cette méthode de calcul est connue sous le nom d'algorithme de Hörner.
\remarque On dit qu'un polynôme $P$ est un polynôme annulateur de
  $x\in\mathcal{A}$ lorsque $P(x)=0$. Par exemple, si $\K=\R$ et
  $\mathcal{A}=\C$, $P\defeq X^2+1$ est un polynôme annulateur de $\ii$. Si $E$
  est un \Kev et si $s\in\Endo{E}$ est une symétrie, alors $P\defeq X^2-1$ est un
  polynôme annulateur de $s$.
\remarque On dit qu'un élément $z\in\C$ est algébrique lorsqu'il existe un polynôme
  non nul $P\in\polyQ$ tel que $P(z)=0$. Par exemple $z_1=\p{1+\sqrt{5}}/2$
  est algébrique car $P_1\defeq X^2-X-1\in\polyQ$
  est un polynôme annulateur de $z_1$. De même,
  $\jj$ est algébrique car $P_2\defeq X^3-1\in\polyQ$
  est un polynôme annulateur de $\jj$. Lorsqu'on
  effectue des calculs avec un nombre algébrique $z$, il est souvent plus économe en
  calculs d'exploiter le fait que $P(z)=0$ plutôt que de remplacer $z$ par une
  expression parfois complexe. Par exemple, si $x\defeq\p{1+\sqrt{5}}/2$, en exploitant le fait que
  $x^2=x+1$, on a
  \[\p{\frac{1+\sqrt{5}}{2}}^3
    =x^3=x\cdot x^2=x\p{x+1}=x^2+x=2x+1=2+\sqrt{5}\]
  Si on souhaite calculer $1/x$, on exploite le fait que $x^2-x-1=0$, ce qui donne $x\p{x-1}=1$, puis $1/x=x-1$. Donc
  \[\frac{1}{\p{\frac{1+\sqrt{5}}{2}}}=\frac{-1+\sqrt{5}}{2}.\]

\remarque On dit qu'un élément de $\C$ est \emph{transcendant} lorsqu'il n'est pas
  algébrique. On peut montrer, mais c'est difficile, que $\e$ et $\pi$ sont
  transcendants.
\end{remarques}

\begin{exoUnique}
\exo Montrer que $1+\sqrt{7}$ et $\sqrt{2}+\sqrt{5}$ sont algébriques.
  \begin{sol}
  On pose $x=1+\sqrt{7}$ et on part de $(x-1)^2=7$. De même avec $x=\sqrt{2}+\sqrt{5}$, on l'élève au carré puis on élève $x^2-7$ au carré.
  Les polynômes minimaux sont $X^2-2X-6$ et $X^4-14X^2+9$.
  \end{sol}
\exo Soit $D\defeq\diag{\lambda_1,\ldots,\lambda_n}\in\mat{n}{\K}$ et $P\in\polyK$. Montrer que
  $P(D)=\diag{P\p{\lambda_1},\ldots,P\p{\lambda_n}}$.  
  \begin{sol}
  L'expliquer et ajouter qu'alors $P=\prod_{k=1}^n(X-\lambda_k)$ est un polynôme annulateur de $D$.
  \end{sol}
\end{exoUnique}

%% Algorithme de Hörner
%%
%% - exemple : calcul de 1+2x-x^2+3x^3 : 3 multiplications entre réels
%%             1+x(2+x(-1+x*3))        : 2 multiplications entre réels
%% - implémentation en Maple
%%   Si P est une liste contenant les coefficients
%%
%%   horner:=proc(P,x)
%%     if P:=[] then
%%       return(0);
%%     else
%%       return(P[1]+x*horner(P[2..-1],x));
%%     end if;
%%   end proc;

\begin{definition}
Soit $P,Q\in\polyK$. On définit le polynôme $P\circ Q$ par
\[P\circ Q\defeq P(Q).\]  
\end{definition}

\begin{remarqueUnique}
\remarque Si $P\in\polyK$, $P(X)=P$. Un polynôme peut donc indifféremment
  être noté $P$ ou $P(X)$.
\end{remarqueUnique}

% \begin{definition}
% Soit $\mathcal{A}$ une $\K$-algèbre et $x\in\mathcal{A}$. Alors,
% il existe une plus petite sous-algèbre de $\mathcal{A}$ contenant $x$; on
% l'appelle algèbre engendrée par $x$ et on la note $\K[x]$. De plus~:
% \[\K[x]=\enstq{P(x)}{P\in\polyK}\]
% En particulier $\K[x]$ est commutative.
% \end{definition}

% \begin{definition}
% Soit $x\in\mathcal{A}$ et $P\in\polyK$. On dit que $P$ est un polynôme
% annulateur de $x$ lorsque $P(x)=0$.
% \end{definition}

\begin{definition}
Soit $P\in\polyK$. On dit que
\begin{itemize}
\item $P$ est \emph{pair} lorsque $P\p{-X}=P(X)$.
\item $P$ est \emph{impair} lorsque $P\p{-X}=-P(X)$.
\end{itemize}
\end{definition}

\begin{proposition}
Soit $\K$ un corps qui n'est pas de caractéristique 2 et $P\in\polyK$. Alors
\begin{itemize}
\item $P$ est pair si et seulement si ses coefficients d'indices impairs sont
  nuls.
\item $P$ est impair si et seulement si ses coefficients d'indices pairs sont
  nuls.
\end{itemize}
\end{proposition}

% \begin{remarques}
% \remarque Cette proposition est fausse lorsque $\K$ est un corps où
%   $1_\K+1_\K=0_\K$
%   (on parle de corps de caractéristique 2) ce qui ne sera jamais le cas en
%   {\sc MPSI}, les corps considérés étant des sous-corps de $\C$.
% \end{remarques}

\subsection{Degré d'un polynôme}

\begin{definition}
Soit $P\in\polyK$. On définit le \emph{degré} de $P$ que l'on note $\deg P$ par
\begin{itemize}
\item Si $P=0$, on pose $\deg P\defeq -\infty$.
\item Sinon, il existe $n\in\N$ et $a_0,\ldots,a_n\in\K$ tels que
  \[P=a_0+a_1X+\cdots+a_n X^n \et a_n\neq 0.\]
  De plus $n$ et les $a_0,\ldots,a_n$ sont uniques; on pose alors $\deg P\defeq n$.
  Le coefficient $a_n$ est appelé \emph{coefficient dominant} de $P$.
\end{itemize}
\end{definition}

\begin{preuve}
L'unicité de $n$ et des $a_i$ provient directement du second axiome de la première définition du cours de $\polyK$, ce qu'on peut démontrer.
\end{preuve}

\begin{remarques}
\remarque Si $P\in\polyK$ est non nul, son coefficient dominant est parfois noté
  ${\rm cd}(P)$.
\remarque Un polynôme $P\in\polyK$ est de degré inférieur ou égal à $n\in\N$
  si et seulement si il existe $a_0,\ldots,a_n\in\K$ tels que
  \[P=\sum_{k=0}^n a_k X^k.\]
\remarque On dit qu'un polynôme $P$ est constant lorsqu'il existe $\lambda\in\K$ tel
  que $P=\lambda$, c'est-à-dire lorsque son degré est inférieur ou égal à 0.
\end{remarques}

\begin{proposition}
Soit $P,Q\in\polyK$ et $n\in\N$.
\begin{itemize}
\item Soit $\lambda,\mu\in\K$. Si $\deg P\leq n$ et $\deg Q\leq n$, alors
  \[\deg\p{\lambda P+\mu Q}\leq n.\]
\item Soit $\lambda\in\Ks$ et $\mu\in\K$. Si $\deg P=n$ et $\deg Q<n$, alors
  \[\deg\p{\lambda P+\mu Q}=n \quad\et\quad {\rm cd}(\lambda P+\mu Q)=\lambda{\rm cd}(P).\]
\end{itemize}
\end{proposition}

\begin{remarqueUnique}
\remarque Lorsque $P$ et $Q$ sont des polynômes de degré $n$, il est possible que
  $P+Q$ soit de degré strictement inférieur à $n$. Par exemple $P\defeq X+1$ et
  $Q\defeq -X$ sont de degré 1 mais $P+Q=1$ est de degré 0.
\end{remarqueUnique}

\begin{exoUnique}
\exo Soit $P\in\polyK$. Calculer le degré de $P\p{X+1}-P(X)$ en fonction de celui de $P$.
  \begin{sol}
  est égal à $\deg(P)-1$ si $\deg P\geq 1$ et à $-\infty$ sinon.  
  \end{sol}
\end{exoUnique}

\begin{definition}
Soit $n\in\N$. On note $\polyK[n]$ l'espace vectoriel des polynômes de degré inférieur ou égal à $n$.
\end{definition}

\begin{remarqueUnique}
\remarque Si $n\geq 1$, $\polyK[n]$ n'est pas stable par produit. En effet,
  $X^n\in\polyK[n]$ mais $X^{2n}=X^{n}\cdot X^n\not\in\polyK[n]$.
\end{remarqueUnique}

% \begin{proposition}
% Soit $n\in\N$. Alors $1,X,\ldots,X^n$ est une base de $\polyK[n]$ appelée base
% canonique. En particulier, $\polyK[n]$ est de dimension finie et~:
% \[\dim \polyK[n]=n+1\]
% \end{proposition}

% \begin{proposition}
% Soit $P_0,\ldots,P_n\in\polyK[n]$ une famille de polynômes tels que~:
% \[\forall k\in\intere{0}{n} \quad \deg P_k=k\]
% Alors $P_0,\ldots,P_n$ est une base de $\polyK[n]$.
% \end{proposition}

\begin{proposition}
Soit $P,Q\in\polyK$. Alors
\[\deg\p{PQ}=\deg P+\deg Q.\]
\end{proposition}

\begin{preuve}
L'égalité de la proposition tient compte de la convention $-\infty+(-\infty)=-\infty$ et $-\infty+n=-\infty$.
On évacue tout d'abord le cas où $P$ ou $Q$ est nul.
Ensuite, on peut poser $p=\deg P$, $q=\deg Q$ et écrire $$P=\sum_{n\in \N}a_nX_n \et  Q=\sum_{n\in \N}b_nX_n \et PQ=\sum_{n\in \N}c_nX_n.$$
Soit $n\geq p+q+1$, on a :
$$c_n=\sum_{k=0}^pa_k\underbrace{b_{n-k}}_{=0 (n-k\geq q+1)}+\sum_{k=p+1}^n\underbrace{a_k}_{=0}b_{n-k}$$
et de plus on montre que $c_{p+q}$ est non nul en séparant la somme entre les termes avant $p-1$ et après $p+1$ qui sont tous nuls et $a_pb_q\neq 0$.
\end{preuve}

\begin{remarques}
\remarque Si $P\in\polyK$ est non nul et si $n\in\N$, alors $\deg\p{P^n}=n\deg P$.
\remarque Si $P\in\polyK$ et $Q\in\polyK$ n'est pas constant,
  alors $\deg\p{P\circ Q}=\deg(P)\deg(Q)$.
\end{remarques}

\begin{proposition}
$\polyK$ est une algèbre intègre
\[\forall P,Q\in\polyK \qsep PQ=0 \quad\implique\quad \cro{P=0 \ou Q=0}.\]
\end{proposition}

\begin{proposition}
Les éléments inversibles de $\polyK$ sont les polynômes de degré 0, c'est-à-dire
les polynômes constants non nuls.
\end{proposition}

\begin{definition}
On dit qu'un polynôme non nul $U$ est unitaire lorsque son coefficient dominant est
égal à 1. Tout polynôme $P$ non nul s'écrit de manière unique sous la forme
$P=\lambda P_u$ où $\lambda\neq 0$ et $P_u$ est unitaire. Lorsque $P=0$, on pose
par convention $P_u\defeq 0$.
\end{definition}


\begin{definition}
  Soit $A,B\in\polyK$ avec $B\neq 0$. Alors, il existe un unique couple
  $\p{Q,R}\in\polyK^2$ tel que
  \[A=QB+R \et \deg R<\deg B.\]
  $Q$ est appelé \emph{quotient} de la division euclidienne de $A$ par $B$, $R$ son
  \emph{reste}.
  \end{definition}
  
  \begin{preuve}
  Pour l'existence, on fait une preuve effective fournissant un algorithme :
  
  Fixons $B\in \polyK$, $B\neq 0$. On définit $H_n : " \forall A \in \polyK[n], \exists (Q,R)\in \polyK^2 \text{ tq } A=BQ+R \et \deg R<\deg B".$.
  \begin{itemize}
  \item[$\bullet$] $H_0$ : Soit $A\in \polyK[0]=\K$. Si $\deg B=0$, $A=(AB^{-1})B+0$ convient. Sinon, $A=0B+A$ convient.
  \item[$\bullet$] Hérédité : On suppose $\deg A=n+1$, sinon, c'est bon d'après l'HR. Si $\deg A<\deg B$, $A=0B+A$ convient.
  \'Ecrivons alors $$A=\sum_{k=0}^{n+1}a_k X^k \et B=\sum_{k=0}^{d}b_k X^k$$ avec $a_{n+1}\neq 0$, $b_d\neq 0$ et $d\leq n+1$.
  
  Posons $A_1=A-\dfrac{a_{n+1}}{b_d}X^{n+1-d}B$. $A_1\in \polyK[n]$ donc d'après l'HR, $A_1=BQ_1+R$ et donc $$A=B\p{Q_1+\dfrac{a_{n+1}}{b_d}X^{n+1-d}}+R.$$
  \end{itemize}
  \end{preuve}
  
  \begin{remarques}
  \remarque Si $B$ est un polynôme annulateur non nul de $x$ et $A\in\polyK$.
    Alors $A(x)=R(x)$ où $R$ est le reste de la division euclidienne de $A$
    par $B$. En effet
    \[A(x)=Q(x)\underbrace{B(x)}_{=0}+R(x)\]
  \exo Il est parfois utile de calculer le reste de la division euclidienne
    de $A$ par $B$ sans calculer son quotient.
      Par exemple, si $A\defeq X^n$
      et $B\defeq \p{X-1}\p{X-2}$, le reste $R$ de la division euclidienne de $A$ par $B$
      est de degré inférieur ou égal à 1 donc il existe $a,b\in\R$ tels que
      $R=aX+b$. Comme $A=QB+R$, on en déduit que
      $A\p{1}=Q\p{1}B\p{1}+R\p{1}$. Comme $B\p{1}=0$, on a $A\p{1}=R\p{1}$. De
      même $A\p{2}=R\p{2}$. Donc
      \[\syslin{a&+b&=&1\hfill\cr
               2a&+b&=&2^n.\hfill}\]
      On en déduit que $a=2^n-1$ et $b=2-2^n$. Donc $R=\p{2^n-1}X+\p{2-2^n}$.
      Cette méthode fonctionne dès que le polynôme $B$, de degré $n$, admet $n$
      racines deux à deux distinctes.\\
      Si $A\defeq X^n$
      et $B\defeq \p{X-1}^2$, le reste $R$ de la division euclidienne de $A$ par $B$
      est de degré inférieur ou égal à 1 donc il existe $a,b\in\R$ tels que
      $R=aX+b$. Comme plus haut, $A\p{1}=R\p{1}$. En dérivant la relation
      $A=QB+R$, on obtient $A'=B'Q+BQ'+R'$. Puisque 1 est racine de $B$ et de $B'$, on
      en déduit que $A'\p{1}=R'\p{1}$. Donc
      \[\syslin{a&+b&=&1\hfill\cr
                a&  &=&n.\hfill}\]
      On en déduit que $a=n$ et $b=1-n$, donc $R=nX+\p{1-n}$.
  \end{remarques}
  
  \begin{exos}
  \exo Calculer $x^5+x^4-1$ où $x\defeq(1+\sqrt{5})/2$.
    \begin{sol}
    On a $X^5+X^4-1=(X^3+2X^2+3X+5)(X^2-X-1)+(8X+4)$. En particulier,
    si $x=(1+\sqrt{5})/2$
    \[\p{\frac{1+\sqrt{5}}{2}}^5+\p{\frac{1+\sqrt{5}}{2}}^4-1=x^5+x^4-1=8x+4=
      8+4\sqrt{5}\]  
    \end{sol}
  \exo Montrer que le polynôme $P\defeq X^3+pX+q\in\polyR$ admet 3 racines réelles
    deux à deux distinctes si et seulement si $4p^3+27q^2<0$.
    \begin{sol}
    On a déjà forcément une racine réelle. Ensuite (faire un dessin), il nous faut regarder les annulations de la dérivée. Pour qu'il y ait trois racines réelles distinctes, il nous faut deux racines pour la dérivée $x_0$ et $x_1$ telles que $P(x_1)P(x_0)<0$.
    
    $P'=3X^2+p$ ne s'annule pas si $p>0$ (dans ce cas, la fonction est strictement croissante et n'admet donc pas trois racines distinctes). Si $p\leq 0$, alors $P'$ s'annule en $x_0$ et $x_1$ vérifiant de plus $x_0+x_1=0$ et $x_0x_1=\frac{p}{3}$. On cherche à calculer $P(x_1)P(x_0)$. On peut faire la DE de $P$ par $P'$ puis on trouve $P(x_1)P(x_0)=\frac{1}{27}\p{4p^3+27q^2}$.
    \end{sol}
      \exo On pose
      \[A\defeq
      \begin{pmatrix}
      4 & -1\\
      2 & 1
      \end{pmatrix}\]
      Vérifier que $A^2-5A+6I_2=0$ puis calculer $A^n$ pour tout $n\in\N$.
      \begin{sol}
      Après avoir vérifié l'égalité, on fait la DE de $X^n$ par $X^2-5X+6=(X-3)(X-2)$ où seul le reste nous intéresse.
      On trouve
      \[A^n=
      \begin{pmatrix}
      -2^n+2\cdot 3^n & 2^n-3^n\\
      -2\cdot 2^n+2\cdot 3^n & 2\cdot 2^n-3^n
      \end{pmatrix}\]
      \end{sol}
  \end{exos}

\subsection{Racines, fonctions polynomiales}

\begin{definition}
Soit $P\in\polyK$. On appelle racine de $P$ tout élément $\alpha\in\K$ tel que
$P\p{\alpha}=0$.
\end{definition}

\begin{remarques}
\remarque Les polynômes de degré 1 admettent une unique racine.
\remarque D'après le théorème des valeurs intermédiaires, tout polynôme
  réel de degré impair admet (au moins) une racine réelle.
\remarque La notion de racine dépend du corps considéré. En effet, si on le
  considère comme élément de $\polyC$, les racines de \mbox{$(X^2-2)(X^2+1)$} sont
  $\sqrt{2},-\sqrt{2},\ii,-\ii$. Considéré comme élément de $\polyR$,
  ses racines sont $\sqrt{2},-\sqrt{2}$. Enfin il n'a aucune racine si
  on le considère comme un élément de $\polyQ$.
\remarque Si $\K$ est un sous-corps de $\KL$, $P\in\polyK$
  et $\alpha\in\KL$, on dit que $\alpha$ est une racine de $P$ sur $\KL$
  lorsque $P\p{\alpha}=0$.
\end{remarques}

\begin{proposition}
Si $n\in\N$, tout polynôme de degré $n$ admet au plus $n$ racines.
\end{proposition}

\begin{preuve}
Se démontre par récurrence sur $n$ avec la proposition comme prédicat. Pour l'hérédité, ou bien il n'y a pas de racines auquel cas c'est tout bon, ou bien il en existe une $\alpha$ et on peut alors écrire $P=P(X)-P(\alpha)$ qui en écrivant avec les coefficients va faire apparaitre des $X^k-\alpha^k$ qu'on peut tous factoriser par $X-\alpha$. On a donc écrit $P=(X-\alpha)Q$. Reste à conclure.
\end{preuve}

\begin{remarques}
\remarque On en déduit qu'un polynôme de degré inférieur ou égal à $n$
  admettant $n+1$ racines deux à deux distinctes est nul. De même, si deux
  polynômes de degrés inférieurs ou égaux à $n$ prennent la même valeur
  en $n+1$ points deux à deux distincts, alors ils sont égaux.
\remarque Un polynôme admettant une infinité de racines est donc nul. De même,
  deux polynômes prenant la même valeur sur un ensemble infini sont
  égaux.
\end{remarques}

\begin{exos}
\exo Montrer que les polynômes de $\polyK$ tels que $P(X)=P(X+1)$ sont les
  polynômes constants.
  \begin{sol}
  On peut utiliser l'exercice d'avant en utilisant les degrés. Ou bien on montre que $P-P(0)$ admet une  infinité de racines.
  \end{sol}
\exo Montrer qu'il n'existe pas de polynôme $P\in\polyC$ tel que, pour tout
  $z\in\C$, $P(z)=\conj{z}$.
  \begin{sol}
  Tous les réels sont racines de $P(X)-X$.
  \end{sol}
% \exo On se donne $n+1$ éléments de $\K$ deux à deux distincts
%   $x_0,x_1,\ldots,x_n$ et $y_0,\ldots,y_n\in\K$. Montrer qu'il existe un unique
%   polynôme $P$ de degré inférieur ou égal à $n$ tel que
%   \[\forall k\in\intere{0}{n} \quad P\p{x_k}=y_k\]
%   On dit que $P$ est le polynôme interpolateur de Lagrange associé aux
%   familles $\p{x_k}$ et $\p{y_k}$.
%   \begin{sol}
%   On a l'unicité en en prenant deux et en montrant que le polynôme de la différence admet trop de racines. 
%   Pour l'existence, on peut poser $\forall k \in \llbracket 0,n \rrbracket$ $$L_k=\prod_{\underset{i\neq k}{i=0}}^n\p{\frac{X-x_i}{x_k-x_i}}.$$
%   On peut le faire sentir car on a besoin de $L_k(x_j)=\delta_{jk}$.
%   \end{sol}
\end{exos}

\begin{proposition}[nom={Polyôme interpolateur de \nom{Lagrange}}]
Soit $x_0,\ldots,x_n\in\K$, $n+1$ éléments deux à deux distincts et $y_0,\ldots,y_n\in\K$. Alors,
il existe un unique polynôme $P$ de degré inférieur ou égal à $n$ tel que
\[\forall i\in\intere{0}{n} \qsep P(x_i)=y_i.\]
On l'appelle polynôme interpolateur de \nom{Lagrange} associé aux familles $\p{x_0,\ldots,x_{n}}$ et $\p{y_0,\ldots,y_{n}}$.
\end{proposition}

\begin{remarqueUnique}
\remarque  Pour tout $i\in\intere{0}{n}$, on note
$L_i$ le polynôme défini par
\[L_i\defeq \prod_{\substack{k=0\\k\neq i}}^{n} \frac{X-x_k}{x_i-x_k}.\]
Alors, le polynôme interpolateur de \nom{Lagrange} associé aux familles $\p{x_0,\ldots,x_{n}}$ et $\p{y_0,\ldots,y_{n}}$ est donné par
\[P=\sum_{i=0}^{n} y_i L_i.\]
\end{remarqueUnique}

% \begin{proposition}
% Soit $x_0,\ldots,x_{n}\in\K$, $n+1$ éléments deux à deux distincts.
% \end{proposition}

\begin{preuve}
On a l'unicité en en prenant deux et en montrant que le polynôme de la différence admet trop de racines. 
  Pour l'existence, on peut poser $\forall k \in \llbracket 0,n \rrbracket$ $$L_k=\prod_{\underset{i\neq k}{i=0}}^n\p{\frac{X-x_i}{x_k-x_i}}.$$
  On peut le faire sentir car on a besoin de $L_k(x_j)=\delta_{jk}$.
  \end{preuve}

% \begin{remarqueUnique}
% \remarque Soit $x_0,\ldots,x_{n}\in\K$ deux à deux distincts, $y_0,\ldots,y_{n}\in\K$ et $P\in\polyK$. Alors
%   \[\forall i\in\intere{0}{n} \qsep P(x_k)=y_k\]
%   si et seulement si il existe un polynôme $Q\in\polyK$ tel que
%   \[P=\sum_{i=0}^{n} y_i L_i + Q\prod_{k=0}^{n} \p{X-x_k}.\]
% \end{remarqueUnique}


\begin{definition}
On dit qu'une application $f:\K\to\K$ est une fonction polynomiale lorsqu'il existe
$P\in\polyK$ tel que
\[\forall x\in\K \qsep f(x)=P(x).\]
% L'ensemble des fonctions polynomiales est une sous-algèbre de
% $\mathcal{F}\p{\K,\K}$.
\end{definition}

\begin{proposition}
Si $\K$ est infini,
l'application de l'algèbre $\polyK$ dans l'algèbre $\mathcal{F}(\K,\K)$, qui au polynôme $P$ associe la fonction polynomiale
$\tilde{P}$, est injective.  
\end{proposition}

\begin{preuve}
Si $P$ et $Q$ ont la même image, alors $P-Q$ admet une infinité (car $\K$ est infini) de racines donc $P=Q$.
\end{preuve} 

\begin{remarques}
% \remarque Lorsque $\K$ est un sous-corps de $\C$, l'algèbre des fonctions polynomiales hérite donc des propriétés de
%   $\polyK$. En particulier, elle est intègre, contrairement à l'algèbre
%   $\mathcal{F}\p{\K,\K}$.
\remarque Cette proposition permet, lorsque $\K$ est infini, d'identifier polynômes et
  fonctions polynomiales. C'est pourquoi certains énoncés se
  permettent de confondre polynômes et fonctions polynomiales, identification
  que nous ne ferons que lorsque l'énoncé le demande explicitement.
\remarque Cette proposition est fausse lorsque le corps $\K$ est fini. En effet,
  si $\K=\ens{a_1,\ldots,a_n}$, le polynôme
  \[P\defeq\prod_{k=1}^n \p{X-a_k}\]
  est non nul car $\deg P=n$, mais la fonction polynomiale associée est nulle.
\end{remarques}

\subsection{Polynôme dérivé}

\begin{definition}
Soit $P=a_0+a_1 X+\cdots+a_n X^n\in\polyK$. On définit le polynôme dérivé de
$P$ par
\begin{eqnarray*}
P' &\defeq& a_1 + 2a_2 X+\cdots+na_nX^{n-1}\\
   &=& \sum_{k=1}^n ka_k X^{k-1}.
\end{eqnarray*}
\end{definition}

\begin{remarqueUnique}
\remarque Dans le cas où $\K=\R$, la fonction polynomiale associée à $P'$ est
  la dérivée de la fonction polynomiale associée à $P$.
\end{remarqueUnique}

\begin{proposition}
Soit $P,Q\in\polyK$ et $\lambda,\mu\in\K$. Alors
\[\p{\lambda P+\mu Q}'=\lambda P'+\mu Q' \qquad
  \p{PQ}'=P'Q+PQ' \et \p{P\circ Q}'=Q'\p{P'\circ Q}.\]
\end{proposition}

\begin{definition}
Soit $P\in\polyK$. On définit par récurrence la dérivée $n$-ième de $P$ par
\begin{itemize}
\item $P^{(0)}\defeq P$
\item $\forall n\in\N \qsep P^{(n+1)}\defeq \cro{P^{(n)}}'$.
\end{itemize}
\end{definition}

\begin{remarqueUnique}
\remarque Soit $n\in\N$. Alors
  \[\forall k\in\N \qsep \p{X^n}^{(k)}=
    \begin{cases}
    \frac{n!}{\p{n-k}!}X^{n-k} &\text{si $k\leq n$}\\
    0 & \text{sinon.}
    \end{cases}\]
  En particulier, si $P=a_0+a_1X+\cdots+a_nX^n\in\polyK$, alors
  \[\forall k\in\N \qsep P^{(k)}\p{0}=k! a_k.\]
\end{remarqueUnique}

\begin{proposition}
Soit $P,Q\in\polyK$ et $n\in\N$
\begin{itemize}
\item Soit $\lambda,\mu\in\K$. Alors
  \[\p{\lambda P+\mu Q}^{(n)}=\lambda P^{(n)}+\mu Q^{(n)}.\]
\item On a
  \[\p{PQ}^{(n)}=\sum_{k=0}^n \binom{n}{k}P^{(n-k)}Q^{(k)}.\]
  Cette formule est appelée formule de \nom{Leibniz}.
\end{itemize}
\end{proposition}

\begin{exoUnique}
\exo Calculer $(X^2P)^{(n)}$ en fonction des dérivées successives de $P$.
\end{exoUnique}

\begin{proposition}
Soit $\K$ est un corps de caractéristique nulle et $P\in\polyK$. Alors
\[\deg P'=
  \begin{cases}
  \deg(P)-1 & \text{si $\deg P\geq 1$,}\\
  -\infty & \text{sinon.}
  \end{cases}\]
\end{proposition}

\begin{preuve}
Notons qu'on se place ici dans un corps de caractéristique nulle. Par exemple, dans $\Z/p\Z$, $P=X^p$ est de degré $p$ mais son polynôme dérivé est nul.
\end{preuve}

\begin{remarques}
\remarque $P'=0$ si et seulement si $P$ est constant.
\remarque Pour tout $P\in\polyK$, $\deg P'\leq\deg(P)-1$.
\remarque Soit $P\in\polyK$ et $n\in\N$. Alors 
\[\deg P^{(n)}=
  \begin{cases}
  \deg(P)-n & \text{si $\deg P\geq n$,}\\
  -\infty & \text{sinon.}
  \end{cases}\]
  En particulier, $\deg P^{(n)}\leq\deg(P)-n$.
\end{remarques}

\begin{proposition}[nom={\nom{Taylor-Young}}]
Soit $\K$ un corps de caractéristique nulle, $P$ un polynôme de degré inférieur ou égal à $n$ et $\alpha\in\K$. Alors
\[P=\sum_{k=0}^n \frac{P^{(k)}\p{\alpha}}{k!}\p{X-\alpha}^k.\]  
\end{proposition}


\begin{preuve}
\begin{eqnarray*}
P&=&\sum_{k=0}^n a_kX^k\\
&=& \sum_{k=0}^n a_k\p{X-\alpha+\alpha}^k\\
&=&\sum_{k=0}^n a_k\sum_{i=0}^k\binom{k}{i}(X-\alpha)^i\alpha^{k-i}\\
&=&\sum_{i=0}^n\p{\sum_{k=i}^n \binom{k}{i}a_k\alpha^{k-i}}(X-\alpha)^i\\
&=&\sum_{i=0}^n\frac{1}{i!}\underbrace{\p{\sum_{k=i}^n \frac{k!}{(k-i)!}a_k\alpha^{k-i}}}_{=P^{(i)}(\alpha)}(X-\alpha)^i
\end{eqnarray*}
\end{preuve}
% \begin{remarques}
% \remarque Si $\K$ est un corps pour lequel il existe $n\in\Ns$ tel que
%   \[\underbrace{1_\K+1_\K+\cdots+1_\K}_{\text{$n$ fois}}=0_\K\]
%   (on parle de corps de caractéristique non nulle), les deux 
%   propositions précédentes deviennent fausses. Cependant, ce ne sera pas le cas
%   en {\sc MPSI}, les corps considérés étant des sous-corps de $\C$.  
% \end{remarques}

%END_BOOK

\end{document}