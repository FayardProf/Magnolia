\documentclass{magnolia}


\magtex{tex_driver={pdftex},
        tex_packages={xypic}}
\magfiche{document_nom={Cours sur le dénombrement},
          auteur_nom={François Fayard},
          auteur_mail={fayard.prof@gmail.com}}
\magcours{cours_matiere={maths},
          cours_niveau={mpsi},
          cours_chapitre_numero={10},
          cours_chapitre={Dénombrement}}
\magmisenpage{}
\maglieudiff{}
\magprocess


\begin{document}

%BEGIN_BOOK
\hfill\includegraphics[width=0.45\textwidth]{../../Commun/Images/maths-cours-gaston.png}

\magtoc

\section{Cardinal}

\subsection{Équipotence}

\begin{definition}
Soit $A$ et $B$ deux ensembles. On dit que $A$ est \emph{équipotent} à $B$
lorsqu'il existe une bijection de $A$ dans $B$.
\end{definition}

\begin{proposition}
La relation \og est équipotent à \fg est une relation d'équivalence.
\end{proposition}

\begin{remarques}
\remarque Une fois que nous aurons défini le cardinal d'un ensemble fini, nous verrons que deux ensembles
  finis sont équipotents si et seulement si ils ont le même nombre d'éléments.
\remarque Il existe des ensembles infinis qui ne sont pas équipotents. Par exemple, on peut montrer que,
  quel que soit l'ensemble $X$, les ensembles $X$ et $\mathcal{P}(X)$ ne sont pas équipotents. En particulier
  $\N$ et $\mathcal{P}(\N)$ ne sont pas équipotents. Il existe donc des ensembles infinis qui ont
  \og plus d'éléments \fg que d'autres.
\remarque Il est possible qu'un ensemble soit équipotent à l'une de ses parties strictes. Par exemple
  l'application $f$ de $\N$ dans $\Ns$ qui à $n$ associe $n+1$ est une bijection, ce qui montre que $\N$
  est équipotent à $\Ns$. Pourtant $\Ns$ est une partie stricte de $\N$. De même, l'application $f$
  de $\interf{0}{1}$ dans $\interf{0}{2}$ qui à $x$ associe $2x$ est une bijection. Pourtant
  $\interf{0}{1}$ est une partie stricte de $\interf{0}{2}$.
\remarque On dit qu'un ensemble est \emph{dénombrable} lorsqu'il est équipotent à $\N$. On peut montrer
  que $\Z$, $\N^n$ (pour $n\in\Ns$) et $\Q$ sont dénombrables. On peut montrer cependant que $\R$ n'est
  pas dénombrable.  On dit qu'un ensemble est \emph{au plus dénombrable} lorsqu'il est fini ou dénombrable. 
\end{remarques}

\subsection{Ensemble fini, cardinal}

On rappelle que pour tout $n\in\N$, on a $\intere{1}{n}\defeq\enstq{k\in\N}{1\leq k\leq n}$.
En particulier $\intere{1}{0}=\emptyset$, $\intere{1}{1}=\ens{1}$,
$\intere{1}{2}=\ens{1, 2}$, $\intere{1}{3}=\ens{1, 2, 3}$, etc. 
\vspace{2ex}
\begin{definition}
On dit qu'un ensemble $A$ est \emph{fini} lorsqu'il existe $n\in\N$ tel que $A$ est
équipotent à $\intere{1}{n}$. On dit qu'il est \emph{infini} dans le cas contraire.
\end{definition}

\begin{definition}
Soit $A$ un ensemble fini. Alors il existe un unique $n\in\N$ tel que $A$ est
équipotent à $\intere{1}{n}$. On l'appelle \emph{cardinal} de $A$ et on le note
$\card(A)$ ou $|A|$.
\end{definition}

\begin{preuve}
On commence par l'admettre sur le dessin. Ensuite~:
\begin{itemize}
\item Existence : par définition
\item Unicité : On dit que A est de cardinal n lorsqu'il existe une bijection
  entre [1,n] et A.
  On va montrer par récurrence sur n que [1,n] ne peut être de
  cardinal n et de cardial p avec n<p.
  \begin{itemize}
  \item $H_0$ : A ne peut être vide et non vide
  \item $H_n \implique H_(n+1)$ : Supposons que A soit de cardinal n+1 et de
     cardinal p avec
     n+1<p. Si x est un élément de A $A\setminus\ens{x}$ est de cardinal n et de
     cardinal p-1 avec n<p-1. C'est absurde
  \end{itemize}
\end{itemize}
\end{preuve}

\begin{remarques}
  \remarque L'ensemble vide est fini et son cardinal est nul.
  \remarque Soit $a,b\in\Z$ tels que $a\leq b + 1$. L'ensemble $\intere{a}{b}$ est de
  cardinal $b-a+1$.
\end{remarques}

\begin{proposition}
Soit $A$ un ensemble fini et $B$ un ensemble. Alors, $A$ et $B$ sont équipotents si et
seulement si $B$ est fini et $\card(A)=\card(B)$.  
\end{proposition}


\begin{exoUnique}
\exo Dénombrer les couples $(a,b) \in \N^2$ tels que $3a + b = 833$.
\exo On a utilisé 6921 chiffres (les caractères d'imprimerie) pour numéroter les
  pages d'un dictionnaire. Combien de pages ce dictionnaire contient-il~?
  Chaque page est numérotée une seule fois, la première portant le numéro 1.
  \begin{sol}
  La réponse est 2007.
  \end{sol}
\end{exoUnique}

\begin{definition}
Soit $A$ une partie de $\N$.
\begin{itemize}
\item Si $A$ est fini, il est l'image d'une unique application
  strictement croissante de $\intere{1}{\card(A)}$ dans $\N$.
\item Sinon, $A$ est infini et il est l'image d'une unique application strictement
  croissante de $\N$ dans $\N$.
\end{itemize}
Une telle application est appelée une \emph{énumération} de $A$.
\end{definition}

\begin{preuve}
Utiliser l'algorithme de construction
\end{preuve}

% \begin{remarqueUnique}
% \remarque Si $\phi$ est une application strictement croissante de $\N$ dans
%   $\N$, alors, pour tout $k\in\N$, $\phi(k) \geq k$.
% \end{remarqueUnique}

\begin{proposition}
Une partie de $\N$ est finie si et seulement si elle est majorée.
\end{proposition}

\begin{preuve}
Si elle est finie, elle est majorée.
Réciproquement, si elle est majorée, elle est incluse dans $[1,n]$ donc il ne
peut y avoir d'application strictement croissante de $N$ dans $A$.  
\end{preuve}

% \begin{proposition}
% Toute partie non vide majorée de $\N$ admet un plus grand élément.
% \end{proposition}

% \begin{exoUnique}
% \exo Soit $P$ un prédicat sur $\N$ tel que
%   \begin{itemize}
%   \item $P(n)$ est vrai pour une infinité de $n$.
%   \item Pour tout $n\in\Ns$, si $P(n)$ est vrai, alors $P(n-1)$ est vrai.
%   \end{itemize}
%   Montrer que $P(n)$ est vrai pour tout $n\in\N$.
% \end{exoUnique}

\begin{proposition}
Soit $E$ un ensemble fini et $A$ une partie de $E$. Alors
\begin{itemize}
\item $A$ est un ensemble fini et $\card(A)\leq\card(E)$.
\item $A=E$ si et seulement si $\card(A)=\card(E)$.
\end{itemize}
\end{proposition}

\begin{preuve}
Admis (voir Tissier, Acx, Desnoux page 33) :
1) On sa ramène à la démonstration du fait que toute partie de [1,n] est
   finie de cardinal inférieur à n. C'est est évident car elle est majorée.
    et on utilise l'application strictement croissante.
\end{preuve}


\begin{proposition}
Soit $E$ et $F$ deux ensembles. Alors
\begin{itemize}
\item Si $F$ est fini, il existe une injection de $E$ dans $F$ si et seulement
  si $E$ est fini et $\card (E)\leq\card (F)$.
\item Si $E$ est fini et $F$ est non vide, il existe une surjection de $E$ dans $F$ si et seulement
  si $F$ est fini et $\card (F)\leq\card (E)$.
\item Si l'un des ensembles est fini, il existe une bijection de $E$ dans $F$ si
  et seulement si l'autre est fini et $\card(E)=\card(F)$.
\end{itemize}
\end{proposition}

\begin{proposition}[nom={Principe des tiroirs}]
Soit $E$ et $F$ deux ensembles finis tels que $\card(F)<\card(E)$ et $f$ une application de $E$ dans $F$. Alors, il existe
$x_1,x_2\in E$ tels que $x_1\neq x_2$ et $f(x_1)=f(x_2)$.
\end{proposition}

% \begin{remarqueUnique}
% % \remarque Si $A$ et $B$ sont finis, il existe une injection de $A$ dans $B$ si
% %   et seulement si $\card A=\card B$.
% \remarque Soit $E$ et $F$ deux ensembles finis et $f\in\mathcal{F}(E,F)$. Si $\card(F)< \card(E)$, alors il existe deux éléments distincts de
%   $E$ qui ont même image par $f$. Cet énoncé est connu sous le nom de \og
%   principe des tiroirs \fg ou \og pigeonhole principle \fg.
% \end{remarqueUnique}

\begin{exos}
\exo Soit $n \geq 2$. En supposant que la relation \og est ami avec \fg est
  symétrique, montrer que dans une assemblée de $n$ personnes, il y en a au moins deux qui
  ont le même nombre d'amis.
% \exo De combien de personnes a-t-on besoin pour pouvoir affirmer avec certitude que $2$ sont nées le même jour de l'année ? De même avec $3$.
% \exo Montrer qu'il existe des entiers $a,b,c$ non tous nuls de valeur absolue strictement inférieure à $10^{6}$ tels que $|a+b \sqrt{2}+ c \sqrt{3}| < 10^{-11}$.
\exo Soit $x_1,\ldots,x_{n+1}\in\interf{0}{1}$. Montrer qu'il existe
  $i,j\in\intere{1}{n+1}$ tels que $i\neq j$ et
  \[\abs{x_i-x_j}\leq\frac{1}{n}.\]
% A mettre dans le cours sur les réels
%

\end{exos}

\begin{proposition}
Soit $E$ un ensemble fini, $F$ un ensemble et $f$ une application de $E$ dans
$F$. Alors
\begin{itemize}
\item $f(E)$ est un ensemble fini et $\card\p{f(E)} \leq \card(E)$
\item $\card\p{f(E)}=\card(E)$ si et seulement si $f$ est injective.
\end{itemize}
Si de plus $F$ est un ensemble fini.
\begin{itemize}
\item $\card\p{f(E)}\leq\card(F)$
\item $\card\p{f(E)}=\card(F)$ si et seulement si $f$ est surjective.
\end{itemize}
\end{proposition}





\begin{proposition}
Soit $E$ et $F$ deux ensembles finis et $f$ une application de $E$ dans $F$.
\begin{itemize}
\item Si $f$ est injective et $\card(E)=\card(F)$, alors $f$ est bijective.
\item Si $f$ est surjective et $\card(E)=\card(F)$, alors $f$ est bijective.
\end{itemize}
Autrement dit, si $\card(E)=\card(F)$, alors
\[\text{$f$ est injective} \quad\ssi\quad \text{$f$ est bijective} \quad\ssi\quad
  \text{$f$ est surjective}.\]
\end{proposition}

\section{Dénombrement}

\subsection{Dénombrement élémentaire}

\begin{proposition}
Soit $E$ un ensemble fini et $A$, $B$ deux parties disjointes de $E$, c'est-à-dire telles
que $A\cap B=\emptyset$. Alors
\[\card\p{A\cup B}=\card (A)+\card(B).\]
\end{proposition}

\begin{remarqueUnique}
\remarque Le \og ou exclusif \fg se traduit donc par un $ + $ en dénombrement.
\remarque Si $A$ et $B$ sont deux parties disjointes, leur réunion est parfois notée $A\sqcup B$.
\end{remarqueUnique}

\begin{proposition}
Soit $E$ un ensemble fini.
\begin{itemize}
\item Si $A$ est une partie de $E$
  \[\card\p{\bar{A}}=\card(E)-\card(A).\]
\item Si $A$ et $B$ sont deux parties de $E$
  \[\card\p{A\cup B}=\card(A)+\card(B)-\card\p{A\cap B}.\]
\item Si $(A_1,\ldots,A_n)$ est une partition de $E$, alors
  \[\card(E)=\card\p{A_1}+\cdots+\card\p{A_n}.\]
\end{itemize}
\end{proposition}

%% Preuve :
%% 1) On pourrait le faire
%% 2) On pourrait le faire : commencer par le cas ou A et B sont disjoints
%% 3) Faire une récurrence sur n


\begin{remarqueUnique}
\remarque L'avantage du passage au complémentaire est qu'il permet de prendre la négation
  de la propriété qui définit l'ensemble. Cela donne parfois une phrase plus simple à
  manipuler et donc un ensemble plus simple à dénombrer.
\end{remarqueUnique}

\begin{exos}
\exo Quel est le nombre d'entiers entre $1$ et $100$ qui ne sont pas divisibles par $3$~?
\exo Dénombrer \[A\defeq\enstq{n \in \intere{1}{100}}{2 |n \text{ ou } 3 |n}.\]
\exo Quel est le nombre de rythmes de $n$ temps que l'on peut composer uniquement avec
  des noires (1~temps) et des blanches (2~temps)~?
\end{exos}

\begin{proposition}[nom={Formule du crible}]
Soit $A_1,\ldots,A_n$ des parties d'un même ensemble fini $E$. Alors
  \[\card\p{\bigcup_{i=1}^n A_n}=\sum_{k=1}^n (-1)^{k+1}
    \sum_{1\leq i_1<\cdots<i_k\leq n}
    \card\p{A_{i_1}\cap\cdots\cap A_{i_k}}.\]
\end{proposition}

\begin{remarqueUnique}
\remarque   Par exemple, pour $n=3$, la formule du crible s'écrit
\begin{eqnarray*}
\card\p{A_1\cup A_2\cup A_3}&=&\card(A_1)+\card(A_2)+\card(A_3)\\
  & & - \cro{\card(A_2\cap A_3)+\card(A_1\cap A_3)+\card(A_1\cap A_2)}\\
& &+\card(A_1\cap A_2\cap A_3).
\end{eqnarray*}
\end{remarqueUnique}

\begin{proposition}
\begin{itemize}
\item Si $A$ et $B$ sont deux ensembles finis, alors $A\times B$ est fini et
  \[\card\p{A\times B}=\card(A)\card(B).\]
  Plus généralement, si $A_1,\ldots,A_n$ sont des ensembles finis,
  $A_1\times\cdots\times A_n$ est fini et
  \[\card(A_1\times \cdots\times A_n)=\prod_{k=1}^n \card(A_k).\]
\item Si $A$ est un ensemble fini et $n\in\N$, alors $A^n$ est fini et
  \[\card\p{A^n}=\card(A)^n.\]
\end{itemize}
\end{proposition}


\begin{remarqueUnique}
\remarque La formule $\card\p{A_1 \times \dots \times A_n}=\card(A_1)\cdots\card(A_n)$ dit
  simplement que lorsque l'on doit faire une succession \og et \fg de choix indépendants,
  on dénombre ces choix et on les multiplie.
\end{remarqueUnique}

\begin{exos}
\exo Montrer que dans un village de 700 personnes, deux au moins ont les mêmes initiales.
%   \begin{sol}
%   Il y en a $26^2$.
%   \end{sol}
% \exo On lance 5 dés. Combien de résultats possibles~?
\exo Combien de menus différents peut-on faire avec 4 entrées, 6 plats et 2 desserts~?
\exo Quel est le nombre de mots de 4 lettres contenant au moins un \og e\fg ? 
% \exo Quel est le nombre de mots de 5 lettres contenant un seul \og f\fg, un seul \og z\fg et pas de \og t\fg ? 
\end{exos}
  

\begin{proposition}
\begin{itemize}
\item Soit $E$ et $F$ deux ensembles finis. Alors $\mathcal{F}\p{E,F}=F^E$ est fini et
  \[\card\p{\mathcal{F}\p{E,F}}=\card\p{F^E}=\card(F)^{\card(E)}.\]
\item Soit $E$ un ensemble fini. Alors $\mathcal{P}(E)$ est fini et
  \[\card\p{\mathcal{P}(E)}=2^{\card(E)}.\]
\end{itemize}
\end{proposition}

\begin{preuve}
\begin{itemize}
\item Utiliser le 3 de la question précédente
\item si $E$ est de cardinal $E$, construire une bijection de $F(E,F)$ dans $F^n$
\item Bijection entre $P(E)$ est les applications de $E$ dans $\{0,1\}$
\end{itemize}
\end{preuve}

\begin{exoUnique}
\exo Quel est le nombre de possibilités de répartir $p$ boules distinctes
  dans $n$ urnes distinctes~?
  \begin{sol}
  Il y en a $n^p$.
  \end{sol}
\exo Une urne contient $n$ boules distinctes. On effectue $p$ tirages
  successifs avec remise (c'est-à-dire que l'on remet la boule dans l'urne
  après chaque tirage). Combien y-a-t-il de possibilités~?
  \begin{sol}
  Il y en a $n^p$.
  \end{sol}
\exo De combien de manières peut-on descendre $n+1$ marches (donc $n$
  \og paliers \fg), en en sautant éventuellement certaines~?
  \begin{sol}
  C'est l'ensemble des paliers qui seront sautés. Donc $2^n$.
  \end{sol}
\end{exoUnique}

\begin{proposition}[nom={Lemme des bergers}]
Soit $E$ et $F$ deux ensembles finis et $f$ une surjection de $E$ dans $F$. On
suppose que
\[\forall y_1,y_2\in F \qsep \card\p{f^{-1}\p{\ens{y_1}}}=
  \card\p{f^{-1}\p{\ens{y_2}}}\]
et on note $p$ cette valeur commune. Alors
\[\card(E)=p\card(F).\]
\end{proposition}

\begin{remarqueUnique}
\remarque Soit $E$ est l'ensemble des pattes de mouton foulant un pré, $F$
  l'ensemble des moutons du près et $f$ l'application de $E$ dans $F$ qui à chaque
  patte associe son propriétaire. Comme chaque mouton a 4 pattes, $p=4$.
  On en déduit que le nombre de pattes foulant le près est égal à quatre fois
  le nombre de moutons. C'est de cet exemple que la proposition précédente
  tire son nom de \og principe des bergers \fg.
\end{remarqueUnique}

% \begin{exoUnique}
% \exo Soit $E$ un ensemble de cardinal $n$. Combien existe-t-il de couples
%   $(a,b)\in E^2$ avec $a \neq b$~? Combien existe-t-il de parties de $E$ de cardinal $2$~?
% \end{exoUnique}
  
\subsection{Arrangements, nombre de parties}

% On commence par compter le nombre de manières de choisir $p$ éléments parmi $n$ lorsque
% l'ordre compte.

\begin{definition}[nom={$p$-listes}]
Soient $E$ un ensemble et $p\in\N$. On appelle \emph{$p$-liste} d'éléments de $E$ tout
$p$-uplet $(a_1,\ldots,a_p)\in E^p$.
\end{definition}

\begin{exoUnique}
\exo Si $E=\ens{1,2,3}$, donner les $2$-listes d'éléments de $E$.
\end{exoUnique}

\begin{remarques}
\remarque Les $p$-listes d'éléments de $E$ sont les fonctions de $\intere{1}{p}$ dans $E$.
\remarque Choisir une $p$-liste, c'est choisir $p$ éléments de $E$ en tenant
  compte de  l'\emph{ordre} et en autorisant les \emph{répétitions}.
\end{remarques}

\begin{proposition}[nom={Nombre de listes}]
Soit $E$ un ensemble de cardinal $n$ et $p\in\N$. Alors, il existe
\[n^p\]
$p$-listes d'éléments de $E$.
\end{proposition}

% On compte maintenant le nombre de manières de choisir $p$ éléments \emph{distincts} parmi $n$ où l'\emph{ordre compte}.

\begin{definition}[nom={$p$-arrangements}]
Soit $E$ un ensemble et $p \in \N$. On appelle \emph{$p$-arrangement} d'éléments de $E$
toute $p$-liste $(a_1,\dots,a_p)\in E^p$ telle que
\[\forall i,j\in\intere{1}{p}\qsep i\neq j\quad\implique\quad a_i \neq a_j.\]
\end{definition}

\begin{exoUnique}
\exo Si $E=\ens{1,2,3,4}$, donner toutes les $2$-listes d'éléments distincts de $E$.
\end{exoUnique}

\begin{remarques}
\remarque Les $p$-arrangements d'éléments de $E$ sont les fonctions injectives $\intere{1}{p}$ dans $E$.
\remarque Choisir un $p$-arrangement, c'est choisir $p$ éléments de $E$ en tenant
  compte de  l'\emph{ordre} et en n'autorisant pas les \emph{répétitions}.
% \remarque C'est aussi le nombre de possibilités de choisir de manière ordonnée $p$
%   objets parmi $n$ objets.
\end{remarques}

\begin{proposition}[nom={Nombre d'arrangements}]
Soit $E$ un ensemble de cardinal $n$ et $p\in\N$. Alors il existe
\[A_n^p\defeq
  \begin{cases}
  \dsp\frac{n!}{\p{n-p}!} & \text{si $p\leq n$}\\
  0 & \text{si $p>n$}
  \end{cases}\]
  $p$-arrangements d'éléments de $E$.
\end{proposition}

\begin{preuve}
Récurrence sur $k$. On démontre facilement en utilisant le principe des bergers
que $A(n,k+1)=n.A(n-1,k)$.
\end{preuve}

\begin{exos}
\exo On répartit $p$ boules distinctes dans $n$ urnes distinctes. Quel est
  le nombre de répartitions pour lesquelles aucune urne ne contient plus d'une
  boule~?
  \begin{sol}
  C'est $A_n^p$.
  \end{sol}
\exo Une urne contient $n$ boules distinctes. On effectue $p$ tirages
  successifs sans remise. Combien y-a-t-il de possibilités~?
  \begin{sol}
  C'est $A_n^p$.
  \end{sol}
\end{exos}

\begin{proposition}
Soit $E$ un ensemble à $n$ éléments. Alors, il existe $n!$ bijections de
$E$ dans $E$.  
\end{proposition}

\begin{preuve}
Prendre $k=n$ dans la proposition précédente.
\end{preuve}

\begin{exoUnique}
\exo Combien d'anagrammes peut-on former avec les mots \og maths \fg, \og chimie \fg et
  \og anagramme \fg~?
% \exo $2n$ personnes doivent se placer autour d'une table ronde. Combien de dispositions
%   sont possibles~? On suppose que ces $2n$ personnes sont constituées de $n$ hommes et $n$
%   femmes. Combien de dispositions sont possibles si l'on veut respecter l'alternance
%   homme/femme~?
%   \begin{sol}
%   Si on numérote les places, il y a $(2n)!$ possibilités. Si on ne les numérote pas, il y a
%   $2n$ rotations possibles, donc $(2n-1)!$ possibilités. Si on considère en plus les
%   symétries par rapport à une droite, et $n\geq 2$, il y a $(2n-1)!/2$ possibilités.\\
%   Si on numérote les places et on place une femme sur la première place, il y a $n!^2$
%   possibilités. Sans la contrainte de la première place, il y a $2 n!^2$. Si on ne
%   numérote plus les place, il y a $(n!)^2/(2n)$ possibilités.
%   \end{sol}
\end{exoUnique}

\begin{definition}[nom={$p$-combinaisons}]
Soit $E$ un ensemble et $p\in\N$. On appelle \emph{$p$-combinaison} d'éléments de $E$ toute
partie de $E$ à $p$ éléments.
\end{definition}

\begin{remarqueUnique}
\remarque Choisir une $p$-combinaison, c'est choisir $p$ éléments de $E$ sans tenir compte
  de l'\emph{ordre} et en n'autorisant pas les \emph{répétitions}.
\end{remarqueUnique}

\begin{proposition}[nom={Nombre de combinaisons}]
Soit $E$ un ensemble de cardinal $n$ et $p\in\N$. Alors il existe
\[\binom{n}{p}=
    \begin{cases}
    \dsp\frac{n!}{p!\p{n-p}!} & \text{si $p\leq n$}\\
    0 & {\text{si $p>n$}}
    \end{cases}\]
$p$-combinaisons d'éléments de $E$.
\end{proposition}


\begin{preuve}
  Utiliser le principe des Bergers avec l'application phi qui a une injection
  de $\intere{1}{k}$ dans $E$ associe son image.
  \end{preuve}

\begin{exos}
\exo Une urne contient $6$ boules numérotées de $1$ à $6$. Dénombrer les tirages
  possibles si on tire $3$ boules
  \begin{itemize}
  \question successivement et avec remise.
  \question successivement et sans remise.
  \question simultanément.
  \end{itemize}
\exo Un code de coffre-fort est composé de 6 chiffres entre $0$ et $9$ dont l'ordre
  compte. Dénombrer 
  \begin{itemize}
  \item tous les codes possibles.
  \item les codes dont tous les chiffres sont distincts.
  \item les codes ne contenant pas $0$.
  \item les codes contenant au plus deux fois le chiffre $1$.
  \item les codes contenant autant de chiffres pairs que de chiffres impairs.
  \item les codes contenant la succession $123$ quelque-part.
  \item les codes strictement croissants.
  \end{itemize}
% \exo On dispose de $n\in\Ns$ joueurs et l'on souhaite constituer une équipe de $p\in\Ns$ joueurs dont l'un sera désigné capitaine.
%   Dénombrer de deux fa\c{c}ons les équipes possibles et en d\'eduire
%   \[\binom{n}{p}=\frac{n}{p}\binom{n-1}{p-1}.\]
  % \exo Montrer que si $q \leq n, p$ alors \[{n+p \choose q}=
  % \sum\limits_{k = 0}^q {n \choose q-k} {p \choose k}.\]
  %   Cette identité est appelée identité de Vandermonde.
% À mettre en exo
% \exo Soit $n\in\Ns$. Combien y a-t-il de parties de $\intere{1}{n}$ ne contenant pas
%   deux entiers consécutifs~?
\exo On lance 5 dés. Combien y a-t-il de possibilités pour faire l'une des
  figures suivantes~?
  \begin{itemize}
  \item Un Yam's~: les 5 dés ont la même valeur.
  \item Une suite~: $(1,2,3,4,5)$ ou $(2,3,4,5,6)$.
  \item Un Full~: 3 dés identiques et les 2 autres identiques mais différents des premiers.
  \end{itemize}
\exo Quel est le nombre de manières de répartir $p$ boules indiscernables
dans $n$ urnes distinctes sachant qu'on ne peut pas mettre plus d'une boule par urne~?
  \begin{sol}
  C'est $\binom{n}{p}$.
  \end{sol}  
\exo Déterminer le nombre de solutions de l'équation $x_1+\cdots+x_n=p$ dans
  $\ens{0,1}^n$.
  % , puis dans $\Ns^n$ (on commencera par considérer l'inéquation
  % $x_1+\cdots+x_n\leq p$)~?
  \begin{sol}
  Pour la première question, il y a $\binom{n}{p}$ solutions. Si on associe
  à une solution de $x_1+\cdots+x_n\leq p$ le $n$-uplet
  $(x_1,x_1+x_2,\ldots,x_1+\cdots+x_n)$ on remarque que le nombre de solution
  de l'inéquation est le nombre de fonctions strictement croissantes de
  $\intere{1}{n}$ dans $\intere{1}{p}$. C'est donc $\binom{p}{n}$. Le nombre
  de solutions cherchées est donc
  $\binom{p}{n}-\binom{p-1}{n}=\binom{p-1}{n-1}$.
  \end{sol}
\exo Combien peut-on former de mots contenant $p$ fois
  la lettre O et $q$ fois la lettre I~?
  \begin{sol}
  C'est $\binom{n}{p}$.
  \end{sol}
\exo Quel est le nombre de manières de répartir $p$ boules indiscernables
  dans $n$ urnes distinctes~?
  \begin{sol}
  On peut modéliser cela par \og $..||...|.|..$ \fg. On est ramené un problème
  d'anagramme. On trouve donc $\binom{n+p-1}{p}$.
  \end{sol}
\exo Déterminer le nombre de solutions de l'équation $x_1+\cdots+x_n=p$ dans
  $\N^n$.
\exo Combien y a-t-il d'applications strictement
  croissantes de $\intere{1}{p}$ dans $\intere{1}{n}$~?
  \begin{sol}
  C'est $\binom{n}{p}$.
  \end{sol}
\exo Combien y a-t-il d'applications croissantes de $\intere{1}{p}$ dans $\intere{1}{n}$~?
  \begin{sol}
  Une application croissante est déterminée par le nombre d'antécédents de
  chaque élément de $\intere{1}{n}$. Le cardinal recherché est
  $\binom{n+p-1}{p}$.
  \end{sol}
\exo Soit $n\in\Ns$. Combien y a-t-il de surjections de $\intere{1}{n+1}$
  dans $\intere{1}{n}$~?
\end{exos}

\begin{proposition}
\begin{eqnarray*}
\forall n\in\N \qsep \forall p\in\intere{0}{n}, & &
  \binom{n}{p}=\binom{n}{n-p}\\
\forall n,p\in\N, & & \binom{n}{p}+\binom{n}{p+1}=\binom{n+1}{p+1}\\
\forall n,p\in\Ns, & & \binom{n}{p}=\frac{n}{p}\binom{n-1}{p-1}.
\end{eqnarray*}
\end{proposition}

\begin{preuve}
Utiliser des considérations ensemblistes
\end{preuve}

\begin{remarqueUnique}
\remarque La dernière formule est parfois appelée \og formule du capitaine \fg ou formule du
  \og comité président \fg.
\end{remarqueUnique}

\begin{proposition}
\begin{eqnarray*}
\forall n\in\N, & & \sum_{p=0}^n \binom{n}{p}=2^n\\
\forall a,b,n\in\N, & & \sum_{p=0}^n \binom{a}{p}\binom{b}{n-p}=\binom{a+b}{n}. 
\end{eqnarray*}
\end{proposition}

\begin{remarqueUnique}
\remarque La seconde formule est appelée formule de \nom{Vandermonde}.
\end{remarqueUnique}

\begin{proposition}[nom={Binôme de \nom{Newton}}]
\[\forall a,b\in\C\qsep \forall n\in\N\qsep \p{a+b}^n =\sum_{p=0}^n \binom{n}{p} a^{n-p} b^p.\]
\end{proposition}


% \subsection{Quelques exercices}

% \begin{exos}
% \exo Soit $k \in \N^*$. Combien existe-t-il de manières de colorier les
% arêtes d'un carré avec $k$ couleurs sans que deux arêtes
% adjacentes aient la même couleur ?
% \exo On lance $5$ dés. Quelle est la probabilité de faire l'une des figures suivantes :
% \begin{enumerate}
% \item Yam's : les $5$ dés ont la même valeur.
% \item Suite : $(1,2,3,4,5)$ ou $(2,3,4,5,6)$.
% \item Full : $3$ dés identiques et les $2$ autres identiques mais différents des premiers.
% \end{enumerate}
% \exo Dans le jeu des tours de Hanoï avec $3$ piquets et $n$ disques, de combien de manières peut-on disposer les $n$ disques ?
% \exo Calculer le nombre de parties de $\intere{1}{n}$ ne contenant pas deux entiers
% consécutifs.
% \exo Soit $n \in \N^*$. Combien existe-t-il de surjections de $I_{n+1}$ dans $I_n$ ?
% \exo Soit $n$ et $p \in \N^*$. Dénombrer les applications de $\intere{1}{p}$ dans $\intere{1}{n}$ :
% \begin{enumerate}
% \item strictement croissantes,
% \item croissantes,
% \item monotones.
% \end{enumerate}
% \exo 
% Soient $E$ et $F$ deux ensembles de cardinaux respectifs $n$ et $p$. On note
% $S_n^p$ le nombre de surjections de $E$ dans $F$.
% \begin{enumerate}
% \item Calculer $S_n^1$, $S_n^n$ et $S_n^p$ pour $p>n$.
% \item On suppose $p \leq n$ et on considère $a$ un élément de $E$. En observant qu'une surjection de $E$ dans $F$ réalise ou ne réalise pas
% une surjection de $E \setminus \{a\}$ dans $F$, montrer que
% $$S_n^p=p \left( S_{n-1}^{p-1} + S_{n-1}^p \right)$$
% \item En déduire \[S_n^p=\sum\limits_{k=0}^p (-1)^{p-k} {p \choose k} k^n.\]
% \end{enumerate}
% \end{exos}
%END_BOOK

% \begin{proposition}
% On a la formule du binôme de Newton~:
% \[\forall z_1,z_2\in\C \quad \forall n\in\N \quad
%   \p{z_1+z_2}^n=\sum_{k=0}^n \binom{n}{k} z_1^{n-k} z_2^k\]  
% \end{proposition}

% \begin{exos}
% \exo Soit $n\in\N$. Calculer
%   \[\sum_{\substack{k=0\\k\equiv 0\ [2]}}^n \binom{n}{k} \et
%     \sum_{\substack{k=0\\k\equiv 1\ [2]}}^n \binom {n}{k}\]
% \end{exos}
%END_BOOK

% \subsection{Problèmes classiques de dénombrement}

% \subsubsection{Applications de $\intere{1}{p}$ dans $\intere{1}{n}$}

% \subsubsection{Injections de $\intere{1}{p}$ dans $\intere{1}{n}$}

% \subsubsection{Parties à $p$ éléments d'un ensemble à $n$ éléments}


\end{document}

%%%%%%%%%%%%%%%%%%%%%%%%%%%%%%%%%%%%%%%%%%%%%%%%%%%%%%%%%%%%%%%%%%%%%%%%%%%%%%%%%%%
%%%%%%%%%%%%%%%%%%%%%%%%%%%%%%%%%%%%%%%%%%%%%%%%%%%%%%%%%%%%%%%%%%%%%%%%%%%%%%%%%%%

% \begin{itemize}
% \item {\bf Applications strictement croissantes de $\intere{1}{p}$ dans
%   $\intere{1}{n}$}~:\\
%   Étant donnés $n,p\in\Ns$, on cherche le cardinal de l'ensemble $E$ des fonctions
%   strictement croissantes de $\intere{1}{p}$ dans $\intere{1}{n}$. De telles
%   applications étant injectives, $\card E=0$ dès que $p>n$. On suppose donc dans
%   la suite que $p\leq n$.On remarque alors que l'application (où
%   $\mathcal{P}_p\p{\intere{1}{n}}$ désigne l'ensemble des parties de
%   $\intere{1}{n}$ à $p$ éléments)~:
%   \[\dspappli{\phi}{E}{\mathcal{P}_p\p{\intere{1}{n}}}{f}{\im f}\]
%   est une bijection. En effet~:
%   \begin{itemize}
%   \item $\phi$ est à valeurs dans $\mathcal{P}_p\p{\intere{1}{n}}$~:\\
%     En effet, soit $f$ une application strictement croissante de $\intere{1}{p}$
%     dans $\intere{1}{n}$. Alors $f$ est injective, donc
%     $\card\p{\im f}=\card \p{f\p{\intere{1}{p}}}=p$.
%   \item $\phi$ est injective~:\\
%     En effet, soit $f_1$ et $f_2$ deux applications strictement croissantes de
%     $\intere{1}{p}$ dans $\intere{1}{n}$ telles que $\im f_1=\im f_2$. Alors $f_1$
%     et $f_2$ sont des énumérations de $Y=\im f_1=\im f_2$, donc $f_1=f_2$.
%   \item $\phi$ est surjective~:\\
%     En effet, soit $Y$ une partie à $p$ éléments de $\intere{1}{n}$. Alors, il
%     existe une énumération de $Y$, c'est-à-dire une application strictement 
%     croissante $f$ de $\intere{1}{p}$ dans $\intere{1}{n}$ dont l'image est $Y$.
%     Alors $f\in E$ et $\phi(f)=Y$.
%   \end{itemize}
%   $\phi$ est bijection, donc~:
%   \[\card E=\card\p{\mathcal{P}_p\p{\intere{1}{n}}}=\binom{n}{p}\]
% \end{itemize}


% \end{document}


